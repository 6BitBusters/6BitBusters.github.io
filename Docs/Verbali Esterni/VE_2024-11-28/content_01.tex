% Insert content here
\section{Informazioni generali}
\subsection{Dettagli sull'incontro}
\begin{itemize}
    \item \textbf{Luogo}: Call in Google Meet
    \item \textbf{Data}: 28-11-2024
    \item \textbf{Ora di inizio}: 17:00
    \item \textbf{Ora di fine}: 17:30
    \item \textbf{Partecipante dell'azienda}: Beggiato Alex
    \item \textbf{Partecipanti}:
    \begin{itemize}
        \item Bergamin Elia
        \item Chilese Elena
        \item Diviesti Filippo
        \item Djossa Edgar
        \item Pincin Matteo 
        \item Soranzo Andrea  
    \end{itemize}
\end{itemize}

\section{Motivo della riunione}
\begin{itemize}
    \item Discussione sull'analisi dei requisiti, con particolare focus sui casi d'uso proposti dal gruppo.
    \item Richiesta di chiarimenti riguardanti le tecnologie da utilizzare e approfondimento delle scelte tecnologiche più adeguate.
\end{itemize}

\section{Resoconto della riunione}

\subsection{Discussione sui casi d'uso}
Durante l'incontro, il gruppo ha presentato al proponente i casi d'uso  elaborati, che sono stati valutati positivamente per la maggior parte. Il proponente ha inoltre suggerito di integrare ulteriori casi d'uso e ha fornito indicazioni utili per migliorarne alcuni, tra cui: 
\begin{itemize}
    \item \textbf{Gestione degli errori}:  
    È fondamentale prevedere una gestione adeguata degli errori, in particolare in caso di fallimento delle chiamate API (ad esempio, in caso di servizio non disponibile).
    
    \item \textbf{Selezione delle celle nella visualizzazione 3D}:  
    La possibilità di selezionare singole celle o elementi all'interno del grafico è stata confermata come utile, ma senza rendere obbligatoria la modifica dei dati selezionati.  

    \item \textbf{Navigazione}:  
    È stato sottolineato il bisogno di fornire un'opzione chiara per tornare alla lista dei dataset o delle API dalla visualizzazione dettagliata.  
\end{itemize}

\subsection{Chiarimenti sulle tecnologie}
\begin{itemize}
    \item \textbf{Scelta delle librerie}:  
    La scelta delle librerie dipende da vari fattori, tra cui il supporto ai requisiti funzionali del progetto, la compatibilità con il framework front-end scelto ( \textbf{React} o \textbf{Angular}). 

    \item \textbf{Differenze tra Three.js e D3.js}:  
    \textbf{D3.js} è progettato per la creazione di grafici, mentre \textbf{Three.js} è focalizzato sulla creazione di ambienti 3D. Nonostante trattino aspetti diversi del progetto, è necessario scegliere soltanto una delle due librerie.  

\end{itemize}
