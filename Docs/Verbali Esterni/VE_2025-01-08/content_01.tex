\section{Informazioni generali}
\subsection{Dettagli sull'incontro}
\begin{itemize}
    \item \textbf{Luogo}: Google Meet
    \item \textbf{Data}: 08-01-2025
    \item \textbf{Ora di inizio}: 17:00
    \item \textbf{Ora di fine}: 17:30
    \item \textbf{Partecipante dell'azienda}: Beggiato Alex
    \item \textbf{Partecipanti}:
    \begin{itemize}
        \item Bergamin Elia
        \item Chilese Elena
        \item Diviesti Filippo
        \item Djossa Edgar
        \item Pincin Matteo 
        \item Soranzo Andrea  
    \end{itemize}
\end{itemize}

\section{Motivo della riunione}
\begin{itemize}
    \item Presentazione PoC;
    \item Aggiornamenti sullo stato di avanzamento della documentazione;
    \item Programmazione della prossima riunione.
\end{itemize}

\section{Resoconto della riunione}
\subsection{Presentazione PoC}
Il punto principale della riunione era la presentazione del PoC prodotto durante lo sprint, con l'obiettivo di ottenere dal proponente considerazioni e pareri sull'operato.
La valutazione finale complessiva è stata molto positiva, anche se durante la presentazione si è evidenziato un piccolo bug visivo riguardante il piano del valore medio, risultante invisibile se
osservato da certe angolazioni del grafico.
A tal punto, particolare enfasi ha avuto la discussione con il proponente sul quando risolvere bug di questo tipo. Il dubbio del team era se questi andassero risolti prima della presentazione finale del PoC e quindi nella RTB oppure
se posticiparli direttamente all'MVP. La risposta riscontrata è di effettuare una valutazione di tipo tecnico sulle cause dei bug, capendo se necessitano di studio e competenze aggiuntive per essere risolti oppure se sono facilmente risolvibili.
Nel primo caso vanno risolti prima della presentazione del PoC in quanto parti integranti di esso, mentre nel secondo caso possono essere rimandati.
\subsection{Aggiornamenti sullo stato di avanzamento della documentazione}
Il proponente è stato aggiornato dal team sullo stato di avanzamento della documentazione prevista per l'RTB, facendo più chiarezza anche sul documento di accompagnamento del PoC che il team dovra produrre.
Inoltre sono stati chiesti degli ulteriori chiarimenti sui requisiti che il prodotto finale dorà soddisfare.
Per quanto riguarda i requisiti di validazione, essi andranno dedotti dal capitolato ed in seguito confermati con il proponente.
Per i requisiti prestazionali invece, sarà il team ad occuparsi della loro definizione ragionando sulle aspettative di un utente finale che utilizza il prodotto finito.
\subsection{Programmazione della prossima riunione}
La prossima riunione è stata definita in data 22 gennaio 2025 alle ore 17:00. E' già stato fornito al team il link per accedere all'incontro che si terrà come di consueto
sulla piattaforma Google Meet. Il punto fondamentale della riunione sarà la revisione dei requisiti prestazionali e di validazione definiti dal team in orientamento all'MVP.


\hfill\signature{Approvazione esterna}{Sanmarco Informatica}
