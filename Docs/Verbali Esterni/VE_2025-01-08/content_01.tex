\section{Informazioni generali}
\subsection{Dettagli sull'incontro}
\begin{itemize}
    \item \textbf{Luogo}: Google Meet
    \item \textbf{Data}: 08-01-2025
    \item \textbf{Ora di inizio}: 17:00
    \item \textbf{Ora di fine}: 17:30
    \item \textbf{Partecipante dell'azienda}: Beggiato Alex
    \item \textbf{Partecipanti}:
    \begin{itemize}
        \item Bergamin Elia
        \item Chilese Elena
        \item Diviesti Filippo
        \item Djossa Edgar
        \item Pincin Matteo 
        \item Soranzo Andrea  
    \end{itemize}
\end{itemize}

\section{Motivo della riunione}
\begin{itemize}
    \item Presentazione PoC;
    \item Aggiornamenti sullo stato di avanzamento della documentazione;
    \item Programmazione della prossima riunione.
\end{itemize}

\section{Resoconto della riunione}
\subsection{Presentazione PoC}
Il focus primario della riunione consisteva nell’illustrazione del PoC sviluppato nel corso dello sprint, con l’intento di ottenere dal proponente feedback sull’operato. 
La valutazione è stata altamente positiva. Durante la presentazione è però emersa un’anomalia visiva. Nello specifico, il piano del valore medio si rivelava non percepibile da determinate prospettive di visualizzazione del grafico.
A seguito di ciò, la discussione si è incentrata sulla determinazione del momento più opportuno per risolvere problematiche di questa natura. La perplessità del team riguardava se intervenire su tali bug prima della presentazione ufficiale del PoC o se procrastinarne la risoluzione alla fase di sviluppo dell’MVP. 
La risposta emersa è quella di analizzare le cause sottostanti ai malfunzionamenti, distinguendo tra problemi che necessitano di approfondimenti e studio da quelli di facile risoluzione. Nel primo caso, la correzione risulta imprescindibile prima della presentazione del PoC, in quanto parte integrante di esso. Nel secondo caso, è invece possibile optare per un rinvio.
\subsection{Aggiornamenti sullo stato di avanzamento della documentazione}
Il proponente è stato aggiornato dal team sullo stato di avanzamento della documentazione, facendo più chiarezza anche sul documento di accompagnamento del PoC.
Inoltre sono stati chiesti degli ulteriori chiarimenti sui requisiti che il prodotto finale dovrà soddisfare.
Per quanto riguarda i requisiti di validazione, essi andranno dedotti dal capitolato ed in seguito confermati con il proponente.
Per i requisiti prestazionali invece, sarà il team ad occuparsi della loro definizione ragionando sulle aspettative di un utente finale.
\subsection{Programmazione della prossima riunione}
La prossima riunione è stata definita in data 22 gennaio 2025 alle ore 17:00. E' già stato fornito al team il link per accedere all'incontro che si terrà come di consueto
sulla piattaforma Google Meet. Il punto fondamentale della riunione sarà la revisione dei requisiti prestazionali e di validazione definiti dal team.


\hfill\signature{Approvazione esterna}{Sanmarco Informatica}
