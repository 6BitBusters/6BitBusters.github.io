\section{Informazioni generali}
\subsection{Dettagli sull'incontro}
\begin{itemize}
    \item \textbf{Luogo}: Google Meet
    \item \textbf{Data}: 22-01-2025
    \item \textbf{Ora di inizio}: 17:00
    \item \textbf{Ora di fine}: 17:30
    \item \textbf{Partecipante dell'azienda}: Beggiato Alex
    \item \textbf{Partecipanti}:
    \begin{itemize}
        \item Bergamin Elia
        \item Chilese Elena
        \item Diviesti Filippo
        \item Djossa Edgar
        \item Soranzo Andrea  
    \end{itemize}
\end{itemize}

\section{Motivo della riunione}
\begin{itemize}
    \item Chiarimenti sui test E2E;
    \item Discussione su come reperire i dati per i grafici;
    \item Pianificazione della prossima riunione.
\end{itemize}

\section{Resoconto della riunione}

\subsection{Chiarimenti sui test E2E}
Durante la riunione si è parlato dell'importanza dei test E2E (End-to-End) per garantire la qualità del software prima del rilascio. 
Questi test dovrebbero includere:
\begin{itemize}
    \item Controlli per correggere eventuali errori prima della pubblicazione;
    \item Verifiche sulle funzionalità principali;
    \item Simulazioni d'uso basate sulle aspettative degli utenti.
\end{itemize}
Un esempio pratico è il seguente: se un utente analizza una serie numerica di valori tra 1 e 10 e calcola la media 5, si aspetta che, selezionando la visualizzazione dei valori superiori alla media, vengano mostrati solo quelli maggiori di 5. 
Un ulteriore esempio riguarda l'uso della rotellina del mouse dove l'utente si aspetta di effettuare uno zoom avanti/indietro.
\\Il proponente ha sottolineato l'importanza di una checklist di controllo per verificare che i test coprano tutti gli scenari previsti.

\subsection{Reperimento dei dati per la costruzione dei grafici}
Un altro argomento affrontato è stato il reperimento dei dati per i grafici. 
Il proponente ci ha illustrato le due principali soluzioni:
\begin{itemize}
    \item Sviluppare un backend per la gestione dei dati e delle loro fonti;
    \item Svilppare un frontend che estrae i dati tramite API. 
\end{itemize}
Per la soluzione basata sul backend, è necessario definire una lista di fonti disponibili, garantendo che siano omogenee. 
Ogni fonte, che può essere un database o un'API, riceverà in input un UID selezionato e restituirà una matrice con le etichette associate. 
Sarà implementata un'interfaccia unificata per accedere ai dati in modo che il frontend non debba preoccuparsi della tipologia della fonte. 
\\
Per la soluzione basata solo sul frontend, il codice includerà parte della logica, organizzata attraverso servizi che interagiranno con API pubbliche senza Key per ottenere i dati. 
Le interfacce saranno definite con implementazioni riutilizzabili dalla vista, semplificando l'integrazione con i dati ricevuti e migliorando la modularità dell'applicazione.

Infine, il proponente ha illustrato che, dopo il caricamento del dataset il punto focale è la sincronizzazione tra grafico, tabella e pulsanti azione presenti nell'interfaccia. 
Per fare ciò dovremmo utilizzare librerie specifiche per React, valutando le migliori soluzioni disponibili.

\subsection{Programmazione della prossima riunione}
La prossima riunione sarà calendarizzata in seguito al risultato dell'RTB.

