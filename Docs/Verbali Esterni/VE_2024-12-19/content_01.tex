\section{Informazioni generali}
\subsection{Dettagli sull'incontro}
\begin{itemize}
    \item \textbf{Luogo}: Google Meet
    \item \textbf{Data}: 19-12-2024
    \item \textbf{Ora di inizio}: 17:30
    \item \textbf{Ora di fine}: 18:30
    \item \textbf{Partecipante dell'azienda}: Beggiato Alex
    \item \textbf{Partecipanti}:
    \begin{itemize}
        \item Bergamin Elia
        \item Chilese Elena
        \item Diviesti Filippo
        \item Djossa Edgar
        \item Pincin Matteo 
        \item Soranzo Andrea  
    \end{itemize}
\end{itemize}

\section{Motivo della riunione}
\begin{itemize}
    \item Chiarimenti su casi d'uso e relativi diagrammi UML;
    \item Discussione sui requisiti, con particolare enfasi su eventuali requisiti prestazionali;
    \item Discussione sul PoC, richiesti consigli su come dividersi i compiti all'interno del gruppo e cosa andare a sviluppare.
\end{itemize}

\section{Resoconto della riunione}
\subsection{Dettagli dei dataset}
Dovendo proporre all'utente una lista di dataset tra cui poter scegliere, è stato chiesto al proponente se i dettagli dovessero essere visibili direttamente dalla lista oppure se dovessero essere nascosti e accessibili in un altro modo.\\
Il proponente ha suggerito che, per migliorare l'usabilità, sarebbe preferibile visualizzare i dettagli principali direttamente nella lista, consentendo all'utente di vederli immediatamente. Per i dettagli più specifici, si potrebbe prevedere un'opzione di espansione della casella relativa al dataset, che permetta di visualizzarli in un secondo momento, con la possibilità di espandere o ridurre la sezione tramite un bottone.
\subsection{Legenda e colore delle barre}  
Abbiamo chiesto al proponente se fosse necessario includere una legenda esplicativa per chiarire gli elementi mostrati nel grafico, come i valori degli assi, l'altezza delle barre e il colore delle barre, e, in tal caso, cosa dovrebbe contenere.\\
Il proponente ha risposto che, essendo il grafico già autoesplicativo grazie alle etichette sugli assi che descrivono i valori corrispondenti, non è necessario implementare una legenda.  
\\Per quanto riguarda la colorazione delle barre, è emerso che, trattandosi di un requisito opzionale, non è richiesta una colorazione specifica. Tuttavia, se decideremo di implementarla, questa dovrà essere studiata con attenzione e non sarà casuale. I colori dovranno avere lo scopo di facilitare la comprensione del grafico da parte dell'utente, ad esempio, variando in base all'altezza delle barre.\\
A tal proposito, il proponente ci ha mostrato un esempio di come Excel gestisce la colorazione in base ai valori. In questo esempio, il colore rosso viene utilizzato per rappresentare il valore minimo, mentre il colore si schiarisce progressivamente fino a raggiungere il valore massimo, che è caratterizzato dal colore verde.
\subsection{Funzione hover sopra le barre}
È stato chiesto se, quando si passa il puntatore del mouse sopra una barra (funzione hover), debba essere mostrato solo il valore numerico corrispondente o anche le etichette degli assi X e Y.\\
Per rendere più chiaro ciò che viene visualizzato, ci è stato suggerito di mostrare sia il valore numerico che le etichette corrispondenti agli assi X e Y.
\subsection{Funzione di filtraggio}
Una funzionalità richiesta è quella di evidenziare un certo numero di barre all'interno del grafico, opacizzando quelle che non rientrano nella selezione tramite filtraggio. Questo filtraggio dovrà prevedere quattro tipi di selezione:

\begin{itemize} 
    \item Mostrare le barre corrispondenti ai bottom N valori;
    \item Mostrare le barre corrispondenti ai top N valori;
    \item Mostrare le barre corrispondenti ai valori inferiori al valore medio globale;
    \item Mostrare le barre corrispondenti ai valori superiori al valore medio globale. 
\end{itemize}
Abbiamo chiesto se queste funzionalità siano mutuamente esclusive o se possano essere selezionate contemporaneamente. Inoltre, abbiamo chiesto se il piano parallelo alla base relativo al valore medio globale debba essere sempre visibile.\\
Dalla risposta del proponente è emerso che le funzionalità di filtraggio sono in relazione di esclusività tra loro, quindi ne verrà selezionata solo una alla volta.\\ La visualizzazione del piano relativo al valore medio globale, invece, è indipendente dai filtri. Ci è stato suggerito di implementare una checkbox che permetta all'utente di decidere se visualizzare o meno il piano.
\subsection{Test E2E}
Pur trattandosi di un requisito opzionale ma gradito, e considerando che i membri del gruppo non ne avevano familiarità, abbiamo chiesto una spiegazione sui test E2E (End-to-End), in particolare su cosa siano e come possano essere implementati.\\
Ci è stato spiegato che il testing E2E è fondamentale per garantire che l'applicazione funzioni correttamente dal punto di vista dell'utente finale. Il proponente ha spiegato che questo tipo di test permette di verificare come l'utente interagisce con l'applicazione e assicura che le funzionalità siano correttamente implementate. Inoltre, ha suggerito l'uso di strumenti come Selenium, Cypress e Playwright per automatizzare questi test, che registrano una serie di azioni dell'utente e le ripetono per verificare che l'impatto sia visibile in altre parti dell'applicazione. Un esempio pratico di test E2E potrebbe essere la verifica della lista di dataset, dove ci si aspetta che gli elementi siano visibili e che i dettagli vengano correttamente mostrati al clic.
\subsection{Requisiti prestazionali e compatibilità con i browser}
Abbiamo chiesto se fossero presenti anche dei requisiti prestazionali misurabili, come ad esempio il tempo di caricamento della pagina, e se dovessimo prendere in considerazione la retrocompatibilità con le versioni precedenti dei vari browser.\\
Per quanto riguarda i requisiti prestazionali, ci è stato detto che non ci sono parametri particolari e definiti da rispettare. Tuttavia, è stato richiesto di immedesimarsi nell'utilizzatore dell'applicazione e porsi la domanda: \textit{"Quanto sarei disposto ad aspettare per il caricamento di questo dataset?"}.\\
Proseguendo nella discussione, si è deciso di basarsi sulla dimensione del dataset per ottenere un tempo di caricamento ragionevole, nell'ordine di grandezza dei secondi.\\
A livello di compatibilità con i browser, ci è stato detto che è sufficiente garantire il funzionamento sull'ultima versione e assicurarsi che il prodotto sia utilizzabile con i browser più comuni.\\
\subsection{Suddivisione del lavoro per PoC}
Prima di terminare l'incontro, abbiamo chiesto al proponente qualche consiglio su come suddividere il lavoro di sviluppo del PoC.\\
Ci è stato indicato che, a differenza dell'MVP (il quale richiede una divisione ben strutturata tra backend, frontend e interazione con il dataset), il PoC può essere sviluppato in parallelo, permettendo di testare singolarmente le tecnologie proposte per la realizzazione del progetto.\\
A tal proposito, ci è stato suggerito di concentrarsi sugli aspetti che potrebbero comportare maggiori difficoltà durante lo sviluppo del prodotto finale e di affrontare i punti critici del progetto.\\
Questo approccio consentirà di arrivare alla RTB con la consapevolezza che tutti i requisiti potranno essere soddisfatti.\\


\hfill\signature{Approvazione esterna}{Sanmarco Informatica}
