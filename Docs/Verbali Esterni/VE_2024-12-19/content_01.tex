\section{Informazioni generali}
\subsection{Dettagli sull'incontro}
\begin{itemize}
    \item \textbf{Luogo}: Google Meet
    \item \textbf{Data}: 19-12-2024
    \item \textbf{Ora di inizio}: 17:30
    \item \textbf{Ora di fine}: 18:30
    \item \textbf{Partecipante dell'azienda}: Beggiato Alex
    \item \textbf{Partecipanti}:
    \begin{itemize}
        \item Bergamin Elia
        \item Chilese Elena
        \item Diviesti Filippo
        \item Djossa Edgar
        \item Pincin Matteo 
        \item Soranzo Andrea  
    \end{itemize}
\end{itemize}

\section{Motivo della riunione}
\begin{itemize}
    \item Chiarimenti su casi d'uso e relativi diagrammi UML;
    \item Discussione sui requisiti, con particolare enfasi su eventuali requisiti prestazionali;
    \item Discussione sul PoC, richiesti consigli su come dividersi i compiti all'interno del gruppo e cosa andare a sviluppare.
\end{itemize}

\section{Resoconto della Riunione}
\subsection{Dettagli dei Dataset}
Per permettere all'utente di scegliere un dataset da una lista, abbiamo chiesto al proponente se i dettagli dovessero essere visibili direttamente nella lista o accessibili in altro modo.
Il proponente ha suggerito di mostrare i dettagli principali direttamente nella lista per migliorare l'usabilità. 
I dettagli più specifici, invece, potrebbero essere visualizzati solo su richiesta, tramite un'opzione di espansione della casella del dataset. 
Questa sezione espandibile potrebbe essere gestita con un bottone per espanderla e ridurla.

\subsection{Legenda e Colore delle Barre}
Abbiamo chiesto al proponente se fosse necessario includere una legenda per chiarire gli elementi del grafico, come i valori degli assi, l'altezza delle barre e il colore delle barre.
Secondo il proponente, il grafico è già autoesplicativo grazie alle etichette sugli assi, quindi una legenda non è necessaria.
Per quanto riguarda i colori delle barre, si è discusso del fatto che, essendo un requisito opzionale, non è obbligatorio definire una colorazione specifica. 
Tuttavia, se si decidesse di implementarla, i colori dovrebbero aiutare l'utente a comprendere meglio il grafico. 
Ad esempio, i colori potrebbero variare in base all'altezza delle barre.
Il proponente ha mostrato un esempio di Excel in cui i colori cambiano in base ai valori: il minimo è rappresentato dal rosso, e il colore diventa progressivamente più chiaro fino al massimo, indicato dal verde.

\subsection{Funzione Hover sopra le Barre}
Abbiamo chiesto se, passando il puntatore del mouse sopra una barra (funzione hover), si debba mostrare solo il valore numerico o anche le etichette degli assi X e Y.
Per migliorare la chiarezza, è stato suggerito di mostrare sia il valore numerico sia le etichette degli assi.

\subsection{Funzione di Filtraggio}
Una funzionalità richiesta prevede di evidenziare alcune barre nel grafico, rendendo opache le altre che non rientrano nella selezione. 
Il filtraggio dovrà includere quattro tipi di selezione:
\begin{itemize} 
    \item Mostrare le barre dei bottom N valori;
    \item Mostrare le barre dei top N valori;
    \item Mostrare le barre dei valori inferiori al valore medio globale;
    \item Mostrare le barre dei valori superiori al valore medio globale. 
\end{itemize}
Abbiamo chiesto se queste opzioni di filtraggio siano mutuamente esclusive o se possano essere selezionate contemporaneamente. 
Inoltre, abbiamo chiesto se il piano parallelo alla base relativo al valore medio globale debba essere sempre visibile.\\
Il proponente ha chiarito che le opzioni di filtraggio sono mutuamente esclusive, quindi è possibile selezionarne solo una alla volta.\\ 
La visualizzazione del piano relativo al valore medio globale, invece, è indipendente dai filtri. 
Ci è stato suggerito di implementare una checkbox che permetta all'utente di decidere se visualizzare o meno il piano.

\subsection{Test E2E}
Anche se i test E2E (End-to-End) sono un requisito opzionale, abbiamo chiesto chiarimenti su cosa siano e come implementarli, dato che il gruppo non ne aveva familiarità.
Ci è stato spiegato che i test E2E sono utili per verificare il corretto funzionamento dell’applicazione dal punto di vista dell’utente finale. 
Questo tipo di test controlla le interazioni dell’utente e assicura che le funzionalità siano implementate correttamente.
Il proponente ha suggerito strumenti come Selenium, Cypress e Playwright per automatizzare i test. 
Questi strumenti registrano una sequenza di azioni e le ripetono, verificando che ogni modifica abbia il risultato atteso in altre parti dell’applicazione. 
Un esempio pratico potrebbe essere il controllo della lista dei dataset, dove è necessario verificare che gli elementi siano visibili e che i dettagli vengano mostrati correttamente al clic.

\subsection{Requisiti Prestazionali e Compatibilità con i Browser}
Abbiamo chiesto informazioni sui requisiti prestazionali, come il tempo di caricamento della pagina, e sulla retrocompatibilità con le diverse versioni dei browser.
Per quanto riguarda le prestazioni, non sono stati indicati parametri precisi. 
Tuttavia, è stato suggerito di considerare l’esperienza dell’utente e chiedersi: "Quanto tempo sarei disposto ad aspettare per caricare questo dataset?". 
Si è deciso di basare il tempo di caricamento sulla dimensione del dataset, mantenendolo ragionevole e nell’ordine di grandezza dei secondi.
Per la compatibilità con i browser, è sufficiente garantire il funzionamento dell’applicazione sull’ultima versione dei browser più comuni.

\subsection{Suddivisione del Lavoro per il PoC}
Abbiamo chiesto al proponente consigli su come suddividere il lavoro per lo sviluppo del PoC.
Ci è stato spiegato che, a differenza dell’MVP, che richiede una chiara separazione tra backend, frontend e gestione del dataset, il PoC può essere sviluppato in parallelo. 
Questo approccio permette di testare singolarmente le tecnologie necessarie per il progetto.
È stato consigliato di concentrarsi sugli aspetti più complessi e sui punti critici che potrebbero creare difficoltà durante lo sviluppo del prodotto finale. 
Questo approccio consente di arrivare alla RTB con la certezza che tutti i requisiti potranno essere soddisfatti.

\hfill\signature{Approvazione esterna}{Sanmarco Informatica}
