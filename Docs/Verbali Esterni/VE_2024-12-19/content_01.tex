% Insert content here
\section{Informazioni generali}
\subsection{Dettagli sull'incontro}
\begin{itemize}
    \item \textbf{Luogo}: Zoom
    \item \textbf{Data}: 19-12-2024
    \item \textbf{Ora di inizio}: 9:00
    \item \textbf{Ora di fine}: 9:30
    \item \textbf{Docenti partecipanti}: 
    \begin{itemize}
        \item Prof. Cardin
    \end{itemize}
    \item \textbf{Partecipanti}:
    \begin{itemize}
        \item Bergamin Elia
        \item Chilese Elena
        \item Diviesti Filippo
        \item Djossa Edgar
        \item Pincin Matteo 
        \item Soranzo Andrea  
    \end{itemize}
\end{itemize}

\section{Motivo della riunione}
Questa riunione è stata richiesta dal team per discutere e avere un feedback rispetto ai diagrammi UML dei casi d'uso inseriti nel documento \textit{Analisi dei requisiti}.

\section{Resoconto}
% insert here all the steps
\subsection{Nomi dell'attore}
Evidenziato che il nome dell'attore deve apportare un valore aggiunto, per cui deve essere rivalutato, facendo intendere in maniera esplicita chi ci aspettiamo interagisca con il nostro sistema.

\subsection{Nomi dei requisiti}
Evidenziato che il nome dei requisiti non apporta sufficiente valore al contesto, sono da rivalutare per rendere più comprensibile cosa l'utente voglia fare interagendo con l'applicazione web.\\
Il gruppo ha deciso di rivedere la logica alla base della rappresentazione dei casi d'uso, utilizzando nomi più chiari e specifici che descrivano meglio ciò che ogni caso d'uso rappresenta e ciò che ci aspettiamo possa essere un'esigenza dell'utente che interagisce con il sistema.
\subsection{Termini tecnici e acronimi}
È stato osservato come un acronimo o un termine tecnico (ad esempio, "PAN") possa risultare ambiguo e assumere significati diversi a seconda del lettore, in particolare se si tratta di una persona esterna al contesto del progetto.\\
Il gruppo ha deciso, su suggerimento del Prof. Cardin, di includere nel glossario questi termini e di fornire una definizione esplicita degli acronimi, al fine di offrire un riferimento chiaro che possa eliminare eventuali ambiguità.
\subsection{Relazioni tra casi d'uso}
È stata sollevata la questione riguardo al tipo di relazione da stabilire tra i casi d'uso e i sottocasi, se di inclusione, estensione o generalizzazione. Inoltre, il Prof. Cardin ci ha suggerito di rivedere attentamente quali casi d'uso debbano essere effettivamente collegati tra loro e quali, invece, debbano rimanere indipendenti.\\
Il gruppo ha quindi definito come obiettivo primario un'analisi approfondita di ciascun caso d'uso, con l'intento di esaminare ogni dettaglio in modo accurato e sistematico. L'obiettivo è quello di identificare e stabilire relazioni più precise e coerenti tra i casi d'uso, al fine di garantire una rappresentazione chiara e corretta dei processi e delle interazioni all'interno del sistema.

