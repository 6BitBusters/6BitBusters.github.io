% Insert content here
\section{Informazioni generali}
\subsection{Dettagli sull'incontro}
\begin{itemize}
    \item \textbf{Luogo}: Discord
    \item \textbf{Data}: 14-11-2024
    \item \textbf{Ora di inizio}: 15:00
    \item \textbf{Ora di fine}: 17:30
    \item \textbf{Partecipanti}:
          \begin{itemize}
              \item Bergamin Elia
              \item Chilese Elena
              \item Diviesti Filippo
              \item Djossa Edgar
              \item Pincin Matteo
              \item Soranzo Andrea
          \end{itemize}
\end{itemize}

\section{Motivo della riunione}
In questo incontro abbiamo discusso il primo sprint\textsuperscript{g}
pianificato da Bergamin Elia. Sono stati assegnati i ruoli, è stato accordato
un giorno della settimana ottimale per le riunioni esterne con il proponente,
rivisto il way of working\textsuperscript{g} circa la
verificazione\textsuperscript{g} e approvazione\textsuperscript{g} dei
documenti, optato per un sistema agile\textsuperscript{g} ed infine accordato
il significato del numero di versione.
\section{Resoconto}
\subsection{Retrospettiva}
Inizialmente ogni membro ha comunicato quanto fatto nello scorso periodo: ciò è
stato propedeutico per capire cosa mancasse da fare e decidere, in base a ciò,
anche i ruoli.
\subsubsection{Assegnazione ruoli}
\begin{enumerate}
    \item \textbf{Responsabile}: Bergamin Elia
    \item \textbf{Analista}: Chilese Elena
    \item \textbf{Amministratori}: Soranzo Andrea, Djossa Edgar
    \item \textbf{Verificatori}: Diviesti Filippo, Pincin Matteo
\end{enumerate}

\section{Riunioni esterne}
Successivamente abbiamo analizzato gli impegni di ciascun membro del team per
decidere il giorno ottimale per cercare in cui fare un incontro con il
proponente.\\ Il giorno migliore è il \textbf{giovedì}, che tuttavia coincide
però con il giorno di riunione interna settimanale: abbiamo quindi deciso che le
riunioni interne possano essere effettuate in un giorno variabile scelto in
base agli impegni di ogni membro, concordandolo di volta in volta in modo tale
da essere ottimale per tutti.
\section{Aggiornamento del way of working}
Di seguito una breve sintesi delle decisioni prese. Si raccomanda, per una
descrizione più specifica, di consultare il documento \textit{Norme di
    progetto}.
\subsection{Verifica e approvazione}
\begin{enumerate}
    \item \textbf{Verifica}: è la fase di controllo errori di un documento e avviene per ogni versione dello stesso.
    \item \textbf{Approvazione}: è la fase finale del documento, dove, dopo una serie di modifiche/aggiunte, verrà aggiunto alla pages\textsuperscript{g} e reso pubblico.
\end{enumerate}
\subsection{Modello di lavoro}
Il team ha deciso di adottare un modello di lavoro agile\textsuperscript{g},
nello specifico il modello \textit{Scrum}\textsuperscript{g}, che prevede la
suddivisione del tempo in "brevi" periodi chiamati sprint\textsuperscript{g}

che abbiamo deciso far durare \textbf{2 settimane}
\subsection{Numero di versione}
Il team ha deciso attribuire i numeri di versione in base al tipo di modifica
che viene fatta ad un documento.
\begin{center}
    \textbf{X.Y.Z}
\end{center}
\begin{enumerate}
    \item \textbf{X o Major\textsuperscript{g}}: Approvazione del documento per la revisione e la pubblicazione sulla page\textsuperscript{g}.
    \item \textbf{Y o Minor\textsuperscript{g}}: Modifiche sostanziali al documento come la rimozione/creazione/modifica di una intera sezione o sottosezione di un documento.
    \item \textbf{Z o HotFix\textsuperscript{g}}: Modifiche secondarie al testo con lo scopo, ad esempio, di rendere più comprensibile una frase o un termine o come l'aggiunta di una parola al \textit{Glossario}
\end{enumerate}
Ad ogni versione Minor\textsuperscript{g} corrisponde un branch\textsuperscript{g} secondario rispetto a quello del documento. Solamente un membro può lavorare nel branch\textsuperscript{g} secondario e a termine del lavoro
potrà eseguire una pull request\textsuperscript{g} per richiedere la verifica. Per le versioni di HotFix\textsuperscript{g} invece è sufficiente eseguire una commit\textsuperscript{g} sul branch\textsuperscript{g} del documento.
Così facendo ogni membro del team può lavorare in modo asincrono e la verifica viene eseguita mano a mano senza aspettare la fine di uno sprint\textsuperscript{g}.
\section{Prossimi obiettivi}
\begin{itemize}
    \item Iniziare analisi dei requisiti
    \item Terminare la redazione della versione attuale dei documenti \textit{Piano di
              progetto}, \textit{Norme di Progetto} e \textit{Glossario}
\end{itemize}
