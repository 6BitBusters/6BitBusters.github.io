% Insert content here
\section{Informazioni generali}
    \subsection{Dettagli sull'incontro}
        \begin{itemize}
            \item \textbf{Luogo}: Discord
            \item \textbf{Data}: 05-12-2024
            \item \textbf{Ora di inizio}: 17:30
            \item \textbf{Ora di fine}: 19:30
            \item \textbf{Partecipanti}:
                \begin{itemize}
                    \item Bergamin Elia
                    \item Chilese Elena
                    \item Diviesti Filippo
                    \item Djossa Edgar
                    \item Pincin Matteo
                    \item Soranzo Andrea
                \end{itemize}
        \end{itemize}

\section{Motivo della riunione}
    In questo incontro abbiamo discusso gli obiettivi raggiunti nel secondo sprint\textsubscript{g},
    e quelli che avevamo pianificato, ma che non abbiamo raggiunto. 
    Abbiamo valutato e cercato di migliorare il wow\textsubscript{g}, in particolare
    nell'ambito della stesura dei documenti. 
    Abbiamo vagliato e scelto le tecnologie da adottare per lo sviluppo del PoC, 
    tra le alternative proposte.
    Abbiamo assegnato i ruoli e pianificato le attività per il terzo sprint.

\section{Resoconto}
    \subsection{Retrospettiva}
        \subsubsection{Piano di qualifica}
            Avevamo pianificato di redigere le sezioni Qualità di processo e Qualità di prodotto 
            del \textit{Piano di qualifica}, ma abbiamo constatato che, anche in seguito alle
            lezioni tenute dal prof. Vardanega, ci mancavano degli elementi per poter completare
            la stesura. Nello specifico, ci siamo informati maggiormente sulle cosiddette
            "Metriche di progetto", a carattere prevalentemente economico, e sulle metriche
            che riguardano il prodotto software.
            Abbiamo quindi deciso di redigere tali sezioni nel prossimo sprint.
        \subsubsection{Stesura documenti}
            Abbiamo osservato che, a causa delle diverse consuetudini di ciascun membro del team 
            nella stesura di documenti, si sono verificate delle incoerenze stilistiche e grammaticali nei nostri
            documenti. Abbiamo quindi deciso che la sezione delle \textit{Norme di progetto} che 
            riguarda la documentazione dovrà essere sostanzialmente ampliata per disciplinare 
            nel dettaglio questo processo. Inoltre, per assicurare concordanza di genere per i 
            termini stranieri, andremo ad aggiungere nel \textit{Glossario}, per ogni vocabolo
            straniero, un'indicazione sulla sua morfologia, ovvero se è un sostantivo,
            un aggettivo o un verbo, e nel caso di un sostantivo, se è maschile o femminile.
            
    \subsection{Tecnologie}
        Le tecnologie alternative proposte nel capitolato di progetto erano:
        \begin{itemize}
            \item D3.js o Three.js per la creazione di grafici 3D interattivi;
            \item Angular o React come framework di sviluppo.       
        \end{itemize}
        A seguito dello studio individuale da parte di alcuni membri del gruppo,
        è emerso che Three.js fornisce degli elementi che favoriscono in particolar
        modo la creazione di ambienti 3D, grazie all'utilizzo di WebGL, 
        anche se trascura in gran parte il concetto di dataset. 
        Al contrario, D3.js è una libreria maggiormente orientata
        alla visualizzazione di dati dinamici, che però trascura la terza dimensione,
        in quanto è basata sostanzialmente su SVG, HTML5 e CSS. 
        Considerato che il focus del progetto ricade sulla visualizzazione 3D dei 
        dati tramite grafici navigabili e interattivi, e non è richiesto di modificare
        tali dati, il gruppo ha scelto Three.js come libreria principale per la creazione 
        della grafica 3D.\\
        Per quanto riguarda la scelta del framework, è emerso che Angular è un framework
        completo e strutturato, con varie funzionalità integrate che riducono la necessità di dipendenze esterne;
        è dunque utilizzato prevalentemente per lo sviluppo di progetti grandi e complessi. 
        Viceversa, React è una libreria per la costruzione di interfacce utente, 
        che lascia maggiore libertà nella scelta delle librerie per le altre funzionalità,
        come, ad esempio, il routing; quindi è preferibile per progetti piccoli o medi, con esigenze specifiche.
        Inoltre, React si integra facilmente con Three.js, grazie alle 
        librerie React Three Fiber e Drei, che forniscono dei componenti che ben si 
        prestano alle esigenze del nostro progetto.
        Quindi, considerato che il progetto non è di grandi dimensioni, in particolare per quanto 
        riguarda la logica sottostante, e considerata la scelta di Three.js, il gruppo ha 
        optato per l'utilizzo di React.        
    \subsection{Assegnazione ruoli}
        Abbiamo assegnato i ruoli secondo il criterio di rotazione stabilito inizialmente,
        senza aggiungere o togliere nessun ruolo rispetto allo sprint precedente.
        \begin{itemize}
            \item \textbf{Responsabile}: Diviesti Filippo
            \item \textbf{Analisti}: Bergamin Elia, Soranzo Andrea
            \item \textbf{Amministratore}: Pincin Matteo
            \item \textbf{Verificatori}: Chilese Elena, Djossa Edgar
        \end{itemize}
    

\section{Prossimi obiettivi}
        \subsection{Norme di progetto}
            \begin{itemize}
                \item Aggiornamento convenzioni
                \item Aggiornamento nome requisiti
                \item Accertamento della qualità
                \item Aggiornamento verifica 
                \item Aggiornamento validazione
            \end{itemize}
        \subsection{Piano di progetto}
            \begin{itemize}
                \item Consuntivo sprint 2
                \item Pianificazione e preventivo sprint 3
            \end{itemize}
        \subsection{Piano di qualifica}
            \begin{itemize}
                \item Qualità di processo
                \item Qualità di prodotto
                \item Testing
            \end{itemize}
        \subsection{Analisi dei requisiti}
            \begin{itemize}
                \item Requisiti funzionali
                \item Requisiti di qualità
                \item Requisiti di vincolo*
                \item Requisiti prestazionali* 
            \end{itemize}   
            \begin{small}*Consultare il proponente\textsubscript{g}.
            \end{small} 
        \subsection{Glossario}
            \begin{itemize}
                \item Inserimento nuovi vocaboli
                \item Morfologia dei vocaboli stranieri
            \end{itemize}
        