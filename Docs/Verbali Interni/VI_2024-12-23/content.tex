% Insert content here
\section{Informazioni generali}
\subsection{Dettagli sull'incontro}
\begin{itemize}
    \item \textbf{Luogo}: Discord
    \item \textbf{Data}: 23-12-2024
    \item \textbf{Ora di inizio}: 14:30
    \item \textbf{Ora di fine}: 16:00
    \item \textbf{Partecipanti}:
    \begin{itemize}
        \item Bergamin Elia
        \item Chilese Elena
        \item Diviesti Filippo
        \item Djossa Edgar
        \item Pincin Matteo 
        \item Soranzo Andrea  
    \end{itemize}
\end{itemize}

\section{Motivo della riunione}
In questo incontro abbiamo discusso gli obiettivi raggiunti nel terzo sprint. Inoltre sono state fatte alcune considerazioni, e di conseguenza dei miglioramenti,
per quanto riguarda il "Wow", in particolare sul nome dei branch dei verbali e sul controllo del contenuto dei documenti tramite calcolo e valutazione dell'indice di Gulpease.
Infine sono stati definiti gli obiettivi che il PoC dovrà soddisfare.

\section{Resoconto}
% insert here all the steps
\subsection{Retrospettiva}

\subsubsection{Codice dei requisiti in NDP e ADR}
In seguito ad una revisione del documento \textit{Analisi dei requisiti}, il team ha deciso di modificare, per un problema di corretezza, il codice dei requisiti eliminando il punto che era presente tra il tipo del requisito e il numero che lo identifica.
Questa modifica andrà successivamente riportata nel documento \textit{Norme di progetto}.

\subsubsection{Nomi dei branch dei verbali}
A causa delle regole di branch protection della repository, il team ha deciso di applicare nuove norme per i nomi dei branch dei verbali.
La nuova nominazione sarà la seguente: \textbf{TIPO\_ANNO\_MESE\_GIORNO\-ISSUE}.

\subsection{Assegnazione ruoli}
Abbiamo assegnato i ruoli seguendo il criterio di rotazione predefinito, introducendo questa volta la figura del programmatore per supportare lo sviluppo del PoC.
Questo ruolo verrà ricoperto da due componenti del team, lasciando quindi un solo verificatore e un solo analista, considerati dal gruppo ruoli minori in questo sprint per priorità e carico di lavoro.
\begin{itemize}
    \item \textbf{Responsabile:} Chilese Elena
    \item \textbf{Amministratore:} Diviesti Filippo
    \item \textbf{Analista:} Djossa Edgar
    \item \textbf{Verificatore:} Soranzo Andrea
    \item \textbf{Programmatore:} Pincin Matteo, Bergamin Elia
\end{itemize}

\section{Prossimi obiettivi}

\subsection{Piano di progetto}
\begin{itemize}
    \item Consuntivo sprint 3
    \item Pianificazione e preventivo sprint 4
\end{itemize}

\subsection{Norme di progetto}
\begin{itemize}
    \item Aggiunta del test automatico indice di Gulpease
    \item Modifica delle norme sui codici dei requisiti
    \item Modifica delle norme sui nomi dei branch dei verbali
\end{itemize}

\subsection{Analisi dei requisiti}
\begin{itemize}
    \item Conclusione descrizione casi d'uso
    \item Modifica codice requisiti
\end{itemize}

\subsection{Glossario}
\begin{itemize}
    \item Inserimento nuovi vocaboli
\end{itemize}

\subsection{PoC}
\begin{itemize}
    \item Selezione barre
    \item Hover
    \item Filtering
    \item Testing (opzionale)
\end{itemize}
