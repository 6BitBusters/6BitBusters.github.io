\section{Informazioni generali}

\subsection{Dettagli sull'incontro}

\begin{itemize}
    \item \textbf{Luogo}: Discord
    \item \textbf{Data}: 10-02-2025
    \item \textbf{Ora di inizio}: 18:15
    \item \textbf{Ora di fine}: 19:30
    \item \textbf{Partecipanti}:
    \begin{itemize}
        \item Bergamin Elia
        \item Chilese Elena
        \item Diviesti Filippo
        \item Djossa Edgar
        \item Pincin Matteo
        \item Soranzo Andrea  
    \end{itemize}
\end{itemize}

\section{Motivo della riunione}

L'incontro aveva lo scopo di fare il punto sul lavoro svolto durante il sesto sprint e organizzare il prossimo. In particolare, sono stati discussi i vari aggiornamenti riguardanti il PoC.

\section{Resoconto}

\subsection{Retrospettiva}

\subsubsection{PoC}

Dopo il primo incontro della RTB con il prof. Cardin, che ha avuto un esito negativo, abbiamo deciso di implementare un backend per facilitare la scalabilità del prodotto. Ad esempio, può essere possibile utilizzare delle API con chiave oppure reperire i dati tramite un database. Dopo diversi studi e prove, abbiamo definito le tecnologie da utilizzare per il backend:

\begin{itemize}
\item \textbf{Node.js}: runtime JavaScript asincrono, utilizzato per la gestione delle richieste lato server;
\item \textbf{Express.js}: framework per Node.js;
\end{itemize}

Per facilitare la distribuzione e la scalabilità, utilizziamo anche:

\begin{itemize}
\item \textbf{Docker}: strumento per creare container.
\end{itemize}

\subsection{Repository e Way of Working}
Sono stati decisi diversi miglioramenti relativi alla nostra repository e al modo di lavorare all'interno di essa, tra cui:
\begin{itemize}
    \item Separare, nel sito, i documenti interni da quelli esterni;
    \item Modificare la protezione dei branch: permettere alle pull request la possibilità di modificare più documenti per evitare i conflitti sui branch di hotfix.
\end{itemize}

\subsection{Assegnazione ruoli}
Abbiamo assegnato i ruoli in base al monte ore rimanente per ogni membro del gruppo.

\begin{itemize}
    \item \textbf{Responsabile:}  Bergamin Elia
    \item \textbf{Amministratore:} Djossa Edgar
    \item \textbf{Analista:} Pincin Matteo
    \item \textbf{Verificatore:} Diviesti Filippo,  Chilese Elena
\end{itemize}

\section{Prossimi obiettivi}

\subsection{PoC}
\begin{itemize}
    \item integrare backend al PoC;
    \item velocizzare il rendering del grafico.
\end{itemize}

\subsection{Analisi dei requisiti}
\begin{itemize}
    \item  Aggiungere i requisiti di vincolo relativi alle tecnologie del backend.
\end{itemize}

\subsection{Piano di qualifica}
\begin{itemize}
    \item Aggiornare i cruscotti delle metriche.
\end{itemize}

\subsection{Piano di progetto}
\begin{itemize}
    \item Consuntivo sprint 6;
    \item Pianificazione sprint 7;
    \item Preventivo sprint 7.
\end{itemize}

\subsection{Glossario}
\begin{itemize}
    \item Inserimento nuovi vocaboli.
\end{itemize}

\subsection{Varie}
\begin{itemize}
    \item Aggiornare la presentazione per l'RTB;
    \item Richiedere un nuovo incontro per l'RTB.
\end{itemize}