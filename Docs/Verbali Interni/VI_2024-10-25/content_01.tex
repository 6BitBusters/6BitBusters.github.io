% Insert content here
\section{Informazioni generali}
\subsection{Dettagli sull'incontro}
\begin{itemize}
    \item \textbf{Luogo}: Incontro in remoto su Discord
    \item \textbf{Data}: 25-10-2024
    \item \textbf{Ora di inizio}: 15:00
    \item \textbf{Ora di fine}: 16:20
    \item \textbf{Partecipanti}:
    \begin{itemize}
        \item Bergamin Elia
        \item Chilese Elena
        \item Diviesti Filippo
        \item Djossa Edgar
        \item Pincin Matteo 
        \item Soranzo Andrea  
    \end{itemize}
\end{itemize}

\section{Resoconto}
    \subsection{Introduzione}
    In questo incontro abbiamo preso visione e discusso delle risposte fornite da Vimar tramite un documento condiviso ai nostri quesiti sul capitolato C2. Successivamente abbiamo realizzato il diario di bordo da presentare lunedì 28-10-2024. Infine abbiamo redatto parzialmente la dichiarazione degli impegni.
    \subsection{Domande e risposte su Vimar GENIALE (C2)}
    \begin{itemize}
        \item \textbf{Qual è l'obiettivo a lungo termine del sistema Vimar: è prevista un'integrazione con i vostri prodotti, oppure il progetto è solo a scopo di dimostrare la potenzialità dei LLM applicato ai database? }\\\\
        L'obiettivo a lungo termine del sistema Vimar è duplice: da un lato, dimostrare la potenzialità dei modelli di linguaggio (LLM) e, dall'altro, esplorare la possibilità di integrare questi modelli con i prodotti Vimar per migliorare l'interazione e il supporto ai clienti, così da comprenderne limiti e costi. A livello sostanziale, all'inizio del documento del capitolato (vedi fine pagina 5) è scritto in chiaro che l'obiettivo – nel breve termine – è creare un \textbf{dimostratore}, e quindi un prodotto il più vicino allo stadio di produzione. Per quelli che sono i nostri progetti futuri, ovviamente non posso rivelare troppo. Diciamo solo che il vostro output finale, se sarà fatto abbastanza bene e sarà fatto con opportuni criteri per eseguirne la manutenzione, possiamo riproporlo anche a utilizzatori interni o a un gruppo di esterni. Quel che posso assicurarvi è che essendo open source avrà una discreta visibilità e per voi tornerà certamente utile per fare “curriculum”.
        
        \item \textbf{Nella presentazione non è stato dato particolare accento alla PoC. Quali dovrebbero essere i principali criteri che essa deve rispettare?} \\\\
        Il Proof of Concept deve dimostrare la fattibilità tecnica del progetto, includendo una primissima implementazione “alla buona” dei criteri di completamento obbligatori tra cui il componente di estrazione, il componente di interrogazione, un minimo di interfaccia grafica con relative API dell'applicativo server, e la capacità di rispondere adeguatamente alle domande degli utenti rispetto al contesto. Essendo un PoC, \textit{il mio consiglio è di \textbf{“smanettare”} per provare a comprendere i vari componenti richiesti}. Inoltre, poiché per voi è la prima volta che vi cimentate in un progetto simile, è normale che venga fuori un Frankenstein di prodotto in questa prima fase. Pertanto, detta in altri termini, \textit{giocate fino al PoC per imparare le tecnologie e poi lavorate di qualità e precisione in ottica di MVP}.
        
        \item \textbf{Ci sono limiti al numero di riunioni che è possibile fare per approfondire le tecnologie da voi richieste?} \\\\
        Non sono specificati limiti al numero di riunioni nel documento. A tal proposito, se avete intenzione di richiedere delle riunione di approfondimento \textit{è bene che illustriate con un po' di anticipo i dubbi e soprattutto cosa vi aspettate di imparare}, così che noi possiamo prepararci adeguatamente per spiegarvelo (e soprattutto \textit{affinché sia \textbf{utile per voi} }). Oltre a questo, vi possiamo fornire anche dei link a delle risorse attendibili. Chiaramente, per cose minori ci saranno le riunioni di allineamento settimanale / bisettimanale dove possiamo discutere qualunque questione e magari possiamo dedicare parte di quelle riunioni per mini-approfondimenti. L'importante per voi – e in secondo luogo per noi – è che il progetto proceda secondo i tempi previsti e all'insorgere di problemi \textit{siate in grado di sapere se e quando alzare la mano per chiedere il nostro supporto}.
    \end{itemize}
    \subsection{Diario di bordo}
    Partendo da una bozza proposta da Pincin Matteo, abbiamo riassunto schematicamente le cose fatte dal momento della formazione del gruppo. In seguito abbiamo approvato gli obiettivi per il prossimo periodo e successivamente abbiamo raccolto alcuni dubbi emersi riguardo l'andamento del progetto.
    \subsection{Dichiarazione degli impegni}
    Abbiamo cominciato a redigere la dichiarazione degli impegni, in particolare stabilendo il numero di ore da assegnare a ciascun ruolo e definendo il costo iniziale del progetto. \\Infine abbiamo stabilito la data entro la quale ci impegniamo a portare a termine il progetto.
\section{Prossimi obiettivi}
   \begin{itemize}
        \item Colloqui con aziende \textit{Sanmarco} e \textit{Vargroup}
        \item Completamento della \textit{Dichiarazione degli impegni}.
        \item Redazione della \textit{Lettera di candidatura}.
    \end{itemize}