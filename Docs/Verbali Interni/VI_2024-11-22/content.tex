% Insert content here
\section{Informazioni generali}
    \subsection{Dettagli sull'incontro}
        \begin{itemize}
            \item \textbf{Luogo}: Discord
            \item \textbf{Data}: 22-11-2024
            \item \textbf{Ora di inizio}: 9:30
            \item \textbf{Ora di fine}: 11:30
            \item \textbf{Partecipanti}:
                \begin{itemize}
                    \item Bergamin Elia
                    \item Chilese Elena
                    \item Diviesti Filippo
                    \item Djossa Edgar
                    \item Pincin Matteo
                    \item Soranzo Andrea
                \end{itemize}
        \end{itemize}

\section{Motivo della riunione}
    In questo incontro abbiamo discusso gli obiettivi raggiunti nel primo sprint\textsubscript{g},
    considerando in particolare \textbf{come} li abbiamo raggiunti. Questo è
    servito per poter valutare e migliorare il wow\textsubscript{g}. Poi abbiamo
    pianificato le attività per il secondo sprint e assegnato i ruoli.
\section{Resoconto}
    \subsection{Retrospettiva}
        \subsubsection{Verifica}
            Abbiamo osservato che il precedente workflow\textsubscript{g} per la verifica\textsubscript{g}
            dei documenti rallentava il processo, costringendo il redattore a chiedere quale verificatore avrebbe verificato il
            documento. Abbiamo dunque deciso che sarà il verificatore stesso a inserire il
            proprio nome nel changelog\textsubscript{g} e nella pagina iniziale dei documenti.
        \subsubsection{Pull request}
            Per evitare conflitti per il nome di una pull request\textsubscript{g}, abbiamo
            deciso che una pull request avrà come nome \textbf{NOME\_DOCUMENTO-ID\_ISSUE} e
            come messaggio di squash\textsubscript{g} \textbf{Update: NOME\_DOCUMENTO-VERSIONE}
        \subsubsection{Rendicontazione ore}
            Per rendere più facile la rendicontazione delle ore di lavoro e, di
            conseguenza, la stesura del consuntivo\textsubscript{g}, abbiamo deciso di aggiungere alle
            issue\textsubscript{g} il campo \textit{Effective} che indica le ore
            effettivamente dedicate allo svolgimento di quella attività. Per motivi del
            tutto analoghi, abbiamo aggiunto i campi \textit{Review
                Estimate} e \textit{Review Effective}, per indicare il tempo preventivato e 
                effettivo per la verifica.
        \subsubsection{Daily scrum}
            Data l'importanza del daily scrum\textsubscript{g} nel
            framework\textsubscript{g} Scrum\textsubscript{g}, abbiamo deciso che lo
            celebreremo su Telegram\textsubscript{g}. In un gruppo creato ad hoc,
            all'inizio di ogni giorno lavorativo, ciascun membro del team risponderà alle
            domande "Cosa ho fatto ieri?", "Cosa farò oggi?", "Cosa mi impedisce di andare
            avanti?".
    \subsection{Assegnazione ruoli}
        Abbiamo assegnato i ruoli secondo il criterio di rotazione stabilito inizialmente,
        aggiungendo un analista e sottraendo un amministratore rispetto allo sprint precedente.
        \begin{itemize}
            \item \textbf{Responsabile}: Djossa Edgar
            \item \textbf{Analisti}: Diviesti Filippo, Pincin Matteo
            \item \textbf{Amministratore}: Chilese Elena
            \item \textbf{Verificatori}: Bergamin Elia, Soranzo Andrea
        \end{itemize}

\section{Prossimi obiettivi}
        \subsection{Norme di progetto}
            \begin{itemize}
                \item Aggiornamento processi di supporto
                \item Aggiornamento processi organizzativi
                \item Aggiunta sottosezioni progettazione e codifica
            \end{itemize}
        \subsection{Piano di progetto}
            \begin{itemize}
                \item Consuntivo sprint 1
                \item Pianificazione e preventivo sprint 2
                \item Aggiunta sezione modello di sviluppo
            \end{itemize}
        \subsection{Piano di qualifica}
            \begin{itemize}
                \item Introduzione
                \item Qualità di processo
                \item Qualità di prodotto
            \end{itemize}
        \subsection{Analisi dei requisiti}
            \begin{itemize}
                \item Introduzione
                \item Descrizione
                \item Casi d'uso* 
            \end{itemize}   
            \begin{small}*In un foglio condiviso, ogni membro scriverà i casi d'uso 
            individuati dalla lettura del capitolato\textsubscript{g} di progetto, 
            per poi discuterli con il proponente\textsubscript{g}.
            \end{small} 
        \subsection{Glossario}
            \begin{itemize}
                \item Inserimento nuovi vocaboli
            \end{itemize}
        