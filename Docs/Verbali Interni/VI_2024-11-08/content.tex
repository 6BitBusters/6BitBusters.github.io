% Insert content here
\section{Informazioni generali}
\subsection{Dettagli sull'incontro}
\begin{itemize}
    \item \textbf{Luogo}: Discord
    \item \textbf{Data}: 08-11-2024
    \item \textbf{Ora di inizio}: 15:00
    \item \textbf{Ora di fine}: 16:30
    \item \textbf{Partecipanti}: 
    \begin{itemize}
        \item Bergamin Elia
        \item Chilese Elena
        \item Diviesti Filippo
        \item Djossa Edgar
        \item Pincin Matteo 
        \item Soranzo Andrea  
    \end{itemize}
\end{itemize}

\section{Motivo della riunione}
In questo incontro abbiamo discusso, definito ed evoluto alcuni principi del way of working, sistemato alcuni template per la creazione di documenti,
cominciato a pianificare le milestones a lungo termine e creato il powerpoint per il secondo diario di bordo.

\section{Resoconto}
\subsection{Retrospettiva}
Inizialmente ci siamo concentrati sull'analisi di attività passate e i nostri errori commessi, sopratutto quelli evidenziati dal Prof. Vardanega Tullio durante la presentazione
delle candidature. Per questo motivo si è deciso di cambiare la struttura del registro delle modifiche aggiungendo una colonna "Verificatore" nella quale verrà inserito il nome del verificatore per quella versione di documento.

\subsection{Aggiornamento del way of working}
\subsubsection{Versionamento Documentazione}
Dopo il consiglio del Prof. Vardanega Tullio circa l'automazione della creazione di documenti e relativa page per mostrarli in modo più efficace, abbiamo ragionato 
sul workflow da adottare per gestire il versionamento di quest'ultimi e cercare di lavorare in modo più asincrono possibile.
In breve il workflow deciso è il seguente:
\begin{enumerate}
    \item Scaricare in locale il template
    \item Aprire un branch primario per un documento
    \item Per ogni modifica aprire un sotto branch derivato dal primario
    \item Compilare/modificare il documento
    \item Una volta terminata la modifica aprire una pull request verso il branch primario
    \item Successivamente un verificatore dovrà accettare la pull request o commentare con eventuali modifiche da apportare al documento
    \item Una volta terminato il documento creare una pull request verso il branch main per approvarlo e pubblicarlo sulla page
\end{enumerate}
Tutto verrà poi specificato nel documento \textit{Norme di Progetto} che è in corso di redazione.
\subsubsection{Backlog}
Oltre al versionamento abbiamo deciso di introdurre un meccanismo di tracciamento delle task/backlog, spostandoci da un normale documento di Google Docs alla board \textit{kanban} di GitHub, la quale è legata
alle issue che mano a mano verranno create.

\subsection{Pianificazione}
Abbiamo cominciato inoltre a pianificare delle milestones a lungo termine, nello specifico la RTB dandoci una data massima da rispettare ovvero il 17/01/2025. Successivamente negli ultimi minuti della riunione
ci siamo focalizzati sulla creazione di template per i documenti di \textit{Piano di progetto} e \textit{Norme di Progetto}.

\section{Prossimi obiettivi}
   \begin{itemize}
        \item Iniziare analisi dei requisiti
        \item Proseguire la redazione dei documenti \textit{Piano di progetto} e \textit{Norme di Progetto}
        \item Avvisare il proponente, definire canali di comunicazione e incontri futuri
    \end{itemize}
