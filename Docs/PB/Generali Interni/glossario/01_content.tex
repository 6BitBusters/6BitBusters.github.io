% Insert content here
\section{A}
\subsection{Action}
(s. c. f.)\\
Combinazione di script e configurazioni software che consentono di eseguire operazioni
specifiche in risposta a determinati eventi, come modifiche al codice.
Vedi anche \nameref{GitHub Actions}.
\subsection{Agile}
(s. p. m. / agg.)
Approccio alla gestione dei progetti che prevede la suddivisione in fasi e sottolinea
l'importanza della collaborazione e del miglioramento continuo tramite feedback.
Il team segue un ciclo di pianificazione, esecuzione e valutazione.
\subsection{API}
(s. c. f.)\\
Abbreviazione di \nameref{Application Programming Interface}.
\subsection{Application Programming Interface}
\label{Application Programming Interface}
(s. c. f.)\\
Insieme di procedure che consentono la comunicazione tra diversi computer, tra diversi
software o tra diversi componenti software.
\subsection{Approvazione}
Attività nella quale un documento viene convalidato dal responsabile di
progetto e, per certi documenti, anche dal proponente. \\ Nell'ambito di
GitHub, l'approvazione di una pull request è l'attività con cui un revisore
conferma le modifiche apportate.
\subsection{Attore}
Ruolo coperto da un certo insieme di entità che interagiscono con il sistema,
inclusi utenti e altri sistemi software. \pagebreak
\section{B}
\subsection{Backlog}
(s. c. m.)\\
Lista prioritaria e ordinata di attività, requisiti o funzionalità che devono
essere completate per sviluppare un prodotto o un progetto.
Vedi anche \nameref{Product backlog} e \nameref{Sprint backlog}.
\subsection{Branch}
(s. c. m.)\\
In Git, ramo del progetto che permette di sviluppare software, documentazione inclusa, su una linea separata
senza interferire con il ramo principale o altri rami del repository.
\subsection{Branch protection}
\label{Branch protection}
(s. c. m.)\\
In GitHub, regola che impedisce a un branch di essere eliminato o di subire modifiche
non autorizzate. Può includere regole di push, pull, merge e delete.
\subsection{Build}
(s. c. f.)\\
Processo che trasforma il codice del progetto in un pacchetto ottimizzato per la produzione.
\subsection{Bundle}
(s. c. m.)\\
File unico che contiene codice combinato e ottimizzato per migliorare le prestazioni, usato da
strumenti come Vite.
\pagebreak
\section{C}
\subsection{Capitolato}
Documento tecnico a cui si fa riferimento per definire le specifiche tecniche
del prodotto che verrà realizzato. Può contenere indicazioni sulle metodologie
e tecnologie da adottare per lo sviluppo.
\subsection{Commit}
(s. c. m.)\\
In Git, istantanea di uno specifico stato del progetto in un determinato momento.
Ogni commit ha un codice univoco che lo identifica,
un riferimento allo stato del repository prima del commit e un messaggio descrittivo
fornito dallo sviluppatore che descrive brevemente le modifiche apportate ai file.
\subsection{Committente}
Chi ordina ad altri l'esecuzione di un lavoro. Nel caso del progetto didattico
sono i docenti.
\subsection{Consuntivo}
Rendiconto dei risultati effettivi di un dato periodo di attività, in termini
di ore e costi, messi a confronto con gli obiettivi pianificati. \pagebreak
\section{D}
\subsection{Daily scrum}
(s. c. m.)\\
Incontro giornaliero a cui partecipano tutti i membri del team, in cui ciascuno
risponde alle seguenti domande:
\begin{itemize}
    \item Cosa hai fatto ieri?
    \item Cosa farai oggi?
    \item C'è qualcosa che ti impedisce di farlo?
\end{itemize}
\subsection{Database}
(s. c. m.)\\
Archivio di dati strutturato in modo da consentire la gestione e
l'organizzazione dei dati.
\subsection{Dataset}
(s. c. m.)\\
Collezione strutturata di dati, generalmente di grandi dimensioni.
\subsection{Deploy}
(s. c. m.)\\
Forma abbreviata di Deployment. Vedi \nameref{Deployment}.
\subsection{Deployment}
(s. c. m.)\\
Processo di rilascio e messa in produzione di un'applicazione su un
server o su un'infrastruttura cloud affinché possa essere utilizzata dagli utenti finali.
\subsection{Diagramma di Gantt}
Strumento di supporto alla gestione dei progetti, costruito partendo da un asse
orizzontale, che rappresenta l'arco temporale totale del progetto, suddiviso in
fasi incrementali (ad esempio giorni o settimane), e da un asse verticale, che
rappresenta le attività che costituiscono il progetto.
\subsection{Discord}
(s. p. m.)\\
Piattaforma di VoIP (Voice over Internet Protocol), messaggistica istantanea
e distribuzione digitale, progettata per la comunicazione tra gruppi di persone.
\subsection{Docker}
(s. p. m.)\\
Software open source che consente di eseguire, distribuire e gestire
applicazioni in contenitori virtualizzati.
\subsection{Docker Compose}
(s. p. m.)\\
Strumento che permette di definire e gestire applicazioni multi-container in Docker
utilizzando un file YAML.
\subsection{DOM}
{s. c. m.}\\
Acronimo di Document Object Model, rappresenta la struttura ad albero di un documento
HTML o XML, permettendo l'accesso e la manipolazione dinamica degli elementi tramite linguaggi
di programmazione.
\subsection{Drei}
(s. p. f.)\\
Vedi \nameref{React Three Drei}.
\pagebreak
\section{E}
\subsection{Efficacia}
Capacità di raggiungere gli obiettivi attesi.
\subsection{Efficienza}
Capacità di utilizzare la minima quantità di risorse necessarie al
raggiungimento di un obiettivo.
\subsection{Express.js}
(s. p. m.)\\
Framework leggero e flessibile per Node.js, utilizzato per la creazione di
applicazioni web e API.
\pagebreak
\section{F}
\subsection{Feature}
(s. c. f.)\\
Vedi \nameref{Funzionalità}.
\subsection{Feedback}
(s. c. m.)\\
Informazione di ritorno, opinione o valutazione che viene fornita come reazione a
un'attività o un prodotto, al fine di migliorare o correggere i risultati futuri.
\subsection{Fiber}
(s. p. f.)\\
Vedi \nameref{React Three Fiber}.
\subsection{Filtering}
(s. c. m.)\\
È il processo di selezione e isolamento dei dati che corrispondono a criteri specifici, escludendo quelli non pertinenti. (Nel nostro caso si applica ai dati del Dataset).
\subsection{Fornitore}
Chi si impegna a svolgere un lavoro o realizzare un prodotto. Nel caso del
progetto didattico è il gruppo di studenti.
\subsection{Framework}
(s. c. m.)\\
Struttura o piattaforma software che fornisce un insieme di strumenti, regole,
librerie e componenti per aiutare gli sviluppatori a costruire applicazioni
in modo più efficiente e standardizzato. Definisce il modo in cui un progetto
deve essere organizzato e sviluppato, riducendo il lavoro ripetitivo e favorendo
la coerenza e la manutenibilità del codice.
\subsection{Funzionalità}
\label{Funzionalità}
Specifica caratteristica o capacità di un prodotto, software, applicazione o sistema,
che ha lo scopo di soddisfare un'esigenza o risolvere un problema per l'utente.
\pagebreak
\section{G}
\subsection{Git}
\label{Git}
(s. p. m.)\\
Sistema software distribuito che tiene traccia delle versioni dei file.
Viene utilizzato per controllare il codice sorgente da parte di programmatori
che stanno sviluppando software in modo collaborativo.
\subsection{GitHub}
\label{GitHub}
(s. p. m.)\\
Piattaforma per sviluppatori che permette di creare, memorizzare, gestire
e condividere il loro codice. È utilizzato principalmente per ospitare
progetti di sviluppo software open source.
\subsection{GitHub Actions}
\label{GitHub Actions}
(s. p. f.)\\
Piattaforma di automazione dei workflow integrata in GitHub che consente
di creare e gestire flussi di lavoro personalizzati direttamente all'interno
di un repository.
\subsection{GitHub Pages}
\label{GitHub Pages}
(s. p. f.)\\
Servizio offerto da GitHub che permette di pubblicare siti web statici direttamente da un repository GitHub,
usato per ospitare la documentazione prodotta.
\subsection{Google Chat}
(s. p. m.)\\
Strumento di comunicazione con funzionalità di messaggistica
istantanea e condivisione di file, integrato con altri servizi Google.
\subsection{Google Meet}
(s. p. m.)\\
Servizio web per videoconferenze e riunioni, con funzionalità di messaggistica,
condivisione schermo e registrazione.
\subsection{Grafico 3D}
Visualizzazione grafica utilizzata per confrontare valori discreti o categorie.
Si compone di barre verticali tridimensionali, la cui altezza rappresenta la
quantità associata a ciascuna categoria, e di assi ortogonali X, Y e Z.
\pagebreak
\section{H}
\subsection{Helper}
(s. c. m.)\\
Entità progettata per fornire supporto ad altre parti del codice,
semplificando l'esecuzione di operazioni comuni o ripetitive.
Può essere una funzione, una classe o un modulo.
\subsection{Hook}
(s. c. m.)\\
Funzione speciale che consente di usare lo stato e altre funzionalità di React all'interno di componenti funzionali.
\subsection{Hotfix}
(s. c. m.)\\
Correzione urgente applicata a un sistema software per risolvere un problema critico,
che richiede un intervento immediato. A differenza delle normali modifiche o aggiornamenti pianificati,
un hotfix viene sviluppato e distribuito rapidamente per minimizzare l'impatto del problema.\\
In caso di hotfix di un documento, viene aumentato di uno il numero di versione più a destra.
\subsection{Hover}
(s. c. m.)\\
Un evento hover del mouse è il passaggio del puntatore sopra un elemento grafico, nel nostro caso una barra del grafico 3D, che innesca una risposta visiva o interattiva da parte del sistema.
\pagebreak
\section{I}
\subsection{Indice Gulpease}
Indice di leggibilità di un testo in lingua italiana. Considera due variabili
linguistiche: la lunghezza della parola e la lunghezza della frase rispetto al
numero delle lettere.
\subsection{Issue}
\label{Issue}
(s. c. f.)\\
Strumento utilizzato per tracciare, discutere e risolvere problemi, richieste di funzionalità,
bug o idee relative a un progetto, facilitando la collaborazione tra i membri di un team.
\subsection{Issue tracking system}
(s. c. m.)\\
Software che gestisce liste di issue, generalmente utilizzato in contesti collaborativi.
Può comprendere l'allocazione delle risorse, la contabilità del tempo, la gestione delle
priorità e il flusso di lavoro per la supervisione.
\pagebreak
\section{J}
\subsection{Job}
(s. c. m.)\\
In GitHub Actions, un job è una sequenza di passaggi che vengono eseguiti
in un ambiente specifico. Un flusso di lavoro può contenere uno o più job.
\pagebreak
\section{K}
\pagebreak
\section{L}
\subsection{LaTeX}
\label{LaTeX}
(s. p. m.)\\
Linguaggio di markup utilizzato per la redazione di documenti di alta qualità tipografica.
\pagebreak
\section{M}
\subsection{Main}
(s. c. m.)\\
Branch principale di sviluppo di un progetto software.
\subsection{Memcached}
(s. p. m.)\\
Sistema di caching distribuito in memoria, progettato per migliorare le prestazioni
delle applicazioni web memorizzando in cache i dati e riducendo il carico su database
o altre sorgenti di dati.
\subsection{Merge}
(s. c. m.)\\
In Git, processo di combinazione di modifiche provenienti da un branch in un altro.
\subsection{Metrica}
Misura quantitativa utilizzata per valutare le caratteristiche o lo stato di un
sistema, processo o entità. Le metriche forniscono dati oggettivi e misurabili
che aiutano a prendere decisioni informate.
\subsection{Milestone}
(s. c. f.)\\
Data che fissa un punto di avanzamento atteso nello svolgimento del progetto.
\pagebreak
\section{N}
\subsection{NestJS}
(s. p. m.)\\
Framework per la creazione di applicazioni lato server in Node.js,
basato su TypeScript e progettato per essere modulare e scalabile.
\subsection{Node.js}
(s. p. m.)\\
Piattaforma per eseguire codice JavaScript lato server.
\pagebreak
\section{O}
\subsection{Overleaf}
(s. p. m.)\\
Piattaforma online per la creazione, modifica e condivisione di documenti
scritti in LaTeX.
\pagebreak
\section{P}
\subsection{Pages}
(s. p. f.)\\
Vedi \nameref{GitHub Pages}.
\subsection{Panning}
(s. c. m.)\\
Nella grafica 3D, movimento della telecamera o della vista che consente
di spostarsi orizzontalmente o verticalmente in una scena senza modificarne
la prospettiva o l'orientamento.
\subsection{PB}
\label{PB}
(s. p. f.)\\
Abbreviazione di \nameref{Product Baseline}.
\subsection{PoC}
(s. p. m.)\\
Abbreviazione di \nameref{Proof of Concept}.
\subsection{Preventivo}
Previsione delle ore e dei costi relativi a un dato periodo di tempo, calcolato
in base alle attività pianificate.
\subsection{Product backlog}
\label{Product backlog}
(s. c. m.)\\
Lista di tutte le funzionalità, modifiche, miglioramenti e correzioni necessarie
per il prodotto finale. Rappresenta una visione globale del lavoro da fare.
\subsection{Product Baseline}
\label{Product Baseline}
(s. p. f.)\\
Seconda e ultima revisione di avanzamento del progetto effettuata dai docenti (committenti)
durante la quale si verificano gli obiettivi funzionali e qualitativi del prodotto.
\subsection{Proof of Concept}
\label{Proof of Concept}
(s. p. m.)\\
Dimostrazione pratica della fattibilità di un progetto o di una soluzione proposta.
\subsection{Prop}
(s. c. m.)\\
In React, attributo passato a un componente per fornire dati o configurazioni.
I prop consentono di personalizzare il comportamento e l'aspetto
di un componente, rendendolo riutilizzabile e modulare.
\subsection{Proponente}
\label{Proponente}
Chi propone un lavoro. Nel progetto didattico svolge il ruolo di cliente
rispetto alle esigenze di prodotto e mentore rispetto alle scelte di sviluppo.
Per il nostro gruppo è l'azienda \textit{Sanmarco Informatica S.p.A.}.
\subsection{Protection}
(s. c. m.)\\
Vedi \nameref{Branch protection}.
\subsection{Pull}
(s. c. m.)\\
In Git, comando per aggiornare un repository locale con le modifiche più recenti provenienti
da un repository remoto. È una combinazione di due comandi: fetch (che scarica le modifiche
da remoto) e merge (che integra le modifiche nel branch corrente).
\subsection{Pull request}
(s. c. f.)\\
Funzionalità comune nelle piattaforme di sviluppo collaborativo, come GitHub, utilizzata per
richiedere l'integrazione del lavoro di uno sviluppatore (di solito su un branch separato)
nel main o in un altro ramo del repository.
\subsection{Push}
(s. c. m.)\\
In Git, comando per trasferire i commit presenti nel proprio repository locale a un
branch specifico in un repository remoto, rendendo le modifiche disponibili
ad altri collaboratori.

\pagebreak
\section{Q}
\pagebreak
\section{R}
\subsection{Raycasting}
(s. c. m.)\\
Tecnica che consiste nel "lanciare" un raggio virtuale (raycast) nello spazio 3D da un punto 
di origine in una direzione specifica per verificare se, dove e cosa colpisce.
\subsection{React}
(s. p. m.)\\
Libreria JavaScript per la la costruzione interfacce utente basata su componenti, che utilizza un
approccio dichiarativo e il Virtual DOM per ottimizzare le prestazioni.
\subsection{React Three Drei}
\label{React Three Drei}
(s. p. f.)\\
Libreria che fornisce un insieme di componenti e helper per
costruire applicazioni 3D utilizzando React e Three.js.
\subsection{React Three Fiber}
\label{React Three Fiber}
(s. p. f.)\\
Libreria basata su Three.js che consente di usare la sintassi
di React nello sviluppo di applicazioni 3D.
\subsection{Redux}
(s. p. m.)\\
Libreria JavaScript per la gestione dello stato delle applicazioni.
\subsection{Redux-Toolkit}
(s. p. m.)\\
Wrapper che fornisce strumenti e convenzioni per semplificare l'uso di Redux,
permettendo di scrivere codice più conciso e leggibile.
\subsection{Repository}
\label{Repository}
(s. c. m.)\\
Struttura che contiene tutti i file, la cronologia delle modifiche e la
configurazione necessaria per gestire un progetto di sviluppo software.
\subsection{Requirements and Technology Baseline}
\label{Requirements and Technology Baseline}
(s. p. f.)\\
Prima revisione di avanzamento del progetto effettuata dai docenti (committenti)
per esaminare i requisiti stabiliti con il proponente. Vengono valutate le
tecnologie e i framework scelti attraverso un \nameref{Proof of Concept}.
L'esito di questa revisione può richiedere modifiche e correzioni.
\subsection{Retrospettiva}
Vedi \nameref{Sprint retrospective}.
\subsection{Routing}
(s. c. m.)\\
Processo di gestione della navigazione in un'applicazione, determinando quale contenuto mostrare in base all'URL.
\subsection{RTB}
(s. p. f.)\\
Abbreviazione di \nameref{Requirements and Technology Baseline}.
\pagebreak
\section{S}
\subsection{Scrum}
(s. p. m.)\\
Framework agile per la gestione del ciclo di sviluppo del software,
iterativo ed incrementale, concepito per gestire progetti e prodotti software
o applicazioni di sviluppo.
\subsection{Sprint}
\label{Sprint}
(s. c. m.)\\
Breve periodo di tempo in cui un team Scrum collabora per completare
una determinata quantità di lavoro. Uno sprint comprende:
sprint planning, daily scrum, lavoro di sviluppo,
sprint review e sprint retrospective. Il team \textit{Six Bix Busters} ha
deciso che ogni sprint durerà due settimane.
\subsection{Sprint backlog}
\label{Sprint backlog}
(s. c. m.)\\
Sotto-lista del product backlog, contenente le attività che il team di sviluppo
si impegna a completare durante uno sprint.
\subsection{Sprint planning}
(s. c. m.)\\
Riunione dedicata alla pianificazione del lavoro per il prossimo sprint.
\subsection{Sprint retrospective}
\label{Sprint retrospective}
(s. c. f.)\\
Riunione in cui il team parla di cosa è andato bene, cosa no e come si può migliorare.
\subsection{Sprint review}
(s. c. f.)\\
Riunione in cui il team mostra ciò che ha realizzato durante uno sprint.
\subsection{Stakeholder}
(s. c. m.)\\
Individuo o gruppo che ha interesse o influenza su un progetto o un'organizzazione.
\subsection{Superset}
(s. c. m.)\\
Un linguaggio o tecnologia che estende un altro, includendone tutte le funzionalità e
aggiungendone di nuove, mantenendo la compatibilità. Ad esempio TypeScript è un superset di JavaScript.
\pagebreak
\section{T}
\subsection{Task}
(s. c. m.)\\
Compito specifico o attività.
\subsection{Team}
(s. c. m.)\\
\label{Team}
Gruppo di persone che lavorano insieme per raggiungere un obiettivo comune.
\subsection{Telegram}
(s. p. m.)\\
Servizio di messaggistica istantanea e VoIP (Voice over Internet Protocol)
basato su cloud.
\subsection{Test E2E}
(s. c. m.)\\
Un test E2E è un metodo di verifica del software che ha lo scopo di testare il funzionamento completo di un'applicazione,
dall'interazione iniziale dell'utente fino alla produzione dell'output finale, assicurandosi che tutti i componenti siano integrati correttamente.
\subsection{Testing}
\label{Testing}
(s. c. m.)\\
Processo di verifica e valutazione di un'applicazione per assicurarsi che funzioni correttamente e
soddisfi i requisiti, attraverso l'esecuzione di test automatici o manuali.
\subsection{Three.js}
(s. p. f.)\\
Libreria JavaScript open source che consente di creare ed eseguire rendering
di grafica 3D nel browser, fornendo agli sviluppatori un'astrazione di alto livello.
\subsection{Tooltip}
(s. c. m.)\\
Elemento grafico che fornisce informazioni aggiuntive su un oggetto 
(nel nostro caso una barra del grafico 3D) quando l'utente ci passa sopra con il mouse. 
\pagebreak
\section{U}
\pagebreak
\section{V}
\subsection{Validazione}
Processo che accerta che il prodotto finale sia pienamente conforme alle
aspettative.
\subsection{Verifica}
Processo che determina se i prodotti software di un'attività soddisfano i
requisiti o le condizioni imposte nelle attività precedenti.
\subsection{Versionamento}
Gestione di versioni multiple di una stessa informazione.
\subsection{Vite}
(s. p. f.)\\
Bundler e strumento di sviluppo per applicazioni web moderne, progettato per
ottimizzare le prestazioni e migliorare l'esperienza di sviluppo.
\pagebreak
\section{W}
\subsection{Way of working}
\label{Way of working}
(s. c. m.)\\
Il modo in cui il gruppo decide di lavorare, ovvero l'insieme di regole
che il gruppo si dà al fine di organizzare al meglio le attività di progetto.
\subsection{Workflow}
(s. c. m.)\\
Sequenza definita di attività o processi attraverso cui un compito, un progetto
o un'operazione viene completata.
\subsection{Wow}
(s. c. m.)\\
Abbreviazione di \nameref{Way of working}.
\subsection{Wrapper}
(s. c. m.)\\
Funzione o classe che incapsula un'altra funzione o classe, fornendo un'interfaccia
semplificata o estesa per l'uso. I wrapper sono spesso utilizzati per
astrarre complessità, migliorare la leggibilità del codice o aggiungere funzionalità.
\pagebreak
\section{X}
\pagebreak
\section{Y}
\pagebreak
\section{Z}
