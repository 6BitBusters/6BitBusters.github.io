% Insert content here
\section{Informazioni generali}
\subsection{Dettagli sull'incontro}
\begin{itemize}
    \item \textbf{Luogo}: Discord
    \item \textbf{Data}: 17-03-2025
    \item \textbf{Ora di inizio}: 9:20
    \item \textbf{Ora di fine}: 11:00
    \item \textbf{Partecipanti}:
    \begin{itemize}
        \item Bergamin Elia
        \item Chilese Elena
        \item Diviesti Filippo
        \item Djossa Edgar
        \item Pincin Matteo 
        \item Soranzo Andrea  
    \end{itemize}
\end{itemize}

\section{Ordine del giorno}
A seguito del colloquio con il professor Cardin Riccardo, il team ha approfondito le tecnologie e relative architetture backend e frontend in JavaScript.
Il team ha deliberato di utilizzare NestJS per l'implementazione del backend, tuttavia la scelta della tecnologia frontend è attualmente in fase di valutazione.
In seguito, l'attenzione è stata rivolta all'analisi approfondita dei risultati della RTB, con l'obiettivo di identificare e risolvere in maniera tempestiva le anomalie riscontrate.
Infine, il team ha deciso di organizzare un incontro con il proponente e un altro con il professor Vardanega Tullio, al fine di ottenere chiarimenti sui problemi emersi durante la fase di RTB.
\section{Resoconto}
% insert here all the steps
\subsection{Retrospettiva}
\subsubsection{Sprint 8}
La retrospettiva relativa allo sprint 8 ha evidenziato che sono stati portati a termine tutti gli obiettivi prefissati.
Tuttavia, dopo aver valutato il tempo rimanente e la scadenza, il team ha deciso di dedicare più risorse al progetto rispetto alla fase iniziale.

\subsection{Assegnazione ruoli}
Abbiamo assegnato i ruoli seguendo il criterio di rotazione predefinito.
\begin{itemize}
    \item \textbf{Responsabile:} Chilese Elena
    \item \textbf{Amministratore:} Soranzo Andrea
    \item \textbf{Analista:} Bergamin Elia
    \item \textbf{Verificatore:} Diviesti Filippo, Pincin Matteo
    \item  \textbf{Progettista:} Bergamin Elia, Edgar Carlos
\end{itemize}
\newpage

\section{Prossimi obiettivi}

\subsubsection{Ricerca hardware minimo}
\begin{itemize}
    \item Testare il PoC su macchine virtuali in modo da definire un hardware minimo congruo al prodotto che stiamo sviluppando;
\end{itemize}

\subsubsection{Piano di qualifica}
\begin{itemize}
    \item Migliore organizzazione delle sezioni;
    \item Aggiornamento cruscotto;
\end{itemize}

\subsubsection{Norme di progetto}
\begin{itemize}
    \item Definire normative per la scrittura del codice;
    \item Aggiornamento del workflow;
\end{itemize}

\subsubsection{Glossario}
\begin{itemize}
    \item Rimozione dei numeri di sezione;
\end{itemize}

\subsubsection{Manuale utente}
\begin{itemize}
    \item Stesura introduzione;
\end{itemize}

\subsubsection{Specifica tecnica}
\begin{itemize}
    \item Stesura introduzione;
    \item Primo diagramma UML delle classi.
\end{itemize}

\subsubsection{Piano di progetto}
\begin{itemize}
    \item Consuntivo sprint 8;
    \item Pianificazione sprint 9;
    \item Preventivo sprint 9;
    \item Aggiunta di un consuntivo RTB e del preventivo a finire.
\end{itemize}

\subsubsection{Incontri}
\begin{itemize}
    \item Organizzare incontro con il professor Vardanega Tullio;
    \item Organizzare incontro con il proponente.
\end{itemize}
