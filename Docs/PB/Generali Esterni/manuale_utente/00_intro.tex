\section{Introduzione}
\subsection{Scopo del documento}
Lo scopo del seguente documento è quello di illustrare le funzionalità fornite dall’applicazione e le istruzioni
per l’utilizzo della stessa. L’utente sarà quindi a conoscenza dei requisiti minimi necessari per il
corretto funzionamento di \textbf{3Dataviz}.

\subsection{Glossario}
Per chiarire i termini tecnici o ambigui si utilizza un glossario disponibile nel file \textit{Glossario}.\\
Tutti i termini che richiedono spiegazioni sono indicati con il pedice “g”. \\
Questa convenzione consente un rapido collegamento tra il testo e la relativa spiegazione dettagliata nel glossario, garantendo coerenza e chiarezza.

\subsection{Browser supportati}
Di seguito viene fornito un breve elenco delle versioni minime di tutti i browser sui quali il funzionamento
del nostro prodotto è garantito:

\subsection{Cos'è 3Dataviz}
\textit{3Dataviz} è un prodotto web ideato dall'azienda \textit{Sanmarco Informatica S.p.A.} per semplificare e rendere più accessibile la visualizzazione dei dati.\\
Esso mira a trasformare i dati in grafici e rappresentazioni visive, sfruttando la capacità del cervello umano di elaborare rapidamente le immagini. 
Questo approccio facilita il processo decisionale e migliora la comprensione delle informazioni.\\

Il sito consente agli utenti di caricare e visualizzare diversi dataset, offrendo funzionalità di filtraggio per un'analisi approfondita.
Il grafico interattivo, dotato di strumenti di navigazione avanzati, facilita l'esplorazione dei dati. 
Inoltre, è disponibile una visualizzazione tabellare completa per un'analisi dettagliata.\\

L'autenticazione non è richiesta per questo prodotto, in quanto progettato per offrire un'esperienza tridimensionale
intuitiva e responsiva tramite l'utilizzo di dataset pubblici.

\subsection{Browser supportati}
\begin{itemize}
    \item Google Chrome, versione 131;
    \item Mozilla Firefox, versione 133;
    \item Opera, versione 115;
    \item Microsoft Edge, versione 131;
\end{itemize}

Per rendere effettivo il funzionamento su tali browser, deve essere abilitato JavaScript.

\subsection{Hardware minimo}
Per sfruttare appieno le funzionalità di rendering 3D, si consiglia di utilizzare una macchina con il seguente hardware \textbf{minimo}:
\begin{itemize}
    \item \textbf{CPU:}
    \item \textbf{GPU:}
    \item \textbf{RAM:}
\end{itemize}
