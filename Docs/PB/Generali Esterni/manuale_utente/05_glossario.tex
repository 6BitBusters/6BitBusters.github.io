\section{Glossario}
\label{glossario}
\subsection{B}
\subsubsection{Barra}
Elemento grafico del grafico 3D a forma di parallelepipedo che rappresenta un
valore numerico. Le barre hanno un'altezza proporzionale al valore che
rappresentano e sono colorate in base alla loro altezza.

\subsection{D}
\subsubsection{Dataset}
Collezione strutturata di dati, generalmente di grandi dimensioni.

\subsection{G}
\subsubsection{Gizmo}
Rappresentazione schematica di un grafico 3D, utilizzata per facilitare la
navigazione e l'interazione con il grafico stesso. Il gizmo è composto da tre
assi ortogonali (X, Y e Z) e consente di ruotare il grafico in modo intuitivo.
\subsubsection{Grafico 3D}
Visualizzazione grafica utilizzata per confrontare valori discreti o categorie.
Si compone di barre verticali tridimensionali, la cui altezza rappresenta la
quantit`a associata a ciascuna categoria, e di assi ortogonali X, Y e Z.

\subsection{P}
\subsubsection{Pinch}
Gesto di zoom in/out eseguito con due dita su un touchpad o su uno schermo
tattile. Per eseguire il pinch, l'utente deve posizionare due dita sul touchpad
o sullo schermo e avvicinarle o allontanarle per ingrandire o ridurre il
grafico.

\subsection{T}
\subsubsection{Tooltip}
Elemento grafico che fornisce informazioni aggiuntive su un oggetto (in
\textit{3Dataviz} una barra del grafico 3D) quando l'utente ci passa sopra con
il mouse.

\subsection{V}
\subsubsection{Valore medio}
Valore calcolato come somma di tutti i valori di un dataset diviso per il
numero totale di valori presenti nel dataset. È anche riferito come valor medio
globale.