\newcommand{\Met}[4]{
    \subsubsection{#1}
    \begin{figure}[h!] \centering
        \includegraphics[scale=0.75]{#2}
        \caption{Proiezione della#1nei vari periodi di progetto}
    \end{figure}
    \subsubsection*{RTB}
    #3
    \subsubsection*{PB}
    #4
    \newpage
}

\newcommand{\MetNoPB}[3]{
    \subsubsection{#1}
    \begin{figure}[h!] \centering
        \includegraphics[scale=0.75]{#2}
        \caption{Proiezione della#1negli ultimi periodi di progetto}
    \end{figure}
    \subsubsection*{PB}
    #3
    \newpage
}

\section{Resoconto delle attività di verifica}

\subsection{Verifica della qualità dei processi}
In questa sezione sono presentati i risultati dell'attività di verifica svolta per valutare la qualità del processo.\\
Le misure riportate sono state calcolate utilizzando le formule e i concetti descritti nel documento \textit{Norme di progetto v1.0.0}, oltre ai dati definiti nel documento \textit{Piano di progetto v1.0.0}.\\
Di seguito sono presentati i grafici relativi a diverse metriche significative del processo nei vari sprint.

\Met
{ % METRICA
    MPC1 - Planned Value
}
{ % GRAFICO
    template/images/PV.png
}
{ % DESCRIZIONE RTB
    Il grafico mostra una crescita dei costi regolare e prevedibile, coerente con una pianificazione accurata che suddivide equamente il budget tra i vari sprint.
    Questo è un valore positivo e significa che il team riesce a pianificare nel modo corretto le ore per ogni ruolo ad ogni sprint.
}
{ % Descrizione PB
    Durante la fase di PB, iniziata a partire dallo sprint 9, il team ha aumentato notevolmente il numero di ore pianificate per ogni sprint al fine di garantire il rispetto delle scadenze finali.
    Al termine del progetto, la pianificazione si è rivelata abbastanza corretta, in quanto il valore finale del Planned Value è rimasto sotto il budget previsto.
}

\Met
{ % METRICA
    MPC2 - Actual Cost
}
{ % GRAFICO
    template/images/AC.png
}
{ % DESCRIZIONE RTB
    Come si nota dal grafico, Actual Cost e Planned Value hanno una forte correlazione, suggerendo che le ore e i costi effettivamente impiegati per ogni ruolo sono in linea con le previsioni iniziali.
    Ciò indica una pianificazione accurata e una buona esecuzione del progetto.
}
{ % Descrizione PB
    Durante la PB il team ha deciso di aumentare il numero di ore pianificate. Di conseguenza il costo effettivo è cresciuto notevolmente durante gli ultimi sprint. A progetto concluso si può vedere che il valore si è avvicinato al costo stimato di completamento, senza superarlo, segnalando quindi un buon controllo delle spese.
}

\Met
{ % METRICA
    MPC3 - Earned Value
}
{ % GRAFICO
    template/images/EV.png
}
{ % DESCRIZIONE RTB
    Questo grafico illustra solamente il valore di Earned Value che rappresenta il valore delle attività realizzate nel progetto.
    Da solo rientra tra i valori limite, e rappresenta un valore positivo, ma spesso viene messo a confronto con le altre metriche.
    Analizzando quindi i tre grafici di Earned Value, Planned Value e Actual Cost, si può notare come siano tutti allineati. 
    Questo indica una gestione del progetto accurata, con una precisa corrispondenza tra le attività pianificate, quelle effettivamente svolte e i costi sostenuti.
}
{ % Descrizione PB
    L'Earned Value mostra un incremento progressivo, in particolare negli ultimi sprint (dal 10 al 12), dove si evidenzia un'accelerazione nello sviluppo. Questo riflette un incremento della produttività e una maggiore quantità di valore generato nel periodo della PB.
}

\Met
{ % METRICA
    MPC4 - Estimated at Completion
}
{ % GRAFICO
    template/images/EaC.png
}
{ % DESCRIZIONE RTB
    Il grafico mostra che l'Estimated at Completion (EAC) è molto vicino al Budget at Completion (BAC).
    Questo indica un'elevata precisione della stima iniziale dei costi e una gestione finanziaria efficace durante i periodi del progetto. 
}
{ % Descrizione PB
    Alla conclusione del progetto, l’EAC si è rivelato sempre vicino al BAC, dimostrando l’accuratezza delle stime iniziali.
}


\Met
{ % METRICA
    MPC5 - Estimate to Complete
}
{ % GRAFICO
    template/images/EtC.png
}
{ % DESCRIZIONE RTB
    Il grafico evidenzia una relazione inversa tra ETC e AC.
    La tendenza all'abbassamento dell'ETC e all'aumento dell'AC è coerente con l'andamento previsto. 
    Questa dinamica suggerisce una buona gestione delle risorse e un avanzamento del progetto conforme alla pianificazione.
}
{ % Descrizione PB
    Durante la PB, si osserva una forte decrescita dell'ETC. Questo è dovuto al fatto che il team ha completato gran parte delle attività pianificate, riducendo significativamente le ore rimanenti necessarie per il completamento del progetto.
}


\Met
{ % METRICA
    MPC6 - Cost Variance
}
{ % GRAFICO
    template/images/CV.png
}
{ % DESCRIZIONE RTB
    Una Cost Variance negativa indica che il progetto sta spendendo più di quanto previsto. 
    Ciò è dovuto principalmente al fatto che i ruoli più ricoperti sono quelli di responsabile, 
    amministratore, e analista che hanno un costo orario superiore alla media.
    Infatti, nello sprint 4, in cui è stato realizzato il PoC dai programmatore, la Cost Variance è stata positiva.
}
{ % Descrizione PB
    La Cost Variance ha mostrato una riduzione significativa nello sprint 9, dovuta a una produttività ridotta in quel periodo. Negli sprint successivi, in particolare durante gli sprint 11 e 12, il valore è nettamente aumentato, mostrando quindi un recupero in termini di efficienza economica.
}

\Met
{ % METRICA
    MPC7 - Schedule Variance
}
{ % GRAFICO
    template/images/SV.png
}
{ % DESCRIZIONE RTB
    L'analisi del grafico evidenzia che, inizialmente, l'assenza di un WoW solido ha impedito al team di rispettare le tempistiche previste. 
    Con l'introduzione e il consolidamento del WoW, il team ha ottimizzato la pianificazione e l'esecuzione delle attività, migliorando significativamente il rispetto delle scadenze.
}
{ % Descrizione PB
    La Schedule Variance ha mostrato valori positivi nella PB, confermando che il team è riuscito a concludere le attività nei tempi previsti.
}

\Met
{ % METRICA
    MPC8 - Budget Variance
}
{ % GRAFICO
    template/images/BV.png
}
{ % DESCRIZIONE RTB
    L'analisi del grafico indica che il team sta impiegando un po' meno risorse rispetto alle previsioni iniziali.
    Questo dimostra una gestione efficiente e un'ottima capacità di adattamento alle esigenze del progetto.
}
{ % Descrizione PB
    Nella fase finale del progetto (Sprint 9-12), la BV si è abbassata notevolmente, raggiungendo il suo valore minimo nelllo sprint 10. Questo è dovuto a un’intensificazione del lavoro in questa fase, che ha portato a un consumo più rapido del budget. Nonostante ciò, il valore è rimasto entro i limiti stabiliti.
}

\Met
{ % METRICA
    MPC9 - Requirements Stability Index
}
{ % GRAFICO
    template/images/RSI.png
}
{ % DESCRIZIONE RTB
    Il grafico mostra un iniziale Requirements Stability Index del 100\% in quanto nei primi due periodi il team si è
    concentrato sulla creazione e successivo aggiornamento di un WoW. 
    È stato solo dal terzo sprint in poi che il team ha iniziato a definire i requisiti.
    Inoltre, nello sprint 8, a seguito delle osservazioni del professor Cardin, il team ha dovuto rivedere alcuni requisiti,
    in particolar modo i requisiti di qualità.
}
{ % Descrizione PB
    Poche modifiche sono state apportate ai requisiti, poiché quasi tutti quelli definiti in fase di RTB sono stati ritenuti sufficienti per il completamento del progetto.
}

\Met
{ % METRICA
    MPC10 - Completezza dei documenti
}
{ % GRAFICO
    template/images/CD.png
}
{ % DESCRIZIONE RTB  
    Come mostrato dal grafico il team ha raramente avuto problemi riguardanti la completezza dei documenti.
    Si può notare che solamente durante lo sprint 3 il team ha avuto delle difficoltà nella redazione del documento \textit{Analisi dei requisiti v1.0.0}.
    In ogni caso, tutti i membri si sono impegnati al massimo per garantire il completamento delle attività assegnate.
}
{ % Descrizione PB
    In PB, tutti i documenti risultano sempre completi in ogni sprint.
}

\Met
{ % METRICA
    MPC11 - Metriche soddisfatte
}
{ % GRAFICO
    template/images/MS.png
}
{ % DESCRIZIONE RTB
    Come mostra il grafico, l'iniziale carenza di competenze ha limitato la capacità del team di soddisfare tutte le metriche.
    Grazie all'apprendimento continuo e all'introduzione di strumenti automatizzati, il team ha colmato queste lacune apportando delle migliorie ai vari documenti.
    L'unica metrica non soddisfatta attualmente è MPC6 - Cost Variance.
}
{ % Descrizione PB
    Il team ha soddisfatto quasi tutte le metriche previste, dimostrando una piena maturazione delle competenze.
}

\Met
{ % METRICA
    MPC15 - Rischi non previsti
}
{ % GRAFICO
    template/images/RISCHI.png
}
{ % DESCRIZIONE RTB
    Il grafico evidenzia l'assenza di rischi non previsti durante l'intero progetto.
    Questo è ovviamente un buon segno in quanto indica che il gruppo è stato in grado 
    di gestire i rischi in modo efficace e che non sono emersi nuovi rischi inaspettati.
}
{ % Descrizione PB
    Anche nella fase conclusiva, non sono emersi nuovi rischi imprevisti. Il team ha mantenuto un controllo efficace su tutte le potenziali criticità.
}

\subsection{Verifica della qualità dei prodotti}
In questa sezione vengono analizzate le attività svolte per garantire la qualità dei prodotti realizzati.
Le analisi effettuate comprendono sia i documenti che il software, al fine di garantire un risultato finale che soddisfi i requisiti stabiliti.\\
Per calcolare le seguenti misure abbiamo utilizzato le formule e le nozioni descritte nel documento di \textit{Norme di progetto v1.0.0}.

\Met
{ % METRICA
    MPD1 - Indice Gulpease
}
{ % GRAFICO
    template/images/GUL.png
}
{ % DESCRIZIONE RTB
    L'analisi del grafico conferma che, nelle prime fasi del progetto, l'Indice di Gulpease non ha raggiunto i 
    valori minimi, a causa della mancanza di conoscenze specifiche da parte del team.
    Solo dopo aver implementato un'automazione per il calcolo dell'indice, è stato possibile 
    migliorarne i risultati nei periodi successivi.
}
{ % Descrizione PB
    Il valore minimo dell'indice Gulpease è stato sempre rispettato e superato durante questa seconda fase del progetto.
}

\Met
{ % METRICA
    MPD2 - Errori grammaticali
}
{ % GRAFICO
    template/images/ERR-G.png
}
{ % DESCRIZIONE RTB
    L'analisi del grafico rivela un numero inizialmente elevato di errori ortografici, 
    in particolare nei documenti \textit{Norme di progetto v1.0.0} e \textit{Piano di progetto v1.0.0}.
    Tuttavia, si osserva una netta diminuzione di tali errori nei periodi e nei 
    documenti successivi, indicando un miglioramento progressivo nella qualità della scrittura.
}
{ % Descrizione PB
    Nella PB, gli errori grammaticali sono stati praticamente assenti, grazie a un processo di revisione ben strutturato.
}

\MetNoPB
{ % METRICA
    MPD3 - Requisiti obbligatori soddisfatti     
}
{ % GRAFICO
    template/images/REQ-O.png
}
{   % DESCRIZIONE PB
    Il grafico evidenzia il pieno soddisfacimento di tutti i requisiti obbligatori, garantendo la conformità del prodotto agli obiettivi minimi concordati con il proponente.
}

\MetNoPB
{ % METRICA
    MPD4 - Requisiti desiderabili soddisfatti    
}
{ % GRAFICO
    template/images/REQ-D.png
}
{   % DESCRIZIONE PB
    Il team non ha individuato requisiti desiderabili durante la fase di analisi, pertanto questa metrica non è applicabile.
}

\MetNoPB
{ % METRICA
    MPD5 - Requisiti opzionali soddisfatti    
}
{ % GRAFICO
    template/images/REQ-OP.png
}
{   % DESCRIZIONE PB
I requisiti opzionali sono stati presi in considerazione sin dall'inizio della fase di sviluppo, il che ha permesso al team di dimostrare efficacia e visione progettuale sin dalle prime fasi del lavoro. Il loro completamento ha contribuito ad arricchire ulteriormente le funzionalità del prodotto, andando oltre gli obiettivi minimi previsti.
}

\MetNoPB
{ % METRICA
    MPD6 - Casi di test superati     
}
{ % GRAFICO
    template/images/TEST.png
}
{   % DESCRIZIONE PB
    Il numero elevato di casi di test superati indica una buona copertura del codice e l’affidabilità del sistema rispetto ai comportamenti attesi.
}

\MetNoPB
{ % METRICA
    MPD7 - Densità errori
}
{ % GRAFICO
    template/images/DE.png
}
{   % DESCRIZIONE PB
La densità di errori ha raggiunto valori sotto la soglia di ottimalità definita dal team negli ultimi due sprint, indicando un significativo miglioramento nella qualità del codice e nella fase di verifica e validazione.
}

\MetNoPB
{ % METRICA
    MPD8 - Tempo di apprendimento    
}
{ % GRAFICO
    template/images/TEMP-A.png
}
{   % DESCRIZIONE PB
    Il tempo di apprendimento per un nuovo utente è risultato contenuto, confermando la facilità d’uso e l’intuitività dell’interfaccia.
}

\MetNoPB
{ % METRICA
    MPD9 - Tempo di caricamento    
}
{ % GRAFICO
    template/images/TEMP-C.png
}
{   % DESCRIZIONE PB
    I tempi di caricamento delle pagine sono stati ottimizzati e si mantengono al di sotto della soglia ottimale, migliorando l’esperienza utente.
}

\MetNoPB
{ % METRICA
    MPD10 - Tempo medio di risposta    
}
{ % GRAFICO
    template/images/TEMP-R.png
}
{   % DESCRIZIONE PB
    Il tempo medio di risposta alle richieste dell’utente è risultato conforme agli standard previsti.
}

\MetNoPB
{ % METRICA
    MPD11 - Complessità ciclomatica 
}
{ % GRAFICO
    template/images/CC.png
}
{   % DESCRIZIONE PB
    La complessità ciclomatica, che misura la complessità del flusso di controllo del codice, è stata mantenuta entro livelli accettabili. Questo approccio favorisce la manutenibilità del codice e facilita la fase di testing.
}

\MetNoPB
{ % METRICA
    MPD12 - Parametri per metodo
}
{ % GRAFICO
    template/images/PM.png
}
{   % DESCRIZIONE PB
    Il numero di parametri per metodo è stato contenuto per migliorare la leggibilità del codice e ridurre l'accoppiamento tra componenti.
}

\MetNoPB
{ % METRICA
    MPD13 -  Linee di codice per metodo
}
{ % GRAFICO
    template/images/LINEE-M.png
}
{   % DESCRIZIONE PB
Inizialmente, alcuni file presentavano un numero elevato di linee di codice per metodo, rendendo il codice meno leggibile. Tuttavia, nel successivo sprint, sono stati rivisitati e ottimizzati, garantendo una distribuzione più equilibrata e migliorando la comprensibilità.
}

\MetNoPB
{ % METRICA
    MPD14 -  Linee di codice per file
}
{ % GRAFICO
    template/images/LINEE-F.png
}
{   % DESCRIZIONE PB
    I file mantengono una dimensione gestibile, favorendo la modularità e la manutenzione del progetto.
}

\MetNoPB
{ % METRICA
    MPD15 -  Densità dei commenti
}
{ % GRAFICO
    template/images/DC.png
}
{   % DESCRIZIONE PB
    La densità dei commenti è risultata adeguata, garantendo una buona documentazione interna al codice senza appesantirlo inutilmente. Grazie alle norme definite, il team ha rispettato le linee guida, inserendo solo i commenti necessari per mantenere il codice chiaro e comprensibile.
}

\MetNoPB
{ % METRICA
    MPD16 -  Supported Browser
}
{ % GRAFICO
    template/images/SB.png
}
{   % DESCRIZIONE PB
    Il prodotto è stato testato con successo su una buona varietà di browser, per assicurare compatibilità e accessibilità per un’ampia base di utenti.
}
