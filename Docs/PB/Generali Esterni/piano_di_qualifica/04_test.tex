\section{Specifica dei test}
L'esecuzione dei test rappresenta un passaggio fondamentale per verificare che
il prodotto, nella sua totalità, soddisfi pienamente tutti i requisiti
specificati nel documento di \textit{Analisi dei requisiti v1.0.0}. I test si
classificano in:
\begin{itemize}
    \item \textbf{Test di unità}: vengono definiti durante la progettazione di
          dettaglio e verificano il funzionamento delle singole unità software;
    \item \textbf{Test di integrazione}: vengono definiti durante la progettazione
          architetturale e verificano le interazioni tra le componenti;
    \item \textbf{Test di sistema}: vengono definiti durante l'analisi dei requisiti
          e verificano che il sistema, nel suo complesso, soddisfi i requisiti software;
    \item \textbf{Test di accettazione}: vengono effettuati insieme al proponente
          durante la fase di collaudo, per accertare il soddisfacimento dei requisiti utente
          e concludere quindi la validazione del prodotto.
\end{itemize}
Per ciascun test, vengono forniti un codice identificativo, una descrizione, e
lo stato del test.
Come specificato nel documento \textit{Norme di progetto v1.0.0}, vengono utilizzate le seguenti
abbreviazioni per identificare lo stato dei test:
\begin{itemize}
    \item \textbf{NI}: Non Implementato;
    \item \textbf{S}: Superato;
    \item \textbf{NS}: Non Superato.
\end{itemize}

\newcommand{\testTable}[1]{
	 

\renewcommand{\arraystretch}{1.5}
\rowcolors{2}{pari}{dispari}
\begin{longtable}{ 
		>{\centering}M{0.15\textwidth} 
		>{\centering}M{0.60\textwidth} 
		>{\centering}M{0.10\textwidth}
		 }
	\rowcolorhead
	\headertitle{Codice} &
	\headertitle{Descrizione} &
	\headertitle{Stato}
	\endfirsthead	
	\endhead
	
	#1

\end{longtable}
\vspace{0.2em}

}
\subsection{Test di unità}
\testTable{
    TU1 & Verificare che la classe \texttt{CacheService} venga istanziata correttamente & S\tabularnewline
    TU2 & Verificare che il metodo \texttt{get} della classe \texttt{CacheService} restituisca il valore inserito con il metodo \texttt{set} della stessa classe & S\tabularnewline
    TU3 & Verificare che il metodo \texttt{get} della classe \texttt{CacheService} restituisca \texttt{null} se non è stato precedentemente inserito alcun valore & S\tabularnewline
    TU4 & Verificare che il metodo \texttt{get} della classe \texttt{CacheService} restituisca \texttt{null} se si verifica un errore & S\tabularnewline
    TU5 & Verificare che il metodo \texttt{set} della classe \texttt{CacheService} non sollevi eccezioni se si verifica un errore & S\tabularnewline
    TU6 & Verificare che la classe \texttt{CacheService} venga distrutta correttamente & S\tabularnewline
    TU7 & Verificare che la classe \texttt{DataSourceController} venga istanziata correttamente & S\tabularnewline
    TU8 & Verificare che il metodo \texttt{getSources} della classe \texttt{DataSourceController} restituisca l'array dei dataset disponibili fornito dalla classe \texttt{DataSourceService} & S\tabularnewline
    TU9 & Verificare che la classe \texttt{DataSourceService} venga istanziata correttamente & S\tabularnewline
    TU10 & Verificare che il metodo \texttt{getSources} della classe \texttt{DataSourceService} restituisca l'array dei dataset disponibili & S\tabularnewline
    TU11 & Verificare che la classe \texttt{DataVisualizationController} venga istanziata correttamente & S\tabularnewline
    TU12 & Verificare che il metodo \texttt{getDataset} della classe \texttt{DataVisualizationController} restituisca il dataset selezionato & S\tabularnewline
    TU13 & Verificare che il metodo \texttt{getDataset} della classe \texttt{DataVisualizationController} invochi il metodo \texttt{getDatasetById} della classe \texttt{DataVisualizationService} con i parametri corretti & S\tabularnewline
    TU14 & Verificare che la classe \texttt{DataVisualizationService} venga istanziata correttamente & S\tabularnewline
    TU15 & Verificare che il metodo \texttt{getDatasetById} della classe \texttt{DataVisualizationService} sollevi un'eccezione se viene passato un ID non valido & S\tabularnewline
    TU16 & Verificare che il metodo \texttt{getDatasetById} della classe \texttt{DataVisualizationService} restituisca il dataset ottenuto dalla cache, se presente & S\tabularnewline
    TU17 & Verificare che il metodo \texttt{getDatasetById} della classe \texttt{DataVisualizationService} restituisca il dataset ottenuto dal \texttt{fetcher} corretto, se non presente in cache & S\tabularnewline
    TU18 & Verificare che il metodo \texttt{getDatasetById} della classe \texttt{DataVisualizationService} salvi il dataset in cache & S\tabularnewline
    TU19 & Verificare che il metodo \texttt{fetchersFactory} ritorni l'array dei \texttt{fetcher} disponibili & S\tabularnewline
    TU20 & Verificare che la classe \texttt{CurrencyApiFetcher} venga istanziata correttamente & S\tabularnewline
    TU21 & Verificare che il metodo \texttt{getName} della classe \texttt{CurrencyApiFetcher} restituisca il valore corretto & S\tabularnewline
    TU22 & Verificare che il metodo \texttt{getSize} della classe \texttt{CurrencyApiFetcher} restituisca il valore corretto & S\tabularnewline
    TU23 & Verificare che il metodo \texttt{getDescription} della classe \texttt{CurrencyApiFetcher} restituisca il valore corretto & S\tabularnewline
    TU24 & Verificare che il metodo \texttt{fetchData} della classe \texttt{CurrencyApiFetcher} restituisca il dataset nel formato corretto e con i valori corretti & S\tabularnewline
    TU25 & Verificare che il metodo \texttt{fetchData} della classe \texttt{CurrencyApiFetcher} sollevi un'eccezione se la chiave API non è definita & S\tabularnewline
    TU26 & Verificare che il metodo \texttt{fetchData} della classe \texttt{CurrencyApiFetcher} sollevi un'eccezione se si verifica un errore generico durante la chiamata API & S\tabularnewline
    TU27 & Verificare che il metodo \texttt{fetchData} della classe \texttt{CurrencyApiFetcher} sollevi un'eccezione se il formato dei dati non è corretto & S\tabularnewline
    TU28 & Verificare che il metodo \texttt{fetchData} della classe \texttt{CurrencyApiFetcher} sollevi un'eccezione se sono state effettuate troppe richieste & S\tabularnewline
    TU29 & Verificare che la classe \texttt{FlightsApiFetcher} venga istanziata correttamente & S\tabularnewline
    TU30 & Verificare che il metodo \texttt{getName} della classe \texttt{FlightsApiFetcher} restituisca il valore corretto & S\tabularnewline
    TU31 & Verificare che il metodo \texttt{getSize} della classe \texttt{FlightsApiFetcher} restituisca il valore corretto & S\tabularnewline
    TU32 & Verificare che il metodo \texttt{getDescription} della classe \texttt{FlightsApiFetcher} restituisca il valore corretto & S\tabularnewline
    TU33 & Verificare che il metodo \texttt{fetchData} della classe \texttt{FlightsApiFetcher} restituisca il dataset nel formato corretto e con i valori corretti & S\tabularnewline
    TU34 & Verificare che il metodo \texttt{fetchData} della classe \texttt{FlightsApiFetcher} sollevi un'eccezione se si verifica un errore generico durante la chiamata API & S\tabularnewline
    TU35 & Verificare che il metodo \texttt{fetchData} della classe \texttt{FlightsApiFetcher} restituisca il dataset corretto se una chiamata API restituisce un errore 404 & S\tabularnewline
    TU35 & Verificare che il metodo \texttt{fetchData} della classe \texttt{FlightsApiFetcher} sollevi un'eccezione se il formato dei dati non è corretto & S\tabularnewline
    TU36 & Verificare che la classe \texttt{PopulationApiFetcher} venga istanziata correttamente & S\tabularnewline
    TU37 & Verificare che il metodo \texttt{getName} della classe \texttt{PopulationApiFetcher} restituisca il valore corretto & S\tabularnewline
    TU38 & Verificare che il metodo \texttt{getSize} della classe \texttt{PopulationApiFetcher} restituisca il valore corretto & S\tabularnewline
    TU39 & Verificare che il metodo \texttt{getDescription} della classe \texttt{PopulationApiFetcher} restituisca il valore corretto & S\tabularnewline
    TU40 & Verificare che il metodo \texttt{fetchData} della classe \texttt{PopulationApiFetcher} restituisca il dataset nel formato corretto e con i valori corretti & S\tabularnewline
    TU41 & Verificare che il metodo \texttt{fetchData} della classe \texttt{PopulationApiFetcher} sollevi un'eccezione se si verifica un errore generico durante la chiamata API & S\tabularnewline
    TU42 & Verificare che il metodo \texttt{fetchData} della classe \texttt{PopulationApiFetcher} sollevi un'eccezione se il formato dei dati non è corretto & S\tabularnewline
    TU43 & Verificare che la classe \texttt{WeatherApiFetcher} venga istanziata correttamente & S\tabularnewline
    TU44 & Verificare che il metodo \texttt{getName} della classe \texttt{WeatherApiFetcher} restituisca il valore corretto & S\tabularnewline
    TU45 & Verificare che il metodo \texttt{getSize} della classe \texttt{WeatherApiFetcher} restituisca il valore corretto & S\tabularnewline
    TU46 & Verificare che il metodo \texttt{getDescription} della classe \texttt{WeatherApiFetcher} restituisca il valore corretto & S\tabularnewline
    TU47 & Verificare che il metodo \texttt{fetchData} della classe \texttt{WeatherApiFetcher} restituisca il dataset nel formato corretto e con i valori corretti & S\tabularnewline
    TU48 & Verificare che il metodo \texttt{fetchData} della classe \texttt{WeatherApiFetcher} sollevi un'eccezione se si verifica un errore generico durante la chiamata API & S\tabularnewline
    TU49 & Verificare che il metodo \texttt{fetchData} della classe \texttt{WeatherApiFetcher} sollevi un'eccezione se il formato dei dati non è corretto & S\tabularnewline
    TU50 & Verificare che lo stato in \texttt{AppSlice} sia quello atteso quando la richiesta è in corso & S\tabularnewline
    TU51 & Verificare che lo stato in \texttt{AppSlice} sia quello atteso quando la richiesta è completata & S\tabularnewline
    TU52 & Verificare che lo stato in \texttt{AppSlice} venga resettato quando viene effettuata una nuova rischiesta & S\tabularnewline
    TU53 & Verificare che lo stato in \texttt{AppSlice} sia quello atteso quando si verifica un errore \texttt{TooManyRequestsError} & S\tabularnewline
    TU54 & Verificare che lo stato in \texttt{AppSlice} sia quello atteso quando si verifica un errore \texttt{NotFoundError} & S\tabularnewline
    TU55 & Verificare che lo stato in \texttt{AppSlice} sia quello atteso quando si verifica un errore \texttt{ServerError} & S\tabularnewline
    TU56 & Verificare che lo stato in \texttt{AppSlice} sia quello atteso quando si seleziona un dataset inesistente \texttt{BadRequestError} & S\tabularnewline
    TU57 & Verificare che il metodo \texttt{selectorAppState} funzioni correttamente & S\tabularnewline
    TU58 & Verificare che lo stato in \texttt{DataSlice} sia quello atteso quando non è selezionato alcun dataset & S\tabularnewline
    TU59 & Verificare che lo stato in \texttt{DataSlice} sia quello atteso quando il dataset selezionato viene caricato con successo & S\tabularnewline
    TU60 & Verificare che lo stato in \texttt{DataSlice} sia quello atteso quando il dataset selezionato non viene caricato correttamente & S\tabularnewline
    TU61 & Verificare che lo stato in \texttt{DataSlice} sia quello atteso quando il dataset selezionato non esiste & S\tabularnewline
    TU62 & Verificare che lo stato in \texttt{DataSlice} sia quello atteso quando si effettua un filtraggio degli N valori più alti & S\tabularnewline
    TU63 & Verificare che lo stato in \texttt{DataSlice} sia quello atteso quando si effettua un filtraggio degli N valori più bassi & S\tabularnewline
    TU64 & Verificare che lo stato in \texttt{DataSlice} sia quello atteso quando si effettua un filtraggio dei valori superiori o uguali al valore medio & S\tabularnewline
    TU65 & Verificare che lo stato in \texttt{DataSlice} sia quello atteso quando si effettua un filtraggio dei valori inferiori o uguali al valore medio & S\tabularnewline
    TU66 & Verificare che lo stato in \texttt{DataSlice} sia quello atteso quando si effettua un filtraggio dei valori superiori o uguali a un valore fissato & S\tabularnewline
    TU67 & Verificare che lo stato in \texttt{DataSlice} sia quello atteso quando si effettua un filtraggio dei valori inferiori o uguali a un valore fissato & S\tabularnewline
    TU68 & Verificare che lo stato in \texttt{DataSlice} sia quello atteso quando si resettano i filtri & S\tabularnewline
    TU69 & Verificare che il metodo \texttt{selectorDataState} funzioni correttamente & S\tabularnewline
    TU70 & Verificare che lo stato in \texttt{DataSourceSlice} sia quello atteso quando non è selezionato alcun dataset & S\tabularnewline
    TU71 & Verificare che lo stato in \texttt{DataSourceSlice} sia quello atteso quando le informazioni dei dataset vengono caricate con successo & S\tabularnewline
    TU72 & Verificare che lo stato in \texttt{DataSourceSlice} sia quello atteso quando le informazioni dei dataset non vengono caricate correttamente & S\tabularnewline
    TU73 & Verificare che lo stato in \texttt{DataSourceSlice} sia quello atteso quando viene selezionato un dataset & S\tabularnewline
    TU74 & Verificare che lo stato in \texttt{DataSourceSlice} sia quello atteso quando viene selezionato un dataset inesistente & S\tabularnewline
    TU75 & Verificare che il metodo \texttt{selectorDatasets} funzioni correttamente & S\tabularnewline
    TU76 & Verificare che il metodo \texttt{selectorCurrentDataset} funzioni correttamente & S\tabularnewline
    TU77 & Verificare che lo stato in \texttt{RaycastHitSlice} sia quello atteso quando il puntatore del mouse si trova sopra una barra del grafico 3D & S\tabularnewline
    TU78 & Verificare che lo stato in \texttt{RaycastHitSlice} sia quello atteso quando il puntatore del mouse clicca su una barra del grafico 3D & S\tabularnewline
    TU79 & Verificare che lo stato in \texttt{RaycastHitSlice} sia quello atteso quando il puntatore del mouse si trova sopra uno spazio vuoto & S\tabularnewline
    TU80 & Verificare che lo stato in \texttt{RaycastHitSlice} sia quello atteso quando il puntatore del mouse clicca su uno spazio vuoto & S\tabularnewline
    TU81 & Verificare che lo stato in \texttt{RaycastHitSlice} sia quello atteso quando il puntatore del mouse si trova sopra più barre consecutivamente & S\tabularnewline
    TU82 & Verificare che lo stato in \texttt{RaycastHitSlice} sia quello atteso quando il puntatore del mouse clicca su più barre consecutivamente & S\tabularnewline
    TU83 & Verificare che il metodo \texttt{selectorRaycastHit} funzioni correttamente & S\tabularnewline
}
\subsection{Test di integrazione}
\testTable{
    TI1 & Verifica che la HomePage venga caricata correttamente & S\tabularnewline
    TI2 & Verifica che i dataset disponibili vengano caricati correttamente & S\tabularnewline
    TI3 & Verifica che, al caricamento del dataset, il nome del dataset corrente sia visualizzato nella pagina dell'ambiente & S\tabularnewline
    TI4 & Verifica che durante il caricamento del dataset, sia visibile il testo di caricamento & S\tabularnewline
    TI5 & Verifica che al termine del caricamento del dataset, il testo di caricamento non sia più visibile & S\tabularnewline
    TI6 & Verifica che la legenda dell'ambiente 3D sia caricata correttamente & S\tabularnewline
    TI7 & Verifica che una cella contenente un valore non filtrato sia visibile e che le venga applicata la classe CSS \texttt{highlight} & S\tabularnewline
    TI8 & Verifica che una cella contenente un valore filtrato non sia visibile e non abbia la classe CSS \texttt{highlight} & S\tabularnewline
    TI9 & Verifica che, dopo aver cliccato una cella, ogni altra cella con un valore numericamente maggiore cambi stile in quanto considerata filtrata & S\tabularnewline
    TI10 & Verifica che, dopo aver cliccato una cella, ogni altra cella con un valore numericamente inferiore cambi stile in quanto considerata filtrata & S\tabularnewline
    TI11 & Verifica che, eseguendo un filtraggio per i valori inferiori al valore medio, tutte le celle della tabella che soddisfano la condizione abbiano la classe CSS appropriata & S\tabularnewline
    TI12 & Verifica che, eseguendo un filtraggio per i valori superiori al valore medio, tutte le celle della tabella che soddisfano la condizione abbiano la classe CSS appropriata & S\tabularnewline
    TI13 & Verifica che, eseguendo un filtraggio per i bottom 2 valori, le due celle della tabella con i valori più bassi abbiano la classe CSS appropriata & S\tabularnewline
    TI14 & Verifica che, eseguendo un filtraggio per i top 2 valori, le due celle della tabella con i valori più alti abbiano la classe CSS appropriata & S\tabularnewline
    TI15 & Verifica che, eseguendo il reset dei filtri, tutte le celle della tabella ritornino ad avere la classe CSS predefinita o quella che avevano prima dell'applicazione dei filtri & S\tabularnewline
    TI16 & Verifica che il bottone di reset della camera sia visibile al caricamento del dataset & S\tabularnewline
    TI17 & Verifica che, al click sul bottone di reset camera, venga eseguita un'animazione di tweening sulla telecamera che la riporta alla sua posizione iniziale & S\tabularnewline
    TI18 & Verifica che l'\texttt{AppStatus} non contenga errori & S\tabularnewline
    TI19 & Verifica che i dati dell'errore vengano caricati correttamente & S\tabularnewline
    TI20 & Verifica che il componente \texttt{ApiSelector} venga caricato correttamente & S\tabularnewline
}
\subsection{Test di sistema}
\testTable{
    TS1 & Verifica che l'utente possa visualizzare la lista dei dataset disponibili & S\tabularnewline
    TS2 & Verifica che l'utente possa visualizzare le informazioni primarie dei dataset & S\tabularnewline
    TS3 & Verifica che l'utente possa visualizzare il nome dei dataset & S\tabularnewline
    TS4 & Verifica che l'utente possa visualizzare la dimensione dei dataset & S\tabularnewline
    TS5 & Verifica che l'utente possa visualizzare i dettagli di un dataset & S\tabularnewline
    TS6 & Verifica che l'utente possa visualizzare la descrizione del contenuto di un dataset & S\tabularnewline
    TS7 & Verifica che l'utente possa caricare nel sistema un dataset della lista & S\tabularnewline
    TS8 & Verifica che, nel caso in cui il caricamento del dataset fallisca,
    venga visualizzato un messaggio di errore & S\tabularnewline
    TS9 & Verifica che l'utente possa visualizzare il dataset in forma tabellare & S\tabularnewline
    TS10 & Verifica che l'utente possa visualizzare le intestazioni della tabella & S\tabularnewline
    TS11 & Verifica che l'utente possa visualizzare i valori del dataset nelle celle della tabella & S\tabularnewline
    TS12 & Verifica che l'utente possa visualizzare il dataset in
    forma di grafico 3D a barre verticali & S\tabularnewline
    TS13 & Verifica che l'utente possa visualizzare gli assi del grafico & S\tabularnewline
    TS14 & Verifica che l'utente possa visualizzare l'asse X con i valori appropriati & S\tabularnewline
    TS15 & Verifica che l'utente possa visualizzare l'asse Y con i valori appropriati & S\tabularnewline
    TS16 & Verifica che l'utente possa visualizzare l'asse Z con i valori appropriati & S\tabularnewline
    TS17 & Verifica che l'utente possa spostare orizzontalmente la vista & S\tabularnewline
    TS18 & Verifica che l'utente possa spostare verticalmente la vista & S\tabularnewline
    TS19 & Verifica che l'utente possa ruotare il grafico  & S\tabularnewline
    TS20 & Verifica che l'utente possa ruotare il grafico attorno all'asse X & S\tabularnewline
    TS21 & Verifica che l'utente possa ruotare il grafico attorno all'asse Y & S\tabularnewline
    TS22 & Verifica che l'utente possa ruotare il grafico attorno all'asse Z & S\tabularnewline
    TS23 & Verifica che l'utente possa compiere un'azione di zoom indipendentemente dalla posizione della telecamera & S\tabularnewline
    TS24 & Verifica che l'utente possa riposizionare la vista alla posizione iniziale & S\tabularnewline
    TS25 & Verifica che l'utente possa visualizzare i dettagli di una barra del grafico & S\tabularnewline
    TS26 & Verifica che l'utente possa visualizzare le etichette della barra del grafico & S\tabularnewline
    TS27 & Verifica che l'utente possa visualizzare il valore rappresentato dalla barra del grafico & S\tabularnewline
    TS28 & Verifica che l'utente possa visualizzare una singola barra del grafico in primo piano & S\tabularnewline
    TS29 & Verifica che l'utente possa filtrare su valori che soddisfano certe condizioni & S\tabularnewline
    TS30 & Verifica che l'utente possa filtrare su un numero arbitrario (N) di valori & S\tabularnewline
    TS31 & Verifica che l'utente possa inserire un numero N di valori su cui filtrare & S\tabularnewline
    TS32 & Verifica che l'utente possa filtrare gli N valori più alti & S\tabularnewline
    TS33 & Verifica che l'utente possa filtrare gli N valori più bassi & S\tabularnewline
    TS34 & Verifica che l'utente possa filtrare su valori inferiori o uguali al valore scelto & S\tabularnewline
    TS35 & Verifica che l'utente possa filtrare su valori inferiori o uguali rispetto al valor medio globale & S\tabularnewline
    TS36 & Verifica che l'utente possa filtrare su valori superiori o uguali al valore scelto & S\tabularnewline
    TS37 & Verifica che l'utente possa filtrare su valori superiori o uguali rispetto al valor medio globale & S\tabularnewline
    TS38 & Verifica che l'utente possa ripristinare lo stato del grafico rimuovendo i filtri precedentemente applicati & S\tabularnewline
    TS39 & Verifica che l'utente possa visualizzare il piano parallelo alla base del grafico che rappresento un valor medio & S\tabularnewline
    TS40 & Verifica che l'utente possa visualizzare il piano parallelo alla base del grafico che rappresenta il valore medio globale & S\tabularnewline
    TS41 & Verifica che l'utente possa visualizzare il piano parallelo alla base del grafico che rappresenta il valore medio di un singolo elemento del dataset & NI\tabularnewline
}
\subsection{Test di accettazione}
\testTable{
    TA1 & Verifica che all'avvio del sistema la lista dei dataset disponibili
    sia visibile all'utente & S\tabularnewline
    TA2 & Verifica che l'utente possa visualizzare le informazioni primarie di
    un dataset & S\tabularnewline
    TA3 & Verifica che l'utente, selezionando un dataset, possa visualizzarne
    i dettagli & S\tabularnewline
    TA4 & Verifica che l'utente possa caricare un dataset nel sistema
    & S\tabularnewline
    TA5 & Verifica che la tabella dei dati sia visibile correttamente
    & S\tabularnewline
    TA6 & Verifica che il grafico 3D a barre verticali sia visibile correttamente
    & S\tabularnewline
    TA7 & Verifica che gli assi del grafico siano visibili e abbiano le
    etichette appropriate & S\tabularnewline
    TA8 & Verifica che l'utente possa spostare la vista del grafico senza
    modificarne l'angolazione (panning) & S\tabularnewline
    TA9 & Verifica che l'utente possa ruotare la vista del grafico attorno
    all'asse X & S\tabularnewline
    TA10 & Verifica che l'utente possa ruotare la vista del grafico attorno
    all'asse Y & S\tabularnewline
    TA11 & Verifica che l'utente possa ruotare la vista del grafico attorno
    all'asse Z & S\tabularnewline
    TA12 & Verifica che l'utente possa compiere un'azione di zoom sul grafico
    & S\tabularnewline
    TA13 & Verifica che l'utente possa riposizionare la vista del grafico alla
    posizione iniziale & S\tabularnewline
    TA14 & Verifica che l'utente, selezionando una cella della tabella, possa
    visualizzare in primo piano la relativa barra & S\tabularnewline
    TA15 & Verifica che l'utente, posizionando il cursore su una barra del
    grafico, possa visualizzarne i dettagli & S\tabularnewline
    TA16 & Verifica che l'utente, inserendo un numero arbitrario (N), possa
    filtrare gli N valori più alti & S\tabularnewline
    TA17 & Verifica che l'utente, inserendo un numero arbitrario (N), possa
    filtrare gli N valori più bassi & S\tabularnewline
    TA18 & Verifica che l'utente possa filtrare i valori superiori (o uguali)
    al valore medio globale & S\tabularnewline
    TA19 & Verifica che l'utente possa filtrare i valori inferiori (o uguali)
    al valore medio globale & S\tabularnewline
    TA20 & Verifica che l'utente, selezionando una barra del grafico o una
    cella della tabella, possa filtrare i valori superiori (o uguali) al valore
    della barra o della cella selezionata & S\tabularnewline
    TA21 & Verifica che l'utente, selezionando una barra del grafico o una
    cella della tabella, possa filtrare i valori inferiori (o uguali) al valore
    della barra o della cella selezionata & S\tabularnewline
    TA22 & Verifica che l'utente possa rimuovere tutti i filtri applicati
    al grafico & S\tabularnewline
    TA23 & Verifica che l'utente possa visualizzare il piano parallelo alla base
    del grafico che rappresenta il valore medio globale & S\tabularnewline
    TA24 & Verifica che l'utente possa visualizzare il piano parallelo alla base
    del grafico che rappresenta la media dei valori relativi a una determinata
    etichetta di un asse & NI\tabularnewline
}
