\section{Tecnologie}
Questa sezione descrive gli strumenti e le tecnologie adottati per lo sviluppo,
il testing e il deployment del software relativo al progetto \textit{3Dataviz}.
In particolare, vengono illustrati il linguaggio di programmazione, le
librerie, i framework e le infrastrutture utilizzate.

\subsection{Linguaggio di programmazione}
\subsubsection{TypeScript}
TypeScript è un linguaggio di programmazione open-source sviluppato da
Microsoft. Si tratta di un superset di JavaScript, che introduce la
tipizzazione statica, consentendo di rilevare errori di programmazione in fase
di sviluppo, ridurre il numero di bug e migliorare la qualità del codice.
\begin{itemize}
    \item \textbf{Versione}: 5.8.2;
    \item \textbf{Documentazione}: \url{https://www.typescriptlang.org/docs/} (consultato:
          25-03-2025).
\end{itemize}
Nel contesto del progetto \textit{3Dataviz}, TypeScript è il linguaggio principale
utilizzato per lo sviluppo sia del frontend che del backend. \\Per il frontend,
si utilizza TSX, un'estensione di Typescript che permette di definire la struttura
dell'interfaccia utente in modo più leggibile e dichiarativo, integrando HTML
direttamente nel codice TypeScript.

\subsection{Formato dei dati}
\subsubsection{JSON}
JSON (JavaScript Object Notation) è un formato di scambio dati leggero e
indipendente dal linguaggio di programmazione. È ampiamente utilizzato nello
sviluppo web e nello scambio di dati tra applicazioni. \\JSON è basato su due
strutture di dati:
\begin{itemize}
    \item \textbf{Oggetti}: insiemi di coppie chiave-valore;
    \item \textbf{Array}: liste ordinate di valori.
\end{itemize}
JSON viene utilizzato per la trasmissione dei dati tra frontend e backend, nonché
per la memorizzazione dei dati nel sistema di cache.

\subsection{Frontend}
\subsubsection{Vite}
Vite è uno strumento di build progettato per offrire un'esperienza di sviluppo
veloce e fluida nei progetti web moderni. Si compone di due parti principali:
\begin{itemize}
    \item Un server di sviluppo, che estende le funzionalità dei moduli ES nativi,
          offrendo, ad esempio, un Hot Module Replacement (HMR) estremamente rapido;
    \item Un comando di compilazione che crea il bundle del codice utilizzando Rollup,
          preconfigurato per generare risorse statiche ottimizzate per la produzione.
\end{itemize}
Vite è estendibile tramite plugin, che consentono di personalizzare
il processo di sviluppo e di compilazione. Nel progetto \textit{3Dataviz}, viene utilizzato il
plugin \texttt{@vitejs/plugin-react-swc}, che abilita la sintassi JSX e ottimizza la compilazione del codice React.

\begin{itemize}
    \item \textbf{Versione}: 6.2.3;
    \item \textbf{Documentazione}: \url{https://vitejs.dev/guide/} (consultato:
          25-03-2025).
\end{itemize}

\subsubsection{Three.js}
Three.js è una libreria JavaScript open-source per la creazione di grafica 3D.
Fornisce un'API ad alto livello per la creazione di scene 3D complesse,
utilizzando WebGL come backend per la grafica hardware-accelerata. \\Three.js
semplifica la creazione di oggetti 3D, luci, materiali e animazioni, offrendo
funzionalità avanzate come il ray tracing e il supporto per shading e texture.
\begin{itemize}
    \item \textbf{Versione}: 0.174.0;
    \item \textbf{Documentazione}: \url{https://threejs.org/docs/} (consultato:
          25-03-2025).
\end{itemize}
Questa è la libreria intorno su cui si basa l'intero progetto, in quanto
viene utilizzata per la visualizzazione dei dati sotto forma di grafico 3D, nonché per la
navigazione e l'interazione con il grafico.

\subsubsection{GSAP}
GSAP (GreenSock Animation Platform) è una libreria JavaScript open-source per
la creazione di animazioni di alta qualità.
\begin{itemize}
    \item \textbf{Versione}: 3.12.7;
    \item \textbf{Documentazione}: \url{https://gsap.com/docs/v3} (consultato:
          25-03-2025).
\end{itemize}
GSAP è necessaria per l'implementazione dei casi d'uso relativi al posizionamento
della camera in punti specifici della scena 3D, nonché per la gestione delle
animazioni di transizione tra un punto e l'altro.

\subsubsection{React}
React è una libreria JavaScript open-source per la creazione di interfacce
utente. Consente di definire componenti riutilizzabili che rappresentano parti
dell'interfaccia, aggiornando automaticamente la vista in base allo stato
dell'applicazione. \\React adotta un approccio dichiarativo per la definizione
dell'interfaccia utente, semplificando la gestione dello stato e la
sincronizzazione con il DOM. \\Non è opinionato nella scelta delle librerie di
routing o di gestione dello stato, permettendo una facile integrazione con
altre tecnologie.
\begin{itemize}
    \item \textbf{Versione}: 19.0.0;
    \item \textbf{Documentazione}: \url{https://react.dev/learn} (consultato:
          25-03-2025).
\end{itemize}

\paragraph{React Router}
Libreria open-source che fornisce un sistema di routing per React. Consente di
definire rotte all'interno dell'applicazione, associando componenti React a URL
specifici. \\React Router gestisce la navigazione tra le diverse rotte,
aggiornando il contenuto dell'interfaccia utente in base all'URL corrente.
\begin{itemize}
    \item \textbf{Versione}: 7.4.0;
    \item \textbf{Documentazione}: \url{https://reactrouter.com/home} (consultato:
          25-03-2025).
\end{itemize}

\paragraph{React Three Fiber}
Libreria open-source che permette di integrare Three.js all'interno di
un'applicazione React. Fornisce un'API dichiarativa per la creazione di scene
3D, semplificando l'integrazione di Three.js con React e consentendo di
definire componenti 3D riutilizzabili.
\begin{itemize}
    \item \textbf{Versione}: 9.1.0;
    \item \textbf{Documentazione}: \url{https://docs.pmnd.rs/react-three-fiber/getting-started/introduction} (consultato:
          25-03-2025).
\end{itemize}

\paragraph{React Three Drei}
Libreria open-source che fornisce componenti e hook riutilizzabili per React
Three Fiber. Contiene una vasta gamma di componenti pronti all'uso per la
creazione di scene 3D, semplificando lo sviluppo di applicazioni 3D basate su
React e Three.js.
\begin{itemize}
    \item \textbf{Versione}: 10.0.5;
    \item \textbf{Documentazione}: \url{https://drei.docs.pmnd.rs/getting-started/introduction} (consultato:
          25-03-2025).
\end{itemize}

\subsubsection{Redux}
Redux è una libreria open-source per la gestione dello stato globale nelle
applicazioni JavaScript. Si basa su due concetti fondamentali:
\begin{itemize}
    \item \textbf{Store}, che rappresenta lo stato dell'applicazione;
    \item \textbf{Azioni}, che definiscono le modifiche allo stato.
\end{itemize}
Redux semplifica la gestione dello stato, fornendo un'API per definire
azioni e reducer, che aggiornano lo stato in base alle azioni ricevute.
\begin{itemize}
    \item \textbf{Versione}: 5.0.1;
    \item \textbf{Documentazione}: \url{https://redux.js.org/introduction/getting-started} (consultato:
          25-03-2025).
\end{itemize}
Redux è utilizzato per gestire lo stato globale dell'applicazione, in particolare per
memorizzare i dati caricati dal backend e per gestire lo stato della
visualizzazione 3D.

\paragraph{Redux Toolkit}
Pacchetto ufficiale per Redux, progettato per semplificare la gestione dello
stato. Viene utilizzato per configurare lo store Redux e per definire le azioni
e i reducer necessari alla gestione dello stato globale dell'applicazione.
\begin{itemize}
    \item \textbf{Versione}: 2.6.1;
    \item \textbf{Documentazione}: \url{https://redux-toolkit.js.org/introduction/getting-started} (consultato:
          25-03-2025).
\end{itemize}

\paragraph{React Redux}
Pacchetto che funge da collegamento tra l'interfaccia utente in React e Redux.
Consente ai componenti React di leggere i dati dallo store Redux e di inviare
azioni per aggiornare lo stato dell'applicazione.
\begin{itemize}
    \item \textbf{Versione}: 9.2.0;
    \item \textbf{Documentazione}: \url{https://react-redux.js.org/introduction/getting-started} (consultato:
          25-03-2025).
\end{itemize}

\subsubsection{Axios}
Axios è un client HTTP basato su Promise per Node.js e browser. È isomorfo,
cioè può essere eseguito sia nel browser che in Node.js utilizzando lo stesso
codice sorgente. Lato server, Axios utilizza il modulo http nativo di Node.js,
mentre lato client usa XMLHttpRequest.
\begin{itemize}
    \item \textbf{Versione}: 1.8.4;
    \item \textbf{Documentazione}: \url{https://axios-http.com/docs/intro} (consultato:
          25-03-2025).
\end{itemize}
Nel progetto \textit{3Dataviz}, Axios viene utilizzato:
\begin{itemize}
    \item Lato frontend, per effettuare richieste HTTP al backend e ottenere i dati da
          visualizzare nel grafico 3D;
    \item Lato backend, per effettuare richieste HTTP alle API esterne, acquisendo i dati
          da elaborare e inviare al frontend.
\end{itemize}

\subsection{Backend}
\subsubsection{Node.js}
Node.js è un runtime JavaScript open-source basato sul motore V8 di Google
Chrome. Consente di eseguire codice JavaScript lato server, utilizzando un
modello di I/O asincrono e orientato agli eventi. Node.js è progettato per
essere leggero e scalabile, adatto per la creazione di applicazioni web ad alte
prestazioni.
\begin{itemize}
    \item \textbf{Versione}: 22.14.0;
    \item \textbf{Documentazione}: \url{https://nodejs.org/en/docs/} (consultato:
          25-03-2025).
\end{itemize}

\subsubsection{NestJS}
NestJS è un framework per Node.js che offre un'architettura modulare e
scalabile per la creazione di applicazioni lato server. Si basa su
Express.js e fornisce funzionalità aggiuntive come dependency injection,
middleware e decorator. NestJS consente di organizzare il codice in moduli
riutilizzabili e di definire facilmente controller, servizi e provider.
\begin{itemize}
    \item \textbf{Versione}: 11.0.12;
    \item \textbf{Documentazione}: \url{https://docs.nestjs.com/} (consultato:
          26-03-2025).
\end{itemize}
Nel progetto \textit{3Dataviz}, NestJS è utilizzato per creare il backend.
In particolare, viene impiegato per definire i controller che gestiscono le richieste HTTP provenienti dal frontend,
i servizi per ottenere ed elaborare i dati dalle API esterne, e i
provider per l'iniezione delle dipendenze.

\subsubsection{Memcached}
Memcached è un sistema di cache open-source ad alte prestazioni, progettato per
accelerare le applicazioni web dinamiche riducendo il carico sul database.
Funziona come un archivio chiave-valore per piccoli dati arbitrari (fino a 1
MB) che possono includere risultati di query a database o chiamate a API.
\begin{itemize}
    \item \textbf{Versione}: 1.6.38;
    \item \textbf{Documentazione}: \url{https://docs.memcached.org/} (consultato:
          26-03-2025).
\end{itemize}
Memcached viene utilizzato nel progetto per memorizzare i dati ottenuti dalle API esterne,
al fine di ridurre i tempi di accesso ai dati e il numero di richieste
effettuate alle API. Ciò è particolarmente utile per le API che impongono limiti sul numero di richieste per unità di tempo.

\subsection{Deployment}
\subsubsection{Docker}
Docker è una piattaforma open-source per lo sviluppo, la distribuzione e
l'esecuzione di applicazioni in contenitori. I contenitori Docker sono unità di software che includono il codice sorgente, le dipendenze e le
configurazioni necessarie per eseguire un'applicazione.
\begin{itemize}
    \item \textbf{Documentazione}: \url{https://docs.docker.com/} (consultato:
          26-03-2025).
\end{itemize}
Per lo sviluppo, il testing e il rilascio del prodotto sono stati utilizzati container Docker per garantire ambienti consistenti e facilmente riproducibili.

\subsection{Testing}
\subsubsection{Vitest}
Vitest è un framework di testing per Vite. Offre funzionalità avanzate come il
supporto per TypeScript, la configurazione tramite file di setup e la
possibilità di eseguire test in modalità watch. Questa modalità è particolarmente utile
durante lo sviluppo attivo e il debugging, poiché riduce i tempi di attesa per
l'esecuzione dei test e aiuta a mantenere il codice sempre funzionante.
\begin{itemize}
    \item \textbf{Versione}: 3.0.9;
    \item \textbf{Documentazione}: \url{https://vitest.dev/guide/} (consultato:
          26-03-2025).
\end{itemize}
Vitest è utilizzato per eseguire i test di unità e di integrazione del frontend,
verificando che i componenti React siano implementati correttamente.

\subsubsection{Cypress}
Cypress è un framework di testing end-to-end per applicazioni web. Consente di
scrivere test in JavaScript, eseguirli in un browser e visualizzare i
risultati in tempo reale.
\begin{itemize}
    \item \textbf{Versione}: 14.2.0;
    \item \textbf{Documentazione}: \url{https://docs.cypress.io/} (consultato:
          26-03-2025).
\end{itemize}
Cypress è utilizzato per testare il frontend dell'applicazione, verificando che
l'interfaccia utente sia correttamente visualizzata e che le principali funzionalità
siano implementate correttamente.

\subsubsection{Jest}
Jest è un framework di testing progettato per garantire la correttezza di
qualsiasi sorgente JavaScript. Consente di scrivere test con un'API
accessibile e ricca di funzionalità, che fornisce risultati rapidamente.
\begin{itemize}
    \item \textbf{Versione}: 29.7.0;
    \item \textbf{Documentazione}: \url{https://jestjs.io/docs/getting-started} (consultato:
          26-03-2025).
\end{itemize}
Jest è utilizzato per eseguire i test di unità e di integrazione del backend,
verificando che i vari componenti siano implementati correttamente.