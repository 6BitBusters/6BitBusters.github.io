\subsection{Architettura di Deployment}

Per facilitare la gestione e la configurazione dei servizi necessari all'applicazione, in particolare l'integrazione di \textbf{Memcached} come sistema di caching, è stato configurato un ambiente basato su Docker.\\
La containerizzazione dell'applicazione permette di avviare contemporaneamente tutti componenti richiesti, semplificando la gestione delle dipendenze e delle configurazioni.\\
Il file Docker Compose \texttt{dc-dep.yaml} definisce tutte le informazioni sulle immagini utilizzate e sulle configurazioni dell'ambiente.\\

Nel nostro ambiente di deploy sono presenti i seguenti container:

\begin{itemize}
    \item \textbf{Container Memcached}:
    \begin{itemize}
        \item Esegue il servizio di caching tramite il sistema distribuito \textbf{Memcached};
        \item Consente la memorizzazione temporanea dei dati per ridurre il carico sulle API esterne e migliorare le performance;
        \item È accessibile al container Server tramite la porta \texttt{11211}.
    \end{itemize}
    
    \item \textbf{Container Server}:
    \begin{itemize}
        \item Esegue il backend;
        \item Si occupa de recupero, dell'elaborazione, dell'invio dei dati e della gestione della cache;
        \item Espone la porta \texttt{3000} per permettere la comunicazione con il frontend.
    \end{itemize}
    
    \item \textbf{Container Client}:
    \begin{itemize}
        \item Esegue l'applicazione frontend;
        \item Comunica con il backend per ottenere i dati da visualizzare all'utente;
        \item Espone la porta \texttt{4173} per l'accesso al frontend tramite browser.
    \end{itemize}
\end{itemize}

Questa organizzazione garantisce una gestione efficace e scalabile dei componenti, semplificando la distribuzione e l'allocazione delle risorse per ciascun servizio. 
L'adozione di Docker consente inoltre di assicurare coerenza tra gli ambienti di sviluppo, testing e produzione, migliorando l'affidabilità e facilitando l'evoluzione futura del sistema.
