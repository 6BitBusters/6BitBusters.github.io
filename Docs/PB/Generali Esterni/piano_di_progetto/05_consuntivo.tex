\section{Consuntivo}
\subsection{Introduzione}
Questa sezione riporta i dati raccolti durante il progetto riguardo alla
ripartizione dei ruoli e alle ore impiegate da ogni componente del gruppo. Tali
dati sono comparati alle previsioni presenti nella sezione di preventivo.

\subsection{RTB}

\subsubsection{Sprint 1}
Di seguito la suddivisione dei ruoli e le ore di lavoro effettive impiegate in
questo sprint:


\newcommand{\memberReportTable}[1]{

	\renewcommand{\arraystretch}{1.5}
	\rowcolors{2}{pari}{dispari}
	\begin{longtable}{ %0.87
		>{\centering}M{0.22\textwidth}
		>{\centering}M{0.08\textwidth}
		>{\centering}M{0.08\textwidth}
		>{\centering}M{0.08\textwidth}
		>{\centering}M{0.08\textwidth}
		>{\centering}M{0.08\textwidth}
		>{\centering}M{0.08\textwidth}
		>{\centering\arraybackslash}M{0.08\textwidth}
		}
		\rowcolorhead
		\headertitle{Membro} &
		\headertitle{Resp.}  &
		\headertitle{Amm.}   &
		\headertitle{An.}    &
		\headertitle{Proge.} &
		\headertitle{Progr.} &
		\headertitle{Ver.}   &
		\headertitle{TOT.}
		\endfirsthead
		\endhead

		#1

	\end{longtable}
	\vspace{1em}

}

\memberReportTable{
    Bergamin Elia       & 5 & - & - & - & - & - & \textbf{5}\tabularnewline
    Diviesti Filippo    & - & - & - & - & - & 6 & \textbf{6}\tabularnewline
    Djossa Edgar        & - & 5 & - & - & - & - & \textbf{5}\tabularnewline
    Chilese Elena       & - & - & 3 (-4) & - & - & - & \textbf{3 (-4)}\tabularnewline
    Pincin Matteo       & - & - & - & - & - & 6 & \textbf{6}\tabularnewline
    Soranzo Andrea      & - & 5 & - & - & - & - & \textbf{5}\tabularnewline
    \midrule[\heavyrulewidth]
    \textbf{TOTALE}     & 5 & 10 & 3 (-4) & - & - & 12 & \textbf{30 (-4)}\tabularnewline

    \rowcolor{white}\caption{Rendiconto effettivo della distribuzione delle ore per lo sprint 1}

}


I costi effettivi del periodo sono i seguenti:


\newcommand{\costReportTable}[1]{

	\renewcommand{\arraystretch}{1.5}
	\rowcolors{2}{pari}{dispari}
	\begin{longtable}{ %0.87
		>{\centering}M{0.30\textwidth}
		>{\centering}M{0.10\textwidth}
		>{\centering}M{0.10\textwidth}
		>{\centering}M{0.10\textwidth}
		>{\centering\arraybackslash}M{0.15\textwidth}
		}
		\rowcolorhead
		\headertitle{Ruolo}            &
		\centering
		\headertitle{Ore}              &
		\headertitle{Costo  (\euro/h)} &
		\headertitle{Costo Totale (\euro)}
		\endfirsthead
		\endhead

		#1

	\end{longtable}
	\vspace{1em}

}

\costReportTable{
    Responsabile & 5 & 30 & 150 \tabularnewline
    Amministratore & 10 & 20 & 200 \tabularnewline
    Analista & 3 (-4) & 25 & 75 (-100) \tabularnewline
    Progettista & - & 25 & - \tabularnewline
    Programmatore & - & 15 & - \tabularnewline
    Verificatore & 12 & 15 & 180 \tabularnewline
    \midrule[\heavyrulewidth]
    \textbf{Totale Consuntivo} & 30 & - & 605 \tabularnewline
    \midrule[\heavyrulewidth]
    \textbf{Totale Preventivo} & 34 & - & 705 \tabularnewline
    \midrule[\heavyrulewidth]
    \textbf{Differenza} & -4 & - & -100 \tabularnewline

    \rowcolor{white}\caption{Consuntivo costi sprint 1}

}


\subsubsubsection{Resoconto}
Nel seguente resoconto vengono analizzate le principali differenze tra le stime iniziali e le ore effettive durante lo sprint:
\begin{itemize}
    \item \textbf{Analista (-4 ore):} Alla fine sono state necessarie meno ore di lavoro da parte
          dell'Analista rispetto al previsto. Questa variazione è legata, oltre che alla nostra inesperienza nel calcolo
          accurato dei tempi necessari, principalmente ad impegni di altre materie universitarie.

\end{itemize}
Nel complesso, il lavoro svolto durante lo sprint è stato in linea con gli obiettivi prefissati, sebbene siano emerse alcune differenze rispetto alle stime iniziali.\\
Per gli sprint successivi, sarà necessario affinare la fase di preventivazione, valutando con
maggiore precisione la complessità delle attività, così da ottenere stime ancora più accurate.\\

Di seguito le ore rimanenti ad ogni componente del gruppo relative ad ogni
ruolo. 
\newcommand{\remainingHoursTable}[1]{

	\renewcommand{\arraystretch}{1.5}
	\rowcolors{2}{pari}{dispari}
	\begin{longtable}{ %0.87
		>{\centering}M{0.22\textwidth}
		>{\centering}M{0.08\textwidth}
		>{\centering}M{0.08\textwidth}
		>{\centering}M{0.08\textwidth}
		>{\centering}M{0.08\textwidth}
		>{\centering}M{0.08\textwidth}
		>{\centering}M{0.08\textwidth}
		>{\centering\arraybackslash}M{0.08\textwidth}
		}
		\rowcolorhead
		\headertitle{Membro} &
		\headertitle{Resp.}  &
		\headertitle{Amm.}   &
		\headertitle{An.}    &
		\headertitle{Proge.} &
		\headertitle{Progr.} &
		\headertitle{Ver.}   
		\endfirsthead
		\endhead

		#1

	\end{longtable}
	\vspace{1em}

}
\remainingHoursTable{
    Bergamin Elia & 4 & 7 & 16 & 20 & 22 & 18 \tabularnewline
    Diviesti Filippo & 9 & 7 & 16 & 20 & 22 & 12 \tabularnewline
    Djossa Edgar & 9 & 2 & 16 & 20 & 22 & 18 \tabularnewline
    Chilese Elena & 9 & 7 & 13 & 20 & 22 & 18 \tabularnewline
    Pincin Matteo & 9 & 7 & 16 & 20 & 22 & 12 \tabularnewline
    Soranzo Andrea & 9 & 2 & 16 & 20 & 22 & 18\tabularnewline
    
    \rowcolor{white}\caption{Ore rimanenti per ogni ruolo dopo lo Sprint 1}
}

\pagebreak
\subsubsection{Sprint 2}
Di seguito la suddivisione dei ruoli e le ore di lavoro effettive impiegate in
questo sprint:


\memberReportTable{
    Bergamin Elia & - & - & - & - & - & 7 & \textbf{7}\tabularnewline
    Diviesti Filippo & - & - & 9 & - & - & - & \textbf{9}\tabularnewline
    Djossa Edgar & 5 & - & - & - & - & - & \textbf{5}\tabularnewline
    Chilese Elena & - & 7 & - & - & - & - & \textbf{7}\tabularnewline
    Pincin Matteo & - & - & 9 & - & - & - & \textbf{9}\tabularnewline
    Soranzo Andrea & - & - & - & - & - & 7 & \textbf{7}\tabularnewline
    \midrule[\heavyrulewidth]
    \textbf{TOTALE}& 5 & 7 & 18 & - & - & 14 &  44\tabularnewline

    \rowcolor{white}\caption{ Rendiconto effettivo della distribuzione delle ore per lo sprint 2}

}


I costi effettivi del periodo sono i seguenti:


\costReportTable{
    Responsabile & 7 & 30 & 210 \tabularnewline
    Amministratore & 14 & 20 & 280 \tabularnewline
    Analista & 18 & 25 & 450 \tabularnewline
    Progettista & - & 25 & - \tabularnewline
    Programmatore & - & 15 & - \tabularnewline
    Verificatore & 14 & 15 & 210 \tabularnewline
    \midrule[\heavyrulewidth]
    \textbf{Totale Consuntivo} & 53 & - & 1150 \tabularnewline
    \midrule[\heavyrulewidth]
    \textbf{Totale Preventivo} & 44 & - & 950 \tabularnewline
    \midrule[\heavyrulewidth]
    \textbf{Differenza} & 9 & - & 200 \tabularnewline

    \rowcolor{white}\caption{ Consuntivo costi sprint 2}

}


\subsubsubsection{Resoconto}
Nel seguente resoconto vengono analizzate le principali differenze tra le stime iniziali e le ore effettive durante lo sprint:
\begin{itemize}
    \item \textbf{Verificatore (-2 ore);}
    \item \textbf{Responsabile (+2 ore);}
    \item \textbf{Amministratore (+5 ore).}
\end{itemize}
Sono state eseguite meno ore da verificatore poichè l'ammontare è stato leggermente sovrastimato.
Inoltre le ore aggiuntive di responsabile e amministratore si evincono dal fatto che si ha avuto più tempo da dedicare al progetto rispetto
a quello preventivato.
\\
Nel complesso, anche grazie al lavoro aggiuntivo svolto, sono stati rispettati tutti gli obiettivi prefissati per questo sprint.\\

Di seguito le ore rimanenti ad ogni componente del gruppo relative ad ogni
ruolo. \remainingHoursTable{
    Bergamin Elia & 2 & 3 & 16 & 20 & 22 & 13 \tabularnewline
    Diviesti Filippo & 9 & 7 & 7 & 20 & 22 & 12 \tabularnewline
    Djossa Edgar & 4 & 2 & 16 & 20 & 22 & 18 \tabularnewline
    Chilese Elena & 9 & 0 & 13 & 20 & 22 & 18 \tabularnewline
    Pincin Matteo & 9 & 7 & 7 & 20 & 22 & 12 \tabularnewline
    Soranzo Andrea & 9 & 1 & 16 & 20 & 22 & 11\tabularnewline
    
    \rowcolor{white}\caption{Ore rimanenti per ogni ruolo dopo lo Sprint 2}
}

\pagebreak
\subsubsection{Sprint 3}
Di seguito la suddivisione dei ruoli e le ore di lavoro effettive impiegate in
questo sprint:


\memberReportTable{
    Bergamin Elia & - & 1 & 5 (-3) & - & - & - & \textbf{6}\tabularnewline
    Diviesti Filippo & 4 (+1) & - & - & - & - & - & \textbf{4}\tabularnewline
    Djossa Edgar & - & 0.5 (+0.5) & - & - & - & 5 (+2) & \textbf{5.5}\tabularnewline
    Chilese Elena & - & - & - & - & - & 4 (+1) & \textbf{4}\tabularnewline
    Pincin Matteo & - & 4 (+2) & - & - & - & - & \textbf{4}\tabularnewline
    Soranzo Andrea & - & - & 6 (-1) & - & - & - & \textbf{6}\tabularnewline
    \midrule[\heavyrulewidth]
    \textbf{TOTALE}& 4 & 5.5 & 11 & - & - & 9 &  29.5 \tabularnewline

    \rowcolor{white}\caption{Rendiconto effettivo della distribuzione delle ore per lo sprint 3}

}


I costi effettivi del periodo sono i seguenti:


\costReportTable{
    Responsabile & 4 (+1) & 30 & 120 \tabularnewline
    Amministratore & 5.5 (+1.5) & 20 & 110 \tabularnewline
    Analista & 11 (-4) & 25 & 275 \tabularnewline
    Progettista & - & 25 & - \tabularnewline
    Programmatore & - & 15 & - \tabularnewline
    Verificatore & 9 (+3) & 15 & 135 \tabularnewline
    \midrule[\heavyrulewidth]
    \textbf{Totale Consuntivo} & 29.5 & - & 640 \tabularnewline
    \midrule[\heavyrulewidth]
    \textbf{Totale Preventivo} & 27 & - & 615 \tabularnewline
    \midrule[\heavyrulewidth]
    \textbf{Differenza} & 2.5 & - & 25 \tabularnewline

    \rowcolor{white}\caption{ Consuntivo costi sprint 3}

}


\subsubsubsection{Resoconto}
Nel seguente resoconto vengono analizzate le principali differenze tra le stime iniziali e le ore effettive durante lo sprint:
\begin{itemize}
    \item \textbf{Responsabile (+1 ora);}
    \item \textbf{Amministratore (+2.5 ore);}
    \item \textbf{Analista (-4 ore);}
    \item \textbf{Verificatore (+3 ore).}
\end{itemize}

Sono state dedicate più ore al ruolo di responsabile e di amministratore
rispetto a quanto inizialmente previsto. Ciò è dovuto ad una maggiore
complessità nella gestione e organizzazione delle attività progettuali. Le ore
destinate al ruolo di analista sono state inferiori alle stime, in quanto
alcune attività di analisi sono risultate meno onerose del previsto e hanno
altresì aiutato il confronto con il proponente e l'incontro con il prof.
Cardin. Infine, il ruolo di verificatore ha richiesto un incremento di ore
rispetto al pianificato, in virtù di un aumento della complessità e
dell'estensione dei documenti da verificare. \\ Complessivamente, le variazioni
riscontrate non hanno compromesso il raggiungimento degli obiettivi prefissati,
permettendo il completamento di tutte le attività previste.\\ Di seguito le ore
rimanenti ad ogni componente del gruppo relative ad ogni ruolo.
\remainingHoursTable{
    Bergamin Elia & 2 & 2 & 11 & 20 & 22 & 13 \tabularnewline
    Diviesti Filippo & 5 & 7 & 7 & 20 & 22 & 12 \tabularnewline
    Djossa Edgar & 4 & 1.5 & 16 & 20 & 22 & 13 \tabularnewline
    Chilese Elena & 9 & 0 & 13 & 20 & 22 & 14 \tabularnewline
    Pincin Matteo & 9 & 3 & 7 & 20 & 22 & 12 \tabularnewline
    Soranzo Andrea & 9 & 1 & 10 & 20 & 22 & 11\tabularnewline
    
    \rowcolor{white}\caption{Ore rimanenti per ogni ruolo dopo lo Sprint 3}
}

\pagebreak
\subsubsection{Sprint 4}
Di seguito la suddivisione dei ruoli e le ore di lavoro effettive impiegate in
questo sprint:


\memberReportTable{
    Bergamin Elia & - & - & - & - & 5 (+2) & 0 (-2.5) & \textbf{5}\tabularnewline
    Diviesti Filippo & - & 4 (+2) & - & - & - & - & \textbf{4}\tabularnewline
    Djossa Edgar & - & - & 3 (+1) & - & - & - & \textbf{3}\tabularnewline
    Chilese Elena & 3 & - & - & - & - & - & \textbf{3}\tabularnewline
    Pincin Matteo & - & - & - & - & 3 & - & \textbf{3}\tabularnewline
    Soranzo Andrea & - & 1 & - & - & - & 4.5 (-0.5) & \textbf{6.5}\tabularnewline
    \midrule[\heavyrulewidth]
    \textbf{TOTALE}& 3 & 6 & 3 & 0 & 8 & 4.5 &  24.5 \tabularnewline

    \rowcolor{white}\caption{Rendiconto effettivo della distribuzione delle ore per lo sprint 4}

}


I costi effettivi del periodo sono i seguenti:


\costReportTable{
    Responsabile & 3 & 30 & 90 \tabularnewline
    Amministratore & 6 (+2) & 20 & 120 \tabularnewline
    Analista & 3 (+1) & 25 & 75 \tabularnewline
    Progettista & - & 25 & - \tabularnewline
    Programmatore & 8 (+2) & 15 & 120 \tabularnewline
    Verificatore & 4.5 (-3) & 15 & 67.5 \tabularnewline
    \midrule[\heavyrulewidth]
    \textbf{Totale Consuntivo} & 24.5 & - & 472.5 \tabularnewline
    \midrule[\heavyrulewidth]
    \textbf{Totale Preventivo} & 22.5 & - & 422.5 \tabularnewline
    \midrule[\heavyrulewidth]
    \textbf{Differenza} & 2 & - & 50 \tabularnewline

    \rowcolor{white}\caption{Consuntivo costi sprint 4}

}


\subsubsubsection{Resoconto}
Nel seguente resoconto vengono analizzate le principali differenze tra le stime iniziali e le ore effettive durante lo sprint:
\begin{itemize}
    \item \textbf{Amministratore (+2)}: la redazione dei verbali ha richiesto più tempo del solito, poiché sono stati trattati più temi nell'incontro interno e si è svolto un incontro con il proponente;
    \item \textbf{Analista (+1)}: richiesto più tempo rispetto al preventivo in quanto l'\textit{Analisi dei requisiti} conteneva più imprecisioni del previsto;
    \item \textbf{Programmatore (+2)}: lo sviluppo del PoC si è rivelato più ostico del previsto e sono state necessarie più ore per realizzare la corretta implementazione delle funzionalità richieste;
    \item \textbf{Verificatore (-3)}: lo sprint, avvenuto durante le festività natalizie, ha comportato una minore produzione di documenti rispetto al solito. Inoltre, la fase di verifica del codice del PoC è stata sovrastimata in fase di preventivo, poiché ci siamo limitati a individuare bug e imperfezioni utilizzando il PoC stesso, il che ha richiesto meno tempo.
\end{itemize}
Nonostante alcune discrepanze rispetto alle ore inizialmente stimate, tutte le attività pianificate sono state completate con successo.\\
Di seguito le ore rimanenti ad ogni componente del gruppo relative ad ogni ruolo.
\remainingHoursTable{
    Bergamin Elia & 2 & 2 & 11 & 20 & 17 & 13 \tabularnewline
    Diviesti Filippo & 5 & 3 & 7 & 20 & 22 & 12 \tabularnewline
    Djossa Edgar & 4 & 1.5 & 13 & 20 & 22 & 13 \tabularnewline
    Chilese Elena & 6 & 0 & 13 & 20 & 22 & 14 \tabularnewline
    Pincin Matteo & 9 & 3 & 7 & 20 & 19 & 12 \tabularnewline
    Soranzo Andrea & 9 & 0 & 10 & 20 & 22 & 6.5\tabularnewline
    
    \rowcolor{white}\caption{Ore rimanenti per ogni ruolo dopo lo Sprint 4}
}

\pagebreak
\subsubsection{Sprint 5}
Di seguito la suddivisione dei ruoli e le ore di lavoro effettive impiegate in
questo sprint:


\memberReportTable{
    Bergamin Elia & - & 2 & - & - & - & 3 & \textbf{5}\tabularnewline
    Diviesti Filippo & - & - & - & - & - & 4 & \textbf{4}\tabularnewline
    Djossa Edgar & - & - & 2 & - & 2 & - & \textbf{4}\tabularnewline
    Chilese Elena & - & - & 2.5(+0.5) & - & - & 3 & \textbf{5.5}\tabularnewline
    Pincin Matteo & 5(-1) & - & - & - & - & - & \textbf{5}\tabularnewline
    Soranzo Andrea & - & - & 7(+1) & - & - & - & \textbf{7}\tabularnewline
    \midrule[\heavyrulewidth]
    \textbf{TOTALE}& 5 & 2 & 11.5 & - & 2 & 10 & 30.5 \tabularnewline

    \rowcolor{white}\caption{Rendiconto effettivo della distribuzione delle ore per lo sprint 5}

}


I costi effettivi del periodo sono i seguenti:


\costReportTable{
    Responsabile & 5(-1) & 30 & 150 \tabularnewline
    Amministratore & 2 & 20 & 40 \tabularnewline
    Analista & 11.5 (+1.5) & 25 & 287.5 \tabularnewline
    Progettista & - & 25 & - \tabularnewline
    Programmatore & 2 & 15 & 30 \tabularnewline
    Verificatore & 10 & 15 & 150 \tabularnewline
    \midrule[\heavyrulewidth]
    \textbf{Totale Consuntivo} & 30.5 & - & 657.5 \tabularnewline
    \midrule[\heavyrulewidth]
    \textbf{Totale Preventivo} & 30 & - & 650 \tabularnewline
    \midrule[\heavyrulewidth]
    \textbf{Differenza} & 0.5 & - & 7.5 \tabularnewline

    \rowcolor{white}\caption{Consuntivo costi sprint 5}

}


\subsubsubsection{Resoconto}
Nel seguente resoconto vengono analizzate le principali differenze tra le stime iniziali e le ore effettive durante lo sprint:
\begin{itemize}
    \item \textbf{Responsabile (-1)};
    \item \textbf{Analista (+1.5)}: La redazione e la creazione dei grafici per il cruscotto di qualità ha richiesto più tempo del previsto perché richiede azioni manuali.
\end{itemize}
In breve, oltre ad alcune difficoltà incontrate per la creazione di grafici utili a rappresentare l'andamento della qualità, tutte le attività pianificate sono state completate con successo.\\
Di seguito le ore rimanenti ad ogni componente del gruppo relative ad ogni ruolo.
\remainingHoursTable{
    Bergamin Elia & 2 & 0 & 11 & 20 & 17 & 10 \tabularnewline
    Diviesti Filippo & 5 & 3 & 7 & 20 & 22 & 8 \tabularnewline
    Djossa Edgar & 4 & 1.5 & 11 & 20 & 20 & 13 \tabularnewline
    Chilese Elena & 6 & 0 & 10.5 & 20 & 22 & 11 \tabularnewline
    Pincin Matteo & 4 & 3 & 7 & 20 & 19 & 12 \tabularnewline
    Soranzo Andrea & 9 & 0 & 3 & 20 & 22 & 6.5\tabularnewline
    
    \rowcolor{white}\caption{Ore rimanenti per ogni ruolo dopo lo Sprint 5}
}

\pagebreak
\subsubsection{Sprint 6}
Di seguito la suddivisione dei ruoli e le ore di lavoro effettive impiegate in
questo sprint:

\memberReportTable{
    Bergamin Elia       & - & - & - & - & - & 1 (-1) & \textbf{1 (-1)}\tabularnewline
    Diviesti Filippo    & - & 1 & - & - & - & - & \textbf{1}\tabularnewline
    Djossa Edgar        & - & - & 0.5 (-0.5) & - & - & - & \textbf{0.5 (-0.5)}\tabularnewline
    Chilese Elena       & - & - & 0 (-2) & - & - & - & \textbf{0 (-2)}\tabularnewline
    Pincin Matteo       & - & - & - & - & - & 0.5 (-1.5) & \textbf{0.5 (-1.5)}\tabularnewline
    Soranzo Andrea      & 2 & - & - & - & - & - & \textbf{2}\tabularnewline
    \midrule[\heavyrulewidth]
    \textbf{TOTALE}     & 2 & 1 & 0.5 (-2.5) & - & - & 1.5 (-2.5) & \textbf{5 (-5)} \tabularnewline

    \rowcolor{white}\caption{Rendiconto effettivo della distribuzione delle ore per lo sprint 6}

}


I costi effettivi del periodo sono i seguenti:

\costReportTable{
    Responsabile & 2 & 30 & 60 \tabularnewline
    Amministratore & 1 & 20 & 20 \tabularnewline
    Analista & 0.5 (-2.5) & 25 & 12.5 (-62.5) \tabularnewline
    Progettista & - & 25 & - \tabularnewline
    Programmatore & - & 15 & - \tabularnewline
    Verificatore & 1.5 (-2.5) & 15 & 22.5 (-37.5) \tabularnewline
    \midrule[\heavyrulewidth]
    \textbf{Totale Consuntivo} & 5 & - & 115 \tabularnewline
    \midrule[\heavyrulewidth]
    \textbf{Totale Preventivo} & 10 & - & 215 \tabularnewline
    \midrule[\heavyrulewidth]
    \textbf{Differenza} & -5 & - & -100 \tabularnewline

    \rowcolor{white}\caption{Consuntivo costi sprint 6}

}


\subsubsubsection{Resoconto}
Nel seguente resoconto vengono analizzate le principali differenze tra le stime iniziali e le ore effettive durante lo sprint:
\begin{itemize}
    \item \textbf{Analista (-2.5)}; 
    \item \textbf{Verificatore (-2.5)}.
\end{itemize}
Lo studio per gli esami universitari ha richiesto più tempo del previsto, riducendo le ore disponibili per il progetto.
Per questo motivo, l'analisi delle metriche di qualità è stata rinviata allo sprint successivo. 
Conseguentemente, anche le ore di verifica sono state inferiori rispetto a quanto preventivato.\\
Di seguito le ore rimanenti ad ogni componente del gruppo relative ad ogni ruolo.
\remainingHoursTable{
    Bergamin Elia & 2 & 0 & 11 & 20 & 17 & 9 \tabularnewline
    Diviesti Filippo & 5 & 2 & 7 & 20 & 22 & 8 \tabularnewline
    Djossa Edgar & 4 & 1.5 & 10.5 & 20 & 20 & 13 \tabularnewline
    Chilese Elena & 6 & 0 & 10.5 & 20 & 22 & 11 \tabularnewline
    Pincin Matteo & 4 & 3 & 7 & 20 & 19 & 11.5 \tabularnewline
    Soranzo Andrea & 7 & 0 & 3 & 20 & 22 & 6.5\tabularnewline
    
    \rowcolor{white}\caption{Ore rimanenti per ogni ruolo dopo lo Sprint 6}
}

\pagebreak
\subsubsection{Sprint 7}
Di seguito la suddivisione dei ruoli e le ore di lavoro effettive impiegate in
questo sprint:

\memberReportTable{
    Bergamin Elia       & 2 & - & - & - & - & - & \textbf{2}\tabularnewline
    Diviesti Filippo    & - & - & - & - & - & 1 & \textbf{1}\tabularnewline
    Djossa Edgar        & - & 1(-1) & - & - & - & - & \textbf{1 (-1)}\tabularnewline
    Chilese Elena       & - & - & - & - & - & 1 & \textbf{1}\tabularnewline
    Pincin Matteo       & - & - & 2.5 (-0.5) & - & - & - & \textbf{2.5 (-0.5)}\tabularnewline
    Soranzo Andrea      & - & - & - & - & 2 & - & \textbf{2}\tabularnewline
    \midrule[\heavyrulewidth]
    \textbf{TOTALE}     & 2 & 1(-1) & 2.5 (-0.5) & - & 2 & 2 & \textbf{9.5 (-1.5)} \tabularnewline

    \rowcolor{white}\caption{Rendiconto effettivo della distribuzione delle ore per lo sprint 7}

}


I costi effettivi del periodo sono i seguenti:

\costReportTable{
    Responsabile & 2 & 30 & 60 \tabularnewline
    Amministratore & 1(-1) & 20 & 20 (-20) \tabularnewline
    Analista & 2.5(-0.5) & 25 & 62.5 (-12.5) \tabularnewline
    Progettista & - & 25 & - \tabularnewline
    Programmatore & 2 & 15 & 30 \tabularnewline
    Verificatore & 2 & 15 & 30 \tabularnewline
    \midrule[\heavyrulewidth]
    \textbf{Totale Consuntivo} & 10.5 & - & 202.5 \tabularnewline
    \midrule[\heavyrulewidth]
    \textbf{Totale Preventivo} & 12 & - & 235 \tabularnewline
    \midrule[\heavyrulewidth]
    \textbf{Differenza} & -1.5 & - & -32.5 \tabularnewline

    \rowcolor{white}\caption{Consuntivo costi sprint 7}

}


\subsubsubsection{Resoconto}
Nel seguente resoconto vengono analizzate le principali differenze tra le stime iniziali e le ore effettive durante lo sprint:
\begin{itemize}
    \item \textbf{Amministratore (-1)}; 
    \item \textbf{Analista (-0.5)}.
\end{itemize}

\remainingHoursTable{
    Bergamin Elia & 0 & 0 & 11 & 20 & 17 & 9 \tabularnewline
    Diviesti Filippo & 5 & 2 & 7 & 20 & 22 & 7 \tabularnewline
    Djossa Edgar & 4 & 0.5 & 10.5 & 20 & 20 & 13 \tabularnewline
    Chilese Elena & 6 & 0 & 10.5 & 20 & 22 & 10 \tabularnewline
    Pincin Matteo & 4 & 3 & 4.5 & 20 & 19 & 11.5 \tabularnewline
    Soranzo Andrea & 7 & 0 & 3 & 20 & 20 & 6.5\tabularnewline
    
    \rowcolor{white}\caption{Ore rimanenti per ogni ruolo dopo lo Sprint 7}
}

\subsubsection{Sprint 8}
Di seguito la suddivisione dei ruoli e le ore di lavoro effettive impiegate in questo sprint:

\memberReportTable{
    Bergamin Elia       & - & - & - & - & - & 2 (+1) & \textbf{2 (+1)}\tabularnewline
    Diviesti Filippo    & - & 1 & - & - & - & - & \textbf{1}\tabularnewline
    Djossa Edgar        & - & - & - & - & - & 1 & \textbf{1}\tabularnewline
    Chilese Elena       & - & - & 3 & - & - & - & \textbf{3}\tabularnewline
    Pincin Matteo       & - & 2 (+1) & - & - & - & - & \textbf{2 (+1)}\tabularnewline
    Soranzo Andrea      & 3.5 & - & - & - & - & - & \textbf{3.5}\tabularnewline
    \midrule[\heavyrulewidth]
    \textbf{TOTALE}     & 3.5 & 3 (+1) & 3 & - & - & 3 (+1) & \textbf{12.5 (+2)} \tabularnewline

    \rowcolor{white}\caption{Rendiconto effettivo della distribuzione delle ore per lo sprint 8}

}


I costi effettivi del periodo sono i seguenti:

\costReportTable{
    Responsabile & 3.5 & 30 & 105 \tabularnewline
    Amministratore & 3 (+1) & 20 & 60 (+20) \tabularnewline
    Analista & 3 & 25 & 75 \tabularnewline
    Progettista & - & 25 & - \tabularnewline
    Programmatore & - & 15 & - \tabularnewline
    Verificatore & 3 (+1) & 15 & 45 (+15) \tabularnewline
    \midrule[\heavyrulewidth]
    \textbf{Totale Consuntivo} & 12.5 & - & 285 \tabularnewline
    \midrule[\heavyrulewidth]
    \textbf{Totale Preventivo} & 10.5 & - & 250 \tabularnewline
    \midrule[\heavyrulewidth]
    \textbf{Differenza} & +2 & - & +35 \tabularnewline

    \rowcolor{white}\caption{Consuntivo costi sprint 8}

}


\subsubsubsection{Resoconto}
Nel seguente resoconto vengono analizzate le principali differenze tra le stime iniziali e le ore effettive durante lo sprint:
\begin{itemize}
    \item \textbf{Amministratore (+1)}; 
    \item \textbf{Verificatore (+1)}.
\end{itemize}
Tali piccole discrepanze sono dovute al perfezionamento di alcuni documenti per il colloquio RTB.

\remainingHoursTable{
    Bergamin Elia & 0 & 0 & 11 & 20 & 17 & 7 \tabularnewline
    Diviesti Filippo & 5 & 1 & 7 & 20 & 22 & 7 \tabularnewline
    Djossa Edgar & 4 & 0.5 & 10.5 & 20 & 20 & 12 \tabularnewline
    Chilese Elena & 6 & 0 & 7.5 & 20 & 22 & 10 \tabularnewline
    Pincin Matteo & 4 & 1 & 4.5 & 20 & 19 & 11.5 \tabularnewline
    Soranzo Andrea & 3.5 & 0 & 3 & 20 & 20 & 6.5\tabularnewline
    
    \rowcolor{white}\caption{Ore rimanenti per ogni ruolo dopo lo sprint 8}
}
Analizzando gli obiettivi iniziali rispetto a quelli effettivamente raggiunti:
\begin{itemize}
    \item Non vi sono stati obiettivi non raggiunti; 
    \item L'unica problematica riscontrata è stato il tempo di attesa prima dei colloqui e per il successivo esito;
    \item Possibili cambiamenti: durante periodi meno intensi il gruppo pianificherà in maniera più accurata il tempo a disposizione per evitare sprechi eccessivi.
\end{itemize}
\pagebreak

\subsubsection{Resoconto finale RTB}
Qui di seguito il resoconto complessivo al termine dell'RTB:
\memberTable{
    Bergamin Elia      & 9 & 7    & 5     & - & 5 & 11 & \textbf{37}\tabularnewline
    Diviesti Filippo   & 4 & 6    & 9     & - & - & 11 & \textbf{30}\tabularnewline
    Djossa Edgar       & 5 & 6.5  & 5.5   & - & 2 & 6 & \textbf{25}\tabularnewline
    Chilese Elena      & 3 & 7    & 8.5   & - & - & 8 & \textbf{26.5}\tabularnewline
    Pincin Matteo      & 5 & 6    & 11.5  & - & 3 & 6.5 & \textbf{32}\tabularnewline
    Soranzo Andrea     & 5.5 & 7  & 13    & - & 2 & 11.5 & \textbf{39}\tabularnewline
    \midrule[\heavyrulewidth]
    \textbf{TOTALE}& 31.5 & 39.5 & 52.5 & - & 12 & 54 & \textbf{189.5}\tabularnewline

    \rowcolor{white}\caption{Ore totali RTB}

}

\costTable{
    Responsabile & 28 & 30 & 840 \tabularnewline
    Amministratore & 36.5 & 20 & 730 \tabularnewline
    Analista & 49.5 & 25 & 1237.50 \tabularnewline
    Progettista & - & 25 & - \tabularnewline
    Programmatore & 12 & 15 & 180 \tabularnewline
    Verificatore & 51 & 15 & 765 \tabularnewline
    \midrule[\heavyrulewidth]
    \textbf{TOTALE} & 177 & - & 3752.50 \tabularnewline

    \rowcolor{white}\caption{Costi effettivi RTB}

}


Considerando il costo finale preventivato del progetto, che ammonta a \euro{11460}, si prevede un notevole aumento delle ore produttive per la realizzazione dell'MVP.\\
Finora, il ruolo del programmatore ha richiesto meno ore del previsto, poiché il PoC prodotto alla fine della prima fase non necessitava di un'architettura complessa, riducendo così il tempo di sviluppo necessario. 
Durante la prima fase del progetto, inoltre, sono state riassegnate le ore rimanenti in quanto il ruolo di amministratore, a causa della produzione di numerosi documenti, ha richiesto più tempo rispetto alle previsioni iniziali.\\\\
In generale il gruppo ha mantenuto un alto livello di intensità, completando tutte le issue pianificate in ogni sprint. 
Il ritardo rispetto al calendario inizialmente previsto è stato principalmente dovuto allo studio e alla selezione delle tecnologie da utilizzare per l'MVP.

\pagebreak

\subsection{PB}

\subsubsection{Sprint 9}
Di seguito la suddivisione dei ruoli e le ore di lavoro effettive impiegate in questo sprint:


\memberReportTable{
    Bergamin Elia       & - & - & 3 & 7 (+1) & - & - & \textbf{10 (+1)}\tabularnewline
    Diviesti Filippo    & - & - & - & - & - & 4 & \textbf{4}\tabularnewline
    Djossa Edgar        & - & 2 (+2) & - & 7 & - & - & \textbf{9 (+2)}\tabularnewline
    Chilese Elena       & 4 & - & - & - & - & - & \textbf{4}\tabularnewline
    Pincin Matteo       & - & - & - & - & - & 4 & \textbf{4}\tabularnewline
    Soranzo Andrea      & - & 6 & - & 8 & - & - & \textbf{14}\tabularnewline
    \midrule[\heavyrulewidth]
    \textbf{TOTALE}     & 4 & 8 (+2) & 3 & 22 (+1) & - & 8 & \textbf{45 (+3)}\tabularnewline

    \rowcolor{white}\caption{Rendiconto effettivo della distribuzione delle ore per lo sprint 9}

}


I costi effettivi del periodo sono i seguenti:

\costReportTable{
    Responsabile & 4 & 30 & 120 \tabularnewline
    Amministratore & 8 (+2) & 20 & 160 (+40) \tabularnewline
    Analista & 3 & 25 & 75 \tabularnewline
    Progettista & 22 (+1) & 25 & 550 (+25) \tabularnewline
    Programmatore & - & 15 & - \tabularnewline
    Verificatore & 8 & 15 & 120 \tabularnewline
    \midrule[\heavyrulewidth]
    \textbf{Totale Consuntivo} & 45 & - & 1025 \tabularnewline
    \midrule[\heavyrulewidth]
    \textbf{Totale Preventivo} & 42 & - & 960 \tabularnewline
    \midrule[\heavyrulewidth]
    \textbf{Differenza} & +3 & - & +65 \tabularnewline

    \rowcolor{white}\caption{Consuntivo costi sprint 9}

}


\subsubsubsection{Resoconto}
Nel seguente resoconto vengono analizzate le principali differenze tra le stime iniziali e le ore effettive durante lo sprint:
\begin{itemize}
    \item \textbf{Progettista (+1)}; 
    \item \textbf{Amministratore (+2)}: È stato necessario aggiornare il glossario con nuovi termini emersi durante la stesura della \textit{Specifica tecnica}.
\end{itemize}

\remainingHoursTable{
    Bergamin Elia       & 0 & 6 & 0 & 11 & 22 & 8 \tabularnewline
    Chilese Elena       & 0 & 6 & 2.5 & 18 & 22 & 12 \tabularnewline
    Diviesti Filippo    & 3 & 7 & 0.5 & 18 & 22 & 4 \tabularnewline
    Djossa Edgar        & 4 & 4 & 3 & 11 & 22 & 12 \tabularnewline
    Pincin Matteo       & 4 & 6 & 2.5 & 18 & 22 & 2 \tabularnewline
    Soranzo Andrea      & 0 & 0 & 2.5 & 10 & 22 & 11 \tabularnewline
    
    \rowcolor{white}\caption{Ore rimanenti per ogni ruolo dopo lo sprint 9}
}



Confrontando gli obiettivi iniziali con i risultati effettivamente ottenuti:
\begin{itemize}
\item Tutti gli obiettivi previsti sono stati raggiunti. In particolare, è stato raggiunto un buon livello di dettaglio nella progettazione, soprattutto per quanto riguarda i diagrammi delle classi, il che ha permesso di pianificare già dal prossimo sprint l’inizio della fase di codifica (pur rimanendo come attività secondaria rispetto al completamento della progettazione), così da ottimizzare i tempi e rimanere in linea con la data di consegna;
\item La principale difficoltà riscontrata è stata la non sempre piena disponibilità di tutti i componenti del team, causando un lieve ritardo nella stesura di alcuni documenti. Tuttavia, grazie alla collaborazione e all’impegno del gruppo, la problematica è stata gestita con successo;
\item Possibili miglioramenti: per evitare ritardi futuri, sarà utile concordare anticipatamente le disponibilità dei membri prima dell'inizio di ogni sprint, così da pianificare meglio le attività e distribuire il carico di lavoro in modo più efficace.
\end{itemize}
