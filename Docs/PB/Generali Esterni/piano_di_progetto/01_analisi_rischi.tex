\section{Analisi dei rischi}
In questa sezione sono riassunti i rischi che il gruppo potrebbe dover affrontare a seguito dell'aggiudicazione del capitolato \textit{3Dataviz}.
Il nostro obiettivo è adottare decisioni e scelte che permettano di gestire il progetto riducendo al minimo i problemi.
Tuttavia, nel caso in cui questi si verifichino, vogliamo essere preparati con soluzioni concrete e immediatamente adottabili.
\\I rischi sono stati suddivisi in quattro differenti categorie e, per ciascun rischio, dopo una breve descrizione, viene presentata l'analisi dei seguenti punti:
\begin{itemize}
    \item \textbf{Nome}: identifica il rischio;
    \item \textbf{Descrizione}: descrizione del rischio atteso;
    \item \textbf{Prob.} (Probabilità di occorrenza): indica la probabilità con cui è possibile si verifichi il rischio (Bassa=B, Media=M, Alta=A);
    \item \textbf{Per.} (Grado di pericolosità): indica il grado di pericolosità del rischio (Basso=B ,Medio=M, Alto=A);
    \item \textbf{Misure preventive}: indica le misure adottate per cercare di evitare il verificarsi del rischio;
    \item \textbf{P.Cont.} (Piano di Contingenza): indica come far fronte al rischio nel caso in cui esso si verifichi. 
\end{itemize}
\newcommand{\riskTable}[1]{
	 

\renewcommand{\arraystretch}{1.5}
\rowcolors{2}{pari}{dispari}
\begin{longtable}{ 
		>{\centering}M{0.2\textwidth} 
		>{\raggedright}M{0.35\textwidth} 
		>{\centering}M{0.05\textwidth}
		>{\centering}M{0.05\textwidth}
		>{\raggedright}M{0.22\textwidth} 
		 }
	\rowcolorhead
	\headertitle{Nome} &
	\headertitle{Descrizione} &
	\centering \headertitle{Prob.} &	
	\headertitle{Per.} &
	\headertitle{Misure preventive}
	\endfirsthead	
	\endhead
	
	#1

\end{longtable}
\vspace{0.2em}

}

\subsection{Rischi riguardanti i requisiti}
\riskTable{
    \textbf{Analisi dei requisiti} & Richiede un impegno significativo da parte del gruppo ed è fondamentale per capire cosa dovrà fare il prodotto. Il rischio principale è tralasciare aspetti importanti, causando discrepanze tra gli impegni preventivati e quelli effettivi.
     & M & A & Analizzare in dettaglio tutti i casi d’uso e i relativi requisiti, mantenendo un dialogo aperto e costante con il proponente.\tabularnewline
     \multicolumn{5}{l}{\textbf{P.Cont.:} Organizzare colloqui con il proponente per discutere sul da farsi.}\tabularnewline
     \textbf{Cambiamento dei requisiti iniziali} & I requisiti iniziali potrebbero subire modifiche e/o aggiunte da parte del proponente.
     & M & M & Realizzare un'analisi dei requisiti il più possibile chiara e precisa.\tabularnewline
     \multicolumn{5}{l}{\textbf{P.Cont.:} Discutere e comprendere appieno ogni cambiamento insieme al proponente.}\tabularnewline
     \captionline\caption{Rischi riguardanti i requisiti}
}
\pagebreak
\subsection{Rischi tecnologici}
\riskTable{
    \textbf{Strumenti e tecnologie sconosciute} & Il progetto richiede l'utilizzo di software per la creazione di grafici 3D, oltre a librerie JavaScript e Framework a noi attualmente poco conosciuti.
     & A & M & Valutare ogni strumento o tecnologia ed esercitarsi per scegliere quello più adatto. Sfruttare le conoscenze pregresse per facilitare la divisione del lavoro e per non concentrarsi troppo su funzionalità meno rilevanti.\tabularnewline
     \multicolumn{5}{p{\dimexpr\textwidth-2\tabcolsep\relax}}{\textbf{P.Cont.:} Se l’utilizzo di uno strumento o di una tecnologia causa ritardi, si procederà con una maggiore suddivisione più efficiente del lavoro e, in casi estremi, si valuterà un’alternativa, sempre confrontando il proponente.}\tabularnewline
     \textbf{Problemi hardware o software} & Problemi causati da PC malfunzionanti e problemi software derivati da guasti.
     & M & B & Condivisione dei file di progetto in un repository comune utilizzando un sistema di versionamento distribuito atto a garantire una gestione collaborativa delle modifiche.\tabularnewline
     \multicolumn{5}{p{\dimexpr\textwidth-2\tabcolsep\relax}}{\textbf{P.Cont.:} Utilizzare dispositivi secondari per continuare il lavoro.}\tabularnewline
     \captionline\caption{Rischi tecnologici}
}

\subsection{Rischi organizzativi}
\riskTable{
    \textbf{Pianificazione inadeguata} & La pianificazione di un intero progetto di questa portata include una corretta individuazione delle attività da svolgere e un'adeguata suddivisione dei compiti. Tuttavia, l'inesperienza dei membri del gruppo può provocare ritardi e spreco di risorse.
     & A & A & Pianificazione su brevi periodi di tempo e con una visione pessimistica.\tabularnewline
     \multicolumn{5}{l}{\textbf{P.Cont.:} Stilare un nuovo piano di lavoro che ottimizzi le risorse e la suddivisione del lavoro.}\tabularnewline
     \textbf{Distribuzione del lavoro} & La cattiva distribuzione del lavoro può causare un sovraccarico per un membro del gruppo, come altresì l'inattività di un membro.
     & M & M & Organizzare il lavoro tenendo conto delle disponibilità e delle capacità di ogni membro.\tabularnewline
     \multicolumn{5}{l}{\textbf{P.Cont.:} Suddivisione del lavoro in maniera più distribuita e uniforme.}\tabularnewline
     \captionline\caption{Rischi organizzativi}
}

\subsection{Rischi personali}
\riskTable{
    \textbf{Disponibilità dei componenti} & Il gruppo è composto da alcuni studenti-lavoratori i cui impegni non sono sempre conciliabili. Inoltre, ogni membro del gruppo è impegnato in altri esami universitari che possono richiedere una quantità significativa di tempo, soprattutto in prossimità della sessione invernale.
     & A & B & Organizzare con anticipo ogni incontro. Prevedere delle settimane di lavoro più intense ed altre meno, soprattutto in concomitanza con gli impegni sopra descritti.\tabularnewline
     \multicolumn{5}{p{\dimexpr\textwidth-2\tabcolsep\relax}}{\textbf{P.Cont.:} Il membro assente deve aggiornarsi sulle decisioni prese e acquisire le informazioni sui progressi in tempi brevi, così da riallinearsi rapidamente con il gruppo. Se l'assenza di uno o più membri causa ritardi significativi, il responsabile dovrà rivedere e ri-pianificare la suddivisione del lavoro.}\tabularnewline
     \textbf{Difficoltà comunicative} & La maggior parte degli incontri si svolgerà in modalità telematica.
     & B & M & Utilizzo di diversi strumenti di comunicazione.\tabularnewline
     \multicolumn{5}{p{\dimexpr\textwidth-2\tabcolsep\relax}}{\textbf{P.Cont.:} Ritrovo in presenza alla fine dell'ora di lezione o in un giorno preventivamente accordato.}\tabularnewline
     \textbf{Mancanza di esperienza personale} & Il progetto richiede l'acquisizione di competenze nuove per ogni membro del gruppo.
     & A & M & Iniziare a capire il funzionamento e sperimentare i nuovi strumenti e tecnologie da utilizzare.\tabularnewline
     \multicolumn{5}{p{\dimexpr\textwidth-2\tabcolsep\relax}}{\textbf{P.Cont.:} Se un membro riscontra parecchie difficoltà, può chiedere supporto agli altri membri e soprattutto al responsabile, il quale avrà il compito di trovare un compromesso atto a risolvere la situazione.}\tabularnewline
     \textbf{Conflitti interni} & I gruppi sono stati formati casualmente e i membri, non conoscendosi tra loro, potrebbero avere idee contrastanti.
     & B & M & Ciascun membro espone educatamente la propria idea, il responsabile sorveglierà ed agirà se necessario.\tabularnewline
     \multicolumn{5}{p{\dimexpr\textwidth-2\tabcolsep\relax}}{\textbf{P.Cont.:} Ogni decisione viene discussa e successivamente approvata a maggioranza.}\tabularnewline
     \captionline\caption{Rischi personali}
}
