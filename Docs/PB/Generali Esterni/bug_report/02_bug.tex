\section{Lista dei bug}
\subsection{Reset dei filtri}
\textbf{Stato:} risolto\\
Nella prima implementazione, il reset del filtro per valore del dataset,
attivato selezionando una barra del grafico o una cella della tabella,
non riportava tutte le barre al loro stato originale: la barra che era
stata selezionata manteneva il colore di selezione.

\subsection{Trasparenza delle barre}
\textbf{Stato:} non risolto\\
Come noto, in caso di filtraggio del dataset, le barre del grafico
che non soddisfano i criteri di selezione vengono rese parzialmente trasparenti.
È lecito aspettarsi che, se due barre sono sovrapposte e la barra più vicina all'osservatore
è parzialmente trasparente, la barra più lontana sia visibile. Tuttavia, in alcuni casi,
una barra parzialmente trasparente può nascondere completamente una barra più lontana.
\begin{figure}[h!]
    \centering
    \includegraphics[width=0.5\textwidth]{template/images/logo-circle.png}
    \caption{Esempio di barre parzialmente trasparenti}
    \label{fig:trasparenza}
\end{figure}

\subsection{Rotazione del grafico}
\textbf{Stato:} non risolto\\
La rotazione del grafico avviene in modo fluido e senza evidenti problemi. Tuttavia,
se nella posizione iniziale il cursore del mouse si trova sopra una barra, e
nella nuova posizione il cursore si trova sopra un'altra (o la stessa) barra,
quest'ultima barra viene selezionata al momento del rilascio del mouse.
Questo comportamento, purtroppo, è intrinsecamente legato alla
gestione dell'evento "onClick" del mouse, che viene attivato al momento del rilascio.