\section{Informazioni generali}
\subsection{Dettagli sull'incontro}
\begin{itemize}
    \item \textbf{Luogo}: Google Meet
    \item \textbf{Data}: 16-04-2025
    \item \textbf{Ora di inizio}: 15:00
    \item \textbf{Ora di fine}: 15:30
    \item \textbf{Partecipante dell'azienda}: Beggiato Alex
    \item \textbf{Partecipanti}:
    \begin{itemize}
        \item Bergamin Elia
        \item Djossa Edgar
        \item Pincin Matteo 
        \item Soranzo Andrea  
        \item Diviesti Filippo
    \end{itemize}
\end{itemize}

\section{Ordine del giorno}
\begin{itemize}
\item Discussione sull'andamento del progetto;
\item Chiarimento di dubbi relativi alla progettazione, in particolare riguardo ai diagrammi UML.
\end{itemize}

\section{Resoconto della riunione}

\subsection{Andamento del progetto}

Durante la riunione, il team ha parlato dei progressi fatti e delle attività ancora da svolgere. L’obiettivo condiviso è completare il progetto entro il prossimo incontro.

\subsection{Progettazione architetturale}

Sono stati chiesti chiarimenti sui diagrammi UML riguardanti il backend. Le indicazioni ricevute sono state utili per migliorare alcune scelte progettuali:

\begin{itemize} 
    \item \textbf{BaseFetcher}: la classe astratta \textit{BaseFetcher} è stata al centro della discussione. Il team ha deciso di trasformarla in un’interfaccia, in modo da definire solo la firma dei metodi. L’implementazione concreta sarà lasciata alle classi specifiche per ogni API. 
    \item \textbf{Dipendenze tra servizi}: è emersa una criticità legata all’eccessiva dipendenza tra i servizi, che compromette la struttura a strati dell’architettura. Il proponente ha suggerito un articolo che spiega come stabilità e astrazione influenzino la qualità del software. Il team approfondirà l’argomento per trovare la soluzione più adatta. 
\end{itemize}

\subsection{Prossimo incontro}

Il prossimo incontro è previsto per lunedì 28 maggio 2025 alle ore 17:00. In quell’occasione, il team e il proponente valuteranno se il progetto può considerarsi concluso o se serviranno ulteriori interventi.