\section{Introduzione}
    \subsection{Scopo del documento}
        Questo documento ha lo scopo di definire le norme e le linee guida operative
        per il team \textit{Six Bit Busters} nello sviluppo del progetto
        \textit{3Dataviz}.\\ In particolare, esso descrive i processi di lavoro,
        le modalità di collaborazione, gli standard di codifica e le pratiche
        di gestione della qualità che il team seguirà per garantire coerenza, 
        efficienza e qualità durante il ciclo di vita del prodotto. 
        Lo scopo è quello di fornire una struttura comune e procedure chiare, per 
        facilitare il lavoro in team e assicurare che tutti i membri operino in 
        linea con gli obiettivi e le specifiche concordate.

    \subsection{Scopo del prodotto}

   3Dataviz è un prodotto ideato dall'azienda \textit{Sanmarco Informatica S.p.A.} per rendere più efficace e accessibile la visualizzazione dei dati in forma tridimensionale. Il progetto mira allo sviluppo di un’interfaccia web avanzata, progettata per trasformare i dati in grafici 3D interattivi e facilmente navigabili.
        
    \subsection{Glossario}
      In tutti i documenti, le parole che richiedono una definizione più precisa saranno identificate con il pedice "g" e riportate nel Glossario v1.0.0 per una spiegazione dettagliata.
        
    \subsection{Riferimenti}
        \subsubsection{Riferimenti normativi}
        \begin{itemize}
            \item Capitolato d'appalto C5 \textit{Sanmarco Informatica SPA} - 3Dataviz : \\ \url{https://www.math.unipd.it/~tullio/IS-1/2024/Progetto/C5.pdf}
            \item Materiale didattico - Corso Ingegneria del software 2024/2025 - Regolamento del Progetto Didattico: \\ \url{https://www.math.unipd.it/~tullio/IS-1/2024/Dispense/PD1.pdf}
        \end{itemize}
        \subsubsection{Riferimenti informativi}
        \begin{itemize}
            \item Materiale didattico - Corso Ingegneria del software 2024/2025 - Processi di ciclo di vita: \\ \url{https://www.math.unipd.it/~tullio/IS-1/2024/Dispense/T02.pdf}
            \item Materiale didattico - Corso Ingegneria del software 2024/2025 - Il ciclo di vita del SW: \\ \url{https://www.math.unipd.it/~tullio/IS-1/2024/Dispense/T03.pdf}
            \item Materiale didattico - Corso Ingegneria del software 2024/2025 - Gestione di progetto: \\ \url{https://www.math.unipd.it/~tullio/IS-1/2024/Dispense/T04.pdf}
            \item Riferimento per alcune metriche di processo: \\ \url{https://it.wikipedia.org/wiki/Metriche_di_progetto}
            \item Indice di Gulpease: \\ \url{https://it.wikipedia.org/wiki/Indice_Gulpease}
            \item Code coverage: \\ \url{https://www.math.unipd.it/~tullio/IS-1/2024/Dispense/T11.pdf}
            \item Complessità ciclomatica: \\ \url{https://www.math.unipd.it/~tullio/IS-1/2024/Dispense/T11.pdf}
            \item Requirements Stability Index (RSI): \\ \url{https://shiyamtj.wordpress.com/2018/09/26/requirement-stability-index/}
        
        \end{itemize}