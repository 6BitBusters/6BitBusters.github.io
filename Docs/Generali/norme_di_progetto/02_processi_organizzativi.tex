\section{Processi organizzativi}
Questa sezione mira a gestire i processi e il loro miglioramento, l'organizzazione degli strumenti
di supporto e la gestione del personale.


    \subsection{Pianificazione}
        \subsubsection{Metodo di Lavoro}
        Il team ha adottato un modello di lavoro Agile, nello specifico il modello Scrum.
        Di conseguenza seguendo questo modello, i compiti derivanti dai processi di sviluppo vengono suddivisi in sprint dalla
        durata di due settimana.\\
        Questo rende l'avanzamento del prodotto più gestibile e più rapido, avendo comunque un tempo abbastanza lungo per implementare diverse
        feature e redarre i documenti necessari.
            \paragraph*{Sprint} ~\\\\
            Per ogni sprint il responsabile assegna i ruoli ad ogni membro creando un diagramma delle attività per stimare le ore necessarie, tenendo traccia, per ogni membro, i giorni
            in cui può lavorare al progetto. Successivamente l'amministratore si occuperà di creare le issue associate allo sprint in modo da rendere il lavoro degli altri membri più semplice e veloce.
            Infine è compito dell'amministratore quello di assicurarsi che sia avvenuta la verifica,in caso di modifica, del Piano
            di Progetto e delle Norme di Progetto prima dell'inizio dello sprint successivo, in modo da avere la
            documentazione adatta ad affrontare il periodo successivo.\\\\
            Le attività dello sprint sono le seguenti:
            \begin{itemize}
                \item Sprint planning:
                \begin{itemize}
                    \item Si stabiliscono in gruppo quali sono le attività da svolgere durante lo sprint;
                    \item Ogni componente del gruppo, durante la riunione, segnala le ore che può mettere a disposizione;
                    \item Il responsabile assegna i ruoli e definisce gli obiettivi dello sprint nel documento \textit{Piano di progetto};
                \end{itemize}
                \item Daily Scrum:
                \item Sprint review:
                \begin{itemize}
                    \item Ogni membro, durante la riunione, riferisce quello che ha svolto e gli eventuali dubbi che ha riscontrato.
                    \item Viene fatta una lista degli obiettivi raggiunti e quelli non raggiunti.
                \end{itemize}
                \item Sprint retrospective: si fa una valutazione di quello che `e andato bene durante lo sprint e di
                quello che `e da migliorare, per capire cosa continuare o smettere di fare allo sprint successivo.
            \end{itemize}
        \subsubsection{Ruoli e responsabilità}
            I membri team \textit{Six Bit Busters} ricopriranno i ruoli principali 
            di un ciclo di vita del prodotto software, ovvero analista, 
            progettista, programmatore, verificatore, amministratore di sistema e responsabile. 
            \\Al fine di garantire una comprensione completa delle diverse fasi 
            e competenze richieste nello sviluppo di un progetto, i membri del team 
            ruoteranno periodicamente tra i ruoli ogni due settimane. Questa rotazione 
            periodica è finalizzata a scopi didattici, permettendo a ciascun membro di 
            acquisire una visione globale del ciclo di vita del prodotto e di sviluppare 
            abilità pratiche in ogni area.
            \begin{itemize}
                \item \textbf{Responsabile}\\
                Colui che possiede la visione d'insieme del progetto, coordina i membri fra i vari sprint e condensa tutte le voci del team
                dialogando con il proponente per rappresentare il progetto.\\
                    \begin{itemize}
                        \item Ad ogni iterazione si ha solo un Responsabile
                        \item Presenta il Diario di Bordo in aula
                        \item Suddivide le attività del gruppo
                        \item Approva i documenti prima di mandarli in produzione
                    \end{itemize}
                \item \textbf{Amministratore di sistema}\\
                Colui che si occupa del funzionamento, mantenimento e sviluppo degli strumenti e ambienti tecnologici
                usati dal gruppo.
                    \begin{itemize}
                        \item Ad ogni iterazione si hanno al massimo due Amministratori
                        \item Gestione delle segnalazioni e problemi dei membri del gruppo riguardanti problemi e malfunzionamenti
                        con gli strumenti tecnologici
                        \item Valuta l'utilizzo di nuove tecnologie e ne fa uno studio preliminare per poter presentare al
                        gruppo i pro e i contro del suo utilizzo;
                        \item Controllo giornaliero della board e issue per garantire una buona organizzazione.
                        \item Controllo se la documentazione è aggiornata.
                    \end{itemize}
                \item \textbf{Analista}\\
                Colui che si occupa di analizzare a fondo il capitolato e le richieste del proponente per estrarne i requisiti.
                    \begin{itemize}
                        \item Ad ogni iterazione si hanno almeno 2 Analisti
                        \item Studiare le risposte del proponente per identificare i requisiti e redarre l' Analisi dei Requisiti.
                    \end{itemize}
                \item \textbf{Progettista}\\
                Colui che trasforma i requisiti, ricavati degli analisti, in una soluzione, determinando una buona 
                architettura. 
                    \begin{itemize}
                        \item Scegliere eventuali pattern architetturali da implementare
                        \item Sviluppa lo schema UML delle classi
                    \end{itemize}
                \item \textbf{Programmatore}\\
                Colui che si occupa di realizzare tramite codice il design presentato dal progettista.
                    \begin{itemize}
                        \item Scrivere il codice atto a implementare lo schema delle classi
                        \item Scrivere eventuali test
                        \item Scrivere la documentazione per la comprensione del codice che scrive.
                    \end{itemize}
                \item \textbf{Verificatore}\\
                Colui che mette in esame tutto ciò che viene fatto  prima che l'oggetto in question non sia caricato in un branch protetto.
                    \begin{itemize}
                        \item Controlla che la documentazione e il codice scritto siano conformi alle \textit{Norme di progetto}
                        \item Propone possibili migliorie da apportare a documenti e/o codice tramite dei commenti, non può modificare direttamente
                    \end{itemize}
            \end{itemize}

            Per l'analisi dei ruoli e la rendicontazione delle ore preventivate si faccia riferimento al documento \textit{Dichiarazione degli impegni}.

    \subsection{Modalità di comunicazione}
        \subsubsection{Interne}
        Si svolgono tra i componenti del gruppo tramite diversi canali di comunicazione.\\
        Si è deciso di utilizzare:

        \begin{itemize}
            \item Telegram: canale di comunicazione asincrono per comunicazioni brevi e poco importanti
            \item Discord: canale di comunicazione sincrono per le riunioni con i soli membri del team.\\ 
        \end{itemize}
        Le comunicazioni con il fornitore verranno invece gestite in modo asincrono 
        utilizzando Google Chat
        \subsubsection{Esterne}
        Si svolgono tra il gruppo e una persona esterna, generalmente il proponente.
        Si è deciso di utilizzare:

        \begin{itemize}
            \item Google Chat: Per messaggi brevi e per organizzare delle riunioni, scritti dal responsabile
            \item Google Meet: Per riunioni in videochiamata. Queste riunioni dovranno essere richieste
            dal gruppo e al termine verrà redatto un Verbale.
        \end{itemize}


    \subsection{Modalità di riunione}
        \subsubsection{Interne}
        Una riunione interna si svolge esclusivamente tra i membri del gruppo utilizzando il canale apposito
        del server Discord. Per ogni riunione il responsabile sarà incaricato di

        \begin{itemize}
            \item Preparare una scaletta degli argomenti da trattare, che potranno essere poi integrati da eventuali
            punti di discussione portati dagli altri membri del gruppo
            \item A fine riunione scrivere il Verbale. In caso fosse impossibilitato
        \end{itemize}

        Le riunioni si svolgeranno a cadenza settimanale, cercando di trovare giorni e orari agevoli a tutti i
        membri del gruppo.
        \subsubsection{Esterne}
        Le riunioni esterne si svolgono tra i membri del gruppo e il proponente. Possono essere richieste
        dal team con un tempo abbastanza ampio di preavviso. Al termine della riunione verrà redatto un Verbale.


    \subsection{Gestione di Infrastrutture}
        \subsubsection{Descrizione}
        In questa sezione sono riportate le norme relative alla gestione delle infrastrutture. Vengono stabiliti gli
        strumenti di cui il gruppo far`a uso e le relative regole di utilizzo.
        \subsubsection{GitHub}
        Servizio di hosting per progetti software che implementa lo strumento di controllo versione Git. 
        Oltre alla copia in remoto del repository di progetto ogni componente del gruppo ha una propria
        copia in locale nella quale può testare e fare prove senza compromettere la struttura del progetto.
        Per prove più complesse è consigliabile eseguire un fork della repository in modo tale da avere una copia uno ad uno anche in 
        remoto sul proprio account personale.\\
        Per ottenere una copia del repository ogni componente ha scaricato lo strumento Git ed eseguendo il
        comando:
        \begin{verbatim}
            git clone <git@github.com:6BitBusters/6BitBusters.github.io.git>

            oppure
            
            git clone <https://github.com/6BitBusters/6BitBusters.github.io.git>
        \end{verbatim}
        da git bash o terminale windows viene creata una cartella collegata alla repository di progetto.
        I componenti del gruppo abituati ad interagire con GitHub da interfaccia grafica possono continuare a
        farne uso.
        
        
        \subsubsection{Overleaf}
        Inizialmente il team si era deciso ad utilizzare Overleaf come ide per la scrittura di documenti online utilizzando latex, successivamente ci siamo accorti
        che non era un modo ottimale per creare documenti avendo implementato un'automazione di build di quest'ultimi direttamente su GitHub.
        In questo momento Overleaf è utilizzato solamente come ide online, di appoggio, per chi non vuole installare LiveTex in locale.
        
        
        \subsubsection{Repository}
        Il repository si può trovare all'indirizzo https://githubg.com/6bitbusters/6bitbusters.github.io ed è pubblico.
        I collaboratori sono i componenti del gruppo 6BitBusters che utilizzano il proprio account GitHubg
        personale per collaborare al progetto.
        La struttura del repository è formata in questo modo:
            \begin{itemize}
                \item .github: Cartella che contiene i file sorgente delle GitHub Action e template per le Issue
                \item 3Dataviz: Cartella che contiene i file sorgente del prodotto
                \item Docs: Cartella che contiene i file di documentazione relativi ai verbali interni ed esterni e alla candidatura al progetto
                    \begin{itemize}
                        \item Candidatura: Cartella che contiene i documenti da presentare durante la candidatura
                        \item Generali: Cartella che contiene tutta la documentazione esterna e interna tranne i verbali
                        \item Verbali esterni: Cartella contenente i verbali esterni, che riportano gli incontri con i proponenti
                        \item Verbali interni: Cartella contenente i verbali interni, relativi agli incontri tra membri del gruppo
                    \end{itemize}
                \item website: Cartella che contiene i file sorgente del sito web in GitHub Pages
            \end{itemize}
        Infine il repository è dotato di sistema di auto-build per la documentazione grazie alle Actions di GitHub. Nello specifico sono stati scritti 2 tipi di Action
        \begin{itemize}
            \item Per il push sul branch "main" che ha il compito di compilare tutta la documentazione creata fino a quel momento e di creare la page
            \item Per un test di compilazione solamente della parte aggiunta nel branch derivato, che deve essere passato con successo in modo da sbloccare l'azione di merge
        \end{itemize}
        Per maggiori informazioni riguardanti i branch seguire le regole descritte nella sezione 4.2.3.3 Commits

        \subsubsection{Branching}
        I branch si dividono in:
        \begin{itemize}
            \item Branches protetti: 
            \begin{itemize}
                \item main: Branch principale che contiene le documentazione e codice approvato del responsabile, non che la parte che viene mostrata sulla page
                \item docs/NOME-DOCUMENTO: Sono più branch che contengono solamente le versioni di documenti verificate\\
                Il nome di questi branch deve essere:
                \begin{center}
                    docs/[NOME-DOCUMENTO]
                \end{center}
            \end{itemize}
            \item Branches derivati: Sono branch utilizzati per aggiungere modifiche e aggiornare un documento o una parte di codice che poi dovrà essere verificata
            attraverso una pull request verso il branch derivato.\\
            I nomi di questi branch derivati si suddividono in 2 casi deve essere:
            \begin{itemize}
                \item Per Verbali e Glossario
                \begin{center}
                    docs/[NOME-DOCUMENTO]-[COGNOME-ASSEGNATARIO]
                \end{center}
                \item Per tutto il resto della documentazione
                \begin{center}
                    docs/[NOME-DOCUMENTO]-[ID-ISSUE]
                \end{center}
            \end{itemize}
            dove:
    
            \begin{itemize}
                \item NOME-DOCUMENTO: indica il nome del documento sul quale se sta lavorando
                \item ID-ISSUE: indica il numero identificativo associato alla issue in questione
                \item COGNOME-ASSEGNATARIO: indica il cognome del membro che ha redatto il documento
            \end{itemize}
            \item branch di hotfix: Sono branch dedicati al hotfix e quindi a correzioni minime, sia per la documentazione che per il codice. Valgono le stesse regole dei
            branch protetti, quindi le modifiche devono comunque essere verificate e approvate.
            Il nome per questi branch deve essere:
            \begin{center}
                hotfix/[NOME-DOCUMENTO] oppure [NOME-FEATURE]
            \end{center}
            dove:

            \begin{itemize}
                \item NOME-DOCUMENTO: indica il nome del documento che viene modificata
                \item NOME-FEATURE: indica la parte di codice che viene modificata
            \end{itemize}
        \end{itemize}

        In ogni branch derivato può lavorarci al massimo un componente del gruppo, al quale
        viene assegnata una issue da risolvere. Una volta che la issue viene risolta il componente deve richiedere una pull request
        verso il branch da cui è stato derivato.
        Notare che se il branch in questione è:
        \begin{itemize}
            \item main $\rightarrow$ il documento deve essere approvato
            \item branch docs/NOME-DOCUMENTO $\rightarrow$ il documento deve essere verificato e successivamente viene eliminato il branch derivato
        \end{itemize}

        \subsubsection{Commits}
        E'preferibile che ogni commit abbia una singola responsabilità per cambiamento.
        I commits non possono essere effettuati direttamente sui branch protetti ma per integrare delle aggiunte o
        modifiche sarà necessario aprire una Pull Request, motivo per il quale abbiamo introdotto i branch derivati.
        I messaggi di commit dovranno seguire la seguenti strutture sintattiche:
        \begin{center}
            add: [NOME-DOCUMENTO]-[NOME-SEZIONE]\\
            change: [COSA]\\
            restructure: [COSA]
        \end{center}
        dove: 
        \begin{itemize}
            \item add: viene utilizzto per il primo un primo commit e push nella repository remota
            \item change: per i successivi commit, nei quali si va a modificare il documento
            \item restructure: indica una ristrutturazione del branch main per quanto riguarda l'organizzazione delle cartelle o la modifica di action o template di issues
            \item NOME-DOCUMENTO: indica il nome del documento sul quale se sta lavorando
            \item NOME-SEZIONE: indica la sezione relativa al documento
            \item COSA: indica una breve descrizione di cosa di è aggiunto e/o fatto
        \end{itemize}

        
        Notare però che all'approvazione di una Pull Request tutti i commit
        relativi verranno raggruppati in un unico commit, il quale titolo deve rispettare la struttura sintattica descritta in
        seguito.
        \begin{center}
            Update: [NOME-DOCUMENTO]-[VERSONE]
        \end{center}
        dove:

        \begin{itemize}
            \item NOME-DOCUMENTO: indica il nome del documento sul quale se sta lavorando
            \item VERSIONE: indica il numero di versione aggiornata
        \end{itemize}
        per quanto riguarda invece il commento facoltativo si lascia quello di default proposto da GitHub, ovvero un elenco puntato di tutte le commit che verranno raggruppate\\
        
        N.B\\
        Per aggiornare direttamente il main si è deciso di disabilitare temporaneamente
        il protection, eseguire un commit delle modifiche con la struttura sintattica elencata sopra e in fine attivare nuovamente il protection.
        Questa soluzione è stata adottata perché il team ritiene sia veloce e semplice apportare modifiche a template e action in questo modo
        rispetto alla creazione di un branch e ad un successivo Pull Request, inoltre il branch main una volta ultimate le automazioni non verrà più modificato se non dai merge
        scatenati delle Pull Request.

        \subsubsection{Pull requests}
        Per effettuare un merge su un branch protetto si deve aprire da GitHub una Pull Request. La Pull
        Request permette di verificare il lavoro svolto prima di integrarlo con il branch desiderato ed eseguire un veloce test di compilazione sulla sezione aggiunta.
        Alla creazione di una Pull Request bisogna associare:
        \begin{itemize}
            \item Title: [NOME-DOCUMENTO] o [NOME-CASO-USO] -[VERSIONE]
            \item verificatori in carica hanno il compito di trovare eventuali errori o mancanze e fornire un feedback
            riguardante il contenuto direttamente su GitHub attraverso un commento sulla stessa Pull Request.
            Non sarà possibile effettuare il merge finchè tutti i commenti di revisione non saranno stati risolti
            e la Pull Request approvata da due verificatori e il test non dia un esito positivo;
            \item Descrizione contenente una lista riassuntiva di cio' che è stato fatto includendo le issue completate, e quindi da chiudere
            con la sintassi 
            \begin{center}
                close ID-ISSUE
            \end{center}
            in modo tale che vengano tutte chiuse quando la Pull request sarà accettata
            \item Gli assegnatari che hanno anche il compito di apportare le modifiche necessarie in fase di verifica;
            \item Le labels associate
        \end{itemize}
        dove:

        \begin{itemize}
            \item NOME-DOCUMENTO: indica il nome del documento sul quale se sta lavorando
            \item NOME-CASO-USO: indica il nomde del caso d'uso nel quale si sta lavorando
            \item VERSIONE: indica il numero di versione aggiornata
            \item ID-ISSUE: indica il numero identificativo associato alla issue in questione
        \end{itemize}

        Per i commit relativi alle Pull Requests seguire le regole descritte nella sezione 4.2.3.3 Commits per i
        branch protetti.

        \subsubsection{Milestones}
        Esistono 2 tipi di milestone:
        \begin{itemize}
            \item Interne: indicano uno sprint
            \item Esterne: indicano un traguardo intermedio significativo per il progetto
        \end{itemize}
        Ad entrambe le milestones possono essere assegnate delle issues per verificarne il raggiungimento e per tenere traccia
        della percentuale di progressione.
        Ogni milestone ha una scadenza che viene fissata da tutto il gruppo. Una delle prime milestone esterne create dal gruppo è
        relativa alla Requirements and Tecnology Baseline e di conseguenza la realizzazione di un PoC


        \subsubsection{Project board}
        Viene utilizzata un unica project board per tracciare tutte le issues della repository, fungendo da backlog del progetto.
        La project board è divisa nelle seguenti sezioni:
        \begin{itemize}
            \item backlog: issue che non sono ancora state iniziate o assegnate
            \item In progress: issue che sono state assegnate e a cui almeno un membro, tra gli assegnatari, ha iniziato a lavorarci
            \item In review: issue che sono state completate ma il documento o il codice richiede la verifica da parte del verificatore
            \item Done: issue terminate e chiuse, con documenti o codice verificati.
        \end{itemize}
        Esempio di work-flow issues:
        \begin{enumerate}
            \item Viene creata l'issue ed inserita all'interno della sezione ”Backlog” della project board
            \item Quando viene presa in carico, l'issue viene spostata nella sezione ”In Progress” fino al suo completamento
            \item L'incaricato, una volta risolta, apre una pull request a cui assegna la issue, come descritto nella sezione "Pull Request", in modo che
                venga chiusa in automatico.
            \item Dopo la verifica la pull request viene approvata la issue verrà chiusa e spostata nella sezione ”Done”;
        \end{enumerate}


        \subsubsection{Issue Tracking System}
        Il gruppo utilizza l'issue tracking system di Github per tenere traccia delle issue create. Le issues verranno
        determinate dall'amministratore, ma la loro assegnazione verrà effettuata dai membri del gruppo, in base
        alla priorità, i ruoli e disponibilità.
        Per marcare le issues secondo criteri di interesse vengono utilizzate delle labels.
        \begin{itemize}
            \item P1: indica una issue o pull request a priorità alta
            \item P2: indica una issue o pull request a priorità media
            \item P3: indica una issue o pull request a priorità bassa
            \item bug: indica una issue o pull request relativa ad un errore/bug nel codice
            \item code: indica una issue o pull request relativa al codice
            \item documentation: indica una issue o pull request relativa alla documentazione
            \item enhancement: indica una issue o pull request relativa al miglioramento di scrittura documentazione o codice
            \item wontfix: indica una issue che verrà ignorata per motivi di tempo o perchè punta ad una azione facoltativa
        \end{itemize}
        Nel caso un membro del gruppo dovesse rendersi conto che l'issue che sta svolgendo potrebbe essere
        suddivisa in ulteriori issue, dovrà rivolgersi al responsabile,per notificarlo di ciò, il quale, se d'accordo,
        darà il compito all'amministratore di modificare/aggiungere delle issue.
        Per rendere la creazione di issue più veloce si sono creati dei template in base al tipo di issue che si vuole inserire, e che richiedono degli input
        specifici per la loro creazione.
        \begin{itemize}
            \item  TO-DO per documenti
            \begin{itemize}
                \item Titolo : [NOME-DOCUMENTO]-[NOME-SEZIONE]
                \item Oggetto di discussione : [NOME-DOCUMENTO]
                \item Codice verbale: il codice del tracciamento delle decisioni riportato a fine di ogni verbale
                \item Link verbale: un collegamento ipertestuale al documento per accedervi direttamente
                \item Ruolo: il ruolo che l'attività puntata della issue richiede
                \item Ore stimate: una stima delle ore totali per completare l'attività
                \item Informazione da implementare: lista in Markdown di tutte le sezioni/funzioni da implementare
            \end{itemize}
            \item  To-DO per codice
            \begin{itemize}
                \item Titolo : [CODICE-CASO-USO]-[NOME-FEATURES]
                \item Oggetto di discussione : [NOME-DOCUMENTO]
                \item Ruolo: il ruolo che l'attività puntata della issue richiede
                \item Ore stimate: una stima delle ore totali per completare l'attività
                \item Informazione da implementare: breve descrizione di cosa la feature aggiunta deve fare o quali requisiti deve implementare
            \end{itemize}
            \item  Bug nel codice
            \begin{itemize}
                \item Titolo : [CODICE-CASO-USO]-[NOME-CASO-USO]
                \item Descrizione bug: breve descrizione del bug in questione
                \item Passi per riprodurlo: lista numerata in Markdown per riassumere i passi da eseguire in modo tale che un'altro programmatore lo possa replicare
                \item Comportamento aspettato: breve descrizione del comportamento aspettato 
                \item Idea sul motivo: se esiste, una veloce descrizione di un possibile motivo in modo tale da accelerare il processo di debug e correzione
            \end{itemize}
            \item  Correzione della documentazione
            \begin{itemize}
                \item Titolo : [NOME-DOCUMENTO]-[NOME-SEZIONE]
                \item Descrizione: breve descrizione di cosa correggere nella sezione indicata
            \end{itemize}
            \item  Miglioramento della documentazione
            \begin{itemize}
                \item Titolo : [NOME-DOCUMENTO]-[NOME-SEZIONE]
                \item Descrizione: breve descrizione di cosa modificare/migliorare e come nella sezione indicata
            \end{itemize}
            \item  Miglioramento del codice
            \begin{itemize}
                 \item Titolo : [CODICE-CASO-USO]-[NOME-CASO-USO]
                 \item Descrizione: breve descrizione di cosa modificare/migliorare e come nella relativa parte di codice indicata
            \end{itemize}
        \end{itemize}
        dove:

        \begin{itemize}
            \item NOME-DOCUMENTO: indica il nome del documento sul quale se sta lavorando
            \item NOME-SEZIONE: indica la sezione relativa al documento
            \item CODICE-CASO-USO: indica il nome del caso d'uso nel quale si sta lavorando
            \item NOME-FEATURES: indica il nome della feature relativa al caso d'uso
        \end{itemize}


        \subsubsection{Discord}
        Strumento utilizzato per la comunicazione tra i componenti del gruppo. Sono creati diversi canali:
        \begin{itemize}
            \item Link : canale testuale per avere dei collegamenti ipertestuali a risorse e riferimenti utili
            \item Utility : canale testuale per la condivisione di media generale come loghi, disegni illustrativi per determinati workflow da adottare e appunti riguardanti le riunioni
            \item General: canale vocale dedicato alle riunioni, al termine delle quali verrà redatto un Verbale
        \end{itemize}

    \subsection{Gestione dei dubbi e conflitti}
        Nel caso sorgano dubbi, questi verranno risolti, in base all'urgenza, 
        tramite comunicazione nel canale Telegram o riunione interna 
        settimanale. \\
        Eventuali conflitti verranno prevalentemente discussi in sede di riunione 
        interna, con l'obiettivo di trovare una soluzione condivisa che consenta 
        al progetto di progredire secondo una visione comune.