\section{Processi organizzativi}
Questa sezione mira a gestire i processi e il loro miglioramento,
l'organizzazione degli strumenti di supporto e la gestione del personale.

\subsection{Pianificazione}
\subsubsection{Metodo di Lavoro}
Il team ha adottato il metodo di lavoro Scrum, una delle metodologie Agile più
diffuse. Di conseguenza i compiti derivanti dai processi di sviluppo vengono
suddivisi in sprint di due settimane. Questo rende l'avanzamento del prodotto
più gestibile e più rapido, avendo però un tempo sufficiente per implementare
diverse feature e redigere i documenti necessari.
\paragraph*{Sprint}\label{inf:sprint} ~\\\\ Per ogni sprint, il responsabile
assegna i ruoli a ciascun membro, creando un diagramma delle attività per
stimare le ore necessarie e tenendo traccia dei giorni in cui ogni membro può
contribuire alla realizzazione del progetto. Successivamente, l'amministratore
si occuperà di creare le issue associate allo sprint in modo da rendere il
lavoro degli altri membri più semplice e veloce. Inoltre assicura che sia
avvenuta la verifica, in caso di modifica, del \textit{Piano di progetto} e
delle \textit{Norme di progetto} prima dell'inizio dello sprint successivo, in
modo da avere sempre la documentazione adatta e aggiornata sotto mano.\\\\ Le
attività dello sprint sono le seguenti:
\begin{itemize}
    \item \textbf{Sprint planning}:
          \begin{itemize}
              \item Il team definisce collettivamente le attività da svolgere durante lo sprint;
              \item Ogni componente del gruppo, durante la riunione, segnala le ore che può mettere
                    a disposizione;
              \item Il responsabile assegna i ruoli e definisce gli obiettivi dello sprint nel
                    documento \textit{Piano di progetto}.
          \end{itemize}
    \item \textbf{Daily scrum}: ogni giorno, i membri del team sono tenuti a condividere, attraverso il gruppo Telegram dedicato,
          un report dettagliato delle attività svolte il giorno precedente,
          quelle pianificate per la giornata in corso segnalando eventuali ostacoli che potrebbero compromettere il lavoro.
          Il responsabile controlla l'andamento dello sprint contattando i componenti del gruppo.
    \item \textbf{Sprint review}:
          \begin{itemize}
              \item Ogni membro, durante la riunione, riferisce quello che ha svolto nel periodo
                    precedente e gli eventuali dubbi che ha riscontrato;
              \item Viene fatta una lista degli obiettivi raggiunti e quelli non raggiunti.
          \end{itemize}
    \item \textbf{Sprint retrospective}: si fa una valutazione di quello che è andato bene durante lo sprint e di
          quello che è da migliorare, per capire come comportarsi per lo sprint successivo.
\end{itemize}

\subsubsection{Ruoli e responsabilità}
I membri del team \textit{Six Bit Busters} ricopriranno i ruoli principali di un
ciclo di vita del prodotto software, ovvero analista, progettista,
programmatore, verificatore, amministratore di sistema e responsabile. \\ Al
fine di garantire una comprensione completa delle diverse fasi e competenze
richieste nello sviluppo di un progetto, i membri del team ruoteranno
periodicamente tra i ruoli ogni due settimane. \\Questa rotazione periodica è
finalizzata a scopi didattici, permettendo a ciascun membro di acquisire una
visione globale del ciclo di vita del prodotto e di sviluppare abilità pratiche
in ogni area.
\begin{itemize}
    \item \textbf{Responsabile}\\
          Colui che coordina i membri del team e rappresenta il progetto verso l'esterno.
          Partecipa al progetto per tutta la sua durata.
          Le sue competenze sono:
          \begin{itemize}
              \item Preparare e presentare il \textit{Diario di bordo};
              \item Suddividere le attività del gruppo;
              \item Pianificare e gestire le risorse;
              \item Valutare rischi, scelte, alternative;
              \item Approvare i documenti prima di eseguire il merge nel branch main, aggiornando
                    la versione del documento e quindi andando ad incrementare il numero più a
                    sinistra.
          \end{itemize}
    \item \textbf{Amministratore di sistema}\\
          Colui che si occupa del funzionamento, mantenimento e sviluppo degli strumenti e ambienti tecnologici
          usati dal gruppo.
          Le sue competenze sono:
          \begin{itemize}
              \item Gestire le segnalazioni e problemi dei membri del gruppo relativi a
                    malfunzionamenti e difficoltà con gli strumenti tecnologici;
              \item Valutare l'utilizzo di nuove tecnologie e farne uno studio preliminare per
                    poter presentare al gruppo i pro e i contro del loro utilizzo;
              \item Controllare giornalmente la board e le issue per garantire una buona
                    organizzazione;
              \item Controllare se la documentazione è aggiornata;
              \item Presentare il \textit{Diario di bordo} in aula nel caso il responsabile non sia
                    presente;
              \item Redigere i verbali.
          \end{itemize}
          Ad ogni sprint si hanno al massimo due amministratori;
    \item \textbf{Analista}\\
          Colui che si occupa di analizzare a fondo il capitolato e le richieste del proponente per estrarne i requisiti.
          Le sue competenze sono:
          \begin{itemize}
              \item Studiare le richieste del proponente per identificare i requisiti e redigere
                    l'\textit{Analisi dei requisiti}.
          \end{itemize}
          Ad ogni sprint si ha almeno un analista;
    \item \textbf{Progettista}\\
          Colui che trasforma i requisiti, ricavati degli analisti, in una soluzione che abbia bassa complessità individuale.
          Le sue competenze sono:
          \begin{itemize}
              \item Prendere decisioni di natura tecnica e tecnologica riguardo lo sviluppo del
                    prodotto;
              \item Definire l'architettura del prodotto, in modo da soddisfare le specifiche;
              \item Sviluppare i diagrammi UML delle classi.
          \end{itemize}
    \item \textbf{Programmatore}\\
          Colui che si occupa di realizzare, tramite codice, il design presentato dal progettista.
          Le sue competenze sono:
          \begin{itemize}
              \item Scrivere il codice atto ad implementare i diagrammi delle classi;
              \item Scrivere eventuali test per il codice;
              \item Scrivere la documentazione per la comprensione del codice che scrive.
          \end{itemize}
    \item \textbf{Verificatore}\\
          Colui che si occupa di controllare che il lavoro degli altri membri del gruppo rispetti le \textit{Norme di progetto}
          prima che sia caricato in un branch protetto.
          Le sue competenze sono:
          \begin{itemize}
              \item Verificare che la documentazione e il codice scritto siano conformi alle
                    \textit{Norme di progetto};
              \item Proporre possibili migliorie da apportare a documenti e/o codice tramite dei
                    commenti.
          \end{itemize}
\end{itemize}

Per l'analisi dei ruoli e la rendicontazione delle ore preventivate si faccia
riferimento ai documenti \textit{Dichiarazione degli impegni} e \textit{Piano
    di progetto}.

\subsubsection{Metriche}
Per perseguire la qualità nel processo riguardo la pianificazione, si è deciso
di adottare le seguenti metriche:
\begin{itemize}
    \item \nameref{M:BV};
    \item \nameref{M:SV};
    \item \nameref{M:UR}.
\end{itemize}

\subsection{Modalità di comunicazione}
\subsubsection{Interne}
Si svolgono tra i soli componenti del gruppo. Si utilizza:
\begin{itemize}
    \item \textbf{Telegram}: canale di comunicazione asincrono per comunicazioni brevi;
    \item \textbf{Discord}: canale di comunicazione sincrono per le riunioni interne.
\end{itemize}
\subsubsection{Esterne}
Si svolgono tra il gruppo e una persona esterna, generalmente il proponente
o il committente. Si utilizza:
\begin{itemize}
    \item \textbf{Google Chat}: canale di comunicazione asincrono con il proponente;
    \item \textbf{Google Meet}: canale di comunicazione sincrono con il proponente;
    \item \textbf{Gmail}: canale di comunicazione asincrono con il committente;
    \item \textbf{Zoom}: canale di comunicazione sincrono con il committente.
\end{itemize}

\subsection{Modalità di riunione}
\subsubsection{Interne}
Le riunioni interne si svolgono esclusivamente tra i membri del gruppo
utilizzando il canale apposito del server Discord. Per ogni riunione il
responsabile sarà incaricato di preparare una scaletta degli argomenti da
trattare, che potranno essere poi integrati da eventuali punti di discussione
portati dagli altri membri del gruppo.\\ Le riunioni si svolgono al termine di ogni sprint, cercando di trovare giorni e orari agevoli per tutti i membri del
gruppo. \\Al termine della riunione verrà redatto un verbale interno.

\subsubsection{Esterne}
Le riunioni esterne si svolgono tra i membri del gruppo e il proponente a
cadenza bisettimanale. Al termine della riunione verrà redatto un verbale
esterno.

\subsection{Gestione delle infrastrutture}
\subsubsection{Descrizione}
In questa sezione sono riportate le norme relative alla gestione delle
infrastrutture, vengono stabiliti gli strumenti di cui il gruppo farà uso e le
relative regole di utilizzo.

\subsubsection{GitHub}
Come servizio di hosting per il progetto il team ha optato per GitHub. \\Oltre
alla copia in remoto del repository del progetto, ogni membro del gruppo ha
una propria copia in locale nella quale può testare e fare prove senza
compromettere la struttura del progetto. \\Per prove più complesse è
consigliabile eseguire un fork del repository, in modo da avere una copia
identica del progetto anche in remoto sul proprio account personale.\\ 
Ogni membro può scaricare una copia del repository utilizzando Git ed
eseguendo uno dei seguenti comandi su Git bash o terminale Windows:
\begin{verbatim}
            git clone <git@github.com:6BitBusters/6BitBusters.github.io.git>
            git clone <https://github.com/6BitBusters/6BitBusters.github.io.git>
        \end{verbatim}
Il primo comando deve essere utilizzato se si impiegano le chiavi SSH per l'autenticazione, 
mentre il secondo è indicato per l'uso dei personal token.\\ Una volta completato il download verrà creata una cartella
collegata al repository del progetto in remoto.\\ Alternativamente, è possibile
eseguire le stesse operazioni tramite interfaccia grafica, con l'applicativo
GitHub Desktop.

\subsubsection{Overleaf}
Inizialmente, il team aveva scelto Overleaf come editor principale per la scrittura collaborativa di documenti online in LaTeX. 
Tuttavia, è emerso che non era la soluzione ottimale per la creazione dei documenti, soprattutto considerando l'automazione del processo di build implementata direttamente su GitHub. 
Attualmente Overleaf viene utilizzato esclusivamente come editor di supporto per i membri che non hanno installato LiveTeX in locale.

\subsubsection{Milestone}
Esistono 2 tipi di milestone:
\begin{itemize}
    \item \textbf{Interne}: indicano uno sprint;
    \item \textbf{Esterne}: indicano un traguardo intermedio significativo per il progetto.
\end{itemize}
Ad entrambe le milestone possono essere assegnate delle issue per verificarne il raggiungimento e per tenere traccia
della percentuale di progressione dello sprint.
Ogni milestone ha una scadenza che viene fissata da tutto il gruppo. Una delle prime milestone esterne create è
relativa alla Requirements and Technology Baseline e, di conseguenza, alla realizzazione di un PoC.

\subsubsection{Project board}\label{inf:pb}
È utilizzata un'unica project board per tracciare tutte le issue del repository, che funge da backlog del progetto. 
La project board è organizzata in quattro pannelli, che mostrano e raggruppano le issue in base alle loro caratteristiche.
\begin{itemize}
    \item \textbf{Board}: rappresenta le issue con un modello kanban;
    \item \textbf{Priority board}: simile alla Board ma ordina le issue in base alla loro priorità;
    \item \textbf{Team items}: rappresenta le issue tramite una lista verticale e rende possibile
          la modifica delle loro proprietà in modo semplice e veloce;
    \item \textbf{Roadmap}: rappresenta le issue in maniera simile a un diagramma di Gantt,
          dove in questo caso i task sono le singole issue.
\end{itemize}
Ad ogni issue è associato inoltre uno stato che viene utilizzato principalmente nei pannelli Board e Priority Board per una migliore gestione.
I possibili stati sono:
\begin{itemize}
    \item \textbf{Backlog}: issue che non sono ancora state iniziate o assegnate;
    \item \textbf{In progress}: issue che sono state assegnate e a cui almeno un membro, tra gli assegnatari, ha iniziato a lavorare;
    \item \textbf{Waiting for review}: issue terminate e che potrebbero essere chiuse definitivamente dopo la verifica delle modifiche apportate al documento;
    \item \textbf{In review}: issue terminate e in corso di verifica;
    \item \textbf{Done}: issue terminate e chiuse, con documenti o codice verificati.
\end{itemize}
Un esempio di workflow potrebbe essere il seguente:
\begin{enumerate}
    \item Viene creata la issue ed inserita all'interno della sezione ”Backlog” della
          Project board;
    \item Quando viene presa in carico, la issue viene spostata nella sezione ”In
          Progress” fino al suo completamento;
    \item L'incaricato, una volta risolta, apre una pull request a cui assegna la issue,
          come descritto nella sezione \nameref{inf:pr};
    \item Dopo l'approvazione della pull request la issue verrà chiusa e spostata in modo
          automatico nella sezione ”Done”.
\end{enumerate}

\subsubsection{Issue Tracking System}\label{inf:its}
Il gruppo utilizza l'issue tracking system di Github per tenere traccia delle
issue create. Le issue verranno create dall'amministratore, ma la loro
assegnazione verrà effettuata dai membri del gruppo in modo autonomo, in base
alla priorità, ruoli e disponibilità. Per marcare le issue secondo criteri di
interesse vengono utilizzate delle label:
\begin{itemize}
    \item \textbf{Correction}: indica una issue o pull request relativa ad una correzione alla documentazione;
    \item \textbf{Bug}: indica una issue o pull request relativa ad un errore/bug nel codice;
    \item \textbf{Code}: indica una issue o pull request relativa al codice;
    \item \textbf{Documentation}: indica una issue o pull request relativa alla documentazione;
    \item \textbf{Enhancement}: indica una issue o pull request relativa al miglioramento di scrittura documentazione o codice;
    \item \textbf{Wontfix}: indica una issue che verrà ignorata per motivi di tempo o perché punta ad una azione facoltativa.
\end{itemize}
Ogni issue, oltre alle label, possiede degli attributi aggiuntivi che permettono di arricchire
notevolmente la rappresentazione visiva sui pannelli, facilitando la comprensione e l'organizzazione del lavoro.
Si faccia riferimento alla sezione \nameref{inf:pb}.
Questi attributi sono:
\begin{itemize}
    \item \textbf{Estimate}: indica le ore stimate per completare il task descritto dalla issue;
    \item \textbf{Effective}: indica le ore effettive per completare il task descritto dalla issue;
    \item \textbf{Priority}: indica la priorità di una issue;
          \begin{itemize}
              \item \textbf{P0}: priorità alta;
              \item \textbf{P1}: priorità media;
              \item \textbf{P2}: priorità bassa.
          \end{itemize}
    \item \textbf{Begin}: indica la data entro la quale si deve iniziare a svolgere il task descritto dalla issue;
    \item \textbf{Due by}: indica la data entro la quale il task descritto dalla issue deve essere terminata.
\end{itemize}

Se un membro del gruppo si rende conto che la issue che sta
svolgendo potrebbe essere suddivisa in ulteriori issue, deve rivolgersi al
responsabile, il quale, se d'accordo, dà all'amministratore
il compito di modificare/aggiungere delle issue. Per rendere la
creazione delle issue più veloce e standardizzata, sono stati creati dei template in base al tipo di
issue che si vuole inserire e che richiedono degli input specifici.
\begin{itemize}
    \item  \textbf{TO-DO per documenti}
          \begin{itemize}
              \item \textbf{Titolo}: [NOME-DOCUMENTO]-[NOME-SEZIONE];
              \item \textbf{Oggetto di discussione} (facoltativo): [NOME-VERBALE];
              \item \textbf{Codice verbale}: il codice del tracciamento delle decisioni riportato a fine di ogni verbale;
              \item \textbf{Link verbale}: un collegamento ipertestuale al documento per accedervi direttamente;
              \item \textbf{Ruolo}: il ruolo che l'attività puntata dalla issue richiede;
              \item \textbf{Informazione da implementare}: lista di tutte le possibili parti della sezione da implementare.
          \end{itemize}
    \item  \textbf{To-DO per codice}
          \begin{itemize}
              \item \textbf{Titolo} : [CODICE-CASO-USO]-[NOME-FEATURE];
              \item \textbf{Oggetto di discussione} (facoltativo): [NOME-VERBALE];
              \item \textbf{Ruolo}: il ruolo che l'attività puntata dalla issue richiede;
              \item \textbf{Informazione da implementare}: breve descrizione di cosa, la feature aggiunta, deve fare o quali requisiti deve soddisfare.
          \end{itemize}
    \item  \textbf{Bug nel codice}
          \begin{itemize}
              \item \textbf{Titolo}: [CODICE-CASO-USO]-[NOME-FEATURE];
              \item \textbf{Descrizione bug}: breve descrizione del bug;
              \item \textbf{Passi per riprodurlo}: lista numerata per riassumere i passi da eseguire in modo tale che un'altro programmatore lo possa replicare;
              \item \textbf{Comportamento aspettato}: breve descrizione del comportamento aspettato;
              \item \textbf{Idea sul motivo} (facoltativo): se esiste, una veloce descrizione di un possibile motivo in modo tale da accelerare il processo di debug e correzione.
          \end{itemize}
    \item  \textbf{Correzione della documentazione}
          \begin{itemize}
              \item \textbf{Titolo} : [NOME-DOCUMENTO]-[NOME-SEZIONE];
              \item \textbf{Descrizione}: breve descrizione di cosa correggere nella sezione indicata.
          \end{itemize}
    \item  \textbf{Miglioramento della documentazione}
          \begin{itemize}
              \item \textbf{Titolo} : [NOME-DOCUMENTO]-[NOME-SEZIONE];
              \item \textbf{Descrizione}: breve descrizione di cosa modificare/migliorare e come, nella sezione indicata.
          \end{itemize}
    \item  \textbf{Miglioramento del codice}
          \begin{itemize}
              \item \textbf{Titolo} : [CODICE-CASO-USO]-[NOME-CASO-USO];
              \item \textbf{Descrizione}: breve descrizione di cosa modificare/migliorare e come, nella relativa parte di codice indicata.
          \end{itemize}
\end{itemize}
dove:

\begin{itemize}
    \item \textbf{NOME-DOCUMENTO}: indica il nome del documento sul quale si sta lavorando;
    \item \textbf{NOME-VERBALE}: indica il nome del verbale nel quale si è discusso della relativa attività rappresentata successivamente tramite una issue;
    \item \textbf{NOME-SEZIONE}: indica la sezione relativa al documento;
    \item \textbf{CODICE-CASO-USO}: indica il nome del caso d'uso nel quale si sta lavorando;
    \item \textbf{NOME-FEATURE}: indica il nome della feature relativa al caso d'uso.
\end{itemize}

\subsubsection{Discord}
Strumento utilizzato per la comunicazione, in modo sincrono, tra i componenti
del gruppo.\\ Sono stati creati diversi canali:
\begin{itemize}
    \item \textbf{Link} : canale testuale per avere dei collegamenti ipertestuali a risorse e riferimenti utili;
    \item \textbf{Utility} : canale testuale per la condivisione di media generici come loghi, disegni illustrativi per determinati workflow e appunti riguardanti le riunioni;
    \item \textbf{General}: canale vocale dedicato alle riunioni interne, al termine delle quali verrà redatto un verbale.
\end{itemize}

\subsubsection{Telegram}
Strumento utilizzato per la comunicazione, in modo asincrono, tra i componenti
del gruppo.\\ Sono stati creati 2 distinti gruppi:
\begin{itemize}
    \item \textbf{Generale} : gruppo dedicato alle comunicazioni brevi e poco importanti;
    \item \textbf{Daily scrum} : gruppo dedicato all'attivita di daily scrum, descritta nella sezione \nameref{inf:sprint}.
\end{itemize}

\subsection{Gestione dei dubbi e conflitti}
Nel caso sorgano dubbi, questi verranno risolti, in base all'urgenza, tramite
comunicazione nel canale Telegram o durante la riunione interna settimanale. \\
I conflitti invece verranno prevalentemente discussi in sede di riunione
interna, con l'obiettivo di trovare una soluzione condivisa che consenta al
progetto di progredire secondo una visione comune.

