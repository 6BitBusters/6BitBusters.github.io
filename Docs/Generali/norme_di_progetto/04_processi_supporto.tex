\section{Processi di Supporto}
I processi di supporto contribuiscono a rendere i processi primari più efficienti ed efficaci.
    \subsection{Documentazione}
        \subsubsection{Descrizione}
        Questa sezione contiene tutte le norme che ogni membro del gruppo dovrà seguire durante 
        la stesura della documentazione.\\
        Fornisce le indicazioni utili per ottenere una forma uniforme dei documenti e permette di operare seguendo linee guida durante tutte le fasi del ciclo di vita di un documento: creazione, stesura, verifica\textsubscript{g} ed 
        eventuali modifiche, fino ad arrivare all'approvazione\textsubscript{g} e quindi alla pubblicazione del documento.
        \subsubsection{Ciclo di vita di un documento}
        \begin{itemize}
            \item \textbf{Creazione}: viene creato un nuovo branch\textsubscript{g}, che prende il nome del documento stesso, e qui verrà caricato il template, in questo branch\textsubscript{g} ci saranno solamente le versioni verificate del documento;
            \item \textbf{Stesura}: per la stesura verrà creato un sotto-branch\textsubscript{g}, caratterizzato dal numero della issue che richiede la stesura, in seguito si procede con la stesura delle sezioni necessarie, tracciando i cambiamenti nel registro delle modifiche e aggiornando la versione;
            \item \textbf{Verifica\textsubscript{g}}: prima di considerarsi completata, ogni modifica al documento deve essere verificata da un verificatore in carica, tramite una Pull Request\textsubscript{g} viene messo in esame quanto aggiunto al documento rispetto alla versione precedente, si presentano due casi:
            \begin{itemize}
                \item \textbf{Pull Request\textsubscript{g} accettata}: il verificatore non trova errori e considera il lavoro 'verificato', viene accettata la Pull Request\textsubscript{g} e viene fatto il merge tra il branch\textsubscript{g} di lavoro e il branch\textsubscript{g} del documento, viene così chiusa la issue assegnata e chiuso il branch di lavoro;
                \item \textbf{Pull Request\textsubscript{g} rifiutata}: il verificatore non considera la stesura adeguata e procede a segnalare errori ed eventuali correzioni, la Pull Request\textsubscript{g} viene rifiutata e si continua a lavorare nel sotto-branch\textsubscript{g};
            \end{itemize}
            \item \textbf{Approvazione\textsubscript{g}}: Una volta che tutte le sezioni sono state verificate, e il documento è pronto per essere pubblicato, si procede con l'approvazione\textsubscript{g}, effettuata dal responsabile, utilizziamo questa procedura:
            \begin{itemize}
                \item Viene aperta una Pull Request\textsubscript{g} per fare il merge del branch\textsubscript{g} dedicato al documento con il main;
                \item Il responsabile rilegge ed analizza il documento nella sua interezza, se riscontra errori o problematiche, rifiuta la Pull Request\textsubscript{g}, segnala ai verificatori le modifiche necessarie e si procede come descritto sopra;
                \item Nel caso andasse tutto bene, il responsabile aggiorna il registro delle modifiche aggiungendo una riga con la versione di pubblicazione, e contrassegna il documento come 'Approvato';
                \item Il documento viene quindi pubblicato nel ramo main, e reso disponibile pubblicamente.
            \end{itemize}
        \end{itemize}
        Solo alla pubblicazione nel ramo main il documento verrà compilato e reso disponibile in formato '.pdf'.
        \newpage
        \subsubsection{Struttura}
        Ogni documento è caratterizzato da un template, che presenta queste caratteristiche:
        \begin{itemize}
            \item \textbf{Intestazione}: la prima pagina di ogni documento, contiene:
            \begin{itemize}
                \item Logo del gruppo;
                \item Indirizzo email del gruppo;
                \item Titolo del documento;
                \item Informazioni sul documento, che comprendono:
                \begin{itemize}
                    \item Versione;
                    \item Stato: in redazione oppure approvato;
                    \item Uso: interno o esterno;
                    \item Approvazione\textsubscript{g}: indica il nome dell'elemento del gruppo che ha approvato il documento;
                    \item Redazione: indica il nome o i nomi di chi si è occupato della stesura;
                    \item Verifica\textsubscript{g}: indica il nome del verificatore o dei verificatori;
                    \item Distribuzione: elenco delle persone o organizzazioni a cui è destinato il documento.
                \end{itemize}
                \item Descrizione: una breve descrizione di cosa contiene il documento;
            \end{itemize}
            \item \textbf{Registro delle modifiche}: contiene la tabella dove verranno registrati i cambiamenti e le versioni che un documento attraversa prima di giungere alla versione finale, include le seguenti colonne:
            \begin{itemize}
                \item Versione;
                \item Data;
                \item Autore;
                \item Descrizione: una descrizione riassuntiva di ciò che è stato aggiunto o modificato;
                \item Verificatore.
            \end{itemize}
            \item \textbf{Indice}: elenco ordinato dei titoli dei capitoli, per facilitare la navigazione;
            \item \textbf{Contenuto}: varia a seconda del documento.
        \end{itemize}
        % + tutte le convenzioni
        \subsubsection{Convenzioni}
        Al fine di ottenere una stesura della documentazione omogenea, e quindi più professionale, verranno seguite queste convenzioni:
        \paragraph{Date}
        Per garantire un ordinamento in ordine cronologico in fase di pubblicazione, dal più recente al meno recente, verrà utilizzato il formato \textbf{yyyy-mm-dd} nel caso la data debba essere indicata nel nome del documento (vale per verbali interni/esterni), 
        per indicare una data all'interno del documento, verrà utilizzato il formato \textbf{dd-mm-yyyy}.
        \paragraph{Nomi di persona} 
        All'interno dei documenti i nomi di persona saranno rappresentati da cognome e nome.
        \paragraph{Elenchi puntati}
        Gli elenchi puntati saranno gestiti in questo modo:
        \begin{itemize}
            \item Ogni elemento dell'elenco deve iniziare con la lettera maiuscola;
            \item Ogni elemento dell'elenco deve terminare con ';', ad eccezione dell'ultimo che terminerà con '.';
            \item Dopo i due punti la prima parola deve iniziare con la lettera minuscola.
        \end{itemize}
        \paragraph{Stile del testo}
        \begin{itemize}
            \item \textbf{Grassetto}: utilizzato per i titoli delle sezioni, per i sottotitoli e per i paragrafi, oltre alle parole ritenute particolarmente significative;
            \item \textbf{Corsivo}: utilizzato quando vengono scritti nomi di documenti, nome del gruppo o indirizzo email del gruppo.
        \end{itemize}
        \subsubsection{Strumenti per la stesura}
        Per la stesura dei documenti verrà usato il linguaggio Latex, un linguaggio di marcatura per la preparazione di testi, basato sul programma di
        composizione tipografica TEX, tutta la documentazione prodotta è contenuta nella cartella 'Docs'. I documenti seguono una struttura comune:
        \begin{itemize}
            \item Cartella 'config' contenente il file 'changelog\_input' che permette di compilare i campi della tabella di registrazione delle modifiche;
            \item Cartella 'template' contenente:
            \begin{itemize}
                \item File 'changelog': questo file contiene la definizione di un comando chiamato \texttt{\char`\\changelogTable}, che serve per generare la tabella formattata;
                \item File 'package': configura pacchetti e comandi per personalizzare l'impaginazione, le tabelle, le intestazioni, la numerazione e la formattazione di testo, inclusi glossari e codici;
                \item Cartella 'Images' contenente le immagini inserite nel documento.
            \end{itemize}
            \item File 'main' include i file e i pacchetti necessari a comporre il file;
            \item File 'titlepage' contenente il template della pagina di intestazione.
        \end{itemize}
        Oltre a questi files, verrà creato un file per ogni sezione, per garantire maggiore ordine all'organizzazione del documento, e facilitare la suddivisione dei compiti.
        \subsubsection{Documentazione interna}
        La documentazione interna è composta da tutti i documenti che contengono informazioni utili per il gruppo,  
        ma saranno comunque resi pubblici all'interno del repository, e nella schermata creata tramite GitHub Pages\textsubscript{g}.\\
        La documentazione interna è composta da:
        \begin{itemize}
            \item \textit{\textbf{Verbali Interni}}: hanno lo scopo di riportare ciò che viene detto e discusso durante la riunione interna, ossia tra i soli membri del gruppo, 
            rispetta la struttura generale dei documenti e le convenzioni, il nome del file deve avere la forma 'VI\_yyyy-mm-dd', per garantire l'ordinamento.
            \\Nei \textit{verbali interni} la struttura del contenuto assume questa forma:
            \begin{itemize}
                \item \textbf{Informazioni generali}: contiene informazioni circa i dettagli sull'incontro, nello specifico:
                \begin{itemize}
                    \item Luogo;
                    \item Data;
                    \item Ora di inizio;
                    \item Ora di fine;
                    \item Partecipanti.
                \end{itemize}
                \item \textbf{Motivo della riunione}: breve descrizione narrativa di cosa è stato trattato in quell'incontro, descrive i motivi per cui è stata fatta la riunione;
                \item \textbf{Resoconto}: descrive i temi trattati nel dettaglio, partendo dal motivo per il quale sono stati sollevati e arrivando alla decisione presa dal gruppo a seguito di una discussione;
                \item \textbf{Prossimi obiettivi}: elenco puntato che descrive gli obiettivi che il gruppo si impegna a portare a termine nel breve periodo, indicativamente prima della riunione successiva;
                \item \textbf{Tracciamento delle decisioni}: tabella che riassume le decisioni prese in quella riunione indicandone:
                \begin{itemize}
                    \item \textbf{Codice}: in formato VI Y.Z, dove Y indica il numero del \textit{verbale} (incrementale rispetto agli altri), Z indica il numero dell'argomento trattato (non per importanza, ma per ordine di discussione);
                    \item \textbf{Descrizione}: descrizione in poche parole dell'argomento trattato.
                \end{itemize} 
                \end{itemize}
        \item \textit{\textbf{Studio di fattibilità}}: documento interno di valutazione dei capitolati proposti dalle aziende per il progetto didattico, con lo scopo di selezionare il progetto migliore a cui candidarsi secondo valutazioni 
        prese da parte del gruppo.\\
        Il documento, per ogni capitolato, espone una breve descrizione del prodotto che l'azienda chiede di sviluppare, seguito da una serie di pro e contro emersi in base a criteri soggettivi dei membri del gruppo.
        \\La valutazione è fatta sulla base di:
        \begin{itemize}
            \item \textbf{Presentazione del capitolato}: una presentazione più curata e dettagliata riscontra maggiore successo in fase di valutazione;
            \item \textbf{Interesse del gruppo al tema del progetto}: un capitolato porta con se un tema, viene valutato quanto ogni capitolato sia interessante, per apportare un impatto positivo allo svolgimento da parte del gruppo;
            \item \textbf{Tecnologie da utilizzare}: il contesto tecnologico di comune interesse porta maggiore produttività ed entusiasmo all'interno del gruppo;
            \item \textbf{Conoscenze pregresse}: un capitolato può risultare più o meno complicato da svolgere a seconda delle conoscenze acquisite nel percorso dai membri del gruppo;
            \item \textbf{Supporto da parte dell'azienda}: maggiore è il supporto offerto dall'azienda, migliore sarà la valutazione del capitolato.
        \end{itemize}  
        Il documento offre una panoramica completa di tutti i capitolati, potendoli valutare minuziosamente prima di esprimere una valutazione.
        \item \textit{\textbf{Norme di progetto}}: documento interno che contiene le norme applicate dai membri del gruppo durante il ciclo di vita del prodotto, rispetta la struttura generale della documentazione, il corpo di questo documento è composto da:
        \begin{itemize}
            \item \textbf{Introduzione}: contiene una breve descrizione dello scopo del documento e del contesto in cui viene applicato.
            \item \textbf{Processi organizzativi}: costituisce la definizione dei processi per quanto riguarda l'organizzazione del gruppo, in termini di:
            \begin{itemize}
                \item Pianificazione: viene definito il metodo di lavoro, i ruoli che verranno assegnati e le responsabilità che da essi ne derivano;
                \item Modalità di comunicazione: vengono definite le modalità attraverso le quali il gruppo comunicherà internamente ed esternamente (qualsiasi comunicazione che comprenda un soggetto esterno al gruppo di lavoro);
                \item Modalità di riunione: vengono definite le modalità con le quali si svolgeranno le riunioni, interne ed esterne;
                \item Gestione di infrastrutture: descrizione delle infrastrutture utilizzate per lo sviluppo del progetto, affinchè venga garantita affidabilità e sicurezza;
                \item Gestione dei dubbi o conflitti.
            \end{itemize}
            \item \textbf{Processi primari}: definizione dei processi primari, nel nostro caso:
            \begin{itemize}
                \item Fornitura: definizione e regole di fornitura e rapporto con il proponente;
                \item Sviluppo: definizione e regole di sviluppo, per quanto riguarda analisi dei requisiti, progettazione e codifica.
            \end{itemize}
            \item \textbf{Processi di supporto}: definizione dei processi di supporto, nel nostro caso:
            \begin{itemize}
                \item Documentazione: descrive le regole e la struttura dei documenti, il ciclo di vita e le convenzioni da utilizzare durante la stesura. Vengono descritti i documenti presenti nel repository.
                \item Gestione della configurazione: descrive come gestire il versionamento dei documenti.
                \item Accertamento della qualità: descrive come vengono verificati i documenti;
                \item Validazione: comprende le regole e la definizione di validazione.
            \end{itemize} 
        \end{itemize}
        \end{itemize}
        \subsubsection{Documentazione esterna}
        La documentazione esterna è composta da tutti i documenti che interessano anche al proponente e al committente.
        \begin{itemize}
            \item \textit{\textbf{Verbali esterni}}: verbali frutto di incontri fra i membri del gruppo e soggetti esterni ad esso.\\
            Seguono la struttura dei verbali interni, il nome del file deve avere la forma 'VE\_yyyy-mm-dd', cambia il processo di approvazione\textsubscript{g} del documento, che oltre ad essere approvato internamente, viene approvato esternamente dal soggetto esterno con il quale 
            si è svolto l'incontro, questo per rendere il documento più professionale e garantire coerenza tra il gruppo e il soggetto esterno.
            \item \textit{\textbf{Lettera di candidatura}}: la \textit{Lettera di candidatura} serve per esprimere la volontà di candidarsi allo svolgimento del capitolato scelto dopo aver discusso e redatto lo \textit{Studio di fattibilità}, 
            è composta da:
            \begin{itemize}
                \item Pagina iniziale: viene dichiarato il capitolato per il quale il gruppo ha deciso di candidarsi, elencando i capitolati che più hanno riscontrato interesse da parte del gruppo, 
                inoltre espone una descrizione di cosa aspettarsi dal contenuto del documento;
                \item Resoconto degli incontri: una breve descrizione degli incontri svolti con le aziende riguardo i capitolati elencati nella pagina iniziale;
                \item Motivazione della scelta: espone il motivo per il quale il gruppo ha scelto il capitolato a cui candidarsi.
            \end{itemize}
            \item \textit{\textbf{Diario di bordo}}: documento informale ad uso esterno che permette di interagire settimanalmente con il committente per riportare aggiornamenti sullo stato di avanzamento del progetto, descrivendo tutto ciò che è stato fatto rispetto al \textit{diario di bordo} precedente, 
            e quello che il gruppo si impegna a fare nel periodo successivo, vengono inoltre riportati dubbi e domande da porre al committente.
            Vengono elaborati tramite la piattaforma online 'Canva.com', che permette di creare presentazioni collaborative, accessibili online.\\ 
            La struttura del \textit{diario di bordo} è composta da 4 slides contenenti:
            \begin{itemize}
                \item \textbf{Slide 1}: slide di presentazione che contiene:
                \begin{itemize}
                    \item Logo;
                    \item Nome del gruppo;
                    \item Indirizzo email del guppo;
                    \item Titolo del documento: il titolo del \textit{diario di bordo} segue una sintassi prefissata ovvero 'Diario di bordo \#N', dove N è un numero che incrementa ad ogni presentazione.
                \end{itemize}
                \item \textbf{Slide 2}: contiene ciò che è stato svolto nel periodo trascorso;
                \item \textbf{Slide 3}: contiene ciò che il gruppo si impegna a portare a termine nel periodo successivo;
                \item \textbf{Slide 4}: contiene dubbi da chiarire e difficoltà incontrate dal gruppo.
            \end{itemize}
            \item \textit{\textbf{Piano di progetto}}: il documento \textit{piano di progetto} ha lo scopo di supportare la gestione delle risorse per quanto riguarda l'avanzamento del progetto, per riuscire a portarlo a termine entro la data decisa.\\
            Il \textit{piano di progetto} ha inoltre la funzione di descrivere il modello di sviluppo adottato, e di monitorarlo tramite la suddivisione in periodi, per analizzare il lavoro svolto e poter apportare miglioramenti con il passare del tempo.
            \\La struttura del documento segue la struttura generale, per quanto riguarda il corpo, è composto da queste sezioni:
            \begin{itemize}
                \item \textbf{Introduzione}: breve introduzione che descrive cosa aspettarsi dal contenuto del documento;
                \item \textbf{Analisi dei rischi}: sezione che riassume i rischi a cui il gruppo si espone a seguito dell'aggiudicazione del capitolato;
                \item \textbf{Modello di sviluppo}: sezione che descrive il modello di sviluppo adottato dal gruppo;
                \item \textbf{Pianificazione}: sezione che descrive lo svolgimento delle attività periodo per periodo;
                \item \textbf{Preventivo}: sezione in cui viene pianificata in dettaglio la suddivisione dei ruoli con le corrispondenti ore di lavoro, per fornire un preventivo rispetto al periodo a cui ci si sta accingendo;
                \item \textbf{Consuntivo}: sezione che riporta i dati raccolti al termine del periodo, per compararli con la previsione presente nella sezione di preventivo;
                \item \textbf{Attualizzazione dei rischi}: sezione dove vengono riportati i rischi che si sono verificati durante lo svolgimento del progetto e le relative misure di mitigazione attuate.
            \end{itemize}
            \item \textit{\textbf{Glossario}}: documento che contiene il significato dei termini chiave utilizzati nel progetto (indicati con il pedice 'g'), organizzati in ordine alfabetico, utile per garantire una comprensione comune fornendo spiegazioni concise e precise.
            \\La struttura del \textit{glossario} è composta da sezioni caratterizzate dalle lettere dell'alfabeto in ordine alfabetico, contenenti le parole che hanno come iniziale quella lettera.
        \end{itemize}

        \subsubsection{Metriche}
        Per perseguire la qualità nel processo riguardo la documentazione si `e deciso di adottare le seguenti metriche:
        \begin{itemize}
                \item \nameref{M:GI};
        \end{itemize}

    \subsection{Gestione delle configurazione}
        \subsubsection{Scopo}
        La gestione della configurazione è un processo che mira a gestire e controllare i cambiamenti apportati
        a un prodotto software o a un sistema durante il suo ciclo di vita. La gestione della configurazione per
        la documentazione descrive come vengono identificate, controllate, tracciate e gestite le versioni di un
        documento.
        \subsubsection{Versionamento}
        Ogni versione del documento è identificata da un codice di versione nel formato \textbf{Z.Y.X} dove:
        \begin{itemize}
            \item \textbf{Z}: il documento è pronto per una delle revisioni, ovvero tutte le modifiche precedenti sono state approvate dal responsabile;
            \item \textbf{Y}: la sezione aggiunta o modificata di un documento è stata verificata;
            \item \textbf{X}: è stata corretta velocemente qualche incoerenza o errore minore.
        \end{itemize}

    \subsection{Accertamento della qualità}
        \subsubsection{Metriche}
        Per perseguire la qualità nel processo di accertamento di qualità si `e deciso di adottare le seguenti metriche:
        \begin{itemize}
                \item \nameref{M:MM};
        \end{itemize}
    \subsection{Verifica\textsubscript{g}}
        \subsubsection{Scopo}
        Lo scopo del processo di verifica\textsubscript{g} è quello di accertare che non siano stati commessi errori nello svolgimento
        delle attività prefissate. Questo processo viene applicato costantemente sia durante la stesura
        della documentazione, che durante lo sviluppo del software.
        \subsubsection{Verifica\textsubscript{g} della documentazione}
        Al momento della necessità di modificare o aggiungere qualcosa a una sezione di un documento, si procederà in questo modo:
        \begin{enumerate}
            \item Verrà creata una issue che specifica l'attività di modifica o aggiunta da svolgere;
            \item Verrà creato un sotto-branch\textsubscript{g} chiamato 'docs/\textit{nome\_del\_documento}-N' dove con N si intende il numero della issue di riferimento;
            \item Rimanendo in questo sotto-branch\textsubscript{g}, verrà aggiornato il documento;
            \item Una volta terminato, per iniziare il processo di verifica\textsubscript{g} verrà aperta una Pull Request\textsubscript{g};
            \item Uno tra i verificatori in carica durante quello sprint si assegnerà la verifica del documento e modificherà il registro delle modifiche compilando il campo 'Verificatore';
            \item Se la Pull Request\textsubscript{g} avrà esito positivo, ci sarà il merge tra il sotto-branch\textsubscript{g} e il branch\textsubscript{g} principale del documento, con conseguente chiusura della issue e del sotto-branch;
            \item In caso di esito negativo, verranno segnalati gli errori da parte del verificatore tramite commenti, e il redattore si occuperà di risolverli continuando a lavorare nel sotto-branch\textsubscript{g}, fino ad avere il documento verificato.
        \end{enumerate}

        \subsubsection{Verifica\textsubscript{g} del codice}

        \subsubsection{Metriche}
        Per perseguire la qualità nel processo di verifica di qualità si `e deciso di adottare le seguenti metriche:
        \begin{itemize}
                \item \nameref{M:COC};
        \end{itemize}
    
    \subsection{Validazione}
        \subsubsection{Scopo}
        Per la documentazione la validazione corrisponde alla pubblicazione del documento nel branch\textsubscript{g} main.\\
        Questo processo avviene dopo l'ultima verifica\textsubscript{g} e sarà il responsabile a stabilire se il prodotto è accettabile o ha bisogno di
        ulteriori verifiche.
