\newcounter{M}

\newcommand{\MPCQ}[1]{
    \stepcounter{M}
    \subsubsection{MPCQ\arabic{M} - #1}
    }
    \newcommand{\MPDQ}[1]{
    \stepcounter{M}
    \subsubsection{MPDQ\arabic{M} - #1}
}

\section{Metriche di qualità}
\subsection{Introduzione}
Le metriche di qualità sono strumenti fondamentali per
valutare e migliorare l’efficacia e l’efficienza nello sviluppo del software. Queste metriche
forniscono indicatori oggettivi e misurabili che consentono di valutare la conformità agli
standard, identificare aree di miglioramento e monitorare la salute complessiva del processo
di sviluppo.
\subsection*{Codifica}
\begin{center}
    \textbf{M[Tipologia][Id numerico]}
\end{center}
dove:
\begin{itemize}
    \item \textbf{Tipologia}: indica il tipo della metrica:
    \begin{itemize}
        \item \textbf{PCQ}: per il processo;
        \item \textbf{PDQ}: per il prodotto.
    \end{itemize}
    \item \textbf{Id numerico}: indica un numero univoco incrementale separato per le due tipologie.
\end{itemize}

\subsection{Metriche per la qualità di processo}
Di seguito sono descritte le metriche di qualità di processo che il gruppo intende adottare.
\MPCQ{Planned Value (PV)}
    Costo pianificato (in \euro) per realizzare le attività di progetto alla data corrente.
    \begin{itemize}
        \item \textbf{Formula}: $PV = TPB * \frac{PH}{THP}$
        \item \textbf{Valore minimo}: $\geq0$ \euro
        \item \textbf{Valore ottimo}: $\leq$ Buget at completion (BAC)
    \end{itemize}  
    Per eseguire il calcolo:
    \begin{itemize}
        \item \textbf{TPB} : il budget totale preventivato;
        \item \textbf{PH}: ore di lavoro pianificate;
        \item \textbf{THP}: ore di lavoro pianificate totali.
    \end{itemize}

\MPCQ{Actual Cost (AC)}
Costo effettivamente sostenuto (in €) alla data corrente. 
\begin{itemize}
    \item \textbf{Formula}: $AC = \sum_{r}^{R}(THR_r*HC_r)$
    \item \textbf{Valore minimo}: $\geq0$ \euro
    \item \textbf{Valore ottimo}: $\leq$ EAC
\end{itemize}  
Per eseguire il calcolo:
\begin{itemize}
    \item \textbf{R}: insieme dei ruoli;
    \item \textbf{THR}: ore di lavoro effettive per ruolo;
    \item \textbf{HC}: costo fisso ad ora per il determinato ruolo.
\end{itemize}

\MPCQ{Estimated at Completion (EAC)}
Revisione del costo (in \euro) stimato per la realizzazione del progetto alla data corrente. 
\begin{itemize}
    \item \textbf{Formula}:$EAC = AC*ETC$
    \item \textbf{Valore minimo}: preventivo - 5\% $\leq$ EAC $\leq$  preventivo + 5\%
    \item \textbf{Valore ottimo}: Costo preventivato
\end{itemize}  
Per eseguire il calcolo:
\begin{itemize}
    \item \textbf{\%}: percentuali del costo preventivato.
\end{itemize}

\MPCQ{Earned Value (EV)}
Valore (in \euro) delle attività realizzate nel progetto fino alla data corrente. 
\begin{itemize}
    \item \textbf{Formula}: $EV=TPB*\frac{EH}{THP}$
    \item \textbf{Valore minimo}: $\geq0$ \euro
    \item \textbf{Valore ottimo}: $\leq$ EAC
\end{itemize}  
Per eseguire il calcolo:
\begin{itemize}
    \item \textbf{TPB} : il budget totale preventivato;
    \item \textbf{EH}: ore di lavoro effettive;
    \item \textbf{THP}: ore di lavoro pianificate totali.
\end{itemize}

\MPCQ{Estimated to Complete (ETC)}
Valore (in \euro) stimato per la realizzazione delle rimanenti attività necessarie al completamento del progetto. 
\begin{itemize}
    \item \textbf{Formula}: $ETC=TPB*EV$
    \item \textbf{Valore minimo}: $\geq0$ \euro
    \item \textbf{Valore ottimo}: $\leq$ EAC
\end{itemize}  
Per eseguire il calcolo:
\begin{itemize}
    \item \textbf{TPB} : il budget totale preventivato;
    \item \textbf{EV}: Earned Value.
\end{itemize}

\MPCQ{Cost Variance (CV)}
Valore che indica se del costo realmente maturato è maggiore, uguale o minore rispetto al costo effettivo.
Se CV > 0 significa che il progetto produce con maggior efficienza (minor costo) rispetto a quanto pianificato, viceversa se negativo. 
\begin{itemize}
    \item \textbf{Formula}: $CV=EV-AC$
    \item \textbf{Valore minimo}: $=0$ \euro
    \item \textbf{Valore ottimo}: $\leq$ EAC
\end{itemize}  
Per eseguire il calcolo:
\begin{itemize}
    \item \textbf{AC} : Actual Cost;
    \item \textbf{EV}: Earned Value.
\end{itemize}

\MPCQ{Schedule Variance (SV)}
Percentuale che indica se si è in linea, in anticipo o in ritardo rispetto alla schedulazione delle attività di progetto pianificate nella baseline. 
Se SV > 0 significa che il progetto sta producendo con maggior velocità a quanto pianificato, viceversa se negativo. 
\begin{itemize}
    \item \textbf{Formula}: $SV=\frac{EV-PV}{PV}*100$
    \item \textbf{Valore minimo}: $\geq$ -15\%
    \item \textbf{Valore ottimo}: 0\%
\end{itemize}  
Per eseguire il calcolo:
\begin{itemize}
    \item \textbf{AC} : Actual Cost;
    \item \textbf{PV}: Planned Value.
\end{itemize}

\MPCQ{Budget Variance (BV)}
Percentuale che indica se alla data corrente si è speso di più o di meno rispetto a quanto previsto a budget alla data corrente. 
Se BV > 0 significa che il progetto sta spendendo il proprio budget con maggior velocità di quanto pianificato, viceversa se negativo.
\begin{itemize}
    \item \textbf{Formula}: $BV=\frac{PV-AC}{PV}*100$
    \item \textbf{Valore minimo}: $\leq10$\%
    \item \textbf{Valore ottimo}: $=0$ \%
\end{itemize}  
Per eseguire il calcolo:
\begin{itemize}
    \item \textbf{AC} : Actual Cost;
    \item \textbf{PV}: Planned Value.
\end{itemize}

\MPCQ{Requirements Stability Index (RSI)}
Indice che traccia la variazione dei requisiti dell'arco del progetto
\begin{itemize}
    \item \textbf{Formula}: $RSI = 1 - \frac{CRN+DRN+ARN}{IR}*100$
    \item \textbf{Valore minimo}: 70\%
    \item \textbf{Valore ottimo}: 100\%
\end{itemize}  
Per eseguire il calcolo:
\begin{itemize}
    \item \textbf{CRN} : numero di requisiti cambiati;
    \item \textbf{DRN}: numero di requisiti eliminati;
    \item \textbf{ARN}: numero di requisiti aggiunti;
    \item \textbf{IR}: numero totale di requisiti iniziali.
\end{itemize}

\MPCQ{Gulpease Index (GI)}
Indice di leggibilità di un testo tarato sulla lingua italiana. I risultati sono compresi tra 0 e
100, dove il valore 100 indica la leggibilità più alta e 0 la leggibilità più bassa. Ai seguenti valori si
associano i seguenti significati:
\begin{itemize}
    \item inferiore a 80 sono difficili da leggere per chi ha la licenza elementare;
    \item inferiore a 60 sono difficili da leggere per chi ha la licenza media;
    \item inferiore a 40 sono difficili da leggere per chi ha un diploma superiore.
\end{itemize}
\begin{itemize}
    \item \textbf{Formula}: $GI=89+\frac{300*PN-10*LN}{WN}$
    \item \textbf{Valore minimo} $\geq50$
    \item \textbf{Valore ottimo}: $\geq70$
\end{itemize}  
Per eseguire il calcolo:
\begin{itemize}
    \item \textbf{PN} : numero di frasi;
    \item \textbf{LN}: numero di lettere;
    \item \textbf{WN}: numero di parole.
\end{itemize}

\MPCQ{Metrics Met (MM)}
Indice percentuale per tenere conto delle metriche soddisfatte. Una
metrica si dice soddisfatta se raggiunge almeno il valore minimo imposto sul \textit{Piano di Qualifica}.
\begin{itemize}
    \item \textbf{Valore minimo}:$\geq70$
    \item \textbf{Valore ottimo}:$\geq90$
\end{itemize}  

\MPCQ{Code Coverage (COC)}
Percentuale di linee di codice del progetto che sono state eseguite dai test dopo
un’esecuzione.
\begin{itemize}
    \item \textbf{Valore minimo}:$\geq80$
    \item \textbf{Valore ottimo}:$\geq95$
\end{itemize}  


\MPCQ{Unexpected Risks (UR)}
Valore intero che indica il numero di rischi non previsti durante il corso del progetto.
\begin{itemize}
    \item \textbf{Valore minimo}: $\geq0$
    \item \textbf{Valore ottimo}: $=0$
\end{itemize}  


\pagebreak
\setcounter{M}{0}
\subsection{Metriche per la qualità di prodotto}
Di seguito sono descritte le metriche di qualità di prodotto che il gruppo intende adottare.

\MPDQ{Met Requirements Percent (MRP)}
Valore percentuale che serve per indicare i requisiti soddisfatti.
\begin{itemize}
    \item \textbf{Valore minimo}: 100\% dei requisiti obbligatori;
    \item \textbf{Valore ottimo}: 100\% di tutti i requisiti (anche quelli facoltativi)
\end{itemize}  

\MPDQ{Fault Density(FD)}
Percentuale di failure o di esecuzioni non andate a buon fine di determinate azioni. Le
eventuali esecuzioni fallite o failure saranno segnate dai programmatori e di conseguenza verrà calcolato
il valore della metrica.
\begin{itemize}
    \item \textbf{Formula}: $FD=\frac{FTN}{TN}*100$
    \item \textbf{Valore minimo}: 20\%
    \item \textbf{Valore ottimo}: 10\%
\end{itemize}  
Per eseguire il calcolo:
\begin{itemize}
    \item \textbf{FTN} : numero di test falliti;
    \item \textbf{TN}: numero di test eseguiti in totale.
\end{itemize}


% CHIEDERE AL PROPONENTE
\MPDQ{Average Response Time (ART)}
Tempo medio impiegato dal software per rispondere a una richiesta utente o svolgere un’attività di sistema. 
\begin{itemize}
    \item \textbf{Valore minimo}:
    \item \textbf{Valore ottimo}:
\end{itemize}  


% CHIEDERE AL PROPONENTE
\MPDQ{Loading Time (LT)}
Tempo medio di attesa per il caricamento del sito.
\begin{itemize}
    \item \textbf{Valore minimo}:
    \item \textbf{Valore ottimo}:
\end{itemize}  


\MPDQ{Ease Of Learning (EOL)}
Misura basata sul tempo che rapprresenta quando, in media, un' utente di impiega per imparare come si utilizza il programma.
\begin{itemize}
    \item \textbf{Valore minimo}: 10 minuti;
    \item \textbf{Valore ottimo}: 5 minuti.
\end{itemize}  

\MPDQ{Cyclomatic Complexity (CYC)}
Metrica utilizzata per misurare la complessità di un programma. Calcolata sul grafo dei cammini linearmente indipendenti percorsi dal software ed i punti decisionali del programma.
\begin{itemize}
    \item \textbf{Formula}: $v(G)=\exp-n+2p$
    \item \textbf{Valore minimo}: $\leq$ 10
    \item \textbf{Valore ottimo}: $\leq$ 4
\end{itemize}  
Per eseguire il calcolo:
\begin{itemize}
    \item \textbf{$\exp$} : numero di archi nel grafo;
    \item \textbf{n}: numero di punti decisionali (nodi) nel grafo;
    \item \textbf{p}: numero di componenti connesse tra loro.
\end{itemize}

\MPDQ{Comment Density (CD)}
Percentuale delle righe di commento sul totale delle righe di codice presenti in un modulo.
\begin{itemize}
    \item \textbf{Formula}: $CD=\frac{CMLN}{CLN}*100$
    \item \textbf{Valore minimo}: 10\%
    \item \textbf{Valore ottimo}: 20\%
\end{itemize}  
Per eseguire il calcolo:
\begin{itemize}
    \item \textbf{CMLN} : numero di righe di commento;
    \item \textbf{CLN}: numero di righe di codice.
\end{itemize}

% CHIEDERE AL PROPONENTE
\MPDQ{Supported Browser(SB)}
Valore percentuale dei browser e loro relativa versione che supportano dal prodotto software.
\begin{itemize}
    \item \textbf{Formula}: $SB=\frac{BN}{TBN}*100$
    \item \textbf{Valore minimo}: 80\%
    \item \textbf{Valore ottimo}: 100\%
\end{itemize}  
Per eseguire il calcolo:
\begin{itemize}
    \item \textbf{BN} : numero di browser in cui il software funziona;
    \item \textbf{TBN}: numero di browser testati.
\end{itemize}

