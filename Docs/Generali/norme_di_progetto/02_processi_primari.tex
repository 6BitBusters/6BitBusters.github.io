\section{Processi Primari}
I processi primari saranno presenti durante l'intero ciclo di vita del prodotto, 
considerato che il progetto assegnato è da considerarsi didattico, i processi 
primari non saranno considerati nella loro totalità ma solamente un sottoinsieme.\\
Infatti non ci dovremo occupare di installazione e manutenzione, ci fermeremo allo sviluppo.

    \subsection{Fornitura}
        \subsubsection{Descrizione}
        La seguente sezione descrive le regole che il team si impegna a seguire 
        per instaurare e mantenere una collaborazione proficua e costruttiva con il proponente, 
        \textit{Sanmarco Informatica.}

        \subsubsection{Scopo}
        Il processo di fornitura, nei processi primari, si occupa di gestire le relazioni 
        con il cliente o committente, assicurando che i requisiti del progetto siano 
        compresi, rispettati e soddisfatti.

        \subsubsection{Rapporto con il proponente}
        Per garantire un allineamento costante con le aspettative del proponente ed evitare eventuali incomprensioni,
        il gruppo \textit{Six Bit Busters} si impegna a mantenere il contatto regolare con il proponente.
        A tal fine verranno organizzati incontri sincroni con cadenza bisettimanale tramite la piattaforma Google Meet. 
        Inoltre sarà disponibile un canale di comunicazione asincrona attraverso Google Chat oppure via e-mail.\\
        
        Le discussioni con il proponente si concentreranno principalmente sui seguenti aspetti:
        
        \begin{itemize}
            \item Aggiornamento sullo stato del progetto;
            \item Raccolta di feedback sul lavoro completato;
            \item Risoluzione di eventuali problematiche incontrate dal team;
            \item Orientamento sulle tecnologie più adeguate al progetto;
            \item Dettaglio sui requisiti funzionali e non funzionali che il prodotto dovrà soddisfare;
            \item Assegnazione delle priorità alle funzionalità;
            \item Valutazione di proposte alternative;
        \end{itemize}

        \subsubsection{Metriche}
        Per perseguire la qualità nel processo di fornitura si è deciso di adottare le seguenti metriche:
       \begin{itemize}
            \item \nameref{M:PV};
            \item \nameref{M:AC};
            \item \nameref{M:EV};
            \item \nameref{M:EAC};
            \item \nameref{M:ETC};
            \item \nameref{M:CV};
            \item \nameref{M:SV};
            \item \nameref{M:BV};
       \end{itemize}

    \subsection{Sviluppo}
        \subsubsection{Scopo}
        Lo sviluppo rappresenta uno degli elementi centrali nella produzione di un software.
        Il suo scopo è quello di trasformare i requisiti raccolti in un 
        prodotto software che soddisfi le aspettative del 
        proponente. Mira a garantire una realizzazione progressiva delle 
        funzionalità richieste e a mantenere uno standard di qualità e di
        manutenibilità elevate attraverso test di verifica e di validazione.

        \subsubsection{Descrizione}
        Il processo di sviluppo è composto da tre principali attività che sono:
        \begin{itemize}
            \item Analisi dei requisiti;
            \item Progettazione;
            \item Codifica.
        \end{itemize}


        \subsubsection{Analisi dei Requisiti}
        L'analisi dei requisiti costituisce la fase iniziale dello sviluppo.
        Questa attività ha la funzione di: 

        \begin{itemize}
            \item Fornire una descrizione dettagliata del prodotto;
            \item Raccogliere i requisiti funzionali e non funzionali che devono essere soddisfatti dal software;
            \item Facilitare la successiva fase di progettazione;
            \item Facilitare il tracciamento dei requisiti;
            \item Offrire un riferimento chiaro per i verificatori durante le fasi di test e di validazione.
        \end{itemize}

        Questo processo si concretizza nel documento \textit{Analisi dei Requisiti}, 
        che contiene tutti i requisiti e i casi d'uso.
        
        \subsubsubsection{Casi d'uso}
        Definiscono uno scenario in cui uno o più attori interagiscono con il sistema. Sono
        identificati nel modo seguente:
        \textbf{
        \[
        UC[\text{Numero caso d'uso}].[\text{Sottocaso}] - [\text{Titolo caso d'uso}]
        \]
        }
        
        \subsubsubsection{Struttura dei requisiti} \vspace{1em}
         Il codice identificativo di ciascun requisito è strutturato come segue:
        \textbf{
        \[
        R[\text{Tipologia}].[ \text{Codice}]
        \]
        }
        
        con: 
        \begin{itemize}
            \item \textbf{Tipologia:}
                \begin{itemize}
                    \item \textbf{F:} requisito funzionale (servizi e funzioni offerti dal sistema);
                    \item \textbf{Q:} requisito di qualità (vincoli di qualità necessari);
                    \item \textbf{V:} requisito di vincolo (vincoli sui servizi offerti dal sistema);
                    \item \textbf{P:} requisito prestazionale (vincoli sulle prestazioni da soddisfare).
                \end{itemize}
            \item \textbf{Codice:} 
                \par Indica l'identificativo del requisito univoco in base alla tipologia, il codice identificativo deve essere lo stesso del caso d'uso da cui il requisito ne deriva.            
        \end{itemize}
        \textbf{Importanza}: il livello di importanza sarà assegnato nella tabella relativa ai requisiti all'interno del documento \textit{Analisi dei Requisiti}, comprendendo due categorie:
        \begin{itemize}
            \item \textbf{Obbligatorio}: requisito obbligatorio per il soddisfacimento del progetto;
            \item \textbf{Desiderabile}: requisito non strettamente necessario, ma che se soddisfatto darebbe al prodotto un valore aggiunto riconoscibile;
            \item \textbf{Opzionale}: requisito preso in carico dopo il compimento di tutti i requisiti obbligatori.
        \end{itemize}
        
        \subsubsection{Progettazione}
        L'attività di progettazione è assegnata al ruolo \textit{progettista} sotto le linee guida indicate nell'\textit{Analisi dei requisiti}. L'obiettivo finale è la definizione di un'architettura di sistema capace di soddisfare i requisiti dell'analisi, 
        attraverso la creazione iniziale di un Proof of Concept (PoC) per la Requirements and Technology Baseline e, successivamente, realizzando una descrizione più dettagliata per la Product Baseline.
        \\La progettazione è formata da due parti:
        \begin{itemize}
            \item \textbf{Progettazione logica}: motiva e giustifica le scelte adottate per la realizzazione del prodotto, dimostrandone l’adeguatezza nel PoC.
            \\Contiene:
            \renewcommand{\labelitemii}{-}
            \begin{itemize}
                \item Framework, librerie e tecnologie utilizzate;
                \item Proof of Concept;
                \item Diagrammi UML.
            \end{itemize}
            \item \textbf{Progettazione di dettaglio}: descrive la struttura architetturale del prodotto in linea con quanto stabilito nella progettazione logica.\\
            Contiene:
            \renewcommand{\labelitemii}{-}
            \begin{itemize}
                \item Diagrammi delle classi;
                \item Tracciamento delle classi;
                \item Unit test per ogni componente.
            \end{itemize}
        \end{itemize}
        
        \subsubsection{Codifica}
        L'attività di codifica è assegnata al ruolo di \textit{programmatore}. Lo scopo della codifica è quello di implementare le specifiche individuate, seguendo quanto descritto nel diagramma delle classi, al fine di realizzare un prodotto utilizzabile.
        \\La codifica dovrà rispettare delle norme ben definite per garantire la leggibilità e la manutenibilità del codice.

        \subsubsection{Metriche}
        Per perseguire la qualità nel processo di sviluppo si è deciso di adottare le seguenti metriche:
        \begin{itemize}
                \item \nameref{M:RSI};
                \item \nameref{M:MR};
                \item \nameref{M:DR};  
                \item \nameref{M:OR};  
        \end{itemize}






