\section{Processi Primari}
I processi primari saranno presenti durante l'intero ciclo di vita del prodotto, 
considerato che il progetto assegnato è da considerarsi didattico, i processi 
primari non saranno considerati nella loro totalità ma solamente un sottoinsieme.\\
Infatti non ci dovremo occupare di installazione e manutenzione, ci fermeremo allo sviluppo.

    \subsection{Fornitura}
        \subsubsection{Descrizione}
        La seguente sezione descrive le regole che il team si impegna a seguire 
        per instaurare e mantenere una collaborazione proficua e costruttiva con il proponente, 
        \textit{Sanmarco Informatica.}

        \subsubsection{Scopo}
        Il processo di fornitura, nei processi primari, si occupa di gestire le relazioni 
        con il cliente o committente, assicurando che i requisiti del progetto siano 
        compresi, rispettati e soddisfatti.

        \subsubsection{Rapporto con il proponente}
        Per garantire un allineamento costante con le aspettative del proponente ed evitare eventuali incomprensioni,
        il gruppo \textit{Six Bit Busters} si impegna a mantenere il contatto regolare con il proponente.
        A tal fine verranno organizzati incontri sincroni con cadenza bisettimanale tramite la piattaforma Google Meet. 
        In oltre sarà disponibile un canale di comunicazione asincrona attraverso Google Chat oppure via e-mail.\\


        Le discussioni con il proponente si concentreranno principalmente sui seguenti aspetti:

        \begin{itemize}
            \item Aggiornamento sullo stato del progetto;
            \item Raccolta di feedback sul lavoro completato;
            \item Risoluzione di eventuali problematiche incontrate dal team;
            \item Orientamento sulle tecnologie più adeguate al progetto;
            \item Dettaglio sui requisiti funzionali e non funzionali che il prodotto dovrà soddisfare;
            \item Assegnazione delle priorità alle funzionalità;
            \item Valutazione di proposte alternative;
        \end{itemize}

    \subsection{Sviluppo}
        \subsubsection{Scopo}
        Lo sviluppo rappresenta uno degli elementi centrali nella produzione di un software.
        Il suo scopo è quello di trasformare i requisiti raccolti in un 
        prodotto software che soddisfi le aspettative del 
        proponente. Mira a garantire una realizzazione progressiva delle 
        funzionalità richieste e a mantenere uno standard di qualità e di
        manutenibilità elevati attraverso test di verifica e di validazione.

        \subsubsection{Descrizione}
        Il processo di sviluppo è composto da tre principali attività che sono:
        \begin{itemize}
            \item Analisi dei requisiti;
            \item Progettazione;
            \item Codifica.
        \end{itemize}


        \subsubsection{Analisi dei Requisiti}
        L'analisi dei requisiti costituisce la fase iniziale dello sviluppo.
        Questa attività ha la funzione di: 

        \begin{itemize}
            \item Fornire una descrizione dettagliata del prodotto;
            \item Raccogliere i requisiti funzionali e non funzionali che devono essere soddisfatti dal software;
            \item Facilitare la successiva fase di progettazione;
            \item Facilitare il tracciamento dei requisiti;
            \item Offrire un riferimento chiaro per i verificatori durante le fasi di test e di validazione.
        \end{itemize}

        Questo processo si concretizza nel documento \textit{Analisi dei Requisiti}, 
        che contiene tutti i requisiti e i casi d'uso.
        
        \subsubsubsection{Casi d'uso}
        Definiscono uno scenario in cui uno o più attori interagiscono con il sistema. Sono
        identificati nel modo seguente:
        \textbf{
        \[
        UC[\text{Numero caso d'uso}].[\text{Sottocaso}] - [\text{Titolo caso d'uso}]
        \]
        }
        
        \subsubsubsection{Struttura dei requisiti} \vspace{1em}
         Il codice identificativo di ciascun requisito è strutturato come segue:
        \textbf{
        \[
        R[\text{Tipologia}].[ \text{Importanza}].[ \text{Codice}]
        \]
        }
        
        con: 
        \begin{itemize}
            \item \textbf{Tipologia:}
                \begin{itemize}
                    \item \textbf{F:} requisito funzionale (servizi e funzioni offerti dal sistema);
                    \item \textbf{Q:} requisito di qualità (vincoli di qualità necessari);
                    \item \textbf{V:} requisito di vincolo (vincoli sui servizi offerti dal sistema);
                    \item \textbf{P:} requisito prestazionale (vincoli sulle prestazioni da soddisfare).
                \end{itemize}
                
            \item \textbf{Importanza:}
                \begin{itemize}
                    \item \textbf{1:} requisito obbligatorio;
                    \item \textbf{2:} requisito desiderabile ma non obbligatorio;
                    \item \textbf{3:} requisito opzionale.
                \end{itemize}
            \item \textbf{Codice:} 
                \par Indica l'identificativo del requisito univoco in base alla tipologia.
            
        \end{itemize}
        
        

    
        \subsubsection{Progettazione}
        \subsubsection{Codifica}



