\section{Processi Primari}
I processi primari accompagnano l'intero ciclo di vita del prodotto. 
Tuttavia, considerando che il progetto assegnato ha finalità didattiche, saranno analizzati solo un sottoinsieme di tali processi. 
In particolare, non saranno trattati i processi di installazione e manutenzione.

\subsection{Fornitura}
\subsubsection{Descrizione}
La seguente sezione descrive le regole che il team si impegna a seguire per
instaurare e mantenere una collaborazione proficua e costruttiva con il
proponente, \textit{Sanmarco Informatica S.p.A.}

\subsubsection{Scopo}
Il processo di fornitura si occupa di gestire le
relazioni con il cliente o committente, assicurando che i requisiti del
progetto siano compresi, rispettati e soddisfatti.

\subsubsection{Rapporto con il proponente}
Per garantire un allineamento costante con le aspettative del proponente ed
evitare eventuali incomprensioni, il gruppo \textit{Six Bit Busters} si impegna
a mantenere un contatto regolare con il proponente. A tal fine verranno
organizzati incontri sincroni con cadenza bisettimanale tramite la piattaforma
Google Meet. Inoltre sarà disponibile un canale di comunicazione asincrona
attraverso Google Chat oppure via email.\\

Le discussioni con il proponente si concentreranno principalmente sui seguenti
aspetti:

\begin{itemize}
    \item Aggiornamento sullo stato del progetto;
    \item Raccolta di feedback sul lavoro completato;
    \item Risoluzione di eventuali problematiche incontrate dal team;
    \item Orientamento sulle tecnologie più adeguate al progetto;
    \item Dettagli sui requisiti funzionali e non funzionali che il prodotto dovrà
          soddisfare;
    \item Assegnazione delle priorità alle funzionalità;
    \item Valutazione di proposte alternative.
\end{itemize}

\subsubsection{Metriche}
Per perseguire la qualità nel processo di fornitura, si è deciso di adottare le
seguenti metriche:
\begin{itemize}
    \item \nameref{M:PV};
    \item \nameref{M:AC};
    \item \nameref{M:EV};
    \item \nameref{M:EAC};
    \item \nameref{M:ETC};
    \item \nameref{M:CV};
    \item \nameref{M:SV};
    \item \nameref{M:BV}.
\end{itemize}

\subsection{Sviluppo}
\subsubsection{Scopo}
Lo sviluppo rappresenta uno degli elementi centrali nella produzione di un
software. Il suo scopo è quello di trasformare i requisiti raccolti in un
prodotto software che soddisfi le aspettative del proponente. Esso mira a garantire
una realizzazione progressiva delle funzionalità richieste e a mantenere 
standard di qualità e di manutenibilità elevati attraverso test di verifica e
validazione.

\subsubsection{Descrizione}
Il processo di sviluppo è composto da tre principali attività:
\begin{itemize}
    \item Analisi dei requisiti;
    \item Progettazione;
    \item Codifica.
\end{itemize}

\subsubsection{Analisi dei requisiti}
L'analisi dei requisiti costituisce la fase iniziale dello sviluppo. Questa
attività ha la funzione di:

\begin{itemize}
    \item Fornire una descrizione dettagliata del prodotto;
    \item Raccogliere i requisiti funzionali e non funzionali che devono essere
          soddisfatti dal software;
    \item Facilitare la successiva fase di progettazione;
    \item Facilitare il tracciamento dei requisiti;
    \item Offrire un riferimento chiaro per i verificatori durante le fasi di test e
          validazione.
\end{itemize}

Questo processo si concretizza nel documento \textit{Analisi dei requisiti},
che include:
\begin{itemize}
    \item Introduzione;
    \item Casi d'uso;
    \item Requisiti.
\end{itemize}
\subsubsubsection{Casi d'uso}\label{inf:UC}
I casi d'uso definiscono uno scenario in cui uno o più attori interagiscono con il sistema. Ogni caso d'uso è
identificato nel modo seguente:
\textbf{
\[
    UC[\text{Numero caso d'uso}].[\text{Sottocaso}] - [\text{Titolo caso d'uso}]
\]
}
Ogni caso d'uso, oltre all'identificativo, deve contenere:
\begin{itemize}
    \item \textbf{Diagramma UML}: rappresentazione grafica del caso d'uso. Non è necessario un diagramma per ogni caso d'uso,
          ma ogni caso d'uso deve essere rappresentato in almeno un diagramma;
    \item \textbf{Attore primario}: entità esterna al sistema che interagisce con esso per raggiungere un obiettivo;
    \item \textbf{Descrizione}: breve descrizione della funzionalità;
    \item \textbf{Precondizioni}: condizioni che devono essere verificate affinché la funzionalità sia disponibile;
    \item \textbf{Postcondizioni}: condizioni che devono essere verificate al termine dello scenario principale;
    \item \textbf{Scenario principale}: sequenza di interazioni tra l'attore primario e il sistema;
    \item \textbf{Estensione}: eventuale scenario alternativo, in cui le postcondizioni possono non essere verificate.
\end{itemize}
\subsubsubsection{Requisiti}\label{inf:reqs}
I requisiti rappresentano specifiche dettagliate che definiscono funzionalità, prestazioni e vincoli che il
software deve soddisfare. Questi requisiti servono da riferimento per lo sviluppo, il testing e la valutazione,
garantendo che il prodotto risponda alle necessità degli utenti e agli obiettivi prefissati.\\
Ciascun caso d'uso è identificato come segue:
\textbf{
\[
    R[\text{Tipologia}].[ \text{Codice}]
\]
}

dove:
\begin{itemize}
    \item \textbf{Tipologia:}
          \begin{itemize}
              \item \textbf{F:} requisito funzionale;
              \item \textbf{Q:} requisito di qualità;
              \item \textbf{V:} requisito di vincolo;
              \item \textbf{P:} requisito prestazionale.
          \end{itemize}
    \item \textbf{Codice:}
          \par Indica l'identificativo del requisito, univoco per la tipologia. Un
          requisito funzionale deve avere come codice il numero del caso d'uso da cui
          deriva.
\end{itemize}
Ogni requisito, oltre all'identificativo, deve contenere:
\begin{itemize}
    \item \textbf{Descrizione}: descrizione chiara e completa del requisito;
    \item \textbf{Importanza}:
          \begin{itemize}
              \item \textbf{Obbligatorio}: requisito irrinunciabile per uno o più stakeholder;
              \item \textbf{Desiderabile}: requisito non strettamente necessario, ma che, se soddisfatto, darebbe al prodotto un valore aggiunto riconoscibile;
              \item \textbf{Opzionale}: requisito preso in carico dopo il compimento di tutti i requisiti obbligatori.
          \end{itemize}
    \item \textbf{Fonte}: indica la fonte da cui è stato ricavato il requisito.
\end{itemize}
\subsubsection{Progettazione}
L'attività di progettazione è assegnata al progettista sotto le linee guida
indicate nell'\textit{Analisi dei requisiti}. L'obiettivo finale è la
definizione di un'architettura di sistema capace di soddisfare i requisiti
dell'analisi, attraverso la creazione iniziale di un Proof of Concept (PoC) per
la Requirements and Technology Baseline e, successivamente, la realizzazione di una
specifica dettagliata per la Product Baseline. \\
La progettazione si compone di due parti:
\begin{itemize}
    \item \textbf{Progettazione logica}: motiva e giustifica le scelte adottate per la realizzazione del prodotto, 
    dimostrandone l'adeguatezza nel PoC. Comprende:
          \begin{itemize}
              \item Framework, librerie e tecnologie utilizzate;
              \item Proof of Concept;
              \item Diagrammi UML.
          \end{itemize}
    \item \textbf{Progettazione di dettaglio}: descrive la struttura architetturale del prodotto in linea con quanto 
    stabilito nella progettazione logica. Comprende:
          \begin{itemize}
              \item Diagrammi delle classi;
              \item Tracciamento delle classi;
              \item Test di unità per ogni componente.
          \end{itemize}
\end{itemize}

\subsubsection{Codifica}
L'attività di codifica è assegnata al programmatore. Lo scopo della codifica è quello di 
implementare le specifiche individuate, seguendo quanto descritto nel
diagramma delle classi, al fine di realizzare un prodotto utilizzabile. \\
La codifica dovrà rispettare delle norme ben definite per garantire la leggibilità
e la manutenibilità del codice.

\subsubsection{Metriche}
Per perseguire la qualità nel processo di sviluppo, si è deciso di adottare le
seguenti metriche:
\begin{itemize}
    \item \nameref{M:RSI};
    \item \nameref{M:MR};
    \item \nameref{M:DR};
    \item \nameref{M:OR}.
\end{itemize}

