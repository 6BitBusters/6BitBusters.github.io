\section{Standard per la qualità}

\subsection{ISO/IEC 12207:1995}
Per definire, misurare e valutare i processi da attuare nello svolgimento
del progetto, il gruppo adotta lo standard ISO/IEC 12207:1995, che
definisce un modello per i processi di ciclo di vita del software. 
Tale modello suddivide i processi in primari, di supporto e organizzativi.
Ogni processo è suddiviso in una serie di attività; ciascuna attività è 
ulteriormente suddivisa in una serie di task.
Nelle sezioni seguenti vengono descritti i processi definiti dallo standard, 
con le relative attività.

\subsubsection{Processi primari}
I processi primari vengono impiegati dalle parti primarie durante il ciclo di 
vita del software. Una parte è detta primaria se avvia o si occupa dello 
sviluppo, dell'operatività o della manutenzione di prodotti software. 
Queste parti sono l'acquirente, il fornitore, lo sviluppatore, l'operatore e il manutentore.

\subsubsubsection{Acquisizione}
Processo che contiene le attività dell'acquirente, ossia l'organizzazione che
acquista un prodotto o servizio software.
\begin{itemize}
    \item Avviamento;
    \item Preparazione della richiesta di proposta;
    \item Preparazione e aggiornamento del contratto;
    \item Controllo dei fornitori;
    \item Accettazione e completamento.
\end{itemize}

\subsubsubsection{Fornitura}
Processo che contiene le attività del fornitore, ossia l'organizzazione che
fornisce il prodotto o servizio software all'acquirente.
\begin{itemize}
    \item Avviamento;
    \item Preparazione della risposta;
    \item Contratto;
    \item Pianificazione;
    \item Esecuzione e controllo;
    \item Revisione e valutazione;
    \item Consegna e completamento.
\end{itemize}

\subsubsubsection{Sviluppo}
Processo che contiene le attività dello sviluppatore, ossia l'organizzazione che
definisce, progetta e implementa il prodotto software.
\begin{itemize}
    \item Implementazione dei processi;
    \item Analisi dei requisiti di sistema;
    \item Progettazione dell'architettura di sistema;
    \item Analisi dei requisiti software;
    \item Progettazione dell'architettura software;
    \item Progettazione dettagliata del software;
    \item Codifica e test del software;
    \item Integrazione software;
    \item Integrazione di sistema;
    \item Installazione del software;
    \item Supporto all'accettazione del software.
\end{itemize}

\subsubsubsection{Processo operativo}
Processo che contiene le attività dell'operatore, ossia l'organizzazione che
fornisce il servizio di gestione del sistema informatico nel suo ambiente 
operativo per i suoi utenti.
\begin{itemize}
    \item Implementazione dei processi;
    \item Test operativo;
    \item Operazioni di sistema;
    \item Supporto utente.
\end{itemize}

\subsubsubsection{Manutenzione}
Processo che contiene le attività del manutentore, ossia l'organizzazione che
fornisce il servizio di manutenzione del prodotto software. Gestisce quindi 
le modifiche al prodotto software per mantenerlo aggiornato e in condizioni operative. 
\begin{itemize}
    \item Implementazione dei processi;
    \item Analisi dei problemi e delle modifiche;
    \item Implementazione delle modifiche;
    \item Revisione/accettazione delle manutenzione;
    \item Migrazione;
    \item Ritiro del software.
\end{itemize}

\subsubsection{Processi di supporto}
Un processo di supporto sostiene un altro processo come parte integrante 
con uno scopo distinto, e contribuisce al successo e alla qualità del progetto software. 
Un processo di supporto è impiegato ed eseguito, se necessario, da un altro processo. 

\subsubsubsection{Documentazione}
Processo per registrare le informazioni prodotte da un'attività o
da un processo di ciclo di vita.
\begin{itemize}
    \item Implementazione del processo;
    \item Progettazione e sviluppo;
    \item Produzione;
    \item Manutenzione.
\end{itemize}

\subsubsubsection{Gestione della configurazione}
Processo che consiste nell'applicare procedure amministrative e tecniche
attraverso il ciclo di vita del software allo scopo di identificare e definire
gli item in un sistema; controllare modifiche e rilasci degli item;
registrare lo stato degli item e le richieste di modifica; assicurare la completezza,
la consistenza e la correttezza degli item.
\begin{itemize}
    \item Implementazione del processo;
    \item Identificazione della configurazione;
    \item Controllo della configurazione;
    \item Resoconto dello stato della configurazione;
    \item Valutazione della configurazione;
    \item Gestione dei rilasci e delle consegne.
\end{itemize}

\subsubsubsection{Accertamento della qualità}
Processi per fornire adeguate garanzie che i processi e i prodotti
siano conformi ai requisiti e aderiscano ai piani stabiliti.
L'accertamento della qualità può utilizzare i risultati di altri
processi di supporto, come la verifica e la validazione.
\begin{itemize}
    \item Implementazione del processo;
    \item Accertamento dei prodotti;
    \item Accertamento dei processi;
    \item Accertamento dei sistemi di qualità.
\end{itemize}

\subsubsubsection{Verifica}
Processo per determinare se i prodotti di una data attività soddisfano
i requisiti e le condizioni ad essi imposti nelle precedenti attività.
\begin{itemize}
    \item Implementazione del processo;
    \item Verifica.
\end{itemize}

\subsubsubsection{Validazione}
Processo per determinare se i requisiti e il sistema software
prodotto soddisfano la specifica destinazione d'uso.
\begin{itemize}
    \item Implementazione del processo;
    \item Validazione.
\end{itemize}

\subsubsubsection{Revisione congiunta}
Processo per valutare lo stato e i prodotti di un'attività, ove opportuno.
Le revisioni congiunte avvengono sia a livello di gestione di progetto che
a livello tecnico.
\begin{itemize}
    \item Implementazione del processo;
    \item Revisione della gestione di progetto;
    \item Revisione tecnica.
\end{itemize}

\subsubsubsection{Audit}
Processo per determinare la conformità con i requisiti, i piani 
e il contratto, ove opportuno.
\begin{itemize}
    \item Implementazione del processo;
    \item Audit.
\end{itemize}

\subsubsubsection{Risoluzione dei problemi}
Processo per analizzare e risolvere i problemi (incluse non conformità),
che si presentano durante l'esecuzione di altri processi. 
\begin{itemize}
    \item Implementazione del processo;
    \item Risoluzione dei problemi.
\end{itemize}

\subsubsection{Processi organizzativi}
I processi organizzativi vengono impiegati da un'organizzazione per stabilire 
e attuare una struttura di base costituita dai processi e dal personale,
e per migliorare costantemente tale struttura.

\subsubsubsection{Gestione organizzativa}
Processo che contiene le attività generiche che possono 
essere impiegate da qualunque parte che deve gestire i propri processi.
\begin{itemize}
    \item Avviamento e definizione dello scopo;
    \item Pianificazione;
    \item Esecuzione e controllo;
    \item Revisione e valutazione;
    \item Chiusura.
\end{itemize}

\subsubsubsection{Infrastruttura}
Processo per costruire e mantenere l'infrastruttura necessaria
agli altri processi.
\begin{itemize}
    \item Implementazione del processo;
    \item Costruzione dell'infrastruttura;
    \item Manutenzione dell'infrastruttura.
\end{itemize}

\subsubsubsection{Miglioramento}
Processo per definire, valutare, misurare, monitorare e migliorare
un processo di ciclo di vita.
\begin{itemize}
    \item Costituzione del processo;
    \item Valutazione del processo;
    \item Miglioramento del processo.
\end{itemize}

\subsubsubsection{Formazione}
Processo per fornire e mantenere personale adeguatamente formato.
\begin{itemize}
    \item Implementazione del processo;
    \item Sviluppo del materiale di formazione;
    \item Implementazione del piano di formazione.
\end{itemize}

\subsection{ISO/IEC 9126}
Per definire, misurare e valutare il software prodotto, il gruppo adotta lo 
standard ISO/IEC 9126, che definisce un modello di qualità del software. Tale 
modello è organizzato in sei caratteristiche principali, 
ognuna delle quali è suddivisa in sottocaratteristiche.

\subsubsection{Funzionalità}
La funzionalità misura la capacità del software di soddisfare i requisiti specificati e di fornire le funzioni,
espresse e implicite, necessarie per operare in un determinato contesto.
\begin{itemize}
    \item \textbf{Appropriatezza}: capacità di eseguire i compiti richiesti;
    \item \textbf{Accuratezza}: livello di correttezza dei risultati;
    \item \textbf{Interoperabilità}: capacità di interagire con uno o più sistemi specificati;
    \item \textbf{Sicurezza}: protezione delle informazioni e dei dati da accessi non autorizzati;
    \item \textbf{Conformità}: adesione a standard, convenzioni e regolamenti relativi alla funzionalità.
\end{itemize}

\subsubsection{Affidabilità}
L'affidabilità misura la capacità del software di svolgere correttamente il suo compito
mantenendo un certo livello di prestazioni quando usato in determinate condizioni.
\begin{itemize}
    \item \textbf{Maturità}: capacità di evitare errori, malfunzionamenti e guasti;
    \item \textbf{Tolleranza agli errori}: capacità di mantenere un certo livello di prestazioni anche in presenza di errori;
    \item \textbf{Recuperabilità}: capacità di ripristinare il livello di prestazioni e di recuperare
        i dati in caso di errori;
    \item \textbf{Conformità}: adesione a standard, convenzioni e regolamenti relativi all'affidabilità.
\end{itemize}

\subsubsection{Efficienza}
L'efficienza misura la capacità del software di offrire prestazioni adeguate in relazione alle risorse impiegate.
Le risorse includono altri prodotti software e la configurazione hardware e software del sistema.
\begin{itemize}
    \item \textbf{Comportamento rispetto al tempo}: velocità di risposta ed elaborazione;
    \item \textbf{Utilizzo delle risorse}: capacità di utilizzare un appropriato numero e tipo di risorse;
    \item \textbf{Conformità}: adesione a standard, convenzioni e regolamenti relativi all'efficienza.
\end{itemize}

\subsubsection{Usabilità}
L'usabilità misura la capacità del software di essere comprensibile, di poter essere
studiato e di risultare attraente per un utente sotto determinate condizioni.
\begin{itemize}
    \item \textbf{Comprensibilità}: capacità di essere compreso facilmente;
    \item \textbf{Apprendibilità}: facilità di apprendimento delle funzionalità;
    \item \textbf{Operabilità}: facilità di essere utilizzato e controllato;
    \item \textbf{Attrattività}: capacità di essere attraente per l'utente;
    \item \textbf{Conformità}: adesione a standard, convenzioni e regolamenti relativi all'usabilità.
\end{itemize}

\subsubsection{Manutenibilità}
La manutenibilità misura la capacità del software di essere adattato per 
correggere errori, implementare miglioramenti o rispondere a cambiamenti 
negli ambienti e nei requisiti.

\begin{itemize}
    \item \textbf{Analizzabilità}: facilità di diagnosticare problemi;
    \item \textbf{Modificabilità}: capacità di essere modificato;
    \item \textbf{Stabilità}: capacità di mantenere il funzionamento dopo modifiche;
    \item \textbf{Testabilità}: facilità di eseguire test sul software;
    \item \textbf{Conformità}: adesione a standard, convenzioni e regolamenti relativi alla manutenibilità.
\end{itemize}

\subsubsection{Portabilità}
La portabilità misura la capacità del software di essere trasferito da un ambiente di lavoro a un altro.
L'ambiente include aspetti organizzativi e tecnologici (hardware e software).
\begin{itemize}
    \item \textbf{Adattabilità}: capacità di adattarsi a diversi ambienti di lavoro;
    \item \textbf{Installabilità}: facilità di installazione;
    \item \textbf{Coesistenza}: capacità di coesistere con altri software;
    \item \textbf{Sostituibilità}: facilità con cui il software può sostituirne un altro, per lo 
        stesso scopo e nello stesso ambiente;
    \item \textbf{Conformità}: adesione a standard, convenzioni e regolamenti relativi alla portabilità.
\end{itemize}