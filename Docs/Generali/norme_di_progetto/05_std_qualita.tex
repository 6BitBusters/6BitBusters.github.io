\section{Standard per la qualità}

\subsection{ISO/IEC 12207:1995}
Per definire, misurare e valutare i processi da attuare nello svolgimento
del progetto, il gruppo adotta lo standard ISO/IEC 12207:1995, che
definisce un modello per i processi di ciclo di vita del software. 
Tale modello suddivide i processi in primari, di supporto e organizzativi.
Ogni processo consiste di più attività, e ogni attività è a sua volta
composta da più task.

\subsubsection{Processi primari}
The primary life cycle processes (clause 5) consist of five processes that serve primary parties during
the life cycle of software. A primary party is one that initiates or performs the development, operation,
or maintenance of software products. 

\subsubsubsection{Acquisizione}
\begin{itemize}
    \item Inizializzazione;
    \item Preparazione;
    \item Preparazione del contratto;
    \item Controllo del fornitore;
    \item Accettazione e completamento.
\end{itemize}

\subsubsubsection{Fornitura}
\begin{itemize}
    \item Inizializzazione;
    \item Preparazione del responso;
    \item Contratto;
    \item Pianificazione;
    \item Esecuzione e controllo;
    \item Revisione e valutazione;
    \item Consegna e completamento.
\end{itemize}

\subsubsubsection{Sviluppo}
\begin{itemize}
    \item Implementazione dei processi;
    \item Analisi dei requisiti di sistema;
    \item Progettazione dell'architettura di sistema;
    \item Analisi dei requisiti software;
    \item Progettazione dell'architettura software;   
    \item Progettazione dettagliata del software;
    \item Codifica e test del software;
    \item Integrazione software;
    \item Integrazione di sistema;
    \item Installazione del software;
    \item Supporto all'accettazione del software.
\end{itemize}

\subsubsubsection{Gestione operativa}
\begin{itemize}
    \item Implementazione dei processi;
    \item Test operativo;
    \item Operazioni di sistema;
    \item Supporto utente.
\end{itemize}

\subsubsubsection{Manutenzione}
\begin{itemize}
    \item Implementazione dei processi;
    \item Analisi dei problemi e delle modifiche;
    \item Implementazione delle modifiche;
    \item Revisione/accettazione delle manutenzione;
    \item Migrazione;
    \item Ritiro del software.
\end{itemize}
\subsubsection{Processi di supporto}
The supporting life cycle processes (clause 6) consist of eight processes. A supporting process
supports another process as an integral part with a distinct purpose and contributes to the success and
quality of the software project. A supporting process is employed and executed, as needed, by another
process. 

\subsubsection{Processi organizzativi}
The organizational life cycle processes (clause 7) consist of four processes. They are employed by an
organization to establish and implement an underlying structure made up of associated life cycle
processes and personnel and continuously improve the structure and processes. They are typically
employed outside the realm of specific projects and contracts; however, lessons from such projects and
contracts contribute to the improvement of the organization.

\subsection{ISO/IEC 9126}
Per definire, misurare e valutare il software prodotto, il gruppo adotta lo 
standard ISO/IEC 9126, che definisce un modello di qualità del software. Tale 
modello è organizzato in sei caratteristiche principali, 
ognuna delle quali è suddivisa in sotto-caratteristiche.

\subsubsection{Funzionalità}
La funzionalità misura la capacità del software di soddisfare i requisiti specificati e di fornire le funzioni,
espresse e implicite, necessarie per operare in un determinato contesto.
\begin{itemize}
    \item \textbf{Appropriatezza}: capacità di eseguire i compiti richiesti;
    \item \textbf{Accuratezza}: livello di correttezza dei risultati;
    \item \textbf{Interoperabilità}: capacità di interagire con uno o più sistemi specificati;
    \item \textbf{Sicurezza}: protezione delle informazioni e dei dati da accessi non autorizzati;
    \item \textbf{Conformità}: adesione a standard, convenzioni e regolamenti relativi alla funzionalità.
\end{itemize}

\subsubsection{Affidabilità}
L'affidabilità misura la capacità del software di svolgere correttamente il suo compito
mantenendo un certo livello di prestazioni quando usato in determinate condizioni.
\begin{itemize}
    \item \textbf{Maturità}: capacità di evitare errori, malfunzionamenti e guasti;
    \item \textbf{Tolleranza agli errori}: capacità di mantenere un certo livello di prestazioni anche in presenza di errori;
    \item \textbf{Recuperabilità}: capacità di ripristinare il livello di prestazioni e di recuperare
        i dati in caso di errori;
    \item \textbf{Conformità}: adesione a standard, convenzioni e regolamenti relativi all'affidabilità.
\end{itemize}

\subsubsection{Efficienza}
L'efficienza misura la capacità del software di fornire prestazioni appropriate rispetto alla quantità di risorse utilizzate.
Le risorse includono altri prodotti software e la configurazione hardware e software del sistema.
\begin{itemize}
    \item \textbf{Comportamento rispetto al tempo}: velocità di risposta ed elaborazione;
    \item \textbf{Utilizzo delle risorse}: capacità di utilizzare un appropriato numero e tipo di risorse;
    \item \textbf{Conformità}: adesione a standard, convenzioni e regolamenti relativi all'efficienza.
\end{itemize}

\subsubsection{Usabilità}
L'usabilità misura la capacità del software di essere comprensibile, di poter essere
studiato e di risultare attraente per un utente sotto determinate condizioni.
\begin{itemize}
    \item \textbf{Comprensibilità}: capacità di essere compreso facilmente;
    \item \textbf{Apprendibilità}: facilità di apprendimento delle funzionalità;
    \item \textbf{Operabilità}: facilità di essere utilizzato e controllato;
    \item \textbf{Attrattività}: capacità di essere attraente per l'utente;
    \item \textbf{Conformità}: adesione a standard, convenzioni e regolamenti relativi all'usabilità.
\end{itemize}

\subsubsection{Manutenibilità}
La manutenibilità misura la capacità del software di essere modificato.
Le modifiche possono includere correzioni, miglioramenti o adattamenti del software
a modifiche negli ambienti, nei requisiti e nelle specifiche funzionali.
\begin{itemize}
    \item \textbf{Analizzabilità}: facilità di diagnosticare problemi;
    \item \textbf{Modificabilità}: capacità di essere modificato;
    \item \textbf{Stabilità}: capacità di mantenere il funzionamento dopo modifiche;
    \item \textbf{Testabilità}: facilità di eseguire test sul software;
    \item \textbf{Conformità}: adesione a standard, convenzioni e regolamenti relativi alla manutenibilità.
\end{itemize}

\subsubsection{Portabilità}
La portabilità misura la capacità del software di essere trasferito da un ambiente di lavoro ad un altro.
L'ambiente include aspetti organizzativi e tecnologici (hardware e software).
\begin{itemize}
    \item \textbf{Adattabilità}: capacità di adattarsi a diversi ambienti di lavoro;
    \item \textbf{Installabilità}: facilità di installazione;
    \item \textbf{Coesistenza}: capacità di coesistere con altri software;
    \item \textbf{Sostituibilità}: facilità con cui il software può sostituirne un altro, per lo 
        stesso scopo e nello stesso ambiente;
    \item \textbf{Conformità}: adesione a standard, convenzioni e regolamenti relativi alla portabilità.
\end{itemize}