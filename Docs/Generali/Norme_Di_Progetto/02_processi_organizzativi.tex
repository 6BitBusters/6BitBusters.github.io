\section{Processi organizzativi}
Questa sezione mira a gestire i processi e il loro miglioramento, l'organizzazione degli strumenti
di supporto e la gestione del personale.

    \subsection{Pianificazione}
        \subsubsection{Metodo di Lavoro}
        % agile o altro
        \subsubsection{Ruoli e responsabilità}
            I membri team \textit{Six Bit Busters} ricopriranno i ruoli principali 
            di un ciclo di vita del prodotto software, ovvero analista, 
            progettista, programmatore, verificatore, amministratore di sistema e responsabile. 
            \\Al fine di garantire una comprensione completa delle diverse fasi 
            e competenze richieste nello sviluppo di un progetto, i membri del team 
            ruoteranno periodicamente tra i ruoli ogni due settimane. Questa rotazione 
            periodica è finalizzata a scopi didattici, permettendo a ciascun membro di 
            acquisire una visione globale del ciclo di vita del prodotto e di sviluppare 
            abilità pratiche in ogni area.
            \begin{enumerate}
                \item \textbf{Analista}\\
                Colui che dialoga con il proponente e analizza l'esposizione che esso fornisce del 
                problema, nel nostro caso il progetto didattico, l'analista si occuperà di 
                analizzare a fondo il capitolato per estrarne i requisiti, per poi raffinarli 
                tramite incontri diretti con il proponente, durante lo svolgimento di questo 
                processo verrà redatto il documento \textit{Analisi dei requisiti}.
                \item \textbf{Progettista}\\
                Colui che trasforma i requisiti in una soluzione, determinando una buona 
                architettura e definendo le scelte realizzative, prendono i requisiti 
                e li uniscono in una soluzione che sia la migliore possibile.
                \item \textbf{Programmatore}\\
                Colui che si occupa di realizzare il design presentato dal progettista, 
                ovvero scrivere il codice quando esisterà una soluzione nota per rispondere 
                ai requisiti.
                \item \textbf{Verificatore}\\
                Colui che mette in esame tutto ciò che viene fatto, dalla documentazione 
                al codice prodotto, controllando che non 
                ci siano errori o che qualcosa possa essere fatto meglio, con il supporto 
                di questa figura le attività potranno essere svolte secondo le attese.
                \item \textbf{Amministratore di sistema}\\
                Colui che agiste affinchè le attività siano agevolate da strumenti digitali, 
                che i prodotti siano persistenti e in luoghi disponibili, facilmente reperibili, definisce e 
                controlla l'ambiente di lavoro.
                \item \textbf{Responsabile}\\
                Colui che condensa tutte le voci rappresentando una sola voce per rappresentare il progetto 
                al cliente, governa il team coordinandolo e gestendo le risorse, pianifica e coordina 
                le relazioni esterne, nel nostro caso con il proponente.\\
            \end{enumerate}

            Per l'analisi dei ruoli e la rendicontazione delle ore preventivate si faccia riferimento al documento \textit{Dichiarazione degli impegni}.

    \subsection{Modalità di comunicazione}
        \subsubsection{Interne}
        Il team ha scelto come canale di comunicazione interno asincrono un canale Telegram,
        mentre per le riunioni periodiche sincrone, una a settimana, ha optato per 
        l'utilizzo di un server Discord.\\
        Le comunicazioni con il fornitore verranno invece gestite in modo asincrono 
        utilizzando le mail e Google Chat, e in modo sincrono, preriodicamente ogni 
        XXX settimane, tramite riunioni in Google Meet.
        \subsubsection{Esterne}
        % come di dialoga con il proponente

    \subsection{Modalità di riunione}
        \subsubsection{Interne}
        % come esprimiamo le nostre idee e ogni quanto ci vediamo
        \subsubsection{Esterne}
        % come esprimiamo le nostre idee e ogni quanto ci vediamo

    \subsection{Gestione di Infrastrutture}
        \subsubsection{Descrizione}
        In questa sezione sono riportate le norme relative alla gestione delle infrastrutture. Vengono stabiliti gli
        strumenti di cui il gruppo far`a uso e le relative regole di utilizzo.
        \subsubsection{GitHub}
        % spiegare come ci lavoriamo singolarmente e in modo asincrono
        \subsubsection{Overleaf}
        % spiegare come ci lavoriamo singolarmente, che uso ne facciamo o abbiamo fatto e com`e` organizzato
        \subsubsection{Repository}
        %  com`e` organizzato e un link al progetto
        \subsubsection{Branching}
        %  a cosa ci servono e come lavoriamo su di essi
        \subsubsection{Commits}
        %  modifiche che si apportano per commit, sintassi di messaggio e come iteragiscono e vengono gestiti
        \subsubsection{Pull requests}
        %  Come le utilizzamo per versionare il codice e come funzionano + sintassi di titolo desc e le sue proprieta` come label etc
        \subsubsection{Milestones}
        %  quali sono, quando, possibili motivi e come vengono create
        \subsubsection{Project board}
        %  com`e' organizzata la board e cosa ci contiene + workflow
        \subsubsection{Issue Tracking System}
        %  elenco di tutte le proprieta` o template di issue (lables, titolo, desc etc)

    \subsection{Gestione dei dubbi e conflitti}
        Nel caso sorgano dubbi, questi verranno risolti, in base all’urgenza, 
        tramite comunicazione nel canale Telegram o nella riunione interna 
        settimanale. \\Eventuali conflitti verranno discussi in sede di riunione 
        interna, con l’obiettivo di trovare una soluzione condivisa che consenta 
        al progetto di progredire secondo una visione comune.
