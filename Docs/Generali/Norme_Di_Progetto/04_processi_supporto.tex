\section{Processi di Supporto}
I processi di supporto supportano quelli primari in modo da renderli pi`u efficienti ed efficaci.
    \subsection{Documentazione}
        \subsubsection{Descrizione}
        \subsubsection{Ciclo di vita di un documento}
        \subsubsection{Struttura}
        % + tutte le convenzioni
        \subsubsection{Strumenti per la stesura}
        \subsubsection{Documentazione interna}
        \subsubsection{Documentazione esterna}

    \subsection{Gestione delle configurazione}
        \subsubsection{Scopo}
        La gestione della configurazione `e un processo che mira a gestire e controllare i cambiamenti apportati
        a un prodotto software o a un sistema durante il suo ciclo di vita. La gestione della configurazione per
        la documentazione descrive come vengono identificate, controllate, tracciate e gestite le versioni di un
        documento.
        \subsubsection{Versionamento}
        Ogni versione del documento `e identificata da un codice di versione nel formato \textbf{Z.Y.X} dove:
        \begin{enumerate}
            \item \textbf{Z}: s
            \item \textbf{Y}: s
            \item \textbf{X}: s
        \end{enumerate}

            \paragraph{Software per il versionamento}
            Per il versionamento il gruppo ha deciso di utilizzare un singolo repository GitHub sia per la documentazione che per il. Per informazioni
            più approfondite sulla struttura e la gestione del repository si veda sezione 4.2.3 GitHub

    \subsection{Accertamento della qualità}
    \subsection{Verifica}
        \subsubsection{Scopo}
        Lo scopo del processo di verifica `e quello di accertare che non siano stati commessi errori nello svolgimento
        delle attività prefissate. Questo processo viene applicato costantemente sia durante la stesura
        della documentazione, che durante lo sviluppo del software.
        \subsubsection{Verifica della documentazione}
        \begin{enumerate}
            \item   \textbf{Analisi statica}:questo tipo di analisi va a valutare la documentazione o il software senza bisogno
                                            di esecuzione. Si controlla che il prodotto sia corretto rispetto a determinate regole dettate da una checklist.
            \item   \textbf{Analisi dinamica}:questo tipo di analisi invece richiede esecuzione, e può essere
                                            applicata solo al codice. Si tratta di verificare se il comportamento del prodotto software durante
                                            la compilazione `e corretto.
        \end{enumerate}
        La verifica viene svolta da due verificatori prima del merge con il branch main all'apertura di
        una pull request su un branch riservato al documento. Il primo verificatore sarà quello che esaminerà il documento seguendo una checklist.
        Il documento verrà esaminato a partire dalle modifiche fatte dopo l'ultima versione verificata, deducibili
        tramite il registro delle modifiche.

        La verifica di un documento prevede di seguire la seguente checklist:
        \begin{enumerate}
            \item Lettura del documento iniziale
            \item Seconda lettura del documento per capirne meglio i concetti
            \item Valutare se il testo è scritto in maniera corretta rispetto all'ortografia e in caso segnalare le righe contenenti tali errori
            \item Valutare se il testo è organizzato in maniera opportuna secondo le \textit{Norme di Progetto}
            \item Valutare se ci sono parti mancanti e in caso segnalarle
            \item Valutare se tutti i termini citati del glossario hanno il pedice
            \item Valutare se il nome del documento rispetta le regole indicate in \textit{Norme di Progetto}
        \end{enumerate}
        In caso non ci sia nulla da segnalare e nel caso in cui l'analisi dinamica vada a buon fine il verificatore dovrà spuntare come approvata la Pull Request nella
        sezione dedicata su GitHub e quindi confermare il merge sul main e quindi si passerà alla compilazione ufficiale del documento che successivamente
        verrà pubblicata in automatico nella page del progetto.

        \subsubsection{Verifica del codice}
    
    \subsection{Validazione}
        \subsubsection{Scopo}
        Lo scopo del processo di validazione è quello di determinare se il prodotto finale sia conforme con le
        aspettative preposte e rispetti i requisiti minimi concordati con il proponente. Questo processo avviene
        dopo quello ultima verifica e sarà il responsabile a stabilire se il prodotto `e accettabile o ha bisogno di
        ulteriori verifiche.