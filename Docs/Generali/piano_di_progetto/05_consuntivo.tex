\section{Consuntivo}
\subsection{Introduzione}
Questa sezione riporta i dati raccolti durante il progetto riguardo alla ripartizione dei ruoli e alle ore impiegate da ogni componente del gruppo. Tali dati sono comparati alle previsioni presenti nella sezione di preventivo.

\subsection{Dettaglio Sprint}

\subsubsection{Sprint 1}
Di seguito la suddivisione dei ruoli e le ore di lavoro effettive impiegate in questo sprint:


\newcommand{\memberReportTable}[1]{

	\renewcommand{\arraystretch}{1.5}
	\rowcolors{2}{pari}{dispari}
	\begin{longtable}{ %0.87
		>{\centering}M{0.22\textwidth}
		>{\centering}M{0.08\textwidth}
		>{\centering}M{0.08\textwidth}
		>{\centering}M{0.08\textwidth}
		>{\centering}M{0.08\textwidth}
		>{\centering}M{0.08\textwidth}
		>{\centering}M{0.08\textwidth}
		>{\centering\arraybackslash}M{0.08\textwidth}
		}
		\rowcolorhead
		\headertitle{Membro} &
		\headertitle{Resp.}  &
		\headertitle{Amm.}   &
		\headertitle{An.}    &
		\headertitle{Proge.} &
		\headertitle{Progr.} &
		\headertitle{Ver.}   &
		\headertitle{TOT.}
		\endfirsthead
		\endhead

		#1

	\end{longtable}
	\vspace{1em}

}

\memberReportTable{
    Bergamin Elia       & 5 & - & - & - & - & - & \textbf{5}\tabularnewline
    Diviesti Filippo    & - & - & - & - & - & 6 & \textbf{6}\tabularnewline
    Djossa Edgar        & - & 5 & - & - & - & - & \textbf{5}\tabularnewline
    Chilese Elena       & - & - & 3 (-4) & - & - & - & \textbf{3 (-4)}\tabularnewline
    Pincin Matteo       & - & - & - & - & - & 6 & \textbf{6}\tabularnewline
    Soranzo Andrea      & - & 5 & - & - & - & - & \textbf{5}\tabularnewline
    \midrule[\heavyrulewidth]
    \textbf{TOTALE}     & 5 & 10 & 3 (-4) & - & - & 12 & \textbf{30 (-4)}\tabularnewline

    \rowcolor{white}\caption{Rendiconto effettivo della distribuzione delle ore per lo sprint 1}

}


I costi effettivi del periodo sono i seguenti:


\newcommand{\costReportTable}[1]{

	\renewcommand{\arraystretch}{1.5}
	\rowcolors{2}{pari}{dispari}
	\begin{longtable}{ %0.87
		>{\centering}M{0.30\textwidth}
		>{\centering}M{0.10\textwidth}
		>{\centering}M{0.10\textwidth}
		>{\centering}M{0.10\textwidth}
		>{\centering\arraybackslash}M{0.15\textwidth}
		}
		\rowcolorhead
		\headertitle{Ruolo}            &
		\centering
		\headertitle{Ore}              &
		\headertitle{Costo  (\euro/h)} &
		\headertitle{Costo Totale (\euro)}
		\endfirsthead
		\endhead

		#1

	\end{longtable}
	\vspace{1em}

}

\costReportTable{
    Responsabile & 5 & 30 & 150 \tabularnewline
    Amministratore & 10 & 20 & 200 \tabularnewline
    Analista & 3 (-4) & 25 & 75 (-100) \tabularnewline
    Progettista & - & 25 & - \tabularnewline
    Programmatore & - & 15 & - \tabularnewline
    Verificatore & 12 & 15 & 180 \tabularnewline
    \midrule[\heavyrulewidth]
    \textbf{Totale Consuntivo} & 30 & - & 605 \tabularnewline
    \midrule[\heavyrulewidth]
    \textbf{Totale Preventivo} & 34 & - & 705 \tabularnewline
    \midrule[\heavyrulewidth]
    \textbf{Differenza} & -4 & - & -100 \tabularnewline

    \rowcolor{white}\caption{Consuntivo costi sprint 1}

}


\subsubsubsection{Resoconto}
Nel seguente resoconto vengono analizzate le principali differenze tra le stime iniziali e le ore effettive durante lo sprint:
\begin{itemize}
    \item \textbf{Analista (-4 ore):} Alla fine sono state necessarie meno ore di lavoro da parte 
    dell'Analista rispetto al previsto. Questa variazione è legata, oltre che alla nostra inesperienza nel calcolo 
    accurato dei tempi necessari, principalmente ad impegni di altre materie universitarie.

\end{itemize}
Nel complesso, il lavoro svolto durante lo sprint è stato in linea con gli obiettivi prefissati, sebbene siano emerse alcune differenze rispetto alle stime iniziali.\\
Per gli sprint successivi, sarà necessario affinare la fase di preventivazione, valutando con 
maggiore precisione la complessità delle attività, così da ottenere stime ancora più accurate.\\



Di seguito le ore rimanenti ad ogni componente del gruppo relative ad ogni ruolo.

\newcommand{\remainingHoursTable}[1]{

	\renewcommand{\arraystretch}{1.5}
	\rowcolors{2}{pari}{dispari}
	\begin{longtable}{ %0.87
		>{\centering}M{0.22\textwidth}
		>{\centering}M{0.08\textwidth}
		>{\centering}M{0.08\textwidth}
		>{\centering}M{0.08\textwidth}
		>{\centering}M{0.08\textwidth}
		>{\centering}M{0.08\textwidth}
		>{\centering}M{0.08\textwidth}
		>{\centering\arraybackslash}M{0.08\textwidth}
		}
		\rowcolorhead
		\headertitle{Membro} &
		\headertitle{Resp.}  &
		\headertitle{Amm.}   &
		\headertitle{An.}    &
		\headertitle{Proge.} &
		\headertitle{Progr.} &
		\headertitle{Ver.}   
		\endfirsthead
		\endhead

		#1

	\end{longtable}
	\vspace{1em}

}
\remainingHoursTable{
    Bergamin Elia & 4 & 7 & 16 & 20 & 22 & 18 \tabularnewline
    Diviesti Filippo & 9 & 7 & 16 & 20 & 22 & 12 \tabularnewline
    Djossa Edgar & 9 & 2 & 16 & 20 & 22 & 18 \tabularnewline
    Chilese Elena & 9 & 7 & 13 & 20 & 22 & 18 \tabularnewline
    Pincin Matteo & 9 & 7 & 16 & 20 & 22 & 12 \tabularnewline
    Soranzo Andrea & 9 & 2 & 16 & 20 & 22 & 18\tabularnewline
    
    \rowcolor{white}\caption{Ore rimanenti per ogni ruolo dopo lo Sprint 1}
}

\subsubsection{Sprint 2}
Di seguito la suddivisione dei ruoli e le ore di lavoro effettive impiegate in questo sprint:


\memberReportTable{
    Bergamin Elia & - & - & - & - & - & 7 & \textbf{7}\tabularnewline
    Diviesti Filippo & - & - & 9 & - & - & - & \textbf{9}\tabularnewline
    Djossa Edgar & 5 & - & - & - & - & - & \textbf{5}\tabularnewline
    Chilese Elena & - & 7 & - & - & - & - & \textbf{7}\tabularnewline
    Pincin Matteo & - & - & 9 & - & - & - & \textbf{9}\tabularnewline
    Soranzo Andrea & - & - & - & - & - & 7 & \textbf{7}\tabularnewline
    \midrule[\heavyrulewidth]
    \textbf{TOTALE}& 5 & 7 & 18 & - & - & 14 &  44\tabularnewline

    \rowcolor{white}\caption{ Rendiconto effettivo della distribuzione delle ore per lo sprint 2}

}


I costi effettivi del periodo sono i seguenti:


\costReportTable{
    Responsabile & 7 & 30 & 210 \tabularnewline
    Amministratore & 14 & 20 & 280 \tabularnewline
    Analista & 18 & 25 & 450 \tabularnewline
    Progettista & - & 25 & - \tabularnewline
    Programmatore & - & 15 & - \tabularnewline
    Verificatore & 14 & 15 & 210 \tabularnewline
    \midrule[\heavyrulewidth]
    \textbf{Totale Consuntivo} & 53 & - & 1150 \tabularnewline
    \midrule[\heavyrulewidth]
    \textbf{Totale Preventivo} & 44 & - & 950 \tabularnewline
    \midrule[\heavyrulewidth]
    \textbf{Differenza} & 9 & - & 200 \tabularnewline

    \rowcolor{white}\caption{ Consuntivo costi sprint 2}

}


\subsubsubsection{Resoconto}
Nel seguente resoconto vengono analizzate le principali differenze tra le stime iniziali e le ore effettive durante lo sprint:
\begin{itemize}
    \item \textbf{Verificatore (-2 ore);} 
    \item \textbf{Responsabile (+2 ore);} 
    \item \textbf{Amministratore (+5 ore).}
\end{itemize}
Sono state eseguite meno ore da verificatore poichè l'ammontare è stato leggermente sovrastimato. Inoltre le ore aggiuntive di responsabile e amministratore si evincono dal fatto che si ha avuto più tempo da dedicare al progetto rispetto
a quello preventivato.
\\
Nel complesso, anche grazie al lavoro aggiuntivo svolto, sono stati rispettati tutti gli obiettivi prefissati per questo sprint.\\

Di seguito le ore rimanenti ad ogni componente del gruppo relative ad ogni ruolo.
\remainingHoursTable{
    Bergamin Elia & 2 & 3 & 16 & 20 & 22 & 13 \tabularnewline
    Diviesti Filippo & 9 & 7 & 7 & 20 & 22 & 12 \tabularnewline
    Djossa Edgar & 4 & 2 & 16 & 20 & 22 & 18 \tabularnewline
    Chilese Elena & 9 & 0 & 13 & 20 & 22 & 18 \tabularnewline
    Pincin Matteo & 9 & 7 & 7 & 20 & 22 & 12 \tabularnewline
    Soranzo Andrea & 9 & 1 & 16 & 20 & 22 & 11\tabularnewline
    
    \rowcolor{white}\caption{Ore rimanenti per ogni ruolo dopo lo Sprint 2}
}



\subsubsection{Sprint 3}
Di seguito la suddivisione dei ruoli e le ore di lavoro effettive impiegate in questo sprint:


\memberReportTable{
    Bergamin Elia & - & 1 & 5 (-3) & - & - & - & \textbf{6}\tabularnewline
    Diviesti Filippo & 4 (+1) & - & - & - & - & - & \textbf{4}\tabularnewline
    Djossa Edgar & - & 0.5 (+0.5) & - & - & - & 5 (+2) & \textbf{5.5}\tabularnewline
    Chilese Elena & - & - & - & - & - & 4 (+1) & \textbf{4}\tabularnewline
    Pincin Matteo & - & 4 (+2) & - & - & - & - & \textbf{4}\tabularnewline
    Soranzo Andrea & - & - & 6 (-1) & - & - & - & \textbf{6}\tabularnewline
    \midrule[\heavyrulewidth]
    \textbf{TOTALE}& 4 & 5.5 & 11 & - & - & 9 &  29.5 \tabularnewline

    \rowcolor{white}\caption{Rendiconto effettivo della distribuzione delle ore per lo sprint 3}

}


I costi effettivi del periodo sono i seguenti:


\costReportTable{
    Responsabile & 4 (+1) & 30 & 120 \tabularnewline
    Amministratore & 5.5 (+1.5) & 20 & 110 \tabularnewline
    Analista & 11 (-4) & 25 & 275 \tabularnewline
    Progettista & - & 25 & - \tabularnewline
    Programmatore & - & 15 & - \tabularnewline
    Verificatore & 9 (+3) & 15 & 135 \tabularnewline
    \midrule[\heavyrulewidth]
    \textbf{Totale Consuntivo} & 29.5 & - & 640 \tabularnewline
    \midrule[\heavyrulewidth]
    \textbf{Totale Preventivo} & 27 & - & 615 \tabularnewline
    \midrule[\heavyrulewidth]
    \textbf{Differenza} & 2.5 & - & 25 \tabularnewline

    \rowcolor{white}\caption{ Consuntivo costi sprint 3}

}


\subsubsubsection{Resoconto}
Nel seguente resoconto vengono analizzate le principali differenze tra le stime iniziali e le ore effettive durante lo sprint:
\begin{itemize}
    \item \textbf{Responsabile (+1 ora);} 
    \item \textbf{Amministratore (+1.5 ore);} 
    \item \textbf{Analista (-4 ore);}
    \item \textbf{Verificatore (+3 ore).}
\end{itemize}

Sono state dedicate più ore al ruolo di responsabile e di amministratore rispetto a quanto inizialmente previsto. Ciò è dovuto ad una maggiore complessità nella gestione e organizzazione delle attività progettuali.
Le ore destinate al ruolo di analista sono state inferiori alle stime, in quanto alcune attività di analisi sono risultate meno onerose del previsto e hanno altresì aiutato il confronto con il proponente e l'incontro con il prof. Cardin. 
Infine, il ruolo di verificatore ha richiesto un incremento di ore rispetto al pianificato, in virtù di un aumento della complessità e dell'estensione dei documenti da verificare.
\\
Complessivamente, le variazioni riscontrate non hanno compromesso il raggiungimento degli obiettivi prefissati, permettendo il completamento di tutte le attività previste.\\
Di seguito le ore rimanenti ad ogni componente del gruppo relative ad ogni ruolo.
\remainingHoursTable{
    Bergamin Elia & 2 & 2 & 11 & 20 & 22 & 13 \tabularnewline
    Diviesti Filippo & 5 & 7 & 7 & 20 & 22 & 12 \tabularnewline
    Djossa Edgar & 4 & 1.5 & 16 & 20 & 22 & 13 \tabularnewline
    Chilese Elena & 9 & 0 & 13 & 20 & 22 & 14 \tabularnewline
    Pincin Matteo & 9 & 3 & 7 & 20 & 22 & 12 \tabularnewline
    Soranzo Andrea & 9 & 1 & 10 & 20 & 22 & 11\tabularnewline
    
    \rowcolor{white}\caption{Ore rimanenti per ogni ruolo dopo lo Sprint 3}
}

\subsubsection{Sprint 4}
Di seguito la suddivisione dei ruoli e le ore di lavoro effettive impiegate in questo sprint:


\memberReportTable{
    Bergamin Elia & - & - & - & - & 5 (+2) & 0 (-2.5) & \textbf{5}\tabularnewline
    Diviesti Filippo & - & 4 (+2) & - & - & - & - & \textbf{4}\tabularnewline
    Djossa Edgar & - & - & 3 (+1) & - & - & - & \textbf{3}\tabularnewline
    Chilese Elena & 3 & - & - & - & - & - & \textbf{3}\tabularnewline
    Pincin Matteo & - & - & - & - & 3 & - & \textbf{3}\tabularnewline
    Soranzo Andrea & - & 1 & - & - & - & 4.5 (-0.5) & \textbf{6.5}\tabularnewline
    \midrule[\heavyrulewidth]
    \textbf{TOTALE}& 3 & 6 & 3 & 0 & 8 & 4.5 &  24.5 \tabularnewline

    \rowcolor{white}\caption{Rendiconto effettivo della distribuzione delle ore per lo sprint 4}

}


I costi effettivi del periodo sono i seguenti:


\costReportTable{
    Responsabile & 3 & 30 & 90 \tabularnewline
    Amministratore & 6 (+2) & 20 & 120 \tabularnewline
    Analista & 3 (+1) & 25 & 75 \tabularnewline
    Progettista & - & 25 & - \tabularnewline
    Programmatore & 8 (+2) & 15 & 120 \tabularnewline
    Verificatore & 4.5 (-3) & 15 & 67.5 \tabularnewline
    \midrule[\heavyrulewidth]
    \textbf{Totale Consuntivo} & 24.5 & - & 472.5 \tabularnewline
    \midrule[\heavyrulewidth]
    \textbf{Totale Preventivo} & 22.5 & - & 422.5 \tabularnewline
    \midrule[\heavyrulewidth]
    \textbf{Differenza} & 2 & - & 50 \tabularnewline

    \rowcolor{white}\caption{Consuntivo costi sprint 4}

}


\subsubsubsection{Resoconto}
Nel seguente resoconto vengono analizzate le principali differenze tra le stime iniziali e le ore effettive durante lo sprint:
\begin{itemize}
    \item \textbf{Amministratore (+2)}: la redazione dei verbali ha richiesto più tempo del solito, poiché sono stati trattati più temi nell'incontro interno e si è svolto un incontro con il proponente;
    \item \textbf{Analista (+1)}: richiesto più tempo rispetto al preventivo in quanto l'\textit{Analisi dei requisiti} conteneva più imprecisioni del previsto;
    \item \textbf{Programmatore (+2)}: lo sviluppo del PoC si è rivelato più ostico del previsto e sono state necessarie più ore per realizzare la corretta implementazione delle funzionalità richieste;
    \item \textbf{Verificatore (-3)}: lo sprint, avvenuto durante le festività natalizie, ha comportato una minore produzione di documenti rispetto al solito. Inoltre, la fase di verifica del codice del PoC è stata sovrastimata in fase di preventivo, poiché ci siamo limitati a individuare bug e imperfezioni utilizzando il PoC stesso, il che ha richiesto meno tempo.
\end{itemize}
Nonostante alcune discrepanze rispetto alle ore inizialmente stimate, tutte le attività pianificate sono state completate con successo.\\
Di seguito le ore rimanenti ad ogni componente del gruppo relative ad ogni ruolo.
\remainingHoursTable{
    Bergamin Elia & 2 & 2 & 11 & 20 & 17 & 13 \tabularnewline
    Diviesti Filippo & 5 & 3 & 7 & 20 & 22 & 12 \tabularnewline
    Djossa Edgar & 4 & 1.5 & 13 & 20 & 22 & 13 \tabularnewline
    Chilese Elena & 6 & 0 & 13 & 20 & 22 & 14 \tabularnewline
    Pincin Matteo & 9 & 3 & 7 & 20 & 19 & 12 \tabularnewline
    Soranzo Andrea & 9 & 0 & 10 & 20 & 22 & 6.5\tabularnewline
    
    \rowcolor{white}\caption{Ore rimanenti per ogni ruolo dopo lo Sprint 4}
}

\subsubsection{Sprint 5}
Di seguito la suddivisione dei ruoli e le ore di lavoro effettive impiegate in questo sprint:


\memberReportTable{
    Bergamin Elia & - & 2 & - & - & - & 3 & \textbf{5}\tabularnewline
    Diviesti Filippo & - & - & - & - & - & 4 & \textbf{4}\tabularnewline
    Djossa Edgar & - & - & 2 & - & 2 & - & \textbf{4}\tabularnewline
    Chilese Elena & - & - & 2.5(+0.5) & - & - & 3 & \textbf{5.5}\tabularnewline
    Pincin Matteo & 5(-1) & - & - & - & - & - & \textbf{5}\tabularnewline
    Soranzo Andrea & - & - & 7(+1) & - & - & - & \textbf{7}\tabularnewline
    \midrule[\heavyrulewidth]
    \textbf{TOTALE}& 5 & 2 & 11.5 & - & 2 & 10 & 30.5 \tabularnewline

    \rowcolor{white}\caption{Rendiconto effettivo della distribuzione delle ore per lo sprint 5}

}


I costi effettivi del periodo sono i seguenti:


\costReportTable{
    Responsabile & 5(-1) & 30 & 150 \tabularnewline
    Amministratore & 2 & 20 & 40 \tabularnewline
    Analista & 11.5 (+1.5) & 25 & 287.5 \tabularnewline
    Progettista & - & 25 & - \tabularnewline
    Programmatore & 2 & 15 & 30 \tabularnewline
    Verificatore & 10 & 15 & 150 \tabularnewline
    \midrule[\heavyrulewidth]
    \textbf{Totale Consuntivo} & 30.5 & - & 657.5 \tabularnewline
    \midrule[\heavyrulewidth]
    \textbf{Totale Preventivo} & 30 & - & 650 \tabularnewline
    \midrule[\heavyrulewidth]
    \textbf{Differenza} & 0.5 & - & 7.5 \tabularnewline

    \rowcolor{white}\caption{Consuntivo costi sprint 5}

}


\subsubsubsection{Resoconto}
Nel seguente resoconto vengono analizzate le principali differenze tra le stime iniziali e le ore effettive durante lo sprint:
\begin{itemize}
    \item \textbf{Responsabile (-1)}; 
    \item \textbf{Analista (+1.5)}: La redazione e la creazione dei grafici per il cruscotto di qualità ha richiesto più tempo del previsto perché richiede azioni manuali.
\end{itemize}
In breve, oltre ad alcune difficoltà incontrate per la creazione di grafici utili a rappresentare l'andamento della qualità, tutte le attività pianificate sono state completate con successo.\\
Di seguito le ore rimanenti ad ogni componente del gruppo relative ad ogni ruolo.
\remainingHoursTable{
    Bergamin Elia & 2 & 0 & 11 & 20 & 17 & 10 \tabularnewline
    Diviesti Filippo & 5 & 3 & 7 & 20 & 22 & 8 \tabularnewline
    Djossa Edgar & 4 & 1.5 & 11 & 20 & 20 & 13 \tabularnewline
    Chilese Elena & 6 & 0 & 10.5 & 20 & 22 & 11 \tabularnewline
    Pincin Matteo & 4 & 3 & 7 & 20 & 19 & 12 \tabularnewline
    Soranzo Andrea & 9 & 0 & 3 & 20 & 22 & 6.5\tabularnewline
    
    \rowcolor{white}\caption{Ore rimanenti per ogni ruolo dopo lo Sprint 5}
}
