\section{Introduzione}
\subsection{Scopo del documento}
Questo documento fornisce una guida chiara ed organizzata per pianificare, eseguire e controllare le attività dello sviluppo del prodotto.
Definisce gli obiettivi del progetto, specifica i risultati attesi e identifica le risorse necessarie. 
Include anche le stime preventive per i budget.
Il documento è pertanto strutturato come segue:
\begin{itemize}
      \item \textbf{Analisi dei rischi:} identifica i potenziali rischi individuati
            dal gruppo, la probabilità che si verifichino, la loro gravità e le relative
            strategie di mitigazione;
      \item \textbf{Modello di sviluppo:} descrive il modello di sviluppo adottato dal
            gruppo, da cui derivano vincoli sulla pianificazione e gestione di progetto;
      \item \textbf{Pianificazione:} offre una visione dettagliata del piano di lavoro
            del gruppo, basata su una chiara suddivisione delle attività. La pianificazione
            è strutturata in sprint, ognuno dei quali rappresenta un'unità di tempo
            incentrata su obiettivi specifici;
      \item \textbf{Preventivo:} indica le ore e i costi che si intende impiegare
            in ogni periodo pianificato;
      \item \textbf{Consuntivo:} indica le ore e i costi effettivamente impiegati
            in ogni periodo pianificato, mettendoli a confronto con i dati preventivati;
      \item \textbf{Attualizzazione dei rischi:} descrive i rischi che si sono verificati
            durante lo svolgimento del progetto e le misure adottate conseguentemente.
\end{itemize}

\subsection{Scopo del prodotto}
\textit{3Dataviz} è un prodotto ideato dall'azienda \textit{Sanmarco Informatica S.p.A.} per semplificare e rendere più accessibile la visualizzazione dei dati.\\
Esso mira a trasformare i dati in grafici e rappresentazioni visive, sfruttando la capacità del cervello umano di elaborare rapidamente le immagini. 
Questo approccio facilita il processo decisionale e migliora la comprensione delle informazioni.\\
L’obiettivo principale è lo sviluppo di un’interfaccia web che trasforma dati provenienti da diverse fonti (come database e REST API) in grafici 3D interattivi e navigabili. 
I dati potranno essere consultati anche in formato tabellare, offrendo una visione alternativa ma altrettanto utile.  

\subsection{Glossario}
Per chiarire i termini tecnici o ambigui si utilizza il glossario disponibile nel file \textit{Glossario}.\\
Tutti i termini che richiedono spiegazioni sono indicati con il pedice “g”. \\
Questa convenzione consente un rapido collegamento tra il testo e la relativa spiegazione dettagliata nel glossario, garantendo coerenza e chiarezza.

\subsection{Riferimenti normativi}
\begin{itemize}
      \item \textit{Norme di progetto} \\ \url{https://6bitbusters.github.io/norme_di_progetto.pdf}
      \item Capitolato d'appalto C5 - \textit{Sanmarco Informatica S.p.A.}: 3Dataviz \\ \url{https://www.math.unipd.it/~tullio/IS-1/2024/Progetto/C5.pdf}
      \item Regolamento del progetto didattico \\ \url{https://www.math.unipd.it/~tullio/IS-1/2024/Dispense/PD1.pdf}
\end{itemize}
\subsection{Riferimenti informativi}
\begin{itemize}
      \item Slide T2 - Corso di Ingegneria del Software - I processi di ciclo di vita del software\\ \url{https://www.math.unipd.it/~tullio/IS-1/2024/Dispense/T02.pdf}
      \item Slide T3 - Corso di Ingegneria del Software - Modelli di sviluppo software \\ \url{https://www.math.unipd.it/~tullio/IS-1/2024/Dispense/T03.pdf}
      \item Slide T4 - Corso di Ingegneria del Software - Gestione di progetto \\ \url{https://www.math.unipd.it/~tullio/IS-1/2024/Dispense/T04.pdf}
\end{itemize}
