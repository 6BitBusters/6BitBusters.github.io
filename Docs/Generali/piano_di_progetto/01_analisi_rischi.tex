\section{Analisi dei rischi}
In questa sezione sono riassunti i rischi a cui il gruppo è esposto a seguito dell'aggiudicazione del capitolato \textit{3Dataviz}.
La nostra aspettativa è quella di riuscire a mettere in pratica decisioni e scelte che ci consentano di affrontare il progetto con il minor numero di problemi possibile.
Tuttavia, nel caso in cui questi si verifichino, vogliamo essere preparati con soluzioni concrete e immediatamente adottabili.
\\I rischi sono stati suddivisi in quattro differenti categorie e, per ciascun rischio, dopo una breve descrizione, viene presentata l'analisi dei seguenti punti:
\begin{itemize}
    \item \textbf{Probabilità di occorrenza}: indica la probabilità con cui è possibile si verifichi il rischio;
    \item \textbf{Grado di pericolosità}: indica il grado di pericolosità del rischio;
    \item \textbf{Misure preventive}: indica le misure adottate per cercare di evitare il verificarsi del rischio;
    \item \textbf{Piano di contingenza}: indica come far fronte al rischio nel caso in cui esso si verifichi. 
\end{itemize}
\subsection{Rischi riguardanti i requisiti}

\begin{enumerate}
    \item \textbf{Analisi dei requisiti}\\
    L'analisi dei requisiti comporta un impegno non indifferente per tutto il gruppo.\\
    Essa è di fondamentale importanza in quanto ci permette di capire al meglio ciò che il prodotto che andremo a sviluppare dovrà essere in grado di fare.\\
    Il rischio maggiore è che si possano tralasciare aspetti che in futuro potrebbero rivelarsi fondamentali creando così discrepanze significative tra gli impegni preventivati e quelli effettivi.
    \begin{itemize}
        \item \textbf{Probabilità di occorrenza:} Media
        \item \textbf{Grado di pericolosità:} Alto
        \item \textbf{Misure preventive:} Analizzare in dettaglio tutti i casi d’uso e i relativi requisiti, mantenendo un dialogo aperto e costante con il proponente.
        \item \textbf{Piano di contingenza:} Organizzare colloqui con il proponente per discutere sul da farsi.
    \end{itemize}

    \item \textbf{Cambiamento dei requisiti iniziali}\\
    I requisiti iniziali potrebbero subire modifiche e/o aggiunte da parte del proponente.
    \begin{itemize}
        \item \textbf{Probabilità di occorrenza:} Media
        \item \textbf{Grado di pericolosità:} Medio
        \item \textbf{Misure preventive:} Realizzare un'analisi dei requisiti il più possibile chiara e precisa. 
        \item \textbf{Piano di contingenza:} Discutere e comprendere appieno ogni cambiamento insieme al proponente.
    \end{itemize}
\end{enumerate}


\subsection{Rischi tecnologici}
\begin{enumerate}
    \item \textbf{Strumenti e tecnologie sconosciute}\\
    Il progetto richiede l'utilizzo di software per la creazione di grafici 3D, oltre a librerie JavaScript e Framework a noi attualmente poco conosciuti.
    \begin{itemize}
        \item \textbf{Probabilità di occorrenza:} Alta
        \item \textbf{Grado di pericolosità:} Medio
        \item \textbf{Misure preventive:} Valutare l’impiego di ogni strumento e tecnologia ed esercitarsi personalmente per comprendere lo strumento/tecnologia da utilizzare. Sfruttare inoltre le eventuali conoscenze pregresse così da facilitare la suddivisione del lavoro ed evitare di concentrarsi eccessivamente su funzionalità meno rilevanti.
        \item \textbf{Piano di contingenza:} Se l’utilizzo di uno strumento o di una tecnologia causa ritardi, si procederà con una maggiore suddivisione più efficiente del lavoro e, in casi estremi, si valuterà un’alternativa, sempre confrontando il proponente.
    \end{itemize}

    \item \textbf{Problemi hardware o software}\\
    Problemi causati da PC malfunzionanti e successivi problemi software derivati da guasti.
    \begin{itemize}
        \item \textbf{Probabilità di occorrenza:} Media
        \item \textbf{Grado di pericolosità:} Basso
        \item \textbf{Misure preventive:} Condivisione dei file di progetto in un repository comune utilizzando un sistema di versionamento distribuito atto a garantire una gestione efficiente e collaborativa delle modifiche.
        \item \textbf{Piano di contingenza:} Utilizzare dispositivi secondari per continuare il lavoro.
    \end{itemize}

\end{enumerate}

\subsection{Rischi organizzativi}
\begin{enumerate}
    \item \textbf{Pianificazione inadeguata}\\
    La pianificazione di un intero progetto di questa portata, che include una corretta individuazione delle attività da svolgere e un'adeguata suddivisione dei compiti, soprattutto a causa dell'inesperienza dei membri del gruppo, può provocare ritardi e spreco di risorse.
    \begin{itemize}
        \item \textbf{Probabilità di occorrenza:} Alta
        \item \textbf{Grado di pericolosità:} Alto
        \item \textbf{Misure preventive:} Pianificazione su brevi periodi di tempo e con una visione pessimistica.
        \item \textbf{Piano di contingenza:} Stilare un nuovo piano di lavoro che ottimizzi le risorse e la suddivisione del lavoro.
    \end{itemize}

\item \textbf{Distribuzione del lavoro}\\
    La mal distribuzione del lavoro può causare un sovraccarico per un componente del gruppo, come altresì l'inattività di un componente.
    \begin{itemize}
        \item \textbf{Probabilità di occorrenza:} Media
        \item \textbf{Grado di pericolosità:} Medio
        \item \textbf{Misure preventive:} Organizzare il lavoro tenendo conto delle disponibilità e delle capacità di ogni componente.
        \item \textbf{Piano di contingenza:} Suddivisione del lavoro in maniera più distribuita e uniforme.
    \end{itemize}
    
\end{enumerate}

\subsection{Rischi personali}
\begin{enumerate}
    \item \textbf{Disponibilità dei componenti}\\
    Il gruppo è composto da alcuni studenti-lavoratori i cui impegni a volte non sono conciliabili con eventuali riunioni straordinarie e, a volte, con quelle settimanali.\\
    Inoltre, ogni componente del gruppo è impegnato in altri esami universitari che possono impegnare una mole significativa di tempo, soprattutto in prossimità della sessione invernale.
    \begin{itemize}
        \item \textbf{Probabilità di occorrenza:} Alta
        \item \textbf{Grado di pericolosità:} Basso
        \item \textbf{Misure preventive:} Organizzare con anticipo ogni incontro. Prevedere delle settimane di lavoro più intense ed altre meno, soprattutto in concomitanza con gli impegni sopra descritti.
        \item \textbf{Piano di contingenza:} Il componente assente deve aggiornarsi sulle decisioni prese e acquisire le informazioni sui progressi compiuti in tempi utili, al fine di riallinearsi tempestivamente con il resto del gruppo.
        Se l'assenza di uno o più membri sta provocando ritardi significativi, sarà compito del \textit{Responsabile} ri-pianificare la suddivisione del lavoro.
    \end{itemize}

    \item \textbf{Difficoltà comunicative}\\
    La maggior parte degli incontri, sia quelli interni che quelli con il proponente, si svolgeranno in modalità telematica.
    \begin{itemize}
        \item \textbf{Probabilità di occorrenza:} Bassa
        \item \textbf{Grado di pericolosità:} Medio
        \item \textbf{Misure preventive:} Utilizzo di diversi strumenti di comunicazione.
        \item \textbf{Piano di contingenza:} Ritrovo in presenza a alla fine dell'ora di lezione o in un giorno preventivamente accordato.
    \end{itemize}

    \item \textbf{Mancanza di esperienza personale}\\
    
    \begin{itemize}
        \item \textbf{Probabilità di occorrenza:} Alta
        \item \textbf{Grado di pericolosità:} Medio
        \item \textbf{Misure preventive:} Iniziare a capire il funzionamento e sperimentare i nuovi strumenti e tecnologie da utilizzare. 
        \item \textbf{Piano di contingenza:} Se un membro riscontra parecchie difficoltà, può chiedere supporto agli altri membri e soprattutto al \textit{Responsabile}, il quale avrà il compito di trovare un compromesso atto a risolvere la situazione.
    \end{itemize}

    \item \textbf{Conflitti interni}\\
    I gruppi sono stati formati casualmente e per questo motivo i membri, non si conoscendosi tra loro, potrebbero avere idee contrastanti.
    \begin{itemize}
        \item \textbf{Probabilità di occorrenza:} Bassa
        \item \textbf{Grado di pericolosità:} Medio
        \item \textbf{Misure preventive:} Ciascun membro espone educatamente la propria idea, il \textit{Responsabile} sorveglierà ed agirà se necessario.
        \item \textbf{Piano di contingenza:} Ogni decisione viene discussa e successivamente approvata a maggioranza.
    \end{itemize}
\end{enumerate}
