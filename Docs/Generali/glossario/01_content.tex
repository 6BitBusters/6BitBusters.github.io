% Insert content here
\section{A}
    \subsection{Action}
    (s. c. f.)\\
    Combinazione di script e configurazioni software che consentono di eseguire operazioni 
    specifiche in risposta a determinati eventi, come modifiche al codice.
    Vedi anche \nameref{GitHub Actions}.
    \subsection{Agile}
    (s. p. m. / agg.)
    Approccio alla gestione dei progetti che prevede la suddivisione in fasi e sottolinea 
    l'importanza della collaborazione e del miglioramento continuo tramite feedback. 
    Il team segue un ciclo di pianificazione, esecuzione e valutazione.
    \subsection{API}
    (s. c. f.)\\
    Abbreviazione di \nameref{Application Programming Interface}.
    \subsection{Application Programming Interface}
    \label{Application Programming Interface}
    (s. c. f.)\\
    Insieme di procedure che consentono la comunicazione tra diversi computer, tra diversi 
    software o tra diversi componenti di software.
    \subsection{Applicazione web}
    \label{Applicazione web}
    Applicazione software accessibile tramite un browser utilizzando una 
    connessione a Internet.
    \subsection{Approvazione}
    Attività nella quale un documento viene convalidato dal responsabile di progetto.
    \subsection{Assignee}
    (s. c. m.)\\
    Letteralmente "Assegnatario", ovvero membro del team a cui è assegnato lo
    svolgimento di una determinata attività.
    \subsection{Attore}
    Ruolo coperto da un certo insieme di entità che interagiscono con il sistema, 
    inclusi utenti e altri sistemi software.
\pagebreak
\section{B}
    \subsection{Backlog}
    (s. c. m.)\\
    Lista prioritaria e ordinata di attività, requisiti o funzionalità che devono 
    essere completate per sviluppare un prodotto o un progetto.
    Vedi anche \nameref{Product backlog} e \nameref{Sprint backlog}.
    \subsection{Branch} 
    (s. c. m.)\\
    In Git, ramo del progetto che permette di sviluppare software, documentazione inclusa, su una linea separata 
    senza interferire con il ramo principale o altri rami del repository.
\pagebreak
\section{C}
    \subsection{Capitolato}
    Documento tecnico a cui si fa riferimento per definire le specifiche tecniche 
    del prodotto che verrà realizzato. Può contenere indicazioni sulle metodologie e
    tecnologie da adottare per lo sviluppo.
    \subsection{Commit}
    (s. c. m.)\\
    In Git, istantanea di uno specifico stato del progetto in un determinato momento.
    Ogni commit ha un codice univoco che lo identifica,
    un riferimento allo stato del repository prima del commit e un messaggio descrittivo 
    fornito dallo sviluppatore che descrive brevemente le modifiche apportate ai file.
    \subsection{Committente}
    Chi ordina ad altri l'esecuzione di un lavoro. Nel caso del progetto didattico
    è il docente.
    \subsection{Consuntivo}
    Rendiconto dei risultati effettivi di un dato periodo di attività, 
    in termini di ore e costi, messi a confronto con gli obiettivi pianificati.
\pagebreak
\section{D}
    \subsection{Daily scrum}
    (s. c. m.)\\
    Incontro giornaliero a cui partecipano tutti i membri del team, in cui ciascuno
    risponde alle seguenti domande:
    \begin{itemize}
        \item Cosa hai fatto ieri?
        \item Cosa farai oggi?
        \item C'è qualcosa che ti impedisce di farlo?
    \end{itemize}
    \subsection{Database}
    (s. c. m.)\\
    Archivio di dati strutturato in modo da consentire la gestione e 
    l'organizzazione dei dati.
    \subsection{Dataset}
    (s. c. m.)\\
    Collezione strutturata di dati, generalmente di grandi dimensioni.
    \subsection{Diagramma di Gantt}
    Strumento di supporto alla gestione dei progetti, costruito partendo da 
    un asse orizzontale, che rappresenta l'arco temporale totale del progetto, 
    suddiviso in fasi incrementali (ad esempio giorni o settimane), e da un asse 
    verticale, che rappresenta le attività che costituiscono il progetto.
    \subsection{Discord}
    (s. p. m.)\\
    Piattaforma di VoIP (Voice over Internet Protocol), messaggistica istantanea 
    e distribuzione digitale, progettata per la comunicazione tra gruppi di persone.
    \subsection{Drei}
    (s. p. f.)\\
    Vedi \nameref{React Three Drei}.
\pagebreak
\section{E}
    \subsection{Efficacia}
    Capacità di raggiungere gli obiettivi attesi.
    \subsection{Efficienza}
    Capacità di utilizzare la minima quantità di risorse necessarie al raggiungimento
    di un obiettivo.
\pagebreak
\section{F}
    \subsection{Feature}
    (s. c. f.)\\
    Vedi \nameref{Funzionalità}.
    \subsection{Feedback}
    (s. c. m.)\\
    Informazione di ritorno, opinione o valutazione che viene fornita come reazione a 
    un'attività o un prodotto, al fine di migliorare o correggere i risultati futuri.
    \subsection{Fiber}
    (s. p. f.)\\
    Vedi \nameref{React Three Fiber}.
    \subsection{Fornitore}
    Chi si impegna a svolgere un lavoro o realizzare un prodotto. Nel caso del progetto didattico
    è il gruppo di studenti.
    \subsection{Framework}
    (s. c. m.)\\
    Struttura o piattaforma software che fornisce un insieme di strumenti, regole, 
    librerie e componenti per aiutare gli sviluppatori a costruire applicazioni 
    in modo più efficiente e standardizzato. Definisce il modo in cui un progetto 
    deve essere organizzato e sviluppato, riducendo il lavoro ripetitivo e favorendo 
    la coerenza e la manutenibilità del codice.
    \subsection{Funzionalità}
    \label{Funzionalità}
    Specifica caratteristica o capacità di un prodotto, software, applicazione o sistema, 
    che ha lo scopo di soddisfare un'esigenza o risolvere un problema per l'utente.
\pagebreak
\section{G}
    \subsection{Git}
    \label{Git}
    (s. p. m.)\\
    Sistema software distribuito che tiene traccia delle versioni dei file. 
    Viene utilizzato per controllare il codice sorgente da parte di programmatori 
    che stanno sviluppando software in modo collaborativo.
    \subsection{GitHub}
    \label{GitHub}
    (s. p. m.)\\
    Piattaforma per sviluppatori che permette di creare, memorizzare, gestire 
    e condividere il loro codice. È utilizzato principalmente per ospitare 
    progetti di sviluppo software open source.
    \subsection{GitHub Actions}
    \label{GitHub Actions}
    (s. p. f.)\\
    Piattaforma di automazione dei workflow integrata in GitHub che consente 
    di creare e gestire flussi di lavoro personalizzati direttamente all'interno 
    di un repository.
    \subsection{GitHub Pages}
    \label{GitHub Pages}
    (s. p. f.)\\
    Servizio offerto da GitHub che permette di pubblicare siti web statici direttamente da un repository GitHub,
    usato per ospitare la documentazione prodotta.
    \subsection{Google Chat}
    (s. p. m.)\\
    Strumento di comunicazione con funzionalità di messaggistica
    istantanea e condivisione di file, integrato con altri servizi Google.
    \subsection{Google Meet}    
    (s. p. m.)\\
    Servizio web per videoconferenze e riunioni, con funzionalità di messaggistica,
    condivisione schermo e registrazione.
    \subsection{Grafico a barre}
    Visualizzazione grafica utilizzata per confrontare valori discreti o categorie. 
    Si compone di barre la cui altezza rappresenta la quantità associata a ciascuna categoria.
\pagebreak
\section{H}
    \subsection{Helper}
    (s. c. m.)\\
    Entità progettata per fornire supporto ad altre parti del codice, 
    semplificando l'esecuzione di operazioni comuni o ripetitive.
    Può essere una funzione, una classe o un modulo.
    \subsection{Hotfix}
    (s. c. m.)\\
    Correzione urgente applicata a un sistema software per risolvere un problema critico, 
    che richiede un intervento immediato. A differenza delle normali modifiche o aggiornamenti pianificati, 
    un hotfix viene sviluppato e distribuito rapidamente per minimizzare l'impatto del problema.\\
    In caso di hotfix di un documento, viene aumentato di uno il numero di versione più a destra.
\pagebreak
\section{I}
    \subsection{Indice Gulpease}
    Indice di leggibilità di un testo in lingua italiana. Considera due variabili 
    linguistiche: la lunghezza della parola e la lunghezza della frase rispetto 
    al numero delle lettere.
    \subsection{Issue}
    \label{Issue}
    (s. c. f.)\\
    Strumento utilizzato per tracciare, discutere e risolvere problemi, richieste di funzionalità, 
    bug o idee relative a un progetto, facilitando la collaborazione tra i membri di un team.
    \subsection{Issue tracking system}
    (s. c. m.)\\
    Software che gestisce liste di issue, generalmente utilizzato in contesti collaborativi.
    Può comprendere l'allocazione delle risorse, la contabilità del tempo, la gestione delle 
    priorità e il flusso di lavoro per la supervisione.
\pagebreak
\section{J}
\pagebreak
\section{K}
\pagebreak
\section{L}
    \subsection{LaTeX}
    \label{LaTeX}
    (s. p. m.)\\
    Linguaggio di markup utilizzato per la redazione di documenti di alta qualità tipografica.
\pagebreak
\section{M}
    \subsection{Main}
    (s. c. m.)\\
    Branch principale di sviluppo di un progetto software.
    \subsection{Merge}
    (s. c. m.)\\
    In Git, processo di combinazione di modifiche provenienti da un branch in un altro.
    \subsection{Metrica}
    Misura quantitativa utilizzata per valutare le caratteristiche o lo stato di un sistema, processo o entità. 
    Le metriche forniscono dati oggettivi e misurabili che aiutano a prendere decisioni informate.
    \subsection{Milestone}
    (s. c. f.)\\
    Data che fissa un punto di avanzamento atteso nello svolgimento del progetto.
    \pagebreak
\section{N}
\pagebreak
\section{O}
    \subsection{Overleaf}
    (s. p. m.)\\
    Piattaforma online per la creazione, modifica e condivisione di documenti 
    scritti in LaTeX.
\pagebreak
\section{P}
    \subsection{Pages}
    (s. p. f.)\\
    Vedi \nameref{GitHub Pages}.
    \subsection{Pan}
    (s. c. m.)\\
    Nella grafica 3D, movimento della telecamera o della vista che consente 
    di spostarsi orizzontalmente o verticalmente in una scena senza modificarne 
    la prospettiva o l'orientamento.
    \subsection{PoC}
    (s. p. m.)\\
    Abbreviazione di \nameref{Proof of Concept}.
    \subsection{Preventivo}
    Previsione delle ore e dei costi relativi a un dato periodo di tempo,
    calcolato in base alle attività pianificate.
    \subsection{Product backlog}
    \label{Product backlog}
    (s. c. m.)\\
    Lista di tutte le funzionalità, modifiche, miglioramenti e correzioni necessarie 
    per il prodotto finale. Rappresenta una visione globale del lavoro da fare.
    \subsection{Proof of Concept}
    \label{Proof of Concept}
    (s. p. m.)\\
    Dimostrazione pratica della fattibilità di un progetto o di una soluzione proposta.
    \subsection{Proponente}
    Chi propone un lavoro. Nel progetto didattico svolge il ruolo di cliente 
    rispetto alle esigenze di prodotto e mentore rispetto alle scelte di sviluppo.
    Per il nostro gruppo è Sanmarco Informatica S.p.A.
    \subsection{Pull}
    (s. c. m.)\\
    In Git, comando per aggiornare un repository locale con le modifiche più recenti provenienti 
    da un repository remoto. È una combinazione di due comandi: fetch (che scarica le modifiche 
    da remoto) e merge (che integra le modifiche nel branch corrente).
    \subsection{Pull request}
    (s. c. f.)\\
    Funzionalità comune nelle piattaforme di sviluppo collaborativo, come GitHub, utilizzata per 
    richiedere l'integrazione del lavoro di uno sviluppatore (di solito su un branch separato) 
    nel main o in un altro ramo del repository.
    \subsection{Push}
    (s. c. m.)\\
    In Git, comando per trasferire i commit presenti nel proprio repository locale a un 
    branch specifico in un repository remoto, rendendo le modifiche disponibili 
    ad altri collaboratori.

\pagebreak
\section{Q}
\pagebreak
\section{R}
    \subsection{React}
    (s. p. m.)\\
    Libreria JavaScript per la la costruzione interfacce utente basata su componenti, che utilizza un 
    approccio dichiarativo e il Virtual DOM per ottimizzare le prestazioni.
    \subsection{React Three Drei}
    \label{React Three Drei}
    (s. p. f.)\\
    Libreria che fornisce un insieme di componenti e helper per 
    costruire applicazioni 3D utilizzando React e Three.js.
    \subsection{React Three Fiber}
    \label{React Three Fiber}
    (s. p. f.)\\
    Libreria basata su Three.js che consente di usare la sintassi
    di React nello sviluppo di applicazioni 3D.
    \subsection{Repository}
    \label{Repository}
    (s. c. m.)\\
    Struttura che contiene tutti i file, la cronologia delle modifiche e la 
    configurazione necessaria per gestire un progetto di sviluppo software.
    \subsection{Retrospettiva}
    Vedi \nameref{Sprint retrospective}.
\pagebreak
\section{S}
    \subsection{Scrum}
    (s. p. m.)\\
    Framework agile per la gestione del ciclo di sviluppo del software, 
    iterativo ed incrementale, concepito per gestire progetti e prodotti software 
    o applicazioni di sviluppo.
    \subsection{Sprint}
    \label{Sprint}
    (s. c. m.)\\
    Breve periodo di tempo in cui un team Scrum collabora per completare 
    una determinata quantità di lavoro. Uno sprint comprende:  
    sprint planning, daily scrum, lavoro di sviluppo, 
    sprint review e sprint retrospective. Il team \textit{Six Bix Busters} ha 
    deciso che ogni sprint durerà due settimane.
    \subsection{Sprint backlog}   
    \label{Sprint backlog}
    (s. c. m.)\\
    Sotto-lista del product backlog, contenente le attività che il team di sviluppo 
    si impegna a completare durante uno sprint.
    \subsection{Sprint planning}
    (s. c. m.)\\
    Riunione dedicata alla pianificazione del lavoro per il prossimo sprint. 
    \subsection{Sprint retrospective}
    \label{Sprint retrospective}
    (s. c. f.)\\
    Riunione in cui il team parla di cosa è andato bene, cosa no e come si può migliorare.
    \subsection{Sprint review}
    (s. c. f.)\\ 
    Riunione in cui il team mostra ciò che ha realizzato durante uno sprint.
\pagebreak
\section{T}
    \subsection{Task}
    (s. c. m.)\\
    Compito specifico o attività.
    \subsection{Team}
    (s. c. m.)\\
    \label{Team}
    Gruppo di persone che lavorano insieme per raggiungere un obiettivo comune.
    \subsection{Telegram}
    (s. p. m.)\\
    Servizio di messaggistica istantanea e VoIP (Voice over Internet Protocol)
    basato su cloud.
    \subsection{Three.js}
    (s. p. f.)\\
    Libreria JavaScript open source che consente di creare ed eseguire rendering 
    di grafica 3D nel browser, fornendo agli sviluppatori un'astrazione di alto livello.
\pagebreak
\section{U}
\pagebreak
\section{V}
    \subsection{Validazione}
    Processo che accerta che il prodotto finale sia pienamente conforme alle aspettative.
    \subsection{Verifica}
    Processo che determina se i prodotti software di un'attività soddisfano
    i requisiti o le condizioni imposte nelle attività precedenti.
    \subsection{Versionamento}
    Gestione di versioni multiple di una stessa informazione.
\pagebreak
\section{W}
    \subsection{Way of working}
    \label{Way of working}
    (s. c. m.)\\
    Il modo in cui il gruppo decide di lavorare, ovvero l'insieme di regole 
    che il gruppo si dà al fine di organizzare al meglio le attività di progetto.
    \subsection{Web application}
    (s. c. f.)\\
    Vedi \nameref{Applicazione web}.
    \subsection{Workflow}
    (s. c. m.)\\
    Sequenza definita di attività o processi attraverso cui un compito, un progetto 
    o un'operazione viene completata.
    \subsection{Wow}
    (s. c. m.)\\
    Abbreviazione di \nameref{Way of working}.
\pagebreak
\section{X}
\pagebreak
\section{Y}
\pagebreak
\section{Z}
