% Insert content here
\section{A}
    \subsection{Approvazione}
    Attività nel quale un documento viene convalidato dal responsabile di progetto.
    \subsection{Agile}
    Approccio alla gestione dei progetti che prevede la suddivisione in fasi e sottolinea 
    l'importanza della collaborazione e del miglioramento continuo tramite feedback. 
    Il team seguono un ciclo di pianificazione, esecuzione e valutazione.

\section{B}
    \subsection{Branch}
    In Git, ramo del progetto che permette di sviluppare software, documentazione inclusa, su una linea separata 
    senza interferire con il ramo principale o altri rami del repository.

\section{C}
    \subsection{Commit}
    In Git, istantanea di uno specifico stato del tuo progetto in un determinato momento.
    Ogni commit ha un codice univoco che lo identifica e include le modifiche apportate ai file,
    un riferimento allo stato del repository prima del commit e un messaggio descrittivo fornito dallo sviluppatore.

\section{D}
\section{E}
\section{F}
\section{G}
    \subsection{GitHub Pages}
    Servizio offerto da GitHub che permette di pubblicare siti web statici direttamente da un repository GitHub,
    usato per ospitare la documentazione prodotta.
\section{H}
    \subsection{Hotfix}
    Correzione urgente applicata a un sistema software per risolvere un problema critico, 
    che richiede un intervento immediato. A differenza delle normali modifiche o aggiornamenti pianificati, 
    un hotfix viene sviluppato e distribuito rapidamente per minimizzare l'impatto del problema.\\
    In caso di hotfix di un documento, viene aumentato di uno il numero di versione più a destra.

\section{I}
\section{J}
\section{K}
\section{L}
\section{M}
    \subsection{Major}
    Numero di versione più a sinistra, il cui incremento indica: nel caso di un documento, che esso è pronto per una delle revisioni, ovvero tutte le modifiche precedenti sono state 
    approvate dal responsabile; per il codice, invece, che è stata introdotta una modifica sostanziale che potenzialmente potrebbe causare problemi di retrocompatibilità.
    \subsection{Minor}
    Numero di versione centrale, il cui suo incremento indica: nel caso di un documento che la sezione aggiunta è stata verificata; per il codice invece che è stata
    introdotta una funzione o feature aggiuntiva che non causa problemi al software preesistente, oppure una correzione ad un bug complesso di importanza elevata.

\section{N}
\section{O}
\section{P}
    \subsection{Pages}
    Vedi GitHub Pages.
    \subsection{Pull request}
    Feature di github che permette di chiedere ad un verificatore o al responsabile se il contenuto di un branch può
    essere unito alla parte principale del progetto, già verificata e pronta per essere mandata in produzione.

\section{Q}
\section{R}
\section{S}
    \subsection{Sprint}
    Periodo di tempo corrispondente 2/3 settimane nel quale il gruppo di lavoro si assegna degli 
    obiettivi, suddivide il lavoro, lo svolge e alla fine verifica i risultati.
    \subsection{Scrum}
    E' uno dei modi di lavoro agile più famosi e utilizzati, il tempo di sviluppo si suddivide in sprint della durata di 2/3 settimane,
    prevede una retrospective dove il team analizza ciò che ha fatto nello sprint precedente e quali sono stati i problemi in modo da migliorarsi
    e una riunione giornaliera per discutere quali sono le attività da svolgere in data odierna.

\section{T}
\section{U}
\section{V}
    \subsection{Verifica}
    Insieme di attività chiamate in causa molte volte dal processo di sviluppo. Ha lo scopo di assicurare
    che un prodotto, sistema o servizio soddisfi i requisiti specificati e ne garantisce la qualità e l’affidabilità
    prevenendo eventuali problemi che possono verificarsi durante il suo utilizzo.

\section{W}
    \subsection{Way of working}
    Termine che riassume tutte le regole di lavoro e organizzazione del team.

\section{X}
\section{Y}
\section{Z}
