\newcounter{UC}
\newcounter{SUC}
\newcounter{SSUC}

\newcommand{\resetCounter}[1]{%
    \setcounter{#1}{0}
}

\newcommand{\UseCase}[1]{%
    \refstepcounter{UC}
    \resetCounter{SUC}
    \resetCounter{SSUC}
    \subsection{UC\arabic{UC} - #1}\label{UC:\arabic{UC}}
}

\newcommand{\SubUseCase}[1]{%
    \stepcounter{SUC}
    \resetCounter{SSUC}
    \subsubsection{UC\arabic{UC}.\arabic{SUC} - #1}
}

\newcommand{\SubSubUseCase}[1]{%
    \stepcounter{SSUC}  
    \paragraph{UC\arabic{UC}.\arabic{SUC}.\arabic{SSUC} - #1}
}

% COME UTILIZZARE I CASI D`USO:
% i comandi di use case vanno utilizzati al posto di subsection, subsubsection e paragraph

% \UseCase => rappresenta lo use case principale (UC1,UC2,UC3)
% \SubUseCase => rappresenta il primo sotto livello dello use case principale (UC1.1,UC2.1,UC3.3)
% \SubSubUseCase => rappresenta il secondo sotto livello dello lo use case principale (UC1.1.1,UC2.1.2,UC3.3.1)

\newcommand{\UCdsc}[5]{
    \begin{itemize}
        \item \textbf{Attore primario:}
         #1
        \item \textbf{Descrizione:} 
         #2
        \item \textbf{Precondizioni:}
         #3
        \item \textbf{Postcondizioni:}
        #4
        \item \textbf{Scenario principale:} 
         #5
    \end{itemize}
}


\GetTitleStringSetup{expand}
\section{Casi d'uso}
% \subsection{Attore}
% Poiché per lo svolgimento del progetto non è necessario gestire permessi differenti per l'accesso alle funzionalità, l'attore che interagisce con il nostro software è unico, denominato "Utente".\\
% \textbf{Utente:} soggetto che utilizza la web application, sfruttandone le funzionalità.


% UC1 - UC11
