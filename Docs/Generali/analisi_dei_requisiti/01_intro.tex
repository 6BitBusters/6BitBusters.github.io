% Insert content here
\section{Introduzione}
\subsection{Scopo del documento}
Questo documento di \textit{analisi dei requisiti} descrive le funzionalità richieste per il progetto, estratte dalla presentazione del capitolato e dagli incontri con il proponente. 
Include un elenco dei casi d'uso, rappresentati con diagrammi UML, e una specifica dei requisiti.\\
Definisce la base per comprendere le esigenze del sistema e garantisce una progettazione e sviluppo coerenti con gli obiettivi del progetto.
\subsection{Scopo del prodotto}
Lo scopo del prodotto proposto è quello di fornire un'interfaccia web avanzata progettata per visualizzare i dati in modo chiaro e interattivo.
La soluzione mostra informazioni provenienti da \textit{database} e \textit{REST API}, usando istogrammi 3D navigabili, offrendo un'esperienza visiva coinvolgente.\\
Oltre alla visualizzazione 3D, il sistema include anche una vista tabellare dei dati garantendo un'analisi completa e versatile.\\
Il prodotto è pensato ad aiutare gli utenti a comprendere e elaborare i dati, supportando decisioni rapide grazie alla \textit{data visualization}.
\textit{data visualization}.
\subsection{Glossario}
Al fine di evitare incomprensioni o ambiguità relative alla terminologia
usata, viene fornito un Glossario nel file
\textit{Glossario V1.0.0}. In questo documento tali termini saranno facilmente identificabili
grazie a una lettera G a pedice.

Al fine di evitare incomprensioni o ambiguità riguardo alla terminologia, viene fornito un Glossario nel file \textit{Glossario V1.0.0}. Contiene definizioni precise dei termini tecnici. In questo documento tali termini sono contrassegnati con una lettera G a pedice.
\subsection{Riferimenti normativi}
\begin{itemize}
      \item {\textit{Norme di progetto}}
      \item \href{https://www.math.unipd.it/~tullio/IS-1/2024/Progetto/C5.pdf}
            {Capitolato d'appalto C5 - Sanmarco Informatica S.p.A.: 3Dataviz}
\end{itemize}
\subsection{Riferimenti informativi}
\begin{itemize}
      \item \href{https://www.math.unipd.it/~tullio/IS-1/2024/Dispense/T05.pdf}
            {Slide T5 - Corso di Ingegneria del Software - Analisi dei requisiti}
      \item \href{https://www.math.unipd.it/~rcardin/swea/2022/Diagrammi\%20Use\%20Case.pdf}
            {Slide P - Corso di Ingegneria del Software - Diagrammi degli use cases}
\end{itemize}