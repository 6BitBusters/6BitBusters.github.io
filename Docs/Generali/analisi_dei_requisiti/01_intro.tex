\section{Introduzione}
\subsection{Scopo del documento}
Questo documento di \textit{analisi dei requisiti} descrive le funzionalità richieste per il progetto, estratte dalla presentazione del capitolato e dagli incontri con il proponente. 
Include un elenco dei casi d'uso, rappresentati con diagrammi UML, e una specifica dei requisiti.\\
Definisce la base per comprendere le esigenze del sistema e garantisce una progettazione e sviluppo coerenti con gli obiettivi del progetto.

\subsection{Scopo del prodotto}
\textit{3Dataviz} è un prodotto ideato dall'azienda \textit{Sanmarco Informatica S.p.A.} per semplificare e rendere più accessibile la visualizzazione dei dati.\\
Il progetto si basa sul concetto di data visualization, che consiste nel trasformare i dati in grafici e rappresentazioni visive, sfruttando la capacità del cervello umano di elaborare rapidamente le immagini. 
Questo approccio facilita il processo decisionale e migliora la comprensione delle informazioni.\\
L’obiettivo principale è lo sviluppo di un’interfaccia web che trasforma dati provenienti da diverse fonti (come database e REST API) in grafici 3D interattivi e navigabili. 
Inoltre, i dati potranno essere consultati anche in formato tabellare, offrendo una visione alternativa ma altrettanto utile.  

\subsection{Glossario}
Per chiarire i termini tecnici o ambigui si utilizza un glossario disponibile nel file \textit{Glossario}.\\
Tutti i termini che richiedono spiegazioni sono indicati con il pedice “g”. \\
Questa convenzione consente un rapido collegamento tra il testo principale e la relativa spiegazione dettagliata nel glossario, garantendo coerenza e chiarezza nella comunicazione.

\subsection{Riferimenti normativi}
\begin{itemize}
      \item {\textit{Norme di progetto}}
      \item \href{https://www.math.unipd.it/~tullio/IS-1/2024/Progetto/C5.pdf}
            {Capitolato d'appalto C5 - Sanmarco Informatica S.p.A.: 3Dataviz}
\end{itemize}

\subsection{Riferimenti informativi}
\begin{itemize}
      \item \href{https://www.math.unipd.it/~tullio/IS-1/2024/Dispense/T05.pdf}
            {Slide T5 - Corso di Ingegneria del Software - Analisi dei requisiti}
      \item \href{https://www.math.unipd.it/~rcardin/swea/2022/Diagrammi\%20Use\%20Case.pdf}
            {Slide P - Corso di Ingegneria del Software - Diagrammi degli use cases}
\end{itemize}
