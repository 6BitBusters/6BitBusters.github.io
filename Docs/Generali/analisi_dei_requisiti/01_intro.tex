\section{Introduzione}
\subsection{Scopo del documento}
Questo documento di \textit{analisi dei requisiti} descrive le funzionalità richieste per il progetto. Le informazioni provengono dal capitolato e dagli incontri con il proponente. Include un elenco dei casi d'uso, rappresentati con diagrammi UML, e una specifica dei requisiti.\\
Il documento è la base per comprendere le esigenze del sistema. Serve a garantire che la progettazione e lo sviluppo siano coerenti con gli obiettivi del progetto.

\subsection{Scopo del prodotto}
Il prodotto è un'interfaccia web avanzata progettata per visualizzare i dati in modo chiaro e interattivo.
La soluzione mostra informazioni provenienti da \textit{database} e \textit{REST API}, usando istogrammi 3D navigabili. Questo offre un'esperienza visiva coinvolgente.\\
Oltre alla visualizzazione 3D, il sistema include anche una vista tabellare dei dati. Questo permette un'analisi completa.\\
Il prodotto aiuta gli utenti a comprendere e elaborare i dati, supportando decisioni rapide grazie alla \textit{data visualization}.

\subsection{Glossario}
Per evitare confusione, è fornito un Glossario nel file \textit{Glossario V1.0.0}. Contiene definizioni precise dei termini tecnici. I termini rilevanti sono contrassegnati con una lettera G a pedice.

\subsection{Riferimenti normativi}
\begin{itemize}
      \item {\textit{Norme di progetto}}
      \item \href{https://www.math.unipd.it/~tullio/IS-1/2024/Progetto/C5.pdf}
            {Capitolato d'appalto C5 - Sanmarco Informatica S.p.A.: 3Dataviz}
\end{itemize}

\subsection{Riferimenti informativi}
\begin{itemize}
      \item \href{https://www.math.unipd.it/~tullio/IS-1/2024/Dispense/T05.pdf}
            {Slide T5 - Corso di Ingegneria del Software - Analisi dei requisiti}
      \item \href{https://www.math.unipd.it/~rcardin/swea/2022/Diagrammi%20Use%20Case.pdf}
            {Slide P - Corso di Ingegneria del Software - Diagrammi degli use cases}
\end{itemize}
