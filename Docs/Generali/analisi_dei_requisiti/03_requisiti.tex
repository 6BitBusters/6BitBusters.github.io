\newcommand{\requirementsTable}[1]{

\renewcommand{\arraystretch}{1.5}
\rowcolors{2}{pari}{dispari}
\begin{longtable}{ %0.87
		>{\centering}M{0.15\textwidth} 
		>{\centering}M{0.5\textwidth}
		>{\centering}M{0.20\textwidth}
		>{\centering\arraybackslash}M{0.15\textwidth} 
		 }
	\rowcolorhead
	\headertitle{Requisito} &
	\centering \headertitle{Descrizione} &	
	\headertitle{Fonte}
	\endfirsthead	
	\endhead
	
	#1

\end{longtable}
\vspace{2em}

}

\newcommand{\SrcToReqTable}[1]{

\renewcommand{\arraystretch}{1.5}
\rowcolors{2}{pari}{dispari}
\begin{longtable}{ %0.87
		>{\centering}M{0.5\textwidth} 
		>{\centering\arraybackslash}M{0.20\textwidth} 
		 }
	\rowcolorhead
	\headertitle{Fonte} &
	\centering\headertitle{Requisito}
	\endfirsthead	
	\endhead
	
	#1

\end{longtable}
\vspace{2em}

}

\newcommand{\ReqToUCTable}[1]{

\renewcommand{\arraystretch}{1.5}
\rowcolors{2}{pari}{dispari}
\begin{longtable}{ %0.87
		>{\centering}M{0.5\textwidth} 
		>{\centering\arraybackslash}M{0.20\textwidth} 
		 }
	\rowcolorhead
	\headertitle{Requisito} &
	\centering\headertitle{Fonte}
	\endfirsthead	
	\endhead
	
	#1

\end{longtable}
\vspace{2em}

}

\newcommand{\requirementsSummaryTable}[1]{

\renewcommand{\arraystretch}{1.5}
\rowcolors{2}{pari}{dispari}
\begin{longtable}{ %0.87
		>{\centering}M{0.3\textwidth} 
		>{\centering}M{0.15\textwidth}
		>{\centering}M{0.15\textwidth}
		>{\centering}M{0.15\textwidth}
		>{\centering\arraybackslash}M{0.15\textwidth} 
		 }
	\rowcolorhead
	\headertitle{Tipologia} &
	\centering \headertitle{Obbligatorio} &	
	\headertitle{Desiderabile} &
	\headertitle{Opzionale} &
	\headertitle{Totale}
	\endfirsthead	
	\endhead
	
	#1

\end{longtable}
\vspace{2em}

}

\section{Requisiti}
\subsection{Introduzione}
Sono stati definiti dei requisiti codificati in base all’ambito di competenza e ad un numero seriale per
tenerne meglio traccia, inoltre nelle tabelle sottostanti sono fornite di descrizione e classificazione di ciascun
requisito. Il codice di ciascuno requisito è formato da:
\begin{itemize}
    \item \textbf{R} : Sta per requisito e serve a definire il dominio del codice rendendo subito intuibile che si tratti di
    un requisito;
    \item Lettera di tipologia:
    \begin{itemize}
        \item \textbf{F}: Funzionale;
        \item \textbf{Q}: Di qualità;
        \item \textbf{D}: Di vincolo;
        \item \textbf{P}: Prestazionale;
    \end{itemize}
\end{itemize}
\subsection{Requisiti funzionali}
\requirementsTable{
    RF1 & L’utente deve poter visualizzare i dataset proposti & Obbligatorio & UC1 \tabularnewline
    RF1.1 & L’utente deve poter visualizzare i dettagli dataset selezionato & Obbligatorio & UC1.1\par UC1.1.1\par UC1.1.2\par UC1.1.3\par \tabularnewline
    RF1.2 & L’utente deve poter caricare il dataset selezionato & Obbligatorio & UC1.2 \tabularnewline

    RF2 & L’utente deve poter visualizzare i dati del dataset selezionato in forma tabellare & Obbligatorio & UC2 \tabularnewline
    
    RF3 & L’utente deve poter visualizzare i dati del dataset selezionato in forma di grafico tridimensionale a barre verticali & Obbligatorio & UC3 \tabularnewline
    
    RF4 & L’utente deve poter visualizzare gli assi del grafico tridimensionale  & Obbligatorio & UC4 \tabularnewline
    RF4.1 & L’utente deve poter visualizzare l`asse globale X del grafico tridimensionale  & Obbligatorio & UC4.1 \tabularnewline
    RF4.2 & L’utente deve poter visualizzare l`asse globale Y del grafico tridimensionale  & Obbligatorio & UC4.2 \tabularnewline
    RF4.3 & L’utente deve poter visualizzare l`asse globale Z del grafico tridimensionale  & Obbligatorio & UC4.3 \tabularnewline
    
    RF5 & L’utente deve poter visualizzare la legenda del grafico tridimensionale  & Obbligatorio & UC5 \tabularnewline
 
    RF6 & L’utente deve poter effettuare un`azione di pan nell` ambiente, spostando la telecamera sugli assi locali Y e X & Obbligatorio & UC6 \tabularnewline
    
    RF7 & L’utente deve poter ruotare la telecamera a suo piacimento per visualizzare meglio il grafico tridimensionale & Obbligatorio & UC7 \tabularnewline
    RF7.1 & L’utente deve poter ruotare la telecamera attorno all`asse globale X & Obbligatorio & UC7.1 \tabularnewline
    RF7.2 & L’utente deve poter ruotare la telecamera attorno all`asse globale Y & Obbligatorio & UC7.2 \tabularnewline
    
    RF8 & L’utente deve poter muovere la telecamera all`interno dell` ambiente & Obbligatorio & UC8 \tabularnewline
    RF8.1 & L’utente deve poter muovere la telecamera sull`asse globale X & Obbligatorio & UC8.1 \tabularnewline
    RF8.2 & L’utente deve poter muovere la telecamera sull`asse globale Y & Obbligatorio & UC8.2 \tabularnewline
    RF8.3 & L’utente deve poter muovere la telecamera sull`asse globale Z & Obbligatorio & UC8.3 \tabularnewline
    
    RF9 & L’utente deve poter riposizionare la telecamera alla sua posizione iniziale & Obbligatorio & UC9 \tabularnewline
    
    RF10 & L’utente deve poter visualizzare i valori del dataset selezionando le barre opportune verticali del grafico tridimensionale & Obbligatorio & UC10 \tabularnewline
    
    RF11 & L’utente deve poter selezionare elementi rappresentanti i dati del dataset & Obbligatorio & UC11 \tabularnewline
    RF11.1 & L’utente deve poter selezionare le barre verticali del grafico tridimensionale & Obbligatorio & UC11.1 \tabularnewline
    RF11.1.1 & L’utente selezionando una barra verticale del grafico tridimensionale fa si che la corrispondente cella della tabella venga evidenziata & Obbligatorio & UC11.1.1 \tabularnewline
    RF11.2 & L’utente deve poter selezionare le celle della tabella & Obbligatorio & UC11.2 \tabularnewline
    RF11.2.1 & L’utente deve poter selezionare una cella della tabella fa si che la corrispondente barra del grafico tridimensionale venga evidenziata  & Obbligatorio & UC11.2.1 \tabularnewline
    RF11.2.2 & L’utente deve poter selezionare una cella della tabella fa si che la telecamera cambi posizione mettendo in primo piano la corrispondente barra del grafico tridimensionale & Obbligatorio & UC11.2.2 \tabularnewline

    RF12 & Il sistema modifica la trasparenza delle barre del grafico secondo azioni dell'utente & Obbligatorio & UC12 \tabularnewline
    RF12.1 & L’utente deve poter selezionare una barra la mette in evidenza andando a rendere piu` trasparenti quelle che hanno un valore maggiore o minore& Obbligatorio & UC12.1 \tabularnewline
    RF12.2 & L’utente deve poter scegliere un numero "X" che verrà utilizzato per rendere piu` trasparenti le barre, i quali valori non rientrano tra i primi o ultimi X & Obbligatorio & UC12.2 \tabularnewline
    RF12.3 & L’utente deve poter scegliere se rendere piu` trasparenti le barre con valori superiori o inferiori rispetto al valor medio globale & Obbligatorio & UC12.3 \tabularnewline
    
    RF13 & L’utente deve poter di scegliere visualizzare il piano parallelo alla base che rappresenta il valor medio globale & Facoltativo & UC13 \tabularnewline
    
    RF14 & L’utente riceve un errore se il caricamento di un dataset fallisce & Obbligatorio & UC14 \tabularnewline
    }

\subsection{Requisiti qualitativi}
    
\requirementsTable{
    RQ1 & Il software deve essere sviluppato seguendo le metriche e il modello di qualità descritti nel documento \textit{Norme di Progetto} & Obbligatorio & Decisione interna \tabularnewline
    RQ2 & Il software deve essere sviluppato pubblicando il codice sorgente sul repository Github dedicato & Obbligatorio & Decisione interna \tabularnewline
    RQ3 & Il software deve essere sviluppato in modo tale da supportare grandi volumi di dati & Obbligatorio & Capitolato \tabularnewline
    RQ4 & L'architettura del software deve permettere con agilità di poter aggiungere nuove funzionalità e possibilità di iterazione con il grafico  & Obbligatorio & Capitolato \tabularnewline
    RQ5 & La suite di test deve essere robusta per garantire ciò specificato in RQ4 & Obbligatorio & Capitolato \tabularnewline
    RQ6 & Il software deve essere testato attraverso test di tipo e2e & Facoltativo & Capitolato \tabularnewline
    % CHIEDERE AL PROPONENTE  
}
        
\subsection{Requisiti di vincolo}
\requirementsTable{
    RV1 & Il software deve essere sviluppato utilizzando Typescript come linguaggio di programmazione primario & Obbligatorio & Decisione interna \tabularnewline
    RV2 & Il software deve essere sviluppato utilizzando la libreria React per la creazione di un interfaccia visiva sempice & Obbligatorio & Decisione interna \tabularnewline
    RV3 & Il software deve essere sviluppato utilizzando la libreria Three.js & Obbligatorio & Decisione interna \tabularnewline
    
    % PLACEHOLDER -> CHIEDERE AL PROPONENTE
    RV4 & l software deve essere compatibile dalla versione 110 del browser Chrome  & Obbligatorio & Decisione interna \tabularnewline
}
\subsection{Requisiti prestazionali}

\subsection{Tracciamento requisiti}
