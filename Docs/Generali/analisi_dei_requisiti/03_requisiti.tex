\newcounter{RF1}
\newcounter{RF2}
\newcounter{RF3}
\newcounter{RV1}
\newcounter{RV2}
\newcounter{RV3}
\newcounter{RP1}
\newcounter{RP2}
\newcounter{RP3}
\newcounter{RQ1}
\newcounter{RQ2}
\newcounter{RQ3}

\newcounter{RF}
\newcounter{RV}
\newcounter{RP}
\newcounter{RQ}

\newcounter{M}
\newcounter{D}
\newcounter{O}

\newcommand{\RFM}{
    \stepcounter{M}
    \stepcounter{RF}
    \stepcounter{RF1}
    RF1.\arabic{RF}
}
\newcommand{\RFD}{
    \stepcounter{D}
    \stepcounter{RF}
    \stepcounter{RF2}
    RF2.\arabic{RF}
}
\newcommand{\RFO}{
    \stepcounter{O}
    \stepcounter{RF}
    \stepcounter{RF3}
    RF3.\arabic{RF}
}
\newcommand{\RVM}{
    \stepcounter{M}
    \stepcounter{RV}
    \stepcounter{RV1}
    RV1.\arabic{RV}
}
\newcommand{\RVD}{
    \stepcounter{D}
    \stepcounter{RV}
    \stepcounter{RV2}
    RV2.\arabic{RV}
}
\newcommand{\RVO}{
    \stepcounter{O}
    \stepcounter{RV}
    \stepcounter{RV3}
    RV3.\arabic{RV}
}
\newcommand{\RPM}{
    \stepcounter{M}
    \stepcounter{RP}
    \stepcounter{RP1}
    RP1.\arabic{RP}
}
\newcommand{\RPD}{
    \stepcounter{D}
    \stepcounter{RP}
    \stepcounter{RP2}
    RP2.\arabic{RP}
}
\newcommand{\RPO}{
    \stepcounter{O}
    \stepcounter{RP}
    \stepcounter{RP3}
    RP3.\arabic{RP}
}
\newcommand{\RQM}{
    \stepcounter{M}
    \stepcounter{RQ}
    \stepcounter{RQ1}
    RQ1.\arabic{RQ}
}
\newcommand{\RQD}{
    \stepcounter{D}
    \stepcounter{RQ}
    \stepcounter{RQ2}
    RQ2.\arabic{RQ}
}
\newcommand{\RQO}{
    \stepcounter{O}
    \stepcounter{RQ}
    \stepcounter{RQ3}
    RQ3.\arabic{RQ}
}


\newcommand{\requirementsTable}[1]{

\renewcommand{\arraystretch}{1.5}
\rowcolors{2}{pari}{dispari}
\begin{longtable}{ %0.87
		>{\centering}M{0.15\textwidth} 
		>{\centering}M{0.5\textwidth}
		>{\centering}M{0.20\textwidth}
		>{\centering\arraybackslash}M{0.15\textwidth} 
		 }
	\rowcolorhead
	\headertitle{Requisito} &
	\centering \headertitle{Descrizione} &	
	\headertitle{Fonte}
	\endfirsthead	
	\endhead
	
	#1

\end{longtable}
\vspace{2em}

}

\newcommand{\SrcToReqTable}[1]{

\renewcommand{\arraystretch}{1.5}
\rowcolors{2}{pari}{dispari}
\begin{longtable}{ %0.87
		>{\centering}M{0.5\textwidth} 
		>{\centering\arraybackslash}M{0.20\textwidth} 
		 }
	\rowcolorhead
	\headertitle{Fonte} &
	\centering\headertitle{Requisito}
	\endfirsthead	
	\endhead
	
	#1

\end{longtable}
\vspace{2em}

}

\newcommand{\ReqToUCTable}[1]{

\renewcommand{\arraystretch}{1.5}
\rowcolors{2}{pari}{dispari}
\begin{longtable}{ %0.87
		>{\centering}M{0.5\textwidth} 
		>{\centering\arraybackslash}M{0.20\textwidth} 
		 }
	\rowcolorhead
	\headertitle{Requisito} &
	\centering\headertitle{Fonte}
	\endfirsthead	
	\endhead
	
	#1

\end{longtable}
\vspace{2em}

}

\newcommand{\requirementsSummaryTable}[1]{

\renewcommand{\arraystretch}{1.5}
\rowcolors{2}{pari}{dispari}
\begin{longtable}{ %0.87
		>{\centering}M{0.3\textwidth} 
		>{\centering}M{0.15\textwidth}
		>{\centering}M{0.15\textwidth}
		>{\centering}M{0.15\textwidth}
		>{\centering\arraybackslash}M{0.15\textwidth} 
		 }
	\rowcolorhead
	\headertitle{Tipologia} &
	\centering \headertitle{Obbligatorio} &	
	\headertitle{Desiderabile} &
	\headertitle{Opzionale} &
	\headertitle{Totale}
	\endfirsthead	
	\endhead
	
	#1

\end{longtable}
\vspace{2em}

}

\section{Requisiti}
\subsection{Introduzione}
Sono stati definiti dei requisiti codificati in base all’ambito di competenza e ad un numero seriale per
tenerne meglio traccia, inoltre nelle tabelle sottostanti sono fornite di descrizione e classificazione di ciascun
requisito. 
Per la codifica dei requisiti si faccia riferimento alle \textit{Norme di Progetto}
\subsection{Requisiti funzionali}
\requirementsTable{
    % VISUALIZZAZIONE DATASET
    \RFM & L’utente deve poter visualizzare i dataset  & \tabularnewline
    \RFM & L’utente deve poter visualizzare le informazioni primarie dei dataset  & \tabularnewline
    \RFM & L’utente deve poter visualizzare i dettagli dei dataset   & \tabularnewline

    % CARICAMENTO DATASET
    \RFM & L’utente deve poter caricare il dataset &  \tabularnewline
    \RFM & L’utente riceve un errore se il caricamento di un dataset fallisce & \tabularnewline

    % VISUALIZZAZIONE DEI DATI IN TABELLA
    \RFM & L’utente deve poter visualizzare i dati del dataset selezionato in forma tabellare   & \tabularnewline
    \RFM & L’utente deve poter visualizzare le proprietà del dato tramite le intestazioni della tabella   & \tabularnewline
    \RFM & L’utente deve poter visualizzare il valore del dato in una cella della tabella   & \tabularnewline
    
    % VISUALIZZAZIONE DEI DATI SUL GRAFICO
    \RFM & L’utente deve poter visualizzare i dati del dataset selezionato in forma di grafico tridimensionale a barre verticali   & \tabularnewline
    \RFM & L’utente deve poter visualizzare gli assi del grafico tridimensionale    & \tabularnewline
    \RFM & L’utente deve poter visualizzare l`asse globale X del grafico tridimensionale    &  \tabularnewline
    \RFM & L’utente deve poter visualizzare l`asse globale Y del grafico tridimensionale    & \tabularnewline
    \RFM & L’utente deve poter visualizzare l`asse globale Z del grafico tridimensionale    & \tabularnewline
    \RFM & L’utente deve poter visualizzare i valori delle barre del grafico tridimensionale nel dettaglio    & \tabularnewline
    %%%%%%%%%%%%%%%\RFO & Le barre de grafico 3D devono avere una colorazione appropriata che rendano i valori più veloci da comprendere   & \tabularnewline
 
    % TELECAMERA
    \RFO & L’utente deve poter muovere la telecamera in prima persona all`interno dell` ambiente &  \tabularnewline
    \RFO & L’utente deve poter muovere la telecamera sul asse globale X   &  \tabularnewline
    \RFO & L’utente deve poter muovere la telecamera sul asse globale Y   &  \tabularnewline
    \RFO & L’utente deve poter muovere la telecamera sul asse globale Z   &  \tabularnewline
    
    % TELECAMERA
    \RFM & L’utente deve poter effettuare un`azione di camera panning all`interno dell` ambiente   &  \tabularnewline
    \RFM & L’utente deve poter muovere la telecamera orizzontalmente spostandola sull'asse locale X  &  \tabularnewline
    \RFM & L’utente deve poter muovere la telecamera orizzontalmente spostandola sull'asse locale Y  &  \tabularnewline
    
    % TELECAMERA
    \RFM & L’utente deve poter eseguire un'azione di zoom indipendentemente dalla posizione della telecamera   & \tabularnewline
    
    % TELECAMERA
    \RFM & L’utente deve poter ruotare la telecamera all`interno dell` ambiente   &\tabularnewline
    \RFM & L’utente deve poter ruotare la telecamera attorno all`asse globale X   & \tabularnewline
    \RFM & L’utente deve poter ruotare la telecamera attorno all`asse globale Y   & \tabularnewline

    % TELECAMERA
    \RFM & L’utente deve poter riposizionare la telecamera alla sua posizione iniziale   & \tabularnewline
    
    % \RFM & L’utente deve poter visualizzare i valori di dettaglio di una singola barra del grafico 3D  &  \tabularnewline
    % \RFM & L’utente deve poter visualizzare il valore numerico che una singola barra del grafico 3D rappresenta &  \tabularnewline
    % \RFM & L’utente deve poter visualizzare le etichette relative al dato che una singola barra del grafico 3D rappresenta &  \tabularnewline
    
    % \RFM & L'utente deve poter occludere alla vista delle barre   &  \tabularnewline
    % \RFM & L'utente deve poter occludere alla vista delle barre con un valore superiore rispetto a quella selezionata  &  \tabularnewline
    % \RFM & L'utente deve poter occludere alla vista delle barre con un valore inferiore rispetto a quella selezionata  &  \tabularnewline
    % \RFM & L'utente deve poter tornare allo stato iniziale del grafico, senza quindi barre occultate  &  \tabularnewline
    
    % FILTRAGGIO VALORI
    \RFM & L'utente può eseguire un filtraggio sui valori e corrispettive barre  &  \tabularnewline

    \RFM & L'utente può eseguire un filtraggio su un numero arbitrario (N) di barre &  \tabularnewline
    \RFM & L'utente può eseguire un filtraggio sulle N barre più alte &  \tabularnewline
    \RFM & L'utente può eseguire un filtraggio sulle le N barre più basse &  \tabularnewline

    \RFM & L'utente può eseguire un filtraggio sul valor medio globale &  \tabularnewline
    \RFM & L'utente può eseguire un filtraggio sulle barre superiori al valor medio globale &  \tabularnewline
    \RFM & L'utente può eseguire un filtraggio sulle barre inferiori al valor medio globale &  \tabularnewline

    \RFM & L'utente può eseguire un filtraggio su una determinata barra &  \tabularnewline
    \RFM & L'utente può eseguire un filtraggio sulle barre superiori ad una determinata barra &  \tabularnewline
    \RFM & L'utente può eseguire un filtraggio sulle barre inferiore ad una determinata barra &  \tabularnewline

    \RFM & L'utente deve poter visualizzare una barra in primo piano &  \tabularnewline

    \RFM & L'utente deve poter ripristinare lo stato del grafico rispetto ai precedenti valori applicati &  \tabularnewline

    % VISUALIZZAZIONE VALOR MEDIO
    \RFM & L’utente deve poter di visualizzare un piano parallelo alla base che rappresenti il valor medio globale  &  \tabularnewline
    \RFD & L’utente deve poter visualizzare il piano parallelo alla base che rappresenta il valor medio di un singolo elemento dell'asse   &  \tabularnewline

    }

\subsection{Requisiti qualitativi}
    
\requirementsTable{
    \RQM & Il software deve essere sviluppato seguendo le metriche e il modello di qualità descritti nel documento \textit{Norme di Progetto}   & Decisione interna \tabularnewline
    \RQM & Il software deve essere sviluppato pubblicando il codice sorgente sul repository Github dedicato   & Decisione interna \tabularnewline
    \RQM & Il software deve essere sviluppato in modo tale da supportare grandi volumi di dati   & Capitolato \tabularnewline
    \RQM & L'architettura del software deve permettere con agilità di poter aggiungere nuove funzionalità e possibilità di iterazione con il grafico    & Capitolato \tabularnewline
    \RQM & La suite di test deve essere robusta per garantire ciò specificato in RQ4   & Capitolato \tabularnewline
    \RQD & Il software deve essere testato attraverso test di tipo e2e   & Capitolato \tabularnewline  
}
 
\subsection{Requisiti di vincolo}
\requirementsTable{
    \RVM & Il software deve essere sviluppato utilizzando Typescript come linguaggio di programmazione primario   & Decisione interna \tabularnewline
    \RVM & Il software deve essere sviluppato utilizzando la libreria React per la creazione di un interfaccia visiva sempice   & Decisione interna \tabularnewline
    \RVM & Il software deve essere sviluppato utilizzando la libreria Three.js   & Decisione interna \tabularnewline
    \RVM & Il software deve essere compatibile dalla versione 131 del browser Chrome  & Decisione interna \tabularnewline
    \RVM & Il software deve essere compatibile dalla versione 133 del browser Firefox  & Decisione interna \tabularnewline
    \RVM & Il software deve essere compatibile dalla versione 115 del browser Opera  & Decisione interna \tabularnewline
    \RVM & Il software deve essere compatibile dalla versione 131 del browser Microsoft Edge  & Decisione interna \tabularnewline
}
\newpage

\subsection{Requisiti prestazionali}
\requirementsTable{
    \RPM & Il software quando carica un dataset di dimensioni 100x100 non deve avere un tempo di caricamento superiore a 2 secondi & Decisione interna \tabularnewline
    \RPM & Il software quando carica un dataset di dimensioni 500x500 non deve avere un tempo di caricamento superiore a 5 secondi & Decisione interna \tabularnewline
    \RPM & Il software quando carica un dataset di dimensioni 1000x1000 non deve avere un tempo di caricamento superiore a 8 secondi & Decisione interna \tabularnewline
    \RPD & Il software quando carica un dataset di dimensioni maggiori di 1000x1000 non deve avere un tempo di caricamento superiore a 15 secondi & Decisione interna \tabularnewline
    \RPM & Il software non deve avere un tempo di risposta superiore a 5 secondi & Decisione interna \tabularnewline
}


\subsection{Tracciamento requisiti}

\subsubsection{Fonte - Requisiti}
\SrcToReqTable{
    Capitolato & RQ1.3\par RQ1.4\par RQ1.5\par RQ2.6\tabularnewline
    Decisione interna & RQ1.1\par
                        RQ1.2\par 
                        RV1.1\par 
                        RV1.2\par 
                        RV1.3\par 
                        RV1.4\par
                        RV1.5\par
                        RV1.6\par
                        RV1.7\par
                        RP1.1\par
                        RP1.2\par
                        RP1.3\par
                        RP2.4\par
                        RP1.5\par
                        \tabularnewline
    UC1         & RF \tabularnewline
    UC1.1       & RF \tabularnewline
    UC1.1.1     & RF \tabularnewline
    UC1.1.2     & RF \tabularnewline
    UC1.1.3     & RF \tabularnewline
    UC1.2       & RF \tabularnewline
    UC2         & RF \tabularnewline
    UC2         & RF \tabularnewline
    UC3         & RF \tabularnewline
    UC4         & RF \tabularnewline
    UC4.1       & RF \tabularnewline
    UC4.2       & RF \tabularnewline
    UC4.3       & RF \tabularnewline
    UC5         & RF \tabularnewline
    UC6         & RF \tabularnewline
    UC7         & RF \tabularnewline
    UC7.1       & RF \tabularnewline
    UC7.2       & RF \tabularnewline
    UC8         & RF \tabularnewline
    UC8.1       & RF \tabularnewline
    UC8.2       & RF \tabularnewline
    UC8.3       & RF \tabularnewline
    UC9         & RF \tabularnewline
    UC10        & RF \tabularnewline
    UC11        & RF \tabularnewline
    UC11.1      & RF \tabularnewline
    UC11.1.1    & RF \tabularnewline
    UC11.2      & RF \tabularnewline
    UC11.2.1    & RF \tabularnewline
    UC11.2.2    & RF \tabularnewline
    UC12        & RF \tabularnewline
    UC12.1      & RF \tabularnewline
    UC12.2      & RF \tabularnewline
    UC12.3      & RF \tabularnewline
    UC13        & RF \tabularnewline
    UC14        & RF \tabularnewline
    }

    \newpage
\subsubsection{Requisiti - Casi d'uso}
\ReqToUCTable{

    RF & UC1 \tabularnewline 
    RF & UC1.1\par UC1.1.1\par UC1.1.2\par UC1.1.3\tabularnewline
    RF & UC1.2 \tabularnewline
    RF & UC2 \tabularnewline
    RF & UC2 \tabularnewline
    RF & UC3 \tabularnewline
    RF & UC4 \tabularnewline
    RF & UC4.1 \tabularnewline
    RF & UC4.2 \tabularnewline
    RF & UC4.3 \tabularnewline
    RF & UC5 \tabularnewline
    RF & UC6 \tabularnewline
    RF & UC7 \tabularnewline
    RF & UC7.1 \tabularnewline
    RF & UC7.2 \tabularnewline
    RF & UC8 \tabularnewline
    RF & UC8.1 \tabularnewline
    RF & UC8.2 \tabularnewline
    RF & UC8.3 \tabularnewline
    RF & UC9 \tabularnewline
    RF & UC10 \tabularnewline
    RF & UC11 \tabularnewline
    RF & UC11.1 \tabularnewline
    RF & UC11.1.1 \tabularnewline
    RF & UC11.2 \tabularnewline
    RF & UC11.2.1 \tabularnewline
    RF & UC11.2.2 \tabularnewline
    RF & UC12 \tabularnewline
    RF & UC12.1 \tabularnewline
    RF & UC12.2 \tabularnewline
    RF & UC12.3 \tabularnewline
    RF & UC13 \tabularnewline
    RF & UC14 \tabularnewline

    RQ1.1 & Decisione interna\tabularnewline
    RQ1.2 & Decisione interna\tabularnewline
    RQ1.3 & Capitolato\tabularnewline
    RQ1.4 & Capitolato\tabularnewline
    RQ1.5 & Capitolato\tabularnewline
    RQ2.6 & Capitolato\tabularnewline

    RV1.1 & Decisione interna\tabularnewline
    RV1.2 & Decisione interna\tabularnewline
    RV1.3 & Decisione interna\tabularnewline
    RV1.4 & Decisione interna\tabularnewline
    RV1.5 & Decisione interna\tabularnewline
    RV1.6 & Decisione interna\tabularnewline
    RV1.7 & Decisione interna\tabularnewline

    RP1.1 & Decisione interna\tabularnewline
    RP1.2 & Decisione interna\tabularnewline
    RP1.3 & Decisione interna\tabularnewline
    RP2.4 & Decisione interna\tabularnewline
    RP1.5 & Decisione interna\tabularnewline

}

\newcounter{T}

\addtocounter{T}{\value{M}}
\addtocounter{T}{\value{D}}
\addtocounter{T}{\value{O}}



\subsubsection{Riepilogo requisiti}
\requirementsSummaryTable{
    Funzionale & \arabic{RF1} & \arabic{RF2} & \arabic{RF3} & \arabic{RF} \tabularnewline
    Di qualità & \arabic{RQ1} & \arabic{RQ2} & \arabic{RQ3} & \arabic{RQ} \tabularnewline
    Di vincolo & \arabic{RV1} & \arabic{RV2} & \arabic{RV3} & \arabic{RV} \tabularnewline
    Prestazionali & \arabic{RP1} & \arabic{RP2} & \arabic{RP3} & \arabic{RP} \tabularnewline
    \bottomrule
    \textbf{Totale requisiti} & \arabic{M} & \arabic{D} & \arabic{O} & \arabic{T}
    
}

