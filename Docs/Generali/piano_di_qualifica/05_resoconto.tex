\newcommand{\Met}[3]{
    \subsubsection{#1}
    \begin{figure}[h!] \centering
        \includegraphics[scale=0.55]{#2}
        \caption{Proiezione della#1nei vari periodi di progetto}
    \end{figure}
    \subsubsection*{RTB}
    #3
    \newpage
}

\section{Resoconto delle attività di verifica}

\subsection{Verifica della qualità dei processi}
In questa sezione sono presentati i risultati dell'attività di verifica svolta per valutare la qualità del processo.\\
Le misure riportate sono state calcolate utilizzando le formule e i concetti descritti nel documento \textit{Norme di Progetto}, oltre ai dati definiti nel documento \textit{Piano di Progetto}.

In seguito sono presentati i grafici relativi a diverse metriche significative del processo nei vari sprint.

\Met
{ % METRICA
    MPC1 - Planned Value
}
{ % GRAFICO
    template/images/PV.png
}
{ % DESCRIZIONE RTB
    Il grafico mostra una crescita dei costi regolare e prevedibile, coerente con una pianificazione accurata che suddivide equamente il budget tra i vari sprint.
    Questo è un valore positivo e significa che il team riesce a pianificare nel modo corretto le ore per ogni ruolo ad ogni print.
}

\Met
{ % METRICA
    MPC2 - Actual Cost
}
{ % GRAFICO
    template/images/AC.png
}
{ % DESCRIZIONE RTB
    Come il grafico fa notare il confronto con il Planned Value rivela una forte correlazione, suggerendo che le ore e i costi effettivamente impiegati per ogni ruolo sono in linea con le previsioni iniziali.
    Ciò indica una pianificazione accurata e una buona esecuzione del progetto.
}

\Met
{ % METRICA
    MPC3 - Earned Value
}
{ % GRAFICO
    template/images/EV.png
}
{ % DESCRIZIONE RTB
    Questo grafico illustra solamente il valore di Erned Value che rappresenta il valore delle attività realizzate nel progetto.
    Da solo rientra tra i valori limite, e rappresenta un valor positivo, ma spesso viene messo a confronto con le altre metriche.
    Analizzando quindi i 3 grafici di Earned Value, Planned Value e Actual Cost si può notare come siano tutti allineati. 
    Questo indica una gestione del progetto accurata, con una precisa corrispondenza tra le attività pianificate, quelle effettivamente svolte e i costi sostenuti.
}

\Met
{ % METRICA
    MPC4 - Estimated at Compleation
}
{ % GRAFICO
    template/images/EaC.png
}
{ % DESCRIZIONE RTB
    Il grafico mostra che l'Estimated at Completion (EAC) è molto vicino al Budget at Completion (BAC).
    Questo indica un'elevata precisione della stima iniziale dei costi e una gestione finanziaria efficace durante i periodi del progetto. 
}

\Met
{ % METRICA
    MPC5 - Estimate to Complete
}
{ % GRAFICO
    template/images/EtC.png
}
{ % DESCRIZIONE RTB
    Il grafico evidenzia una relazione inversa tra ETC e AC.
    La tendenza all'abbassamento dell'ETC e all'aumento dell'AC è coerente con l'andamento previsto. 
    Questa dinamica suggerisce una buona gestione delle risorse e un avanzamento del progetto conforme alla pianificazione.
}

\Met
{ % METRICA
    MPC6 - Cost Variance
}
{ % GRAFICO
    template/images/CV.png
}
{ % DESCRIZIONE RTB
    L'analisi del grafico evidenzia una sovrastima iniziale delle ore necessarie per completare le attività.
    Con il progredire del progetto, il team ha affinato le proprie stime, acquisendo una maggiore consapevolezza dei tempi di esecuzione e delle competenze individuali.
}

\Met
{ % METRICA
    MPC7 - Schedule Variance
}
{ % GRAFICO
    template/images/SV.png
}
{ % DESCRIZIONE RTB
    L'analisi del grafico evidenzia che, inizialmente, l'assenza di un WoW solido ha impedito al team di rispettare le tempistiche previste. 
    Con l'introduzione e il consolidamento del WoW, il team ha ottimizzato la pianificazione e l'esecuzione delle attività, migliorando significativamente il rispetto delle scadenze.
}

\Met
{ % METRICA
    MPC8 - Budget Variance
}
{ % GRAFICO
    template/images/BV.png
}
{ % DESCRIZIONE RTB
    L'analisi del grafico indica che il team sta impiegando un po' meno risorse rispetto alle previsioni iniziali.
    Questo dimostra una gestione efficiente e un'ottima capacità di adattamento alle esigenze del progetto.
}

\Met
{ % METRICA
    MPC9 - Requirements Stability Index
}
{ % GRAFICO
    template/images/RSI.png
}
{ % DESCRIZIONE RTB
    Il grafico mostra un iniziale Requirements Stability Index del 100\% in quanto nei primi due periodi il team si è
    concentrato sulla creazione e successivo aggiornamento di un WoW. 
    È stato solo dal terzo sprint in poi che il team ha iniziato ad analizzare e successivamente raffinare i requisiti, rendendoli più chiari e comprensibili per facilitare le successive fasi di progettazione.
}

\Met
{ % METRICA
    MPC10 - Completezza dei documenti
}
{ % GRAFICO
    template/images/CD.png
}
{ % DESCRIZIONE RTB  
    Come mostrato dal grafico il team non ha quasi mai avuto problemi riguardante la completezza dei documenti.
    Si può notare che solamente durante lo sprint 3 il team ha avuto delle difficoltà nella redazione del documento \textit{Analisi dei requisiti}.
    A ogni modo ad ogni sprint tutti i membri si sono impegnati al massimo per garantire il completamento delle attività assegnate.
}

\Met
{ % METRICA
    MPC11 - Metriche soddisfatte
}
{ % GRAFICO
    template/images/MS.png
}
{ % DESCRIZIONE RTB
    Come mostra il grafico l'iniziale carenza di competenze ha limitato la capacità del team di soddisfare tutte le metriche.
    Grazie all'apprendimento continuo e all'introduzione di strumenti automatizzati, il team ha colmato queste lacune apportando delle migliorie ai vari documenti.
}

\Met
{ % METRICA
    MPC15 - Rischi non previsti
}
{ % GRAFICO
    template/images/RISCHI.png
}
{ % DESCRIZIONE RTB
    Il grafico evidenzia una notevole stabilità dei rischi non previsti durante l'intero progetto.
    Questo è ovviamente un ottimo segno in quanto indica che il gruppo è stato in grado di gestire i rischi in modo efficace e che non sono 
    emersi nuovi rischi inaspettati.
}

\subsection{Verifica della qualità dei prodotti}
In questa sezione vengono analizzate le attività svolte per garantire la qualità dei prodotti realizzati.
 Le analisi effettuate comprendono sia i documenti che il software, al fine di garantire un risultato finale che soddisfi i requisiti stabiliti.\\
Per calcolare le seguenti misure abbiamo utilizzato le formule e le nozioni descritte nel documento di \textit{Norme di Progetto}.

\Met
{ % METRICA
    MPD1 - Indice Gulpease
}
{ % GRAFICO
    template/images/GUL.png
}
{ % DESCRIZIONE RTB
    L'analisi del grafico conferma che, nelle prime fasi del progetto, l'Indice di Gulpease non ha raggiunto i valori minimi, a causa della mancanza di conoscenze specifiche da parte del team.
    Solo dopo aver implementato un'automazione per il calcolo dell'indice, è stato possibile migliorarne i risultati nei periodi successivi.
}

\Met
{ % METRICA
    MPD2 - Errori grammaticali
}
{ % GRAFICO
    template/images/ERR-G.png
}
{ % DESCRIZIONE RTB
    L'analisi del grafico rivela un numero inizialmente elevato di errori ortografici, in particolare nei documenti \text{Norme di progetto} e \textit{Piano di progetto}.
    Tuttavia, si osserva una netta diminuzione di tali errori nei periodi e documenti successivi, indicando un miglioramento progressivo nella qualità della scrittura.
}



