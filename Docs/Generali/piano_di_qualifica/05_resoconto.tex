\section{Resoconto delle attività di verifica}

\subsection{Verifica della qualità dei processi}
In questa sezione sono presentati i risultati dell'attività di verifica svolta per valutare la qualità del processo.\\
Le misure riportate sono state calcolate utilizzando le formule e i concetti descritti nel documento \textit{Norme di Progetto}, oltre ai dati definiti nel documento \textit{Piano di Progetto}.

% \newcommand{\metricsResultsTable}[1]{
	 

\renewcommand{\arraystretch}{1.5}
\rowcolors{2}{pari}{dispari}
\begin{longtable}{ 
		>{\centering}M{0.40\textwidth} 
		>{\centering}M{0.25\textwidth} 
		>{\centering}M{0.25\textwidth}
		 }
	\rowcolorhead
	\headertitle{Metrica} &
	\headertitle{Valore} &
	\headertitle{Esito}
	\endfirsthead	
	\endhead
	
	#1

\end{longtable}
\vspace{0.2em}

}
% \metricsResultsTable{
%     Planned Value & - & -\tabularnewline
%     Actual Cost & - & -\tabularnewline 
%     Earned Value & - & -\tabularnewline  
%     Estimate at Completion & - & -\tabularnewline  
%     Estimate to Complete & - & -\tabularnewline  
%     Cost Variance & - & -\tabularnewline  
%     Schedule Variance & - & -\tabularnewline  
%     Budget Variance & - & -\tabularnewline  
%     Requirements Stability Index & - & -\tabularnewline  
%     Completezza dei documenti & - & -\tabularnewline  
%     Metriche soddisfatte & - & -\tabularnewline  
%     Code Coverage & - & -\tabularnewline
%     Statement Coverage & - & -\tabularnewline 
%     Branch Coverage & - & -\tabularnewline 
%     Rischi non previsti & - & -\tabularnewline  
% }

In seguito sono presentati i grafici relativi a diverse metriche significative del processo nei vari sprint.

\subsubsection{Planned Value}

\subsubsection{Actual Cost}

\subsubsection{Earned Value}

\subsubsection{Cost Variance}

\subsubsection{Schedule Variance}

\subsubsection{Budget Variance}

\subsubsection{Requirements Stability Index}

\subsubsection{Completezza dei documenti}

\subsubsection{Metriche soddisfatte}

\subsubsection{Rischi non previsti}

\subsection{Verifica della qualità dei prodotti}
In questa sezione vengono analizzate le attività svolte per garantire la qualità dei prodotti realizzati. Le analisi effettuate comprendono sia i documenti che il software, al fine di garantire un risultato finale che soddisfi i requisiti stabiliti.\\
Per calcolare le seguenti misure abbiamo utilizzato le formule e le nozioni descritte nel documento di \textit{Norme di Progetto}.

\subsubsection{Indice di Gulpease}

\subsubsection{Errori ortografici}
% Nella tabella sottostante sono riportati gli indici di Gulpease calcolati sulle versioni più recenti dei documenti elencati.\\
% Per ottenere tali valori, sono state considerate sezioni di testo in paragrafi, celle di tabelle e liste.

% \newcommand{\gulpeaseTable}[1]{
	 

\renewcommand{\arraystretch}{1.5}
\rowcolors{2}{pari}{dispari}
\begin{longtable}{ 
		>{\centering}M{0.45\textwidth} 
		>{\centering}M{0.25\textwidth} 
		>{\centering}M{0.25\textwidth}
		 }
	\rowcolorhead
	\headertitle{Documento} &
	\headertitle{Valore} &
	\headertitle{Esito}
	\endfirsthead	
	\endhead
	
	#1

\end{longtable}
\vspace{0.2em}

}
% \gulpeaseTable{
%     VE 2024-12-19 & 47 & Non superato\tabularnewline
%     VE 2024-11-28 & 52 & Superato\tabularnewline
%     VE 2024-10-30 & 47 & Non superato\tabularnewline
%     VE 2024-10-28 & 45 & Non superato\tabularnewline
%     VE 2024-10-24 & 56 & Superato\tabularnewline
%     VI 2024-12-13 & 47 & Non superato\tabularnewline
%     VI 2024-12-05 & 49 & Non superato\tabularnewline
%     VI 2024-11-22 & 60 & Superato\tabularnewline
%     VI 2024-11-14 & 63 & Superato\tabularnewline
%     VI 2024-11-08 & 48 & Non superato\tabularnewline
%     VI 2024-11-04 & 39 & Non superato\tabularnewline
%     VI 2024-10-25 & 53 & Superato\tabularnewline
%     VI 2024-10-19 & 45 & Non superato\tabularnewline
%     VI 2024-10-16 & 48 & Non superato\tabularnewline
    
% }


