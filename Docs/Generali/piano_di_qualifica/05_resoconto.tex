\newcommand{\Met}[3]{
    \subsubsection{#1}
    \begin{figure}[h!] \centering
        \includegraphics[scale=0.55]{#2}
        \caption{Proiezione della#1nei vari periodi di progetto}
    \end{figure}
    \subsubsection*{RTB}
    #3
    \newpage
}

\section{Resoconto delle attività di verifica}

\subsection{Verifica della qualità dei processi}
In questa sezione sono presentati i risultati dell'attività di verifica svolta per valutare la qualità del processo.\\
Le misure riportate sono state calcolate utilizzando le formule e i concetti descritti nel documento \textit{Norme di Progetto}, oltre ai dati definiti nel documento \textit{Piano di Progetto}.

In seguito sono presentati i grafici relativi a diverse metriche significative del processo nei vari sprint.

\Met
{ % METRICA
    MPC1 - Planned Value
}
{ % GRAFICO
    template/images/PV.png
}
{ % DESCRIZIONE RTB
    Si può notare dal grafico che i costi crescono in maniera quasi lineare senza sbalzi o curve sostanziali
    il che indica che il costo pianificato per ogni sprint viene scelto in modo corretto.
}

\Met
{ % METRICA
    MPC2 - Actual Cost
}
{ % GRAFICO
    template/images/AC.png
}
{ % DESCRIZIONE RTB
    Confontandolo con il Planned Value è molto simile e si può dedurre che le ore e successivamente i costi complessivi
    per ruolo non variano di molto suggerendo che il lavoro completato corrisponde spesso alla pianificazione.
    Questa coincidenza implica un processo positivo rispetto alla pianificazione del progetto.
}

\Met
{ % METRICA
    MPC3 - Earned Value
}
{ % GRAFICO
    template/images/EV.png
}
{ % DESCRIZIONE RTB
    Controntandolo con il Planned Value e Actual Cost si può notare come anche esso è simile a quest'ultimi 
    andando quindi a confermare un'ulteriore volta un processo positivo  rispetto alla pianificazione e successiva realizzazione del progetto.
}

\Met
{ % METRICA
    MPC4 - Estimated at Compleation
}
{ % GRAFICO
    template/images/EaC.png
}
{ % DESCRIZIONE RTB
    Come si vede dal grafico l' Estimated at compleation è molto vicina al valore ottimo non che il Budget at Compleation (BAC)
    Ciò significa che il costo prefentivato ad inizio progetto è corretto. 
}

\Met
{ % METRICA
    MPC5 - Estimate to Complete
}
{ % GRAFICO
    template/images/EtC.png
}
{ % DESCRIZIONE RTB
    Dall' analisi del grafico corrente e quell del Actual Cost si nota come l'Estimate to Complete tende a decrescere, come atteso,
    metre l'AC mostra un incremento proporzionale alla descrescita dell'ETC.
}

\Met
{ % METRICA
    MPC6 - Cost Variance
}
{ % GRAFICO
    template/images/CV.png
}
{ % DESCRIZIONE RTB
    Osservando il grafico si può notare che all'inizio del progetto il team ha sovrastimato un le ore di lavoro necessarie per completare
    le varie task. Si può notare che con il passare dei periodi il team ha imparato a conoscere i tempi e impegni dei singoli membri e quindi a stimare
    con più precisione le ore di lavoro per ogni singolo ruolo.
}

\Met
{ % METRICA
    MPC7 - Schedule Variance
}
{ % GRAFICO
    template/images/SV.png
}
{ % DESCRIZIONE RTB
    Osservando il grafico anche in questo caso si può notare che all'inizio del progetto il team non avendo a disposizione un WoW solido e strutturato 
    non abbia soddisfatto il requisito minimo di Schedule Variance. Con l'avvanzare dei periodi e versioni del WoW il team ha cominciato a lavorare in modo
    più stutturato e più efficace.
}

\Met
{ % METRICA
    MPC8 - Budget Variance
}
{ % GRAFICO
    template/images/BV.png
}
{ % DESCRIZIONE RTB
    Analizzando il grafico si capisce subito come il team stia utilizzando un piò meno risorse, ma non troppo, da quanto pianificato, ciò è un fattore positivo 
    in quanto dimostra che il team sta lavorando in modo efficiente.
}

\Met
{ % METRICA
    MPC9 - Requirements Stability Index
}
{ % GRAFICO
    template/images/RSI.png
}
{ % DESCRIZIONE RTB
    Si può notare dal grafico come inizialmente il Requirements Stability Index sia 100\% in quanto nei primi 2 periodi il team si è
    focalizzato sulla creazione e successivo aggiornamento di un WoW. Solo durante lo sprint 3 il team ha cominciato a analizzare quali fossero
    i requisiti del progetto e successivamente a raffinarli in modo tale da essere facilmente comprensibili per successivamente facilitare quanto più possibile la progettazione 
}

\Met
{ % METRICA
    MPC10 - Completezza dei documenti
}
{ % GRAFICO
    template/images/CD.png
}
{ % DESCRIZIONE RTB
    Come descritto dal grafico il team non ha quasi mai avuto problemi riguardante la completezza dei documenti, infatti ad ogni sprint tutti i membri si sono impegnati
    al massimo per garantire il completamento delle attività assegnate, nonche la redazione di più sezioni dei documenti.
}

\Met
{ % METRICA
    MPC11 - Metriche soddisfatte
}
{ % GRAFICO
    template/images/MS.png
}
{ % DESCRIZIONE RTB
    Nel corso dei primi periodi è evidente che non tutte le metriche fossero soddisfatte in quanto mancava tutta una serie di conoscenze
    che solo successivamente il team ha appreso. Inoltre con la creazione di automazioni il team è riuscito a soddisfare più metriche in quanto 
    velocizzano la creazione di feedback e di conseguenza la correzione o miglioramento dei prodotti.
}

\Met
{ % METRICA
    MPC15 - Rischi non previsti
}
{ % GRAFICO
    template/images/RISCHI.png
}
{ % DESCRIZIONE RTB
    Il grafico molstra come i rischi non previsti siano rimasti costanti durante tutti i periodi del progetto.
    Questo è ovviamente un ottimo segno in quanto indica che il gruppo è stato in grado di gestire i rischi in modo efficace e che non sono 
    emersi nuovi rischi inaspettati.
}


\subsubsection{Rischi non previsti}

\subsection{Verifica della qualità dei prodotti}
In questa sezione vengono analizzate le attività svolte per garantire la qualità dei prodotti realizzati. Le analisi effettuate comprendono sia i documenti che il software, al fine di garantire un risultato finale che soddisfi i requisiti stabiliti.\\
Per calcolare le seguenti misure abbiamo utilizzato le formule e le nozioni descritte nel documento di \textit{Norme di Progetto}.

\Met
{ % METRICA
    MPD1 - Indice Gulpease
}
{ % GRAFICO
    template/images/GUL.png
}
{ % DESCRIZIONE RTB
    Dalla valutazione del grafico si nota che, anche come detto in precedenza, all'inizio del progetto non avendo le conoscenze necessarie
    il team non ha rispettato l'Indice di Gulpease minimo.
    Con la successiva creazione di un'automazione si è riuscito a calcolare l'indice per tutti i documenti per successivamente migliorarne il valore nei periodi successivi.
}

\Met
{ % METRICA
    MPD2 - Errori grammaticali
}
{ % GRAFICO
    template/images/ERR-G.png
}
{ % DESCRIZIONE RTB
    Si nota come il numero di errori ortografici sia inizialmente alto soprattutto per il documento \textit{Norme di progetto} e 
    \textit{Piano di progetto} ma che successivamente tende a diminuire.
    Questo è dovuto al fatto che questi 2 sono stati i primi documenti scritti e che solo successivamente il team ha iniziato a prestare maggiore
    attenzione alla scrittura dei documenti in generale, motivo per il quale gli altri documenti tendono ad essere privi di errori.
}



