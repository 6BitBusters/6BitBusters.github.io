    \section{Introduzione}
    \subsection{Scopo del documento}
        Questo documento descrive i principi guida e le attività svolte dal team \textit{Six Bit Busters} per garantire la qualità dei processi e dei prodotti sviluppati nel progetto \textit{3Dataviz}.  
        Sono presentati i risultati delle analisi quantitative condotte per valutare le performance del team. Vengono inoltre evidenziate le eventuali criticità e descritte le azioni correttive adottate.
        L'obiettivo del documento è assicurare trasparenza e migliorare il lavoro del team.
        
    \subsection{Scopo del prodotto}
        \textit{3Dataviz} è un prodotto ideato dall'azienda \textit{Sanmarco Informatica S.p.A.} per semplificare e rendere più accessibile la visualizzazione dei dati.\\
        Esso mira a trasformare i dati in grafici e rappresentazioni visive, sfruttando la capacità del cervello umano di elaborare rapidamente le immagini. 
        Questo approccio facilita il processo decisionale e migliora la comprensione delle informazioni.\\
        L’obiettivo principale è lo sviluppo di un’interfaccia web che trasforma dati provenienti da diverse fonti (come database e REST API) in grafici 3D interattivi e navigabili. 
        I dati potranno essere consultati anche in formato tabellare, offrendo una visione alternativa ma altrettanto utile.  
        
    \subsection{Glossario}
        Per chiarire i termini tecnici o ambigui si utilizza un glossario disponibile nel file \textit{Glossario}.\\
        Tutti i termini che richiedono spiegazioni sono indicati con il pedice “g”. \\
        Questa convenzione consente un rapido collegamento tra il testo e la relativa spiegazione dettagliata nel glossario, garantendo coerenza e chiarezza.

    \subsection{Riferimenti}
        \subsubsection{Riferimenti normativi}
        \begin{itemize}
            \item Capitolato d'appalto C5, \textit{Sanmarco Informatica S.p.A.} - 3Dataviz:  \\
            \url{https://www.math.unipd.it/~tullio/IS-1/2024/Progetto/C5.pdf}
            \item Materiale didattico - Corso Ingegneria del Software 2024/2025 - Regolamento del Progetto Didattico: \\ 
            \url{https://www.math.unipd.it/~tullio/IS-1/2024/Dispense/PD1.pdf}
        \end{itemize}

        \subsubsection{Riferimenti informativi}
        \begin{itemize}
            \item Materiale didattico - Corso Ingegneria del Software 2024/2025 - Processi di ciclo di vita:  \\
            \url{https://www.math.unipd.it/~tullio/IS-1/2024/Dispense/T02.pdf} 
            \item Materiale didattico - Corso Ingegneria del Software 2024/2025 - Qualità del Software: \\ 
            \url{https://www.math.unipd.it/~tullio/IS-1/2024/Dispense/T07.pdf} 
            \item Materiale didattico - Corso Ingegneria del Software 2024/2025 - Qualità di Processo:  \\
            \url{https://www.math.unipd.it/~tullio/IS-1/2024/Dispense/T08.pdf}
        \end{itemize}
