\section{Specifica dei test}
L'esecuzione dei test rappresenta un passaggio fondamentale per verificare che 
il prodotto, nella sua totalità, soddisfi pienamente tutti i requisiti specificati
nel documento di \textit{Analisi dei Requisiti}.
I test si classificano in:
\begin{itemize}
    \item \textbf{Test di unità}: vengono stabiliti durante la progettazione di
    dettaglio e verificano il funzionamento delle singole unità software;
    \item \textbf{Test di integrazione}: vengono stabiliti durante la progettazione
    architetturale e verificano le interazioni tra le componenti;
    \item \textbf{Test di sistema}: vengono stabiliti durante l'analisi dei requisiti
    e verificano che il sistema nel suo complesso soddisfi i requisiti software;
    \item \textbf{Test di accettazione}: vengono effettuati insieme al proponente 
    durante la fase di collaudo, per accertare il soddisfacimento dei requisiti utente.
\end{itemize}
Per ciascun test, vengono forniti un codice identificativo, una descrizione, e
lo stato del test. 
Come specificato nel documento \textit{Norme di progetto}, vengono utilizzate le seguenti 
abbreviazioni per identificare lo stato dei test:
\begin{itemize}
    \item \textbf{NI}: Non Implementato;
    \item \textbf{S}: Superato;
    \item \textbf{NS}: Non Superato.
\end{itemize}

\newcommand{\testTable}[1]{
	 

\renewcommand{\arraystretch}{1.5}
\rowcolors{2}{pari}{dispari}
\begin{longtable}{ 
		>{\centering}M{0.15\textwidth} 
		>{\centering}M{0.60\textwidth} 
		>{\centering}M{0.10\textwidth}
		 }
	\rowcolorhead
	\headertitle{Codice} &
	\headertitle{Descrizione} &
	\headertitle{Stato}
	\endfirsthead	
	\endhead
	
	#1

\end{longtable}
\vspace{0.2em}

}
\subsection{Test di unità}
\testTable{
    TU1 & - & NI\tabularnewline
}
\subsection{Test di integrazione}
\testTable{
    TI1 & - & NI\tabularnewline
}
\subsection{Test di sistema}
\testTable{
    TS1 & Verifica che l'utente possa visualizzare la lista dei dataset disponibili & NI\tabularnewline
    TS2 & Verifica che l'utente possa visualizzare le informazioni primarie dei dataset & NI\tabularnewline
    TS3 & Verifica che l'utente possa visualizzare il nome dei dataset & NI\tabularnewline
    TS4 & Verifica che l'utente possa visualizzare la dimensione dei dataset & NI\tabularnewline
    TS5 & Verifica che l'utente possa visualizzare i dettagli di un dataset & NI\tabularnewline
    TS6 & Verifica che l'utente possa visualizzare la descrizione del contenuto di un dataset & NI\tabularnewline
    TS7 & Verifica che l'utente possa caricare nel sistema un dataset nella lista & NI\tabularnewline
    TS8 & Verifica che, in caso di fallito caricamento del dataset, 
        venga visualizzato un messaggio di errore & NI\tabularnewline
    TS9 & Verifica che l'utente possa visualizzare il dataset in forma tabellare & NI\tabularnewline
    TS10 & Verifica che l'utente possa visualizzare le intestazioni della tabella & NI\tabularnewline
    TS11 & Verifica che l'utente possa visualizzare i dati della tabella & NI\tabularnewline
    TS12 & Verifica che l'utente possa visualizzare il dataset in 
        forma di grafico 3D a barre verticali & NI\tabularnewline
    TS13 & Verifica che l'utente possa visualizzare gli assi del grafico & NI\tabularnewline
    TS14 & Verifica che l'utente possa visualizzare l'asse X con i valori appropriati & NI\tabularnewline
    TS15 & Verifica che l'utente possa visualizzare l'asse Y con i valori appropriati & NI\tabularnewline
    TS16 & Verifica che l'utente possa visualizzare l'asse Z con i valori appropriati & NI\tabularnewline
    TS17 & Verifica che l'utente possa spostare la telecamera lungo una linea retta senza modificare l'angolazione della telecamera & NI\tabularnewline
    TS18 & Verifica che l'utente possa spostare orizzontalmente la telecamera & NI\tabularnewline
    TS19 & Verifica che l'utente possa spostare verticalmente la telecamera & NI\tabularnewline
    TS20 & Verifica che l'utente possa compiere un'azione di zoom & NI\tabularnewline
    TS21 & Verifica che l'utente possa ruotare il grafico  & NI\tabularnewline
    TS22 & Verifica che l'utente possa ruotare il grafico attorno all'asse X & NI\tabularnewline
    TS23 & Verifica che l'utente possa ruotare il grafico attorno all'asse Y & NI\tabularnewline
    TS24 & Verifica che l'utente possa ruotare il grafico attorno all'asse Z & NI\tabularnewline
    TS25 & Verifica che l'utente possa riposizionare la telecamera alla sua posizione iniziale & NI\tabularnewline
    TS26 & Verifica che l'utente possa visualizzare i dettagli di una barra del grafico & NI\tabularnewline
    TS27 & Verifica che l'utente possa visualizzare l'altezza della barra & NI\tabularnewline
    TS28 & Verifica che l'utente possa visualizzare le etichette della barra & NI\tabularnewline
    TS29 & Verifica che l'utente possa filtrare le barre che soddisfano certe condizioni & NI\tabularnewline
    TS30 & Verifica che l'utente possa filtrare su un numero arbitrario (N) di barre & NI\tabularnewline
    TS31 & Verifica che l'utente possa filtrare le N barre più alte & NI\tabularnewline
    TS32 & Verifica che l'utente possa filtrare le N barre più basse & NI\tabularnewline
    TS33 & Verifica che l'utente possa filtrare sul valor medio globale & NI\tabularnewline
    TS34 & Verifica che l'utente possa filtrare le barre con altezza 
        superiore al valor medio globale & NI\tabularnewline
    TS35 & Verifica che l'utente possa filtrare le barre con altezza 
        inferiore al valor medio globale & NI\tabularnewline
    TS36 & Verifica che l'utente possa filtrare su una determinata barra & NI\tabularnewline
    TS37 & Verifica che l'utente possa filtrare le barre con altezza 
        superiore a una determinata barra & NI\tabularnewline
    TS38 & Verifica che l'utente possa filtrare le barre con altezza 
        inferiore a una determinata barra & NI\tabularnewline
    TS39 & Verifica che l'utente possa visualizzare una determinata barra in primo piano & NI\tabularnewline
    TS40 & Verifica che l'utente possa visualizzare la barra in primo piano & NI\tabularnewline
    TS41 & Verifica che l'utente possa rimuovere il filtro applicato al grafico & NI\tabularnewline
    TS42 & Verifica che l'utente possa visualizzare il piano parallelo alla 
        base del grafico che rappresenta il valor medio globale & NI\tabularnewline
    TS43 & Verifica che l'utente possa visualizzare il piano parallelo alla
        base del grafico che rappresenta il valor medio di un singolo elemento dell'asse & NI\tabularnewline
}
\subsection{Test di accettazione}
\testTable{
    TA1 & - & NI\tabularnewline
}
