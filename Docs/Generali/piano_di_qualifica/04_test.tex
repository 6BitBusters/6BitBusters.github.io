\section{Specifica dei test}
L'esecuzione dei test rappresenta un passaggio fondamentale per verificare che 
il prodotto, nella sua totalità, soddisfi pienamente tutti i requisiti specificati
nel documento di \textit{Analisi dei Requisiti}.
I test si classificano in:
\begin{itemize}
    \item \textbf{Test di unità}: vengono stabiliti durante la progettazione di
    dettaglio e verificano il funzionamento delle singole unità software;
    \item \textbf{Test di integrazione}: vengono stabiliti durante la progettazione
    architetturale e verificano le interazioni tra le componenti;
    \item \textbf{Test di sistema}: vengono stabiliti durante l'analisi dei requisiti
    e verificano che il sistema nel suo complesso soddisfi i requisiti software;
    \item \textbf{Test di accettazione}: vengono effettuati insieme al proponente 
    durante la fase di collaudo, per accertare il soddisfacimento dei requisiti utente.
\end{itemize}
Per ciascun test, vengono forniti un codice identificativo, una descrizione, e
lo stato del test. 
Come specificato nel documento Norme di progetto, vengono
utilizzate le seguenti abbreviazioni per identificare lo stato dei test:
