% Insert content here
\section{Informazioni generali}
\subsection{Dettagli sull'incontro}
\begin{itemize}
    \item \textbf{Luogo}: Discord
    \item \textbf{Data}: 13-12-2024
    \item \textbf{Ora di inizio}: 16:30
    \item \textbf{Ora di fine}: 17:15
    \item \textbf{Partecipanti}:
    \begin{itemize}
        \item Bergamin Elia
        \item Chilese Elena
        \item Diviesti Filippo
        \item Djossa Edgar
        \item Pincin Matteo 
        \item Soranzo Andrea  
    \end{itemize}
\end{itemize}

\section{Motivo della riunione}
Questo incontro è servito per confrontarci sul lavoro svolto a metà del terzo sprint, riportare eventuali dubbi da parte dei membri del team 
e discuterli per arrivare a una soluzione.\\
In particolare, abbiamo deciso come riportare le metriche nelle \textit{Norme di progetto}, abbiamo discusso sui diagrammi UML relativi ai casi d'uso, come gestire l'automazione del glossario e come ristrutturare il repository dopo la RTB.
\\Inoltre, è stato discusso come organizzare il prossimo sprint, considerando che si svolgerà durante il periodo delle festività natalizie, al fine di garantire una pianificazione efficace e il rispetto delle scadenze.

\section{Resoconto}
% insert here all the steps
\subsection{Metriche}
Ci siamo chiesti se la sezione delle metriche dovesse essere riportata come sezione autonoma o come sottosezione all'interno delle \textit{Norme di progettto}, abbiamo deciso che inseriremo una sezione dedicata alle metriche.

\subsection{Sprint 4}
Poiché il prossimo sprint si terrà dal 23 dicembre 2024 al 3 gennaio 2025, in concomitanza con le festività natalizie, abbiamo discusso le disponibilità di ciascun membro del team per gestire al meglio l'assegnazione dei ruoli e delle ore di lavoro. In particolare, il ruolo di programmatore sarà affidato ai membri che hanno acquisito maggiore dimestichezza con le tecnologie React e Three.js e che daranno disponibilità nel periodo indicato, così da avviare il PoC al più presto.

\subsection{Diagrammi UML dei Casi d'uso}
Sono emerse alcune incertezze da parte dei membri del team sulla qualità dei diagrammi UML relativi ai casi d'uso e sulla loro adeguatezza per una corretta analisi dei requisiti. Per affrontare questi dubbi, si è deciso di fissare due incontri: uno con il proponente e uno con il Prof. Cardin. L'obiettivo è ottenere un riscontro sul lavoro svolto, porre eventuali domande per migliorare la qualità del nostro lavoro e proseguire al meglio con la stesura dei casi d'uso.  

\subsection{Automazione Glossario}
Durante la discussione sul glossario, è emersa la necessità di prevedere alcuni giorni prima della consegna della RTB per testare il funzionamento dell'inserimento automatico dei pedici. 
\\Considerando la possibilità che l'automazione possa fallire e non si riuscisse a risolvere il problema in tempo, il team ha stabilito che, come soluzione alternativa, si procederà manualmente con l'inserimento dei pedici per garantire comunque la consegna puntuale dei documenti. 
Contestualmente, è stato deciso di assegnare alta priorità alla risoluzione del problema dell'automazione subito dopo la consegna, per evitare simili inconvenienti in futuro.

\subsection{Organizzazione directory del repository}
Per garantire maggior organizzazione e struttura all'interno del repository, è stato deciso che, dopo la consegna della RTB, il repository e la Github Page verranno riorganizzate per garantire una chiara distinzione tra il lavoro svolto per la RTB e quello per la PB. \\In particolare, la documentazione e i prodotti software relativi alle due baseline saranno separati in cartelle dedicate, così da migliorare l'ordine e la tracciabilità del lavoro.

\section{Prossimi obiettivi}
   \begin{itemize}
        \item Avanzamento delle attività relative allo Sprint 3;
        \item Effettuare incontri con proponente e Prof. Cardin.
    \end{itemize}