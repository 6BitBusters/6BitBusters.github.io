\section{Stima dei costi e dichiarazione degli impegni}

\setcounter{subsection}{0}
\subsection{Tabella dei costi}
Il gruppo ha deciso di dividersi equamente le ore di lavoro per ciascun ruolo da assumere durante svolgimeto del progetto. Di seguito tabella che indica le ore di
lavoro dei vari ruoli che si occupano del progetto e dei loro relativi costi.


\newcommand{\roleTable}[1]{

	\renewcommand{\arraystretch}{1.5}
	\rowcolors{2}{pari}{dispari}
	\begin{longtable}{ %0.87
		>{\centering}M{0.20\textwidth}
		>{\centering}M{0.10\textwidth}
		>{\centering}M{0.10\textwidth}
		>{\centering}M{0.10\textwidth}
		>{\centering\arraybackslash}M{0.15\textwidth}
		}
		\rowcolorhead
		\headertitle{Ruolo}           &
		\centering
		\headertitle{Ore ind.}        &
		\headertitle{Ore tot.}        &
		\headertitle{Costo (\euro/h)} &
		\headertitle{Costo Totale (\euro)}
		\endfirsthead
		\endhead

		#1

	\end{longtable}
	\vspace{1em}

}

    \roleTable{
    Responsabile& 9 & 54 & 30 &  1620\tabularnewline
    Amministratore& 7 & 42 & 20 &  840\tabularnewline
    Analista& 16 & 96 & 25 &  2400\tabularnewline
	Progettista& 20 & 120 & 25 &  3000\tabularnewline
	Programmatore & 22 & 132 & 15 &  1980 \\
    	Verificatore& 18 & 108 & 15 &  1620\tabularnewline
     \midrule[\heavyrulewidth] 
     \textbf{TOTALE}& 92 & 552 & - &  11460\tabularnewline

     \rowcolor{white}\caption{Distribuzione delle ore e costi totali}
    }




\noindent Il costo finale calcolato in base alle tariffe orarie dei ruoli e alle ore preventivate risulta di: \euro{11460}.\\
In totale il monte ore è pari a 552 ore, quindi 92 ore a testa per ciascun membro del gruppo.\\

\subsection{Percentuale di lavoro per ruolo}
\newcommand{\percentPie}[1]{
    \begin{center}
    \begin{tikzpicture}
    #1
    \end{tikzpicture}
    \end{center}
    \vspace{1em}
    
}

\percentPie {
    \pie{9.8/Responsabile, 7.6/Amministratore, 17.4/Analista, 21.7/Progettista,23.9/Programmatore, 19.6/Verificatore}
}

\newpage

\subsection{Analisi dei ruoli}
\noindent Riprendendo le lezioni svoltasi durante il corso, sono state chiare fin da subito le seguenti caratteristiche dei vari ruoli:
\begin{itemize}
    \item \textbf{Responsabile}: è un ruolo importante ma più di "rappresentanza" (ed è anche il ruolo più costoso). Per tali motivi abbiamo scelto di assegnare circa il 10\% delle ore totali, prevedendo quindi 9 ore a componente, per un totale di 54 ore.
    \item \textbf{Amministratore}: anch'esso è un ruolo più di "alto livello" in quanto volto ad assicurare l'efficacia di tecnologie e strumenti utilizzati durante lo sviluppo del progetto. Ad esso abbiamo assegnato 7 ore a componente, per un totale di 42 ore.
    \item \textbf{Analista}: poiché l'analisi iniziale verrà svolta in collaborazione con l'azienda (e successivamente rielaborata dal gruppo), anche a tale ruolo abbiamo voluto dare una certa importanza, in quanto c'è la necessità di comprendere e descrivere nel dettaglio i requisiti del progetto. A tale ruolo abbiamo quindi assegnato 16 ore individuali, per un totale di 96 ore.
    \item \textbf{Progettista}: si prevedono 20 ore individuali (per un totale di 120 ore totali) in quanto tale ruolo comprende l'ideazione e la strutturazione del funzionamento corretto del software che si andrà a sviluppare. Il quantitativo di ore assegnate ci è sembrato quindi congruo a coprire gli impegni che prevediamo di dover affrontare.
    \item \textbf{Programmatore}: tale ruolo è preponderante rispetto agli altri in termini quantitativi, ma non in maniera netta. Per questo abbiamo assegnato 22 ore produttive individuali, per un totale di 132 ore.
    \item \textbf{Verificatore}: ruolo fondamentale per il controllo della documentazione e del codice prodotto. A tale ruolo abbiamo assegnato 18 ore produttive individuali, per un totale di 108 ore.
\end{itemize}

\vspace{0.5cm}

\subsection{Rotazione dei ruoli}
Il gruppo si impegna a eseguire una rotazione dei ruoli ogni \textbf{2 settimane}. \\ Questa decisione è stata presa dal gruppo ragionando approfonditamente su un intervallo di tempo che permettesse a ciascun membro di lavorare continuativamente, ricoprendo ogni singolo ruolo, per un monte ore tale da permettere un buon approfondimento del ruolo stesso e con la possibilità di raggiungere dei progressi concreti.\\\\

\noindent Legenda abbreviazioni utilizzate nella prossima tabella:
\vspace{-0.5\baselineskip}
\begin{itemize}[noitemsep]
    \item Resp: \textit{Responsabile}
    \item Amm: \textit{Amministratore}
    \item An: \textit{Analista}
    \item Proge: \textit{Progettista}
    \item Progr: \textit{Programmatore}
    \item Ver: \textit{Verificatore}
\end{itemize}

\newpage

\newcommand{\memberTable}[1]{

\renewcommand{\arraystretch}{1.5}
\rowcolors{2}{pari}{dispari}
\begin{longtable}{ %0.87
		>{\centering}M{0.22\textwidth} 
		>{\centering}M{0.08\textwidth}
		>{\centering}M{0.08\textwidth}
		>{\centering}M{0.08\textwidth} 
        >{\centering}M{0.08\textwidth} 
        >{\centering}M{0.08\textwidth} 
        >{\centering}M{0.08\textwidth} 
		>{\centering\arraybackslash}M{0.08\textwidth} 
		 }
	\rowcolorhead
	\headertitle{Membro} &	
    \headertitle{Resp.} &
	\headertitle{Amm.} &
	\headertitle{An.} & 
	\headertitle{Proge.} &
    \headertitle{Progr.} &
    \headertitle{Ver.} &
    \headertitle{TOT.}
	\endfirsthead	
	\endhead
	
	#1

\end{longtable}
\vspace{1em}

}
\memberTable{
    Bergamin Elia & 9 & 7 & 16 & 20 & 22 & 18 & \textbf{92}\tabularnewline
    Diviesti Filippo & 9 & 7 & 16 & 20 & 22 & 18 & \textbf{92}\tabularnewline
    Djossa Edgar & 9 & 7 & 16 & 20 & 22 & 18 & \textbf{92}\tabularnewline
    Chilese Elena & 9 & 7 & 16 & 20 & 22 & 18 & \textbf{92}\tabularnewline
    Pincin Matteo & 9 & 7 & 16 & 20 & 22 & 18 & \textbf{92}\tabularnewline
    Soranzo Andrea & 9 & 7 & 16 & 20 & 22 & 18 & \textbf{92}\tabularnewline
}

\section{Consegna progetto}
Il gruppo si impegna a terminare e successivamente a consegnare il progetto entro il \textbf{giorno 21/03/2025}.\\

